\PassOptionsToPackage{unicode=true}{hyperref} % options for packages loaded elsewhere
\PassOptionsToPackage{hyphens}{url}
%
\documentclass[oneside,8pt,french,]{extbook} % cjns1989 - 27112019 - added the oneside option: so that the text jumps left & right when reading on a tablet/ereader
\usepackage{lmodern}
\usepackage{amssymb,amsmath}
\usepackage{ifxetex,ifluatex}
\usepackage{fixltx2e} % provides \textsubscript
\ifnum 0\ifxetex 1\fi\ifluatex 1\fi=0 % if pdftex
  \usepackage[T1]{fontenc}
  \usepackage[utf8]{inputenc}
  \usepackage{textcomp} % provides euro and other symbols
\else % if luatex or xelatex
  \usepackage{unicode-math}
  \defaultfontfeatures{Ligatures=TeX,Scale=MatchLowercase}
%   \setmainfont[]{EBGaramond-Regular}
    \setmainfont[Numbers={OldStyle,Proportional}]{EBGaramond-Regular}      % cjns1989 - 20191129 - old style numbers 
\fi
% use upquote if available, for straight quotes in verbatim environments
\IfFileExists{upquote.sty}{\usepackage{upquote}}{}
% use microtype if available
\IfFileExists{microtype.sty}{%
\usepackage[]{microtype}
\UseMicrotypeSet[protrusion]{basicmath} % disable protrusion for tt fonts
}{}
\usepackage{hyperref}
\hypersetup{
            pdftitle={SAINT-SIMON},
            pdfauthor={Mémoires - Tome IV},
            pdfborder={0 0 0},
            breaklinks=true}
\urlstyle{same}  % don't use monospace font for urls
\usepackage[papersize={4.80 in, 6.40  in},left=.5 in,right=.5 in]{geometry}
\setlength{\emergencystretch}{3em}  % prevent overfull lines
\providecommand{\tightlist}{%
  \setlength{\itemsep}{0pt}\setlength{\parskip}{0pt}}
\setcounter{secnumdepth}{0}

% set default figure placement to htbp
\makeatletter
\def\fps@figure{htbp}
\makeatother

\usepackage{ragged2e}
\usepackage{epigraph}
\renewcommand{\textflush}{flushepinormal}

\usepackage{indentfirst}
\usepackage{relsize}

\usepackage{fancyhdr}
\pagestyle{fancy}
\fancyhf{}
\fancyhead[R]{\thepage}
\renewcommand{\headrulewidth}{0pt}
\usepackage{quoting}
\usepackage{ragged2e}

\newlength\mylen
\settowidth\mylen{...................}

\usepackage{stackengine}
\usepackage{graphicx}
\def\asterism{\par\vspace{1em}{\centering\scalebox{.9}{%
  \stackon[-0.6pt]{\bfseries*~*}{\bfseries*}}\par}\vspace{.8em}\par}

\usepackage{titlesec}
\titleformat{\chapter}[display]
  {\normalfont\bfseries\filcenter}{}{0pt}{\Large}
\titleformat{\section}[display]
  {\normalfont\bfseries\filcenter}{}{0pt}{\Large}
\titleformat{\subsection}[display]
  {\normalfont\bfseries\filcenter}{}{0pt}{\Large}

\setcounter{secnumdepth}{1}
\ifnum 0\ifxetex 1\fi\ifluatex 1\fi=0 % if pdftex
  \usepackage[shorthands=off,main=french]{babel}
\else
  % load polyglossia as late as possible as it *could* call bidi if RTL lang (e.g. Hebrew or Arabic)
%   \usepackage{polyglossia}
%   \setmainlanguage[]{french}
%   \usepackage[french]{babel} % cjns1989 - 1.43 version of polyglossia on this system does not allow disabling the autospacing feature
\fi

\title{SAINT-SIMON}
\author{Mémoires - Tome IV}
\date{}

\begin{document}
\maketitle

\hypertarget{chapitre-premier.}{%
\chapter{CHAPITRE PREMIER.}\label{chapitre-premier.}}

1702

~

{\textsc{Le roi de Pologne défait par le roi de Suède qui y perd le duc
d'Holstein-Gottorp, son beau-frère.}} {\textsc{- Landau investi par les
Impériaux.}} {\textsc{- Désertion du prince d'Auvergne, pendu en Grève
en effigie.}} {\textsc{- Artifices inutiles des Bouillon.}} {\textsc{-
Siège de Landau par le prince Louis de Bade, défendu par Mélac, où le
roi des Romains arrive et le prend.}} {\textsc{- Électeur de Bavière se
déclare pour la France et l'Espagne.}} {\textsc{- Mort du comte de
Soissons\,; son caractère et sa famille.}} {\textsc{- Canaples et son
mariage avec M\textsuperscript{lle} de Vivonne.}} {\textsc{- Mort du duc
de Coislin\,; son caractère\,; ses singularités.}} {\textsc{- Duc de
Coislin et Novion, premier président du parlement, à une thèse.}}
{\textsc{- Novion premier président.}} {\textsc{- Mélac\,; sa
récompense\,; son caractère\,; sa fin.}} {\textsc{- Mort de Petit,
médecin de Monseigneur\,; Boudin en sa place.}} {\textsc{- Maréchal de
Villeroy libre sans rançon.}} {\textsc{- Madame à la comédie publique.}}

~

Il y avait longtemps que la Pologne était le théâtre des plus fâcheux
troubles. Les succès du roi de Suède, à qui le czar, allié du roi de
Pologne, n'avait pu résister, firent naître à ce jeune conquérant le
dessein de détrôner son ennemi. Il remporta sur lui une victoire
complète, vers la mi-août, à dix lieues de Cracovie, qui achemina fort
ce grand dessein, et le roi de Pologne ne s'y croyant plus en sûreté, se
hâta de gagner la Saxe avec peu de suite. La victoire fut sanglante, et
acheva d'irriter le roi de Suède par la mort du duc d'Holstein-Gottorp,
son beau-frère, tué à ses côtés, qu'il aimait uniquement, et dont, dans
le transport de sa douleur, il jura de tirer la plus grande vengeance.

Le roi ne recevait pas de meilleures nouvelles du Rhin que de Flandre.
Brisach, Fribourg, le fort de Kehl, Philippsbourg, rendus par la paix de
Ryswick, resserraient extrêmement notre armée, et le palatin, beau-frère
de l'empereur, intimement lié à lui et mal avec le roi, qui avait
protégé hautement contre lui les droits de Madame, avait farci son pays
deçà le Rhin de troupes, et favorisé les retranchements du Spirebach
qu'on a vus si glorieux au maréchal de Choiseul, et qui présentement
nous arrêtaient tout court, et ôtaient à notre armée la communication de
Landau et la subsistance des vastes et fertiles plaines qui de là
s'étendent jusqu'à Mayence. Le marquis d'Huxelles et Mélac, gouverneur
de Landau, en avaient écrit tout l'hiver voyant ces préparatifs. Landau
ne valait rien\,; on l'avait augmenté, par l'avis de M. le maréchal de
Lorges d'un ouvrage sur une hauteur qui commandait, mais avec cela la
place était encore mauvaise. Huxelles vint lui-même remontrer le danger
de laisser accommoder le Spirebach aux ennemis, et de ne pas mieux
garnir Landau, dont la garnison n'était presque que de régiments
nouveaux. On était encore dans ce désir effréné de paix qui en donnait
espérance contre toute raison, et, pour le Rhin comme pour la Flandre,
dans cette léthargie qui devint sitôt après funeste. On répondit au
marquis d'Huxelles qu'on n'était en peine de rien de ce côté-là, et
qu'on était bien assuré que le siège de Landau était une chimère à
laquelle il ne serait seulement pas songé. On s'y trompa comme sur la
Flandre.

Catinat n'eut pas plutôt assemblé sa médiocre armée sous Strasbourg, que
sur la fin de juin il apprit que Landau était investi, et qu'il sut que
le Spirebach était une barrière qui de la montagne au Rhin lui ôtait
toute communication avec cette place, et ne lui laissait d'espace à se
promener que le court espace depuis Strasbourg jusqu'à ce retranchement
accommodé et garni à ne rien craindre. Ce fut donc à y pirouetter, et à
subsister aux dépens de la basse Alsace, que Catinat passa la campagne.

Le prince d'Auvergne servait dans cette armée avec son régiment de
cavalerie\,: c'était un gros garçon fort épais et fort désagréable,
extrêmement rempli de sa naissance et des chimères nouvelles de sa
famille. De quatre frères, il était, pour ainsi dire, le seul par
l'exhérédation, et tout à l'heure par la mort de l'aîné, et par la
prêtrise des deux autres. Son père avait avec lui des procédés fort
durs, et bien que juridiquement condamné en plusieurs tribunaux de faire
raison à ses enfants des biens de leur mère, ils n'en pouvaient rien
arracher. Une visite que le prince d'Auvergne alla faire au cardinal de
Bouillon dans son exil, en entrant en campagne, lui tourna apparemment
la tête. Un beau jour qu'il était de piquet, il alla visiter les gardes
du camp, et quand il y fut, piqua des deux et déserta aux ennemis comme
un cavalier. Il avait laissé sur sa table une lettre pour Chamillart,
par laquelle d'un style haut et troublé il lui marquait que, ne pouvant
obtenir de quoi vivre, il s'en allait en chercher en Bavière, auprès de
la sœur de son père, veuve sans enfants d'un oncle paternel de
l'électeur. Ce n'était pas pourtant qu'il n'eût six mille livres de
pension du roi. Il alla en effet à Munich\,; il y fut peu, passa en
Hollande, et dans le cours de l'hiver fut fait major général dans les
troupes de la république.

S'il ne se fût agi que de subsistance, il aurait pu représenter sa
situation au roi, lui en demander, ou la permission d'aller vivre à
Berg-op-Zoom sans servir contre lui. Mais les chimères de son oncle
l'avaient séduit. Il voyait trois fils au duc de Bouillon. Il pouvait
être dangereux de trop multiplier une suite de cadets, dont le rang de
prince étranger pourrait fatiguer, et qui serait mal soutenu par des
établissements. Celui de Berg-op-Zoom, qui n'était rien en France qu'un
revenu en temps de paix, avait une décoration en Hollande, par l'étendue
et la dignité de ce marquisat. Le prince d'Auvergne l'illustrait encore
par le rang que sa maison avait en France, et par les établissements de
son père et de ses oncles. Il se flattait surtout d'y être distingué par
sa parenté avec le feu roi Guillaume et le prince de Nassau, gouverneur
héréditaire de Frise, étant arrière-petit-fils de la maréchale de
Bouillon, fille du célèbre prince d'Orange, fondateur de la république
des Provinces-Unies. Enfin il comptait de rassembler en sa faveur les
créatures du roi Guillaume dans les troupes et dans l'État, et d'y
pouvoir être aidé et décoré par les nombreux parents de la maison de
Hohenzollern, dont était sa mère, répandus dans la basse Allemagne. Il
espéra de faire aisément une figure considérable avec tous ces appuis,
et pour se concilier la faveur du pensionnaire Heinsius, maître en
Hollande, et des autres principales créatures du roi Guillaume, qui lui
étaient unies, et qui comme Heinsius avaient hérité de la haine de leur
stathouder pour le roi et pour la France, et ôter de plus toute sorte
d'ombrage, il préféra la voie de la désertion à toute autre de s'aller
établir en Hollande.

J'avance ici de près d'une année la suite de cette désertion, pour
n'avoir plus à y revenir. Elle fit grand bruit\,; les Bouillon la
blâmèrent, mais plaignirent son malheur. Ils appuyèrent sur sa retraite
à Munich, pour la rendre moins criminelle\,; ils trouvèrent que la
manière n'était que sottise sans mauvaise intention. Le roi, qui ne crut
pas y perdre grand'chose, et qui aimait M. de Bouillon, laissa tomber la
chose, et le monde, séduit par cet exemple et par les amis des Bouillon,
se tourna à la compassion et bientôt au silence. Il se rompit quand on
le vit au service de Hollande\,; le roi en fut piqué. Cette démarche lui
fut présentée par M. de Bouillon, comme le comble de leur douleur, mais
en même temps comme l'effet d'une jeunesse brave, et honteuse de
l'oisiveté au milieu des feux de la guerre, et toujours parmi des gens
de guerre. Avec ce tour adroit, la colère du roi fut émoussée\,; mais
bientôt après, le prince d'Auvergne se lâcha en propos fort licencieux
pour plaire à ses nouveaux maîtres, se montra plus cruel qu'aucun des
ennemis au sac de Venloo, qu'ils reprirent cette même campagne, et
allait partout montrant son épée, qu'il criait être celle de M. de
Turenne, et qu'il rendrait aussi fatale à la France qu'elle y avait été
victorieuse. Ce coup ne put être paré, et le roi voulut que le procès
fût fait et parfait à ce déserteur.

Les Bouillon hors d'espérance de l'empêcher, et accoutumés à tirer des
honneurs et des distinctions des félonies et des ignominies, osèrent
travailler à obtenir que ce procès fût fait en forme de pairie, ou au
moins avec des différences d'un particulier. C'est ce qui était inconnu
au parlement et contre toutes ses règles. Le rang de prince étranger,
accordé par l'échange de Sedan, était le principal obstacle qui en avait
jusqu'alors empêché l'enregistrement au parlement, qui ne reconnaît la
qualité de prince que dans les princes du sang, ni de rang et de
distinction que ceux du royaume. Cette barrière n'ayant pu s'enfreindre,
MM. de Bouillon se rabattirent à faire pitié au roi par leur douleur, et
par celle qui se renouvellerait longtemps tous les jours, si l'affaire,
d'abord instruite et jugée au Châtelet, puis portée au parlement, leur
en faisait essuyer toutes les longueurs, et firent si bien par leur
artifice qu'elle alla droit au parlement. Elle n'y dura pas\,: il y fut
rendu un arrêt qui condamna ce déserteur, dans les termes les plus
communs à tous les plus simples particuliers, à être pendu, et, en
attendant qu'il pût être appréhendé au corps, à être pendu en effigie,
ce qui fut exécuté en place de Grève, en plein jour\,; et le tableau
inscrit de son nom et de l'arrêt y demeura trois jours à la potence.
Mais pour que MM. de Bouillon ne pussent tirer avantage d'avoir évité le
Châtelet, le premier président, avisé par ses amis les Noailles, de
longue main en procès et ennemis des Bouillon, fit écrire sur les
registres du parlement, que ce procès criminel avait été directement
porté à la grand'chambre, et jugé par elle et la Tournelle\footnote{La
  Tournelle était une chambre criminelle du parlement de Paris, qui
  tirait son nom de ce que les membres qui la composaient étaient
  fournis à tour de rôle par les autres chambres du parlement de Paris.}
assemblées seulement, ce qui se pratique à l'égard de tout noble accusé
de crime, non par aucune distinction particulière, mais eu égard à la
qualité du crime, comme on en use aussi pour celui de duel\,: tellement
que MM. de Bouillon n'eurent que les deux potences des deux fils du
comte d'Auvergne, à peu d'années de distance l'une de l'autre, sans que
leur hardiesse et leur intrigue en ait pu tirer aucun fruit.

Le siège de Landau n'avançait pas autant que le prince Louis de Bade qui
le faisait l'avait espéré, et Mélac, gouverneur de la place, profitait
de tout pour en allonger la défense. On se repentit trop tard de n'y
avoir pas pourvu à temps. On voulut le réparer. Villars eut ordre de
mener un très gros détachement de l'armée de Flandre à Catinat, et
celui-ci de tout tenter pour secourir la place. Lé roi des Romains y
était arrivé pour faire à ce siège ses premières armes, et, suivant la
coutume allemande, la reine son épouse l'avait accompagné et alla tenir
sa cour à Heidelberg, en attendant la fin de la campagne. Catinat et
Villars cherchèrent tous les moyens possibles de pénétrer jusqu'à
Landau, mais le Spirebach, de longue main bien retranché et garni du
Rhin jusqu'aux montagnes, leur parut impénétrable. Ils ne trouvèrent pas
plus de facilité par derrière les montagnes\,; tellement qu'ils
mandèrent à la cour qu'il n'y fallait pas songer. Là-dessus Catinat
reçut ordre d'envoyer Villars vers Huningue avec la plus grande partie
de son armée, pour donner de la jalousie aux Impériaux et entreprendre
même ce que l'occasion lui pourrait offrir. L'électeur de Bavière venait
de se déclarer\,: il offrait d'amener vingt-cinq mille hommes sur les
bords du Rhin\,; on voulait le favoriser et le joindre\,; ce fut l'objet
de cette division de l'armée de Catinat vers le haut Rhin. Cependant
Landau, à bout de tout, et ouvert de toutes parts, capitula le 10
septembre, ayant tenu plus d'un mois au delà de toute espérance. Les
conditions furent telles que Mélac les proposa, et les plus honorables
et avantageuses en considération de son admirable défense. Le roi des
Romains lui fit l'honneur de le faire manger à sa table, et voulut qu'il
vît son armée et qu'elle lui rendît tous ceux des feld-maréchaux. Peu de
jours après, il retourna à Vienne avec la reine sa femme.

De part et d'autre le siège fut meurtrier, et le comte de Soissons y
mourut en peu de jours d'une blessure qu'il y reçut. Il était frère aîné
du prince Eugène et neveu paternel et cadet de ce fameux muet le prince
de Carignan\,; le prince Louis de Bade et le comte de Soissons étaient
enfants du frère et de la sœur. Le comte de Soissons père était fils du
prince Thomas, qui a fait tant de bruit et de mouvements en France et en
Savoie, fils et frère de ses ducs, et mari de la dernière princesse du
sang de la branche de Soissons, sœur du comte de Soissons tué à la
bataille de la Marfée, dite de Sedan, qu'il venait de gagner. Le comte
de Soissons Savoie, neveu de ce prince du sang, attiré en France par les
biens de sa mère et les établissements que son père y avait eus, y avait
épousé une Mancini, nièce du cardinal Mazarin, pour laquelle, au mariage
du roi, il inventa la charge de surintendante de la maison de la
reine\,; et en même temps de la reine mère qui, non plus que toutes les
autres reines, n'en avait jamais eu, pour son autre nièce Martinozzi,
femme du prince de Conti. La brillante faveur, les disgrâces, les
étranges aventures de la comtesse de Soissons qui la firent fuir à
Bruxelles ne sont pas de mon sujet. Elle fut fort accusée d'avoir fait
empoisonner son mari à l'armée, en 1673. Il était gouverneur de
Champagne et colonel général des Suisses et Grisons. Sa sœur la
princesse de Bade fut longtemps dame du palais de la reine, qui n'eut et
ne prétendit jamais aucune préférence sur les duchesses et les
princesses de la maison de Lorraine, qui étaient aussi dames du palais,
et qui toutes roulaient ensemble sans distinction entre elles et
faisaient le même service. Elle eut part à la disgrâce de la princesse
de Carignan sa mère, et fut remerciée. Le prince Louis de Bade, si connu
à la tête des armées de l'empereur, était filleul du roi, et avait passé
en France sa première jeunesse.

Le comte de Soissons, sans père et ayant sa mère en situation de n'oser
jamais revenir en France, y fut élevé par la princesse de Carignan, sa
grand'mère, avec le prince Eugène et d'autres frères qu'il perdit, et
ses sœurs dont j'ai parlé lors du mariage de M\textsuperscript{me} la
duchesse de Bourgogne. C'était un homme de peu de génie, fort adonné à
ses plaisirs, panier percé qui empruntait volontiers et ne rendait
guère. Sa naissance le mettait en bonne compagnie, son goût en mauvaise.
À vingt-cinq ans, amoureux fou de la fille bâtarde de La
Cropte-Beauvais\footnote{Voy., à la fin de ce volume, la note
  rectificative remise à M. le duc de Saint-Simon par M. le comte de
  Chantérac. Elle établit qu'Uranie de La Cropte-Beauvais était fille
  légitime de l'écuyer de M. le Prince et de Charlotte-Martel.}, écuyer
de M. le Prince le héros, il l'épousa au désespoir de la princesse de
Carignan, sa grand'mère, et de toute sa parenté. Elle était belle comme
le plus beau jour, et vertueuse, brune, avec ces grands traits qu'on
peint aux sultanes et à ces beautés romaines, grande, l'air noble, doux,
engageant, avec peu ou point d'esprit. Elle surprit à la cour par
l'éclat de ses charmes qui firent en quelque manière pardonner presque
au comte de Soissons\,; l'un et l'autre doux et fort polis.

Elle était si bien bâtarde, que M. le Prince, sachant son père à
l'extrémité, à qui on allait porter les sacrements, monta à sa chambre
dans l'hôtel de Condé pour le presser d'en épouser la mère\,; il eut
beau dire et avec autorité et avec prières, et lui représenter l'état
où, faute de ce mariage, il laissait une aussi belle créature que la
fille qu'il en avait eue, Beauvais fut inexorable, maintint qu'il n
avait jamais promis mariage à cette créature, qu'il ne l'avait point
trompée, et qu'il ne l'épouserait point\,; il mourut ainsi. Je ne sais
où, dans la suite, elle fut élevée ni où le comte de Soissons la vit. La
passion de l'un et la vertu inébranlable de l'autre fit cet étrange
mariage.

On a vu en son temps comment le comte de Soissons était sorti de France,
et comment il avait été rebuté partout où il avait offert ses services.
Ne sachant plus où donner de la tête, il eut recours à son cadet le
prince Eugène et à son cousin le prince Louis de Bade, qui le firent
entrer au service de l'empereur, où il fut tué presque aussitôt après.
Sa femme, qui fut inconsolable et qui était encore belle à surprendre,
se retira en Savoie dans un couvent éloigné de Turin où M. de Savoie
enfin voulut bien la souffrir. Leurs enfants, dont le prince Eugène
voulait faire les siens, sont tous morts à la fleur de leur âge, en
sorte que le prince Eugène, qui avait deux abbayes et n'a point été
marié, a fini cette branche sortie du fameux duc Charles-Emmanuel,
vaincu par Louis XIII en personne au célèbre pas de Suse.

Canaples, frère du feu duc et maréchal de Créqui, était le dernier de
cette branche de la maison de Blanchefort depuis la mort du marquis de
Créqui son neveu. Son père était puîné des ducs de Lesdiguières et frère
du grand-père du duc de Lesdiguières, resté aussi seul de cette branche,
et neveu à la mode de Bretagne de Canaples. Le duc de Créqui n'avait
laissé que la duchesse de La Trémoille, et son duché-pairie, érigé pour
lui en 1663, auquel ses frères n'avaient point été appelés, éteint,
celui de Lesdiguières passait à toute la branche de Créqui qui en
sortait, et Canaples en assurant ses biens aux enfants du maréchal de
Créqui son frère, s'était opiniâtrement réservé ses droits à cet égard.
Il était cadet du duc de Créqui, et aîné du maréchal\,; il avait
soixante-quinze ans lorsque la branche du maréchal de Créqui fut éteinte
par la mort du marquis de Créqui à Luzzara. Tout aussitôt Canaples, plus
riche qu'il n'était par cette succession, et ayant toujours le duché de
Lesdiguières en tête, malgré la jeunesse et la santé de celui qui en
était revêtu, et de sa femme, fille du maréchal de Duras, qui n'avaient
point encore d'enfants, voulut se marier pour continuer la race. C'était
un homme si borné que jamais ses frères n'en avaient pu rien faire. Le
maréchal de Villeroy, fils d'une Créqui de la branche de Lesdiguières et
son cousin germain, lui procura le commandement de son gouvernement de
Lyon à la mort de l'archevêque son oncle qui l'avait eu toute sa vie.
Canaples n'y sut jamais ce qu'il faisait, jusque-là que les dames qui
allèrent au-devant de M\textsuperscript{me} la duchesse de Bourgogne au
pont Beauvoisin, et qui séjournèrent quelque temps à Lyon, me contèrent
au retour qu'elles y avaient rencontré Canaples dans les rues allant au
pas et donnant des bénédictions à droite et à gauche. Il l'avait vu
faire ainsi à l'archevêque Saint-Georges qui y était lors, et avait
succédé à l'oncle de Villeroy. Canaples croyait de son droit d'en faire
autant, et prétendait aussi donner les dimissoires et se mêler de la
discipline intérieure du clergé. Il fit enfin tant de sottises, quoique
le meilleur homme du monde, qu'il fallut bien l'en retirer. Il revint
donc ennuyer la cour et la ville et toujours fort paré.

Il songea, voulant se marier sur la mort de son neveu, à
M\textsuperscript{lle} de Vivonne, qui n'était plus jeune, et qui
n'avait que beaucoup d'esprit, de vertu et de naissance, et pas un
denier vaillant. Le maréchal de Vivonne, frère de M\textsuperscript{me}
de Montespan, était mort, tellement ruiné que sa veuve, dont il avait eu
des biens immenses, et qui avait aussi bien mangé de son côté, vivait à
grand'peine dans la maison de son intendant. M\textsuperscript{lle} de
Vivonne, sœur du feu duc de. Mortemart gendre de M. Colbert, et sœur de
la duchesse d'Elbœuf et de M\textsuperscript{me} de Castries, était
auprès de M\textsuperscript{me} de Montespan qui l'avait retirée chez
elle, et qui lui donnait jusqu'à ses habits\,; elle la trouva trop
heureuse d'épouser ce vieillard pour avoir du pain, et fit le mariage.
Comme il commençait à s'ébruiter, le cardinal de Coislin en parla à
Canaples qu'il trouvait bien vieux pour se marier. Il lui dit qu'il
voulait avoir des enfants. «\,Des enfants\,! monsieur, lui répliqua le
cardinal\,; mais elle est si vertueuse\,!» La compagnie éclata de rire,
d'autant plus que le cardinal, très pur dans ses mœurs, l'était
singulièrement aussi dans ses discours. Le sien fut vrai, et le mariage
fut stérile.

Le duc de Coislin mourut fort peu après, qui fut une grande affliction
pour le cardinal son frère, et une perte pour tous les honnêtes gens.
C'était un très petit homme sans mine, mais l'honneur, la vertu, la
probité même et la valeur même, qui, avec de l'esprit, était un
répertoire exact et fidèle avec lequel il y avait infiniment et très
curieusement à apprendre, d'une politesse si excessive qu'elle désolait,
mais qui laissait place entière à la dignité. Il avait été lieutenant
général avec réputation et mestre de camp général de la cavalerie après
Bussy-Rabutin, de la disgrâce duquel il ne voulut pas profiter pour la
fixation du prix, et qu'il vendit et quitta le service brouillé avec M.
de Louvois. C'était, avec tant de bonnes qualités qui lui conservèrent
toujours une véritable considération et de la distinction du roi, un
homme si singulier que je ne puis me refuser d'en rapporter quelques
traits.

Un des rhingraves, prisonnier à un combat où se trouva le duc de
Coislin, lui échut\,; il lui voulut donner son lit, par composition un
matelas. Tous deux se complimentèrent tant et si bien qu'ils couchèrent
tous deux par terre des deux côtés du matelas. Revenu à Paris, le
rhingrave, qui avait eu liberté d'y venir, le fut voir. Grands
compliments à la reconduite\,; le rhingrave, poussé à bout, sort de la
chambre et ferme la porte par dehors à double tour. M. de Coislin n'en
fait point à deux fois\,; son appartement n'était qu'à quelques marches
du rez-de-chaussée\,; il ouvre la fenêtre, saute dans la cour et se
trouve à la portière du rhingrave avant lui, qui crut que le diable
l'avait porté là. Il était vrai pourtant qu'il s'en démit le pouce\,;
Félix, premier chirurgien du roi, le lui remit. Étant guéri, Félix
retourna voir comment cela allait, et trouva la guérison parfaite. Comme
il sortait, voilà M. de Coislin à vouloir lui ouvrir la porte, Félix à
se confondre et à se défendre. Dans ce conflit, tirant tous deux la
porte, le duc quitta prise subitement et remue sa main\,; c'est que son
pouce s'était redémis\,; et il fallut que Félix y travaillât
sur-le-champ. On peut croire qu'il en fit le conte au roi, et qu'on en
rit beaucoup.

On ne tarirait point sur ses civilités outrées. Nous le rencontrâmes à
un retour de Fontainebleau, M\textsuperscript{me} de Saint-Simon et moi,
à pied avec M. de Metz, son fils, sur le pavé de Ponthierry, où son
carrosse s'était rompu\,; nous envoyâmes les prier de monter avec nous.
Les messages ne finissant point, je fus contraint de mettre pied à terre
malgré la boue, et de l'aller prier de monter dans mon carrosse. M. de
Metz rageait de ses compliments, et enfin le détermina. Quand il eut
consenti, et qu'il n'y eut plus qu'à gagner mon carrosse, il se mit à
capituler et à protester qu'il n'ôterait point la place à ces
demoiselles\,; je lui dis que ces demoiselles étaient deux femmes de
chambre, bonnes de reste à attendre que son carrosse fût raccommodé, et
à venir dedans\,; nous eûmes beau faire, M. de Metz et moi, il lui
fallut promettre qu'il en demeurerait une avec nous. Arrivés au
carrosse, ces femmes de chambre descendirent, et pendant les
compliments, qui ne furent pas courts, je dis au laquais qui tenait la
portière de la fermer dès que je serais monté, et d'avertir le cocher de
marcher sur-le-champ. Cela fut fort bien exécuté\,; mais à l'instant
voilà M. de Coislin à crier qu'il s'allait jeter si on n'arrêtait pour
prendre cette demoiselle, et tout aussitôt à l'exécuter si étrangement
que j'eus peine à me jeter à temps à la ceinture de ses chausses pour le
retenir\,; et lui, le visage contre le panneau de la portière en dehors,
criait qu'il se jetterait, et tirait contre moi. À cette folie, je criai
d'arrêter\,; il se remit à peine et maintint qu'il se serait jeté. La
demoiselle femme de chambre fut rappelée, qui, en allant au carrosse
rompu, avait amassé force crotte qu'elle nous apporta et qui pensa nous
écraser, M. de Metz et moi, dans ce carrosse à quatre.

Son frère, le chevalier de Coislin, rustre, cynique et chagrin, tout
opposé à lui, se vengea bien un jour de l'ennui de ses compliments. Les
trois frères, avec un quatrième de leurs amis, étaient à un voyage du
roi. À chaque logis les compliments ne finissaient point, et le
chevalier s'en désespérait. Il se trouva à une couchée une hôtesse de
bel air et jolie, chez qui ils furent marqués. La maison bien meublée,
et la chambre d'une grande propreté. Grands compliments en arrivant,
plus encore en partant. M. de Coislin alla voir son hôtesse dans la,
chambre où elle s'était mise. Ils crurent qu'ils ne partiraient point.
Enfin les voilà en carrosse et le chevalier de Coislin beaucoup moins
impatient qu'à son ordinaire. Ses frères crurent que la gentillesse de
l'hôtesse et l'agrément du gîte lui avaient pour cette fois adouci les
mœurs. À trois lieues de là et qu'il pleuvait bien fort, voilà tout à
coup le chevalier de Coislin qui se met à respirer au large et à rire.
La compagnie, qui n'était pas accoutumée à sa belle humeur, demande à
qui il en a\,; lui à rire encore plus fort. À la fin il déclare à son
frère qu'au désespoir de tous ses compliments à tous les gîtes, et
poussé à bout par ceux du dernier, il s'était donné la satisfaction de
se bien venger, et que, pendant qu'il était chez leur hôtesse, il s'en
était allé dans la chambre où son frère avait couché et y avait tout au
beau milieu poussé une magnifique selle, qui l'avait d'autant plus
soulagé qu'on ne pouvait douter dans la maison qu'elle ne fût de celui
qui avait occupé cette chambre. Voilà le duc de Coislin outré de colère,
les autres morts de rire. Mais le duc furieux, après avoir dit tout ce
que le désespoir peut inspirer, crie au cocher d'arrêter, et au valet de
chambre d'approcher, veut monter son cheval et retourner à l'hôtesse se
laver du forfait ou accuser et déceler le coupable. Ils virent longtemps
l'heure qu'ils ne pourraient l'empêcher, et il en fut plusieurs jours
tout à fait mal avec son frère\footnote{Quoique Saint-Simon ait déjà
  raconté cette anecdote (t. II, p.~255), nous n'avons pas cru devoir
  supprimer, comme les anciens éditeurs, un récit qui présente des
  variantes nombreuses avec le précédent.}.

À un voyage que le roi fit à Nancy, il lui arriva deux aventures d'une
autre espèce. Le duc de Créqui, qui n'était point en année, se trouva
mal logé en arrivant à Nancy. Il était brutal et accoutumé à l'être bien
davantage par l'air de faveur et d'autorité où il s'était mis à la
cour\,; il s'en alla déloger le duc de Coislin, qui, en arrivant un
moment après, trouva ses gens sur le pavé, dont il apprit la cause. Les
choses alors étaient sur un autre pied\,: M. de Créqui était son ancien,
il ne dit mot\,; mais de ce pas, il s'en va avec tous ses gens à la
maison marquée pour le maréchal de Créqui, lui fait le mémé trait qu'il
venait d'essuyer de son frère et s'établit\,; arrive le maréchal de
Créqui, dont l'impétuosité s'alla jeter sur la maison de Cavoye, qu'il
délogea à son tour, pour lui apprendre à faire les logements de manière
à éviter ces cascades.

Le duc de Coislin avait la fantaisie de ne pouvoir souffrir qu'on lui
donnât le dernier, plaisanterie qui fait courre après celui qui l'a
donné et qui ne passe guère la première jeunesse. M. de Longueville, en
ce même lieu de Nancy où la cour séjourna quelque temps, donna le mot à
deux de ses pages qui lui portaient des flambeaux\,; et, comme chacun se
retirait là à pied du coucher du roi, touche le duc de Coislin, lui dit
qu'il a le dernier et se met à courir, et le duc de Coislin après. Le
devant un peu gagné, M. de Longueville se jette dans une porte, voit
passer devant M. de Coislin courant tant qu'il pouvait, et s'en va
tranquillement se coucher, tandis que les pages avec leurs flambeaux
menèrent M. de Coislin aux quatre coins et au milieu de la ville, tant
que n'en pouvant plus il quitta prise et s'en alla chez lui tout en
eau\,; ce fut une plaisanterie dont il fallut bien rire, mais qui ne lui
plut pas trop.

Une aventure plus sérieuse, et à laquelle il n'y avait pas moyen de
s'attendre, montra qu'il savait bien prendre son parti. Le second fils
de M. de Bouillon, qui par la mort de son aîné fut duc de Bouillon après
son père, et avait en attendant porté le nom du duc d'Albret, père du
duc de Bouillon d'aujourd'hui, était élevé pour l'Église et soutenait
une thèse en Sorbonne en grand apparat. En ces temps-là les princes du
sang allaient aux cérémonies des personnes distinguées. M. le Prince, M.
le Duc, depuis prince de Condé, et MM. les princes de Conti, les deux
frères enfants, étaient à cette thèse. M. de Coislin y arriva
incontinent après, et, comme il était alors tout à la queue des ducs, il
laissa plusieurs fauteuils entre lui et le coin aboutissant à celui des
prélats. Les princes du sang avaient les leurs hors de rang, en face de
la chaire de celui qui présidait à la thèse. Arrive Novion, premier
président, avec plusieurs présidents à mortier, qui, complimentant les
princes du sang, se glisse au premier fauteuil joignant le coin susdit.
Le duc de Coislin, bien étonné de cette folie, le laisse asseoir, et
comme en s'asseyant il tourne la tête vers le cardinal de Bouillon,
assis dans le fauteuil joignant ce même coin à la tête du côté des
prélats, M. de Coislin se lève, prend un fauteuil, le plante devant le
premier président et s'assied\,; cela se fit si brusquement qu'il fut
plus tôt exécuté qu'aperçu. Aussitôt grande rumeur, et M. de Coislin à
serrer le premier président du derrière de sa chaise à l'empêcher de
remuer, et se tenant bien ferme dans le sien. Le cardinal de Bouillon
essaya de s'entremettre\,; M. de Coislin répondit qu'il était où il
devait être, puisque le premier président oubliait ce qu'il lui devait,
qui, interdit de l'affront et de la rage de l'essuyer sans pouvoir
branler, ne savait que faire. Les présidents à mortier, bien
effarouchés, murmuraient fort entre eux\,; enfin le cardinal de Bouillon
d'un côté, et ses frères par le bas bout où ils faisaient les honneurs,
allèrent à M. le Prince le supplier de vouloir bien faire en sorte de
terminer cette scène, qui cependant faisait taire l'argument. M. le
Prince alla au duc de Coislin qui lui fit excuse de ce qu'il ne se
levait point, mais qui ne voulait point désemparer son homme. M. le
Prince blâma fort le premier président ainsi en présence, puis proposa à
M. de Coislin de se lever pour laisser au premier président la liberté
de se lever aussi et de sortir. M. de Coislin résista et ne menaçait pas
moins que de le tenir là toute la thèse. Vaincu enfin par les prières de
M. le Prince et des Bouillon, il consentit à se lever, à condition que
M. le Prince se rendrait garant que le premier président sortirait à
l'instant, et qu'en se levant il n'aurait point quelque autre tour de
passe-passe à en craindre, ce fut le terme dont il se servit. Novion
balbutiant en donna sa parole\,; le duc dit qu'il la méprisait trop et
lui aussi pour la recevoir et qu'il voulait celle de M. le Prince\,; il
la donna. Aussitôt M. de Coislin se lève, range son fauteuil en disant
au premier président\,: «\,Allez-vous-en, monsieur, allez-vous-en\,;»
qui sortit aussi dans la dernière confusion, et alla regagner son
carrosse avec les présidents à mortier, en même temps M. de Coislin prit
sa chaise, la porta où elle était d'abord et s'y remit.

M. le Prince aussitôt lui vint faire compliment, les trois autres
princes du sang aussi, et tout ce qu'il y avait là de plus considérable
à leur exemple. J'oubliais que d'abord MM. de Bouillon avaient employé
la ruse et fait avertir M. de Coislin qu'on le demandait à la porte pour
quelque chose de pressé, et qu'il répondit, en montrant le premier
président derrière lui\,: «\,Rien de si pressé que d'apprendre à M. le
premier président ce qu'il me doit, et rien ne nie fera sortir d'ici,
que M. le premier président que voilà derrière moi n'en sorte le
premier.\,» Le duc de Coislin demeura là encore un argument entier, puis
s'en alla chez lui. Les quatre princes du sang l'allèrent voir le jour
même, et la plupart de tout ce qui avait vu ou su son aventure, en sorte
que sa maison fut pleine jusque fort tard.

Le lendemain il alla au lever du roi, qui, par des gens revenus de Paris
après la thèse, avait su ce qui s'était passé. Dès qu'il vit le duc de
Coislin, il lui en parla, et, devant toute la cour, le loua de ce qu'il
avait fait, et blâma le premier président en le taxant d'impertinent qui
s'oubliait, terme fort éloigné de la mesure des paroles du roi. Son
lever fini, il fit entrer le duc dans son cabinet, et se fit non
seulement conter, mais figurer la chose\,; cela finit par dire au duc de
Coislin qu'il lui ferait justice\,; puis manda le premier président à
qui il lava la tête, lui demanda où il avait pris qu'il pût disputer
quoi que ce fût aux ducs hors la séance du parlement, sur quoi il ne
décidait rien encore, et lui ordonna d'aller chez le duc de Coislin à
Paris lui demander pardon, et le trouver, non pas aller simplement à sa
porte. Il est aisé de comprendre la honte et le désespoir où se sentit
Novion d'avoir à faire une démarche si humiliante et après ce qui venait
de lui arriver\,; il fit parler au duc de Coislin par le duc de Gesvres
et par d'autres, et fit si bien en vingt-quatre heures que le duc de
Coislin', content de son avantage et d'être le maître de faire subir au
premier président toute la rigueur du commandement qu'il avait reçu à
son égard, eut la générosité de l'en dispenser et de se charger encore
envers le roi d'avoir fermé sa porte au premier président, qui, sûr de
n'être pas reçu, alla chez lui avec moins de répugnance. Le roi loua
fort le duc de Coislin de ce procédé, qui {[}fut cause{]} que le premier
président n'osa se plaindre.

C'était la vérité même que le duc de Coislin. Il était fort des amis de
mon père, il me recevait avec bonté, amitié, et parlait volontiers
devant moi. Je lui ai ouï faire ce récit entre beaucoup d'autres
anecdotes curieuses, et ce récit même plusieurs fois à moi, puis devant
moi à d'autres personnes. C'était un homme tellement sensible, que le
cardinal son frère obtint sa survivance de premier aumônier pour l'abbé
de Coislin, sans avoir jamais laissé apercevoir à son frère qu'il
songeât à la demander, dans la crainte que, s'il était, refusé, il n'en
fût trop fortement touché\,; et qu'il avait aussi obtenu du roi, par la
même raison, de ne jamais refuser son frère pour Marly, en sorte qu'il
ne demandait jamais sans y aller. La vérité est qu'il n'en abusait pas.
Il n'était pas fort vieux, mais perdu de goutte, qu'il avait quelquefois
jusqu'aux yeux, au nez et a la langue, et dans cet état sa chambre ne
désemplissait pas de la meilleure compagnie de la cour et de la ville,
et dès qu'il pouvait marcher, il allait à la ville et à la cour, où il
était aimé généralement et, considéré et compté. Il était fort pauvre,
sa mère très riche l'ayant survécu. Il ne laissa que deux fils et la
duchesse de Sully, et il vit toute la fortune de son frère et de son
second fils.

Ce premier président de Novion était un homme vendu à l'iniquité, à qui
l'argent et les maîtresses obscures faisaient tout faire. On gémit
longtemps au palais de ses caprices, et les plaideurs de ses injustices.
Devenu plus hardi, il se mit à changer les arrêts en les signant, et à
prononcer autrement qu'il n'avait été opiné à l'audience. À la fin, des
conseillers, surpris que tout un côté eût opiné comme ils avaient ouï
prononcer, en demandèrent raison à leurs confrères. Ceux-ci à leur tour
furent étrangement surpris ayant cru que ce côté avait pris l'opinion
qui avait formé l'arrêt, lequel se trouva ainsi de la seule voix du
premier président\,; leur attention se réveilla, et ils trouvèrent que
la même chose n'était plus rare. Ils s'informèrent aux rapporteurs et
aux greffiers. Ces derniers s'étaient bien souvent aperçus de quelque
chose, mais ils n'avaient osé parler. Enfin, encouragés par les
conseillers, ils revirent les arrêts des procès par écrit, signés par le
premier président, ils les montrèrent aux rapporteurs\,; il s'en trouva
plusieurs d'extrêmement altérés. Les plaintes en furent portées au roi,
et si bien prouvées, qu'il commanda à Novion de se retirer, et tout à la
fin de 1689 Harlay fut mis en sa place. Il avait succédé à Lamoignon en
1678, de la femme duquel il était cousin germain. Il vécut encore quatre
ans dans l'abandon et dans l'ignominie, et mourut à sa campagne sur la
fin de 1693, à soixante-treize ans. Nous verrons son petit-fils en la
même place, très indigne de toutes celles par lesquelles il passa.

La cour était à Fontainebleau du 19 septembre. Mélac y arriva et salua
le roi le 4 octobre, et, le lendemain au soir, fut longtemps avec le roi
et Chamillart chez M\textsuperscript{me} de Maintenon. Chamillart le
mena de là chez lui, et lui détailla ce que le roi lui donnait, qui avec
la continuation de ses appointements de gouverneur de Landau, et quinze
mille livres de pension pour l'avoir si bien défendu, montait à
trente-huit mille livres de rente. Mélac, loué et caressé du roi,
applaudi de tout le monde, crut avoir mérité des honneurs. Il insista
encore plus lorsqu'il les vit donner incontinent après, comme je vais le
rapporter, à qui n'eût pas eu le temps de les aller chercher de l'autre
côté du Rhin, si Landau n'eût tenu plus de six semaines au delà de toute
espérance. Mélac outré de douleur se retira à Paris. Il n'avait ni femme
ni enfants. Il s'y retira avec quatre ou cinq valets, et s'y consuma
bientôt de chagrin dans une obscurité qu'il ne voulut adoucir par aucun
commerce.

C'était un gentilhomme de Guyenne, de beaucoup d'esprit, même fort orné,
de beaucoup d'imagination, et dont le trop de feu nuisait quelquefois à
ses talents pour la guerre, et souvent à sa conduite particulière, bon
partisan, hardi dans ses projets, et concerté dans son exécution,
surtout fort désintéressé. Il n'avait de patrie que l'armée et les
frontières, et toute sa vie avait fait la guerre, été et hiver, presque
toujours en Allemagne. La manie de se rendre terrible aux ennemis
l'avait rendu singulier\,; il avait réussi à faire peur de son nom par
ses fréquentes entreprises, et à tenir alerte vingt lieues à sa portée
de pays ennemi. Il se divertissait à se faire croire sorcier à ces
peuples, et il en plaisantait le premier. Il était assez épineux et très
fâcheux à ceux qu'il soupçonnait de ne lui vouloir pas de bien, et trop
facile à croire qu'on manquait d'égards pour lui. D'ailleurs, doux et
très bon homme, et qui souffrait tout de ses amis\,; fort commode et
jamais incommode à un général et à tous ses supérieurs, mais fort peu
aux intendants\,; sans intrigue et sans commerce avec le secrétaire
d'État de la guerre, et comme il avait les mains fort nettes, fort libre
sur ce qui ne les avait pas\,; sobre, simple et particulier\,; toujours
ruminant ou parlant guerre avec une éloquence naturelle, et un choix de
termes qui surprenait, sans en chercher aucun. Il était particulièrement
attaché à MM. de Duras et de Lorges, surtout à mon beau-père, qui me le
recommanda autant que je le pourrais, quand il ne serait plus. Il prit
de travers une politesse du chevalier d'Asfeld chez le maréchal de
Choiseul, contre lequel il s'emporta étrangement en présence de
plusieurs officiers généraux. M. de Chamilly m'en vint avertir. J'allai
trouver le maréchal, qui aurait pu le punir et de la chose et du manque
de respect chez lui, mais qui voulut bien ne pas songer à ce qui le
regardait. Je vis après Mélac, et je ne puis mieux témoigner combien il
était endurant pour ses amis que de dire que je ne le ménageai point,
jusqu'à en être honteux à mon âge et seulement colonel, et lui
lieutenant général ancien et en grande réputation. Il m'avoua son tort
et fit tout ce que je voulus. Chamilly, le marquis d'Huxelles et
plusieurs autres continrent le chevalier d'Asfeld, depuis maréchal de
France comme eux, et parvinrent à faire embrasser Mélac et lui, et
jamais depuis il n'en a été mention entre eux. À tout prendre, Mélac
était un excellent homme de guerre, et un bon et honnête homme\,;
pauvre, sobre et frugal, et passionné pour le bien public.

Pelletier de Sousy, tiercelet de ministre par sa direction des
fortifications qui lui donnait un logement partout, jusqu'à Marly, pour
son travail réglé seul avec le roi, le devint encore davantage par la
place distinguée d'un des deux conseillers au conseil royal des
finances, qui vaqua par la mort de Pomereu de l'opération de la taille.
Ce dernier était fort considéré, fort droit, et celui des conseillers
d'État qui avait le plus d'esprit et de capacité, d'ailleurs grand
travailleur, bon homme et honnête homme. Il était extrêmement des amis
de mon père et était demeuré des miens. C'était un feu qui animait tout
ce qu'il faisait, mais allait quelquefois trop loin, et il y avait des
temps où sa famille faisait en sorte qu'il ne voyait personne. Après
cela il n'y paraissait pas. C'est le premier intendant qu'on ait hasardé
d'envoyer en Bretagne et qui trouva moyen d'y apprivoiser la province.

Une autre mort serait ridicule à mettre ici sans des raisons qui y
engagent. C'est celle de Petit, qui était fort vieux, et depuis grand
nombre d'années médecin de Monseigneur. Il avait de l'esprit, du savoir,
de la pratique et de la probité, et cependant il est mort sans avoir
jamais voulu admettre la circulation du sang. Cela m'a pari assez
singulier pour ne le pas omettre. L'autre raison est que sa charge fut
donnée à Boudin, duquel il n'est pas temps de rien dire, mais dont il
n'y aura que trop à parler, et pour des choses très importantes.

Le roi reçut à Fontainebleau la nouvelle de la liberté du maréchal de
Villeroy. Peu après que l'empereur fut informé du cartel réglé en
Italie, il lui fit mander qu'il était libre et ne voulut point galamment
qu'il payât sa rançon, qui avait à cinquante mille livres. Cette liberté
coûta cher doublement à la France, mais elle fut très agréable au roi.
Le maréchal eut ordre d'attendre un officier chargé de le conduire de la
part de l'empereur à travers l'armée du prince Eugène.

On vit à Fontainebleau une nouveauté assez étrange\,: Madame à la
comédie publique dans la deuxième année de son deuil de Monsieur. Elle
en fit d'abord quelque façon\,; mais le roi lui dit que ce qui se
passait chez lui ne devait pas être considéré comme le sont les
spectacles publics.

\hypertarget{chapitre-ii.}{%
\chapter{CHAPITRE II.}\label{chapitre-ii.}}

1702

~

{\textsc{Situation de Catinat.}} {\textsc{- Disposition de Villars.}}
{\textsc{- Bataille de Friedlingen.}} {\textsc{- Villars fait seul
maréchal de France.}} {\textsc{- Retour de Catinat et sa retraite.}}
{\textsc{- Caractère de Villars.}} {\textsc{- Mort de M. le maréchal de
Lorges.}} {\textsc{- Son éloge.}}

~

Catinat avait eu grande occasion de s'apercevoir, à la tête de l'armée
du Rhin, des suites d'un éclaircissement qui lui avait mérité les plus
grandes louanges du roi, mais qui avait convaincu son ministre et commis
M\textsuperscript{me} de Maintenon. Tous les moyens lui manquèrent, et
le dépit de faire malgré lui une campagne honteuse le rendit mystérieux
et chagrin jusqu'à mécontenter les officiers généraux, et les plus
distingués d'entre les particuliers de son armée. La nécessité de
secourir l'électeur de Bavière déclaré et molesté par les Impériaux,
celle aussi d'en être secouru, fit résoudre de tenter le passage du
Rhin\,; il fut proposé à Catinat, peut-être avec peu de moyens et de
troupes, je dis peut-être, parce que je ne le sais pas, et que je ne
fais que le soupçonner, sur le refus qu'il fit de s'en charger. À son
défaut, Villars, qui vit la fortune au bout de ce passage, l'accepta,
sûr de ne rien risquer, en manquant même ce que Catinat avait refusé de
tenter\,; mais en habile homme, il voulut être en force, et outre ce qui
était venu de Flandre qu'il avait été recevoir de Chamarande à
mi-chemin, Blainville lui amena encore un gros détachement de la même
armée de Flandre. Il y joignit ce qu'il voulut de l'armée du Rhin, qui
devenue par là un détachement elle-même, se retrancha sous Strasbourg,
et peu à peu s'y trouva réduite à dix bataillons et à fort peu
d'escadrons, en sorte que Catinat se mit dans Strasbourg, en attendant
tristement le succès du passage que Villars allait tenter, le départ du
roi des Romains pour retourner à Vienne, et ce que deviendrait son armée
après la prise de Landau.

Villars marcha droit à Huningue, visita les bords du Rhin, et choisit
l'établissement de son pont vis-à-vis d'Huningue, à l'endroit d'une île
assez spacieuse pour s'en servir utilement, le grand bras du Rhin entre
lui et l'île, et le plus petit entre elle et l'autre côté du Rhin où
était la petite ville de Neubourg, retranchée et tenue par les Impériaux
qui avaient là un camp volant, et qui avaient donné pendant toute la
campagne, l'inquiétude à Catinat de passer le Rhin et de faire le siège
d'Huningue, sans toutefois avoir songé à l'exécuter, pour ne rien
détacher de celui de Landau. Ce parti pris, Villars fit travailler tout
à son aise, mais fort diligemment, à son pont jusqu'à l'île. Il était
arrivé le 30 septembre\,; ce pont fut l'affaire de moins de vingt-quatre
heures. Le 1er octobre, à midi, il fit passer dessus quarante pièces de
canon avec Champagne et Bourbonnais qu'il établit dans l'île, et fit
travailler à son autre pont. Dès qu'il fut achevé, il fit passer des
travailleurs soutenus par ses grenadiers qui tirèrent une ligne
parallèle au Rhin à la tête du pont, malgré les faibles efforts des
ennemis pour l'empêcher, incommodés du feu de l'artillerie et des quinze
cents hommes qui étaient dans l'île, et de force petits bateaux chargés
de grenadiers. Dans cette posture, Villars, maître d'achever de passer
le Rhin, voulut attendre des nouvelles de l'électeur de Bavière, et
cependant le prince Louis de Bade et la plupart de ses officiers
généraux vinrent se retrancher à Friedlingen. Le 12 octobre Laubanie,
avec un détachement de la garnison du Neuf-Brisach, passa le Rhin dans
de petits bateaux, et emporta la petite ville de Neubourg l'épée à la
main, s'y établit et y fut suivi par notre pont de M. de Guiscard avec
vingt escadrons et dix bataillons. Le prince Louis sur cette nouvelle ne
douta pas que Villars ne voulût faire là son passage, quitta Friedlingen
et marcha à Neubourg le 14 au matin. Ce même matin, à sept heures,
Villars, averti de cette marche, sortit de Huningue, fit diligemment
passer tout ce qu'il avait de troupes en deçà par son pont dans l'île.
La cavalerie passa à gué l'autre petit bras du Rhin et l'infanterie sur
le second pont, qu'il avait remué à temps et porté vis-à-vis Friedlingen
avec son artillerie.

Là-dessus le prince Louis, qui était en marche, fit retourner toutes ses
troupes, qui étaient quarante-deux escadrons avec son infanterie\,;
cinq, de ses escadrons firent le tour d'une petite montagne escarpée de
notre côté, pour en gagner la crête par derrière, et les trente-sept
autres marchèrent à Villars plus tôt qu'il ne s'attendait à les voir. Il
n'avait que trente-quatre escadrons, parce qu'il en avait détaché six
pour aller joindre Guiscard à Neubourg. Trois charges mirent en désordre
la cavalerie impériale, qui fut reçue par six bataillons frais qui la
soutinrent. Leurs autres bataillons s'étaient postés sur la montagne,
dont il fallut les déloger en allant à eux par les vignes et
l'escarpement qui était de notre côté. Ainsi ce fut un combat bizarre ou
la cavalerie et l'infanterie de part et d'autre agit tout à fait
séparément.

Cette attaque de la montagne, conduite par Desbordes, lieutenant
général, qui avait été gouverneur de Philippsbourg, et qui y fut tué, ne
put l'être qu'avec quelque désordre par les coupures et la raideur de la
montagne, tellement que les troupes, essoufflées et un peu rompues en
arrivant, ne purent soutenir une infanterie ensemble et reposée, qui
leur fit perdre du terrain et regagner le bas avec plus de désordre
qu'elles n'avaient monté. Avec les dispositions tout cela prit du temps,
de manière que Villars, qui était demeuré au bas de la montagne et avait
perdu de vue sa cavalerie entière qui était alors à demi lieue de lui
après celle de l'empereur, crut la bataille perdue, et perdit lui-même
la tramontane, sous un arbre où il s'arrachait les cheveux de désespoir,
lorsqu'il vit arriver Magnac, premier lieutenant général de cette armée,
qui accourait seul au galop avec un aide de camp après lui. Alors
Villars, ne doutant plus que tout ne fût perdu, lui cria\,: «\,Eh
bien\,! Magnac, nous sommes donc perdus\,?» À sa voix, Magnac poussa à
l'arbre, et bien étonné de voir Villars en cet état\,: «\,Eh, lui
dit-il, que faites-vous donc là et où en êtes-vous\,? ils sont battus et
tout est à nous.\,» Villars à l'instant recogne ses larmes, court avec
Magnat à l'infanterie qui combattait celle des ennemis qui l'avait
suivie du haut de cette petite montagne, criant tous deux victoire.
Magnat avait mené la cavalerie, avait battu et poursuivi l'impériale
près de demi lieue jusqu'à ces six bataillons frais qui l'avaient
protégée, mais qui n'ayant pu soutenir la furie de nos escadrons,
s'étaient retirés peu à peu avec les débris de la cavalerie impériale,
et Magnac alors, n'ayant plus à les pousser dans les défilés qui se
présentaient, inquiet de notre infanterie dont il n'avait ni vent ni
nouvelles, était revenu de sa personne la chercher et voir ce qu'il s'y
passait, enragé de ne l'avoir pas à portée de ces défilés pour achever
sa victoire, et d'y voir échapper les débris de la cavalerie impériale
et ces six bataillons qui l'avaient sauvée. Lui et Villars avec leurs
cris de victoire rendirent un nouveau courage à notre infanterie, devant
laquelle, après plusieurs charges, celle des ennemis se retira et fut
assez longtemps poursuivie. Villars paya d'effronterie\,; et Magnac
n'osa conter leur bizarre aventure que tout bas\,; mais quand il vit que
Villars se donnait tout l'honneur, et plus encore quand il lui en vit
recevoir la récompense sans y participer en rien, il éclata à l'armée,
puis à la cour, où il fit un étrange bruit\,; mais Villars, qui avait le
prix de la victoire et M\textsuperscript{me} de Maintenon pour lui, n'en
fit que secouer l'oreille. On verra parmi les Pièces le compte qu'il en
rendit au roi, aussitôt après l'action, qui s'appela la bataille de
Friedlingen, qu'il ajuste comme il peut\footnote{Voy., page 11, parmi
  les Pièces, la lettre de Villars au roi. (\emph{Note de Saint-Simon}.)
  On trouvera cette lettre à la fin du volume.}. Outre Desbordes,
lieutenant général tué, Chavanne, brigadier d'infanterie, le fut
aussi\,; et parmi les blessés, le duc d'Estrées, Polignac, Chamarande,
lieutenant général, Coetquen et le fils du comte du Bourg, la plupart
légèrement.

Villars, qui sentit le besoin qu'il avait d'appui, fit un trait de
courtisan. Le lendemain de la bataille, il fut joint par quelques
régiments de cavalerie de ce qui restait autour de Strasbourg, que
Catinat lui envoyait encore. De ce nombre était le comte d'Ayen\,;
Villars lui proposa de porter au roi les drapeaux et les étendards, et
le comte d'Ayen l'accepta, malgré tout ce que Biron lui put dire du
ridicule de porter les dépouilles d'un combat où il ne s'était point
trouvé. Mais tout était bon et permis au neveu de M\textsuperscript{me}
de Maintenon, dont la faveur n'empêcha pas la huée de toute l'armée,
dont les lettres à Paris se trouvèrent pleines de l'aventure de Magnac
et de moqueries sur le comte d'Ayen. Mais elles arrivèrent trop tard,
leur affaire était faite. Choiseul, qui avait épousé une sœur de
Villars, fut chargé de la nouvelle et de sa lettre pour le roi\,; il
arriva le matin du mardi 17 octobre à Fontainebleau, et combla le roi de
joie de sa victoire, d'avoir un passage sur le Rhin, et de pouvoir
compter sur une prompte jonction avec l'électeur de Bavière. Le
lendemain matin le comte d'Ayen arriva aussi, et par le détail, les
drapeaux et les étendards augmenta fort la joie. Mais quand on sut qu'il
ne s'était point trouvé à l'action, le ridicule fut grand, et sa faveur
contraignit peu les brocards. Choiseul eut force louanges du roi du
compte qu'il avait rendu. Il eut le régiment qu'avait le chevalier de
Scève et mille pistoles. Il n'était que capitaine de cavalerie.

Le 20 octobre un courrier de Villars soutint habilement la bonne humeur
du roi. Il lui manda la perte des ennemis bien plus grande qu'on ne la
croyait, tous les villages des environs de Friedlingen pleins de leurs
blessés, sept pièces de canon trouvées abandonnées, le prince d'Anspach,
deux princes de Saxe, et le fils de l'administrateur de Wurtemberg
blessés et prisonniers, enfin leur armée tellement dispersée qu'elle
n'avait pas mille hommes ensemble. Biron détaché avec trois mille
chevaux au-devant de l'électeur de Bavière, et Villars occupé à établir
des forts et des postes au delà du Rhin, et à y rétablir la redoute
vis-à-vis d'Huningue détruite par la paix de Ryswick.

Le samedi matin, 21 octobre, le comte de Choiseul fut redépêché à
Villars avec un paquet du roi. On a vu en son lieu la source impure mais
puissante de la protection de M\textsuperscript{me} de Maintenon pour
lui. Le roi à son dîner le même jour le déclara seul maréchal de France.
Il y voulut ajouter du retour. Le dessus du paquet fut suscrit\,:
«\,\emph{M. le marquis de Villars}, et dedans une lettre de la propre
main du roi, fermée et suscrite\,: \emph{À mon cousin le maréchal de
Villars}. Choiseul en eut la confidence avec défense de la faire à
personne, pas même à son beau-frère en lui remettant le paquet. Le roi
voulut qu'il ne sût l'honneur qu'il lui faisait que par l'inspection du
second dessus. On peut juger de sa joie.

Celle de Catinat relaissé et délaissé dans Strasbourg ne fut pas la
même. N'ayant plus rien à faire, ou plutôt n'étant plus rien, il obtint
son congé et revint dans son carrosse à fort petites journées, comme un
homme qui craint d'arriver. Il salua le roi le 17 novembre, qui le reçut
médiocrement, lui demanda des nouvelles, de sa santé, et ne le vit point
en particulier. Il n'alla point chez Chamillart. Il demeura un jour à
Versailles et fort peu à Paris. Il se retira sagement en sa maison de
Saint-Gratien, près Saint-Denis, où il ne vit plus que quelques amis
particuliers, et ne sortit presque point de cette retraite\,; heureux
s'il n'en était point sorti et qu'il eût su résister aux cajoleries du
roi, pour reprendre le commandement d'une armée et se défier des suites
d'un éclaircissement d'autant plus dangereux qu'il fut victorieux.

Le prince Louis, fort éloigné de la dissipation où Villars l'avait
représenté, reparut incontinent avec une armée qui donna souvent de
l'inquiétude de passer en deçà du Rhin. Le reste de la campagne se passa
à s'observer et à chercher ses avantages. Parmi ceux du nouveau maréchal
la jonction ne se fit point avec l'électeur de Bavière\,: ce prince
avait pris Memmingen et plusieurs petites places pour s'élargir et se
donner des contributions et des subsistances. Les armées se retirèrent
dans leurs quartiers d'hiver\,; la nôtre repassa le Rhin, et bientôt
après Villars eut ordre de demeurer à Strasbourg à veiller sur le Rhin.

Cet enfant de la fortune va si continuellement faire désormais un
personnage si considérable qu'il est à propos de le faire connaître.
J'ai parlé de sa naissance à propos de son père\,: on y a vu que ce
n'est pas un fonds sur lequel il pût bâtir. Le bonheur et un bonheur
inouï y suppléa pendant toute sa longue vie. C'était un assez grand
homme, brun, bien fait, devenu gros en vieillissant, sans en être
appesanti, avec une physionomie vive, ouverte, sortante, et
véritablement un peu folle, à quoi la contenance et les gestes
répondaient. Une ambition démesurée qui ne s'arrêtait pas pour les
moyens\,; une grande opinion de soi, qu'il n'a jamais guère communiquée
qu'au roi\,; une galanterie dont l'écorce était toujours romanesque\,;
grande bassesse et grande souplesse auprès de qui le pouvait servir,
étant lui-même incapable d'aimer ni de servir personne, ni d'aucune
sorte de reconnaissance. Une valeur brillante, une grande activité, une
audace sans pareille, une effronterie qui soutenait tout et ne
s'arrêtait pour rien, avec une fanfaronnerie poussée aux derniers excès
et qui ne le quittait jamais. Assez d'esprit pour imposer aux sots par
sa propre confiance\,; de la facilité à parler, mais avec une abondance
et une continuité d'autant plus rebutante, que c'était toujours avec
l'art de revenir à soi, de se vanter, de se louer, d'avoir tout prévu,
tout conseillé, tout fait, sans jamais, tant qu'il put, en laisser de
part à personne. Sous une magnificence de Gascon, une avarice extrême,
une avidité de harpie, qui lui a valu des monts d'or pillés à la guerre,
et quand il vint à la tête des armées, pillés haut à la main et en
faisant lui-même des plaisanteries, sans pudeur d'y employer des
détachements exprès, et de diriger à cette fin les mouvements de son
armée. Incapable d'aucun détail de subsistance, de convoi, de fourrage,
de marche qu'il abandonnait à qui de ses officiers généraux en voulait
prendre la peine\,; mais s'en donnant toujours l'honneur. Son adresse
consistait à faire valoir les moindres choses et tous les hasards. Les
compliments suppléaient chez lui à tout. Mais il n'en fallait rien
attendre de plus solide. Lui-même n'était rien moins. Toujours occupé de
futilités quand il n'en était pas arraché par la nécessité imminente des
affaires. C'était un répertoire de romans, de comédies et d'opéras dont
il choit à tout propos des bribes, même aux conférences les plus
sérieuses. Il ne bougea tant qu'il put des spectacles avec une indécence
de filles de ces lieux et du commerce de leur vie et de leurs galants
qu'il poussa publiquement jusqu'à sa dernière vieillesse, déshonorée
publiquement par ses honteux propos.

Son ignorance, et s'il en faut dire le mot, son ineptie en affaires,
était inconcevable dans un homme qui y fut si grandement et si longtemps
employé\,; il s'égarait et ne se retrouvait plus\,; la conception
manquait, il y disait tout le contraire de ce qu'on voyait qu'il voulait
dire. J'en suis demeuré souvent dans le plus profond étonnement et
obligé à le remettre et à parler pour lui plusieurs fois, depuis que je
fus avec lui dans les affaires pendant la régence\,; aucune, tant qu'il
lui était possible, ne le détournait du jeu, qu'il aimait parce qu'il y
avait toujours été heureux et y avait gagné très gros, ni des
spectacles. Il n'était occupé que de se maintenir en autorité et laisser
faire tout ce qu'il aurait dû faire ou voir lui-même. Un tel homme
n'était guère aimable, aussi n'eut-il jamais ni amis ni créatures, et
jamais homme ne séjourna dans de si grands emplois avec moins de
considération.

Le nom qu'un infatigable bonheur lui a acquis pour des temps à venir m'a
souvent dégoûté de l'histoire, et j'ai trouvé une infinité de gens dans
cette même réflexion. Les siens ont eu l'imprudence de laisser paraître
fort tôt après lui des Mémoires qu'on ne peut méconnaître de lui\,; il
n'y a qu'à voir sa lettre au roi sur sa bataille de Friedlingen. Un
récit embarrassé, mal écrit, sans exactitude, sans précision,
expressément confus, voile tant qu'il peut le désordre qui pensa perdre
son infanterie\,; son ignorance de ce que fit sa cavalerie\,; ne peint
ni la situation, ni les mouvements, ni l'action, encore moins ce qui en
fit la décision et la fin\,; et ses louanges générales et universelles,
qui ne louent personne en ne marquant rien de particulier de personne,
données au besoin qu'il se sentait de tous\,; n'en peuvent flatter
aucun. Ses Mémoires ont la même confusion, et s'ils ont plus de détail,
c'est pour faire plus de mensonges dont il se donne sans cesse pour le
héros. J'étais bien jeune, et seulement mestre de camp d'un régiment de
cavalerie en 1594 et les années suivantes\,; mais à la première, j'étais
gendre du général de l'armée, et les autres dans la plus intime
confiance du maréchal de Choiseul, qui succéda à mon beau-père. C'en est
assez pour avoir très distinctement vu que les vanteries de ses Mémoires
sur ces campagnes-là n'ont pas seulement la moindre apparence, et que
tout ce qu'il y dit de lui est un roman. J'ai su des officiers
principaux qui ont servi avec lui et sous lui dans les autres campagnes
qu'il raconte, que tout y est mensonge, la plupart des faits entièrement
controuvés, ou avec un fondement dont tout le reste est ajusté à ses
louanges, et au blâme de ceux qui y ont le plus mérité pour leur dérober
le mérite et se l'approprier. Il s'y trouve même des traits dont la
hardiesse pue tellement la fausseté, qu'on est indigné de l'audace pour
soi-même et que le héros prétendu ait osé espérer de se faire si
grossièrement des dupes et des admirateurs. La soif d'en avoir l'a rendu
coupable des plus noirs larcins de la gloire des maîtres, devant qui je
l'ai vu ramper, et des calomnies les plus audacieuses et les plus
follement hasardées.

À l'égard de ses négociations en Bavière et à Vienne, qu'il y décrit
avec de si belles couleurs, j'en ai demandé des nouvelles à M. de Torcy,
à qui lors il en rendait compte, et sur les ordres et les instructions
duquel il avait uniquement à se régler. Torcy m'a protesté qu'il en
avait admiré le roman, que tout y est mensonge, et qu'aucun fait et
aucun mot n'en est véritable\,; il était lors ministre et secrétaire
d'État des affaires étrangères, par qui elles passaient toutes, et le
seul qui se fût préservé de partager, ou plutôt de soumettre son
département à M\textsuperscript{me} de Maintenon. Sa droiture, sa
probité, sa vérité n'ont jamais été douteuses en France ni dans les pays
étrangers, et sa mémoire toujours exacte et nette.

Telle a été la vanité de Villars d'avoir voulu être un héros en tout
genre dans la postérité, aux dépens des mensonges et des calomnies qui
font tout le tissu du roman de ses Mémoires, et la folie de ceux qui se
sont hâtés de les donner avant la mort des témoins des choses et des
spectateurs d'un homme si merveilleux, qui avec tout son art, son
bonheur sans exemple, les plus grandes dignités et les premières places
de l'État, n'y a jamais été qu'un comédien de campagne, et plus
ordinaire encore qu'un bateleur monté sur des tréteaux.

Tel fuit en gros Villars, à qui ses succès de guerre et de cour
acquerront dans la suite un grand nom dans l'histoire, quand le temps
l'aura fait perdre de vue lui-même et que l'oubli aura effacé ce qui
n'est guère connu qu'aux contemporains. Il se retrouvera si souvent dans
la suite de ces Mémoires qu'il y aura lieu de le reconnaître à divers
traits de ce portrait, plus fidèle que la gloire qu'il a dérobée, et
qu'à l'exemple du roi il a transmise à la postérité, non par des
médailles et des statues, il était trop avare, mais par des tableaux
dont il a tapissé sa maison, et où il n'a pas même oublié les choses les
plus simples et jusqu'à sa séance tenant les états de Languedoc,
lorsqu'il a commandé dans cette province. Je ne dis rien du ridicule
extrême de ses jalousies, et des voyages de sa femme traînée sur les
frontières. Il faut voiler ces misères\,; mais il est triste qu'elles
influent sur l'État et sur les plus importantes opérations de la guerre,
comme la Bavière le lui reprochera à jamais.

Parmi tant et de tels défauts, il ne serait pas juste de lui nier des
parties. Il en avait de capitaine. Ses projets étaient hardis, vastes,
presque toujours bons, et nul autre plus propre à l'exécution et aux
divers maniements des troupes, de loin pour cacher son dessein et les
faire arriver juste, de près pour se poster et attaquer. Le coup d'œil,
quoique bon, n'avait pas toujours une égale justesse, et dans l'action
la tête était nette, mais sujette à trop d'ardeur, et par là même à
s'embarrasser. L'inconvénient de ses ordres était extrême, presque
jamais par écrit, et toujours vagues, généraux, et sous prétexte
d'estime et de confiance, avec des propos ampoulés se réservant toujours
des moyens de s'attribuer tout le succès, et de jeter les mauvais sur
les exécuteurs. Depuis qu'il fut arrivé à la tête des armées, son audace
ne fut plus qu'en paroles. Toujours le même en valeur personnelle, mais
tout différent en courage d'esprit. Étant particulier, rien de trop
chaud pour briller et pour percer. Ses projets étaient quelquefois plus
pour soi que pour la chose, et par là même suspects\,; ce qui ne fut pas
depuis pour ceux dont il devait être chargé de l'exécution, qu'il
n'était pas fâché de rendre douteuse aux autres, quand c'était sur ceux
qu'elle devait rouler. À Friedlingen il y allait de tout pour lui, peu à
perdre, ou même à différer si le succès ne répondait pas à son audace,
dans une exécution refusée par Catinat\,; le bâton à espérer s'il
réussissait\,; mais quand il l'eut obtenu, le matamore fut plus réservé,
dans la crainte des revers de fortune, laquelle il se promettait de
pousser au plus haut, et il lui a été reproché depuis, plus d'une fois,
d'avoir manqué des occasions uniques, sûres et qui se présentaient
d'elles-mêmes. Il se sentait alors d'autres ressources.

Parvenu au suprême honneur militaire, il craignait d'en abuser à son
malheur\,; il en voyait des exemples. Il voulut conserver la verdeur des
lauriers qu'il avait dérobés par la main de la fortune, et se réserver
ainsi l'opinion de faire la ressource des malheurs, ou des fautes des
autres généraux. Les intrigues ne lui étaient pas inconnues\,; il savait
prendre, le roi par l'adoration, et se conserver M\textsuperscript{me}
de Maintenon par un abandon à ses volontés sans réserve et sans
répugnance\,; il sut se servir du cabinet dont elle lui avait ouvert la
porte\,; il y ménagea les valets les plus accrédités\,; hardiesse auprès
du roi, souplesse et bassesse avec cet intérieur, adresse avec les
ministres\,; et porté par Chamillart, dévoué à M\textsuperscript{me} de
Maintenon, cette conduite suivie en présence, et suppléée par lettres,
il se la crut plus utile que les hasards des événements de la guerre,
comme aussi plus sûre. Il osa dès lors prétendre aux plus grands
honneurs où les souterrains conduisent mieux que tout autre chemin,
quand on est arrivé à persuader les distributeurs qu'on en est
susceptible. Je ne puis mieux finir ce trop long portrait, où je crois
pourtant n'avoir rien dit d'inutile, et dans lequel j'ai scrupuleusement
respecté le joug de la vérité\,; je ne puis, dis-je, l'achever mieux que
par cet apophtegme de la mère de Villars, qui, dans l'éclat de sa
nouvelle fortune, lui disait toujours\,: «\,Mon fils, parlez toujours de
vous au roi, et n'en parlez jamais à d'autres.\,» Il profita utilement
de la première partie de cette grande leçon, mais non pas de l'autre, et
il ne cessa jamais d'étourdir et de fatiguer tout le monde de soi.

L'époque de cette bataille de Friedlingen me fut celle d'une des plus
sensibles afflictions que je pusse recevoir, par la perte que je fis de
mon beau-père, à soixante-quatorze ans. Au milieu d'une santé d'ailleurs
parfaite, il fut attaqué de la pierre, aux symptômes de laquelle on se
méprit d'abord, ou plutôt on voulut bien se méprendre, dans le désir que
ce ne la fût pas. Les derniers six mois de sa vie il ne put plus sortir
de chez lui, où l'affection publique lui forma toujours plutôt une cour,
par le nombre et la distinction des personnes, qu'une compagnie assidue.
Le mal venu au point de ne le pouvoir méconnaître, la réputation d'un
certain frère Jacques séduisit et le fit préférer aux chirurgiens pour
l'opération. Ce n'était ni un moine ni un ermite, mais un homme
bizarrement encapuchonné de gris, qui avait inventé une manière de faire
la taille par à côté de l'endroit ordinaire, qui avait l'avantage d'être
plus promptement faite et de ne laisser après aucune des fâcheuses
incommodités qui sont très souvent les suites de cette opération faite à
l'ordinaire. Tout est mode en France\,; cet homme-là y était lors
tellement qu'on ne parlait que de lui. On fit suivre ses opérations
pendant trois mois, et sur vingt personnes qu'il tailla il en mourut
fort peu.

Pendant ce temps-là M. le maréchal de Lorges se dérobait au monde, et se
préparait avec une grande fermeté et une résignation vraiment
chrétienne. Le désir de sa famille et de conserver sa charge de
capitaine des gardes du corps à son fils eurent plus de part que
lui-même à cette résolution. Elle fut exécutée le jeudi 19 octobre à
huit heures du matin, ayant la veille fait ses dévotions. Frère Jacques
ne voulut ni conseil ni secours, que Milet, chirurgien-major de la
compagnie des gardes du corps de M. le maréchal de Lorges, auquel il
était fort attaché. Il se trouva une petite pierre, puis de gros
champignons, et, dessous, une fort grosse pierre. Un chirurgien qui eût
su autre chose qu'opérer de la main aurait tiré la petite pierre et en
serait demeuré là pour lors. Il aurait fondu par des onguents ces
excroissances de chair adhérentes à la vessie, qui s'en seraient allées
par les suppurations, après quoi il aurait tiré la grosse pierre. La
tête tourna au frère Jacques, qui n'était que bon opérateur de la main.
Il arracha ces champignons. L'opération dura trois quarts d'heure, et
fut si cruelle, que frère Jacques n'osa aller plus loin et remit à tirer
la grosse pierre. M. le maréchal de Lorges la soutint avec un courage
qui fut toujours tranquille. Fort peu après, M\textsuperscript{me} sa
femme, qui fut la seule qu'on lui laissa voir de sa famille, s'étant
approchée de lui, il lui tendit la main\,: «\,Me voilà, lui dit-il, dans
l'état où on m'a voulu,\,» et, sur sa réponse pleine d'espérance\,:
«\,Il en sera, ajouta-t-il, tout ce qu'il plaira à Dieu.\,» Toute la
famille et quelques amis étaient dans la maison, qui augurèrent mal
d'une opération si étrange. Le duc de Grammont, qui avait été depuis peu
taillé par Maréchal, força la porte, annonça les accidents qui
arriveraient coup sur coup, où il n'y aurait point de remède, et insista
inutilement pour qu'on fît venir Maréchal ou d'autres chirurgiens.
Jamais frère Jacques ne voulut, et la maréchale, qui craignait de le
troubler, n'osa appeler personne. Le duc de Grammont ne fut que trop bon
prophète\,; bientôt après frère Jacques lui-même demanda du secours. Il
l'eut à l'instant, mais tout fut inutile. M. le maréchal de Lorges
mourut le samedi 22 octobre, sur les quatre heures du matin, ayant
toujours eu auprès de lui l'abbé Anselme, alors directeur et prédicateur
fameux.

Le spectacle de cette maison fut terrible\,; jamais homme si tendrement
et si universellement regretté, ni si véritablement regrettable. Outre
ma vive douleur, j'eus à soutenir celle de M\textsuperscript{me} de
Saint-Simon, que je crus perdre bien des fois\,; rien de comparable à
son attachement pour son père, et à la tendresse qu'il avait pour
elle\,; rien aussi de plus parfaitement semblable que leur âme et leur
cœur. Il m'aimait comme son véritable fils, et je l'aimais et le
respectais comme le meilleur père, avec la plus entière et la plus douce
confiance.

Né troisième cadet d'une nombreuse famille, ayant perdu son père à l'âge
de cinq ans, il porta les armes à quatorze. M. de Turenne, frère de sa
mère, prit soin de lui comme de son fils, et dans la suite lui donna
tous ses soins et toute sa confiance. L'attachement du neveu répondit
tellement à l'amitié de l'oncle, qu'ils vécurent toujours ensemble, et
furent considérés de tout le monde comme un père et un fils les plus
étroitement unis. Des malheurs de temps et des engagements de famille
entraînèrent M. de Lorges dans le parti de M. le Prince. Il le suivit
même aux Pays-Bas\,; il servit sous lui de lieutenant général avec de
grandes distinctions et s'acquit entièrement son estime. Instruit déjà
par M. de Turenne, il se perfectionna sous M. le Prince et revint sous
son oncle, qui se fit un plaisir et une étude de le rendre capable de
commander dignement les armées, en l'employant dans les siennes à tout
ce qu'il y avait de plus difficile et de plus important, M. de Lorges,
jeune et bien fait, galant, fort dans le grand monde, pensait néanmoins
sérieusement. Élevé dans le sein des protestants où il était né, et lié
de la plus proche parenté et amitié avec leurs principaux personnages,
il passa la moitié de sa vie sans se défier qu'ils pussent être trompés
et pratiquant exactement leur religion. Mais à force de la pratiquer les
réflexions vinrent, puis les doutes. Les préjugés de l'éducation et de
l'habitude le retenaient\,: il était encore maîtrisé par l'autorité de
sa mère qui en était une de l'Église protestante et par celle de M. de
Turenne plus forte qu'aucune. Il était intimement lié d'amitié avec la
duchesse de Rohan, l'âme du parti et le reste de ses derniers chefs, et
avec ses célèbres filles, et son extrême tendresse pour la comtesse de
Roye sa sœur, qui était infiniment attachée à sa religion, le
contraignit extrêmement. Mais, parmi ces combats, il voulut être
éclairci. Il trouva un grand secours dans un homme médiocre qui lui
était attaché d'amitié, et qui, en étant fort estimé, s'était fait
catholique. Mais M. de Lorges voulut voir par lui-même, quand il fut
parvenu au point de se défier tout à fait de ce qu'il avait cru
jusqu'alors.

Il prit donc le parti de feuilleter lui-même et de proposer ses doutes
au célèbre Bossuet, depuis évêque de Meaux, et à M. Claude, ministre de
Charenton et le plus compté parmi eux. Il ne les consultait que
séparément, à l'insu l'un de l'autre, et leur portait comme de soi-même
leurs réciproques réponses, pour démêler mieux la vérité. Il passa de la
sorte toute une année à Paris, tellement occupé à cette étude qu'il
avait comme disparu du monde, et que ses plus intimes, jusqu'à M. de
Turenne, en étaient inquiets, et lui faisaient des reproches de ce
qu'ils ne pouvaient parvenir à le voir. Sa bonne foi et la sincérité de
sa recherche mérita un rayon de lumière. M. de Meaux lui prouva
l'antiquité de la prière pour les morts, et lui montra dans
Saint-Augustin que ce docteur de l'Église avait prié pour sainte Monique
sa mère. M. Claude ne le satisfit point là-dessus, et ne s'en tira que
par des défaites qui choquèrent la droiture du prosélyte et achevèrent
de le déterminer. Alors il s'ouvrit au prélat et au ministre, du
commerce qu'il avait depuis longtemps avec eux à l'insu l'un de
l'autre\,; il les voulut voir aux mains, mais toujours dans le plus
profond secret. Cette lutte acheva de convaincre son esprit par la
lumière, et son cœur par les échappatoires peu droites qu'il remarqua
souvent dans M. Claude, sur lesquelles après, tête à tête, il n'en put
tirer de meilleures solutions.

Convaincu alors, il prit son parti, mais les considérations de ses
proches l'arrêtèrent encore. Il sentait qu'il allait plonger le poignard
dans le cœur des trois personnes qui lui étaient les plus chères, sa
mère, sa sœur et M. de Turenne à qui il devait tout, et de qui il tenait
tout jusqu'à sa subsistance. Cependant ce fut par lui qu'il crut devoir
commencer. Il lui parla avec toute la tendresse, toute la
reconnaissance, tout le respect du meilleur fils au meilleur père\,; et,
après un préambule dont il sentit tout l'embarras, il lui fit toute la
confidence de cette longue retraite dont il lui avoua enfin le fruit, et
il assaisonna cette déclaration de tout ce qui en pouvait adoucir
l'amertume. M. de Turenne l'écouta sans l'interrompre d'un seul mot,
puis, l'embrassant tendrement, lui rendit confidence pour confidence, et
l'assura qu'il avait d'autant plus de joie de sa résolution, que
lui-même en avait pris une pareille après y avoir travaillé longtemps
avec le même prélat que lui. On ne peut exprimer la surprise, le
soulagement, la joie de M. de Lorges. M. de Meaux lui avait fidèlement
caché qu'il instruisait M. de Turenne depuis longtemps, et à M. de
Turenne ce qu'il faisait avec M. de Lorges. Fort peu de temps après, la
conversion de M. de Turenne éclata. La délicatesse de M. de Lorges ne
lui permit pas de se déclarer sitôt. Le respect du monde le contint
encore cinq ou six mois dans la crainte qu'on ne le crut entraîné par
l'exemple d'un homme de ce poids auquel tant de liens l'attachaient.
Sans avoir jamais fait une profession particulière de piété distinguée,
M. de Lorges regarda tout le reste de sa vie sa conversion comme son
plus précieux bonheur. Il redoubla d'estime, d'amitié et de commerce
avec M. Cotton qui en avait été la première cause\,; il vit tant qu'il
vécut M. de Meaux très familièrement, et avec vénération et grande
reconnaissance. Il abhorrait la contrainte sur la religion, mais il se
portait avec zèle à persuader les protestants à qui il pouvait parler,
et fut jusqu'à la mort régulier et même religieux dans sa conduite et
dans la pratique de la religion qu'il avait embrassée, et ami des gens
de bien. Il eut la douleur que la comtesse de Roye en pensa mourir de
regret. Il n'y avait que la religion que tous deux se préférassent. Elle
fut si outrée de ce changement, qu'elle ne le voulut voir qu'à {[}la{]}
condition, qu'ils tinrent, de ne s'en parler jamais.

M. de Lorges porté par l'estime de M. le Prince et de M. de Turenne, et
par son propre mérite, eut après les maréchaux de France les
commandements les plus importants de la guerre de Hollande\,; il ne tint
qu'à lui après le retour du roi de l'avoir en chef. Il en reçut la
patente et l'ordre de faire arrêter le maréchal de Bellefonds, dont
l'opiniâtreté était tombée en plusieurs désobéissances formelles coup
sur coup aux ordres qu'il avait eus de la cour. M. de Lorges évita l'un
et sauva l'autre, qui ne le sut que longtemps après, et d'ailleurs, et
qui ne l'a jamais oublié. Je ne rougirai point de dire que toute
l'Europe admira et célébra le combat et la savante retraite d'Altenheim,
et la gloire de M. de Lorges qui y commandait en chef\,; en même temps
qu'elle retentit de la mort de M. de Turenne. C'est un fait attesté par,
toutes les histoires, les Mémoires et les lettres de ce temps-là. M. le
Prince voulut bien la rehausser encore. «\,J'osa avouer, dit-il alors au
milieu de l'armée de Flandre qu'il commandait, et d'où il eut ordre
d'aller prendre la place de M. de Turenne, j'ose avouer que j'ai
quelques actions, mais je dis avec vérité que j'en donnerais plusieurs
de celles-là, et avoir fait celle que le comte de Lorges vient de faire
à Altenheim.\,» Après un aussi grand témoignage, et qui fait autant
d'honneur à M. le Prince qu'à M. de Lorges, ce serait affaiblir l'action
d'Altenheim que s'y étendre\,; mais je ne puis m'empêcher de remarquer
le grand homme en laissant le capitaine, et le grand homme que les
Romains eussent également admiré. On trouvera que je ne dis pas trop, si
on se représente la situation, l'étonnement, la désertion de l'armée de
M. de Turenne au coup de canon qui l'emporta, la douleur extrême et
subite de la perte de ce grand homme, dont M. de Lorges fut pénétré, et
dont la sensibilité le devait rendre l'homme de toute l'armée le plus
stupide et le plus incapable de penser et d'agir. Qu'on ajoute à tout ce
que l'amitié, la tendresse, la reconnaissance, la confiance, la
vénération fit d'impression à l'excellent cœur de ce neveu si chéri, ce
qu'y durent opérer après les réflexions les plus tristes de la privation
d'un tel appui à la porte de la fortune dont M. de Lorges n'avait pas
reçu encore la moindre faveur et sans nul patrimoine, avec la
perspective de la toute-puissance de Louvois, ennemi déclaré de M. de
Turenne, et le sien particulier à cause de lui, il n'y en avait que trop
sans doute pour terrasser le cœur et l'esprit d'un homme ordinaire, et
pour confondre même les opérations d'un homme au-dessus du commun,
devenu général tout à coup dans de si cruelles conjonctures.

Comblé d'honneur et de gloire, et l'étonnement de Montécuculli, M. de
Lorges vit peu de jours après faire plusieurs maréchaux de France sans
en être, et arriver quelques-uns d'eux à la suite de M. le Prince, à qui
il remit le commandement de l'armée. On peut imaginer quelle fut pour
lui cette amertume. Il eut la consolation que les armées et la cour
crièrent publiquement à l'iniquité, et qu'aucun des nouveaux maréchaux,
venus avec M. le Prince, n'osa lui donner l'ordre, ni prendre aucun
commandement sur lui.

Le bruit extrême que fit cette injustice inquiéta Louvois qui en était
l'auteur. Vaubrun, lieutenant général, avait été tué au combat
d'Altenheim, et laissait vacant le commandement en chef d'Alsace, de
plus de cinquante mille livres de rente. Louvois ne douta pas que ce
morceau ne fût du goût d'un homme qui n'avait rien vaillant, et l'envoya
à M. de Lorges\,; mais il fut étonné de se le voir rapporter par le même
courrier, avec cette courte réponse, que ce qui était bon pour un cadet
de Nogent ne l'était pas pour un cadet de Duras. Avec ce refus M. de
Lorges avait pris son parti\,; c'était d'achever, comme il fit, la
campagne dans l'éloignement, de ne s'y mêler de rien, avec hauteur, mais
avec modestie, et dès qu'après son retour il aurait salué le roi et vu
ses amis quelques jours, de se retirer à l'institution des pères de
l'Oratoire, et là d'achever sa vie avec trois valets uniquement, dans
une entière retraite et dans la piété. La campagne s'allongea jusque
vers la fin de l'année. Il hâta peu son retour, et fut reçu comme le
méritait sa gloire et son malheur. M. de La Rochefoucauld, son ami
intime, et lors dans le fort de sa faveur, en prit occasion d'en parler
au roi avec tant de force, que Louvois ne put parer le coup, et que M.
de Lorges, qui ne l'avait pas voulu aller voir, fut fait maréchal de
France seul, le 21 février 1676, presque aussitôt qu'il fut arrivé, avec
un applaudissement qui n'a guère eu de semblable.

Alors il fallut changer de résolution, et se livrer à la fortune. Le
bâton fut le premier bienfait qu'il en reçut\,; mais avec la gloire qui
le lui procura il ne portait que douze mille livres de rente\,: c'était
tout l'avoir du nouveau maréchal, sans aucune autre ressource. Il fut
nommé en même temps pour être un des maréchaux de France qui devaient
commander l'armée sous le roi en personne, qui avait résolu se rendre en
Flandre, au commencement d'avril. Il fallait un équipage, et de quoi
soutenir une dépense convenable et pressée. Cette nécessité le fit
résoudre à un mariage étrangement inégal, mais dans lequel il trouvait
les ressources dont il ne se pouvait passer pour le présent, et pour
fonder une maison. Il y rencontra une épouse qui n'eut des yeux que pour
lui malgré la différence d'âge, qui sentit toujours avec un extrême
respect l'honneur que lui faisait la naissance et la vertu de son époux,
et qui y répondit par la sienne, sans soupçon et sans tache, et par le
plus tendre attachement. Lui aussi oublia toute différence de ses
parents aux siens, et donna toute sa vie le plus grand exemple du plus
honnête homme du monde avec elle, et avec toute sa famille, dont il se
fit adorer. Il trouva de plus dans ce mariage une femme adroite pour la
cour et pour ses manèges, qui suppléa à la roideur de sa rectitude, et
qui, avec une politesse qui montrait qu'elle n'oubliait point ce qu'elle
était née, joignait une dignité qui présentait le souvenir de ce qu'elle
était devenue, et un art de tenir une maison magnifique, les grâces d'y
attirer sans cesse la meilleure et la plus nombreuse compagnie, et, avec
cela, le savoir-faire de n'y souffrir ni mélange, ni de ces commodités
qui déshonorent les meilleures maisons, sans toutefois cesser de rendre
la sienne aimable, par le respect et la plus étroite bienséance qu'elle
y sut toujours maintenir et mêler avec la liberté.

Incontinent après ce mariage, M. le maréchal de Lorges en sentit la
salutaire utilité\,; la fortune qui l'avait tant fait attendre sembla
vouloir lui en payer l'intérêt. Le maréchal de Rochefort, capitaine des
gardes du corps, mourut. Il était le favori de M. de Louvois, qui à la
mort de M. de Turenne l'avait fait faire maréchal de France avec les
autres, dont le François, fertile en bons mots, disait que le roi avait
changé une pièce d'or en monnaie. Quoique M. de Duras fût déjà capitaine
des gardes du corps, M. son frère fut choisi pour la charge qui vaqua et
qu'il n'aurait pu payer, ni même y songer sans son mariage. Ainsi les
deux frères, maréchaux de France, furent aussi tous deux capitaines des
gardes du corps, égalité et conformité de fortune sans exemple.

Ce n'était pas que M. le maréchal de Lorges l'eût méritée par sa
complaisance. Le roi à la tête de son armée couvrait Monsieur qui
assiégeait Bouchain, et s'avança jusqu'à la cense\footnote{Le mot
  \emph{cense} désignait quelquefois une terre soumise à une certaine
  redevance, appelée \emph{cens}. Ces terres portaient aussi le nom de
  \emph{censive}.} d'Harrebise. Le prince d'Orange se trouva campé tout
auprès, sans hauteur, ravin ni ruisseau qui séparât les deux armées.
Celle du roi était supérieure, et reçut encore un renfort très à propos
de l'armée devant Bouchain. Il semblait qu'il n'y avait qu'à marcher aux
ennemis, pour orner le roi d'une importante victoire. On balança, on
coucha en bataille, et le matin suivant, M. de Louvois fit tenir au roi
un conseil de guerre, le cul sur la selle avec les maréchaux de France
qui se trouvèrent présents, et deux ou trois des premiers et des plus
distingués d'entre les lieutenants généraux\,; ils étaient en cercle, et
toute la cour et les officiers généraux à une grande distance laissée
vide. M. de Louvois exposa le sujet\,: de la délibération à prendre, et
opina pour se tenir en repos. Il savait à qui il avait affaire, et il
s'était assuré des maréchaux de Bellefonds, d'Humières et de La
Feuillade. M. le maréchal de Lorges opina pour aller donner la bataille
au prince d'Orange, et il appuya ses raisons, de manière qu'aucun de ce
conseil n'osa les combattre\,; mais regardant M. de Louvois dont ils
prirent une seconde fois l'ordre de l'œil, ils persistèrent. M. le
maréchal de Lorges insista, et de toutes ses forces représenta la
facilité du succès, la grandeur des suites à une ouverture de campagne,
et tout ce qui se pouvait tirer d'utile et de glorieux de la présence du
roi, et il réfuta aussi les inconvénients allégués, avec une solidité
qui n'eut aucune réplique. Le résultat fut que le roi lui donna force
louanges, mais {[}dit{]} qu'avec regret il se rendait à la pluralité des
avis. Il demeura donc là, sans rien entreprendre, tandis qu'il arriva du
renfort au prince d'Orange.

Je ne sais quoi engagea à envoyer un trompette aux ennemis, et à
préférer celui d'entre eux qui en avait le plus d'habitude. Il ne fut
pas vingt-quatre heures\,; il rapporta au roi que le prince d'Orange lui
avait fait voir son armée, et lui avait dit qu'il n'avait jamais eu si
belle peur, ni plus de certitude d'être attaqué. Il se plut à lui
expliquer les raisons de sa crainte, et de ce qu'il était perdu à coup
sûr. Apparemment pour en donner plus de regret, et pour le plaisir de
montrer à quel point il était tôt et bien informé, il le chargea de dire
à M. le maréchal de Lorges de sa part qu'il savait combien il avait
disputé pour engager la bataille, en peu de mots, les raisons qu'il en
avait apportées, que s'il avait été cru, il était battu et perdu sans
aucune ressource. Le trompette fut assez imprudent pour raconter tout
cela au roi et à M. de Louvois, en présence de force généraux et
seigneurs\,; et n'y ayant pas remarqué M. le maréchal de Lorges, il
l'alla chercher, et s'acquitta de ce dont le prince d'Orange l'avait
chargé pour lui. Le maréchal, de plus en plus outré de n'avoir pas été
cru, sentit le poids de ce témoignage. Il en commanda bien expressément
le secret au trompette, mais il n'était plus temps\,; et une heure
après, son rapport fut la nouvelle et l'entretien de toute l'armée\,;
sur cela, Monsieur arriva venant de prendre Bouchain, et le roi laissa
son armée à ses généraux, et partit avec Monsieur pour retourner à
Versailles, où, à peine arrivés, Louvois qui le suivit eut la douleur
d'apprendre la mort du maréchal de Rochefort, son ami, et le dépit de
voir donner sa charge à M. le maréchal de Lorges.

Ce ministre, n'était pas homme à pardonner, ni M. le maréchal de Lorges
à se ployer à aucune recherche. Il demeura donc à faire sa charge auprès
du roi. Il ne pouvait se plaindre étant le dernier des maréchaux de
France. La convenance élu comte de Feversham, son frère, grand
chambellan de la reine d'Angleterre, femme de Charles II, grand maître
de la garde-robe, et capitaine des gardes du corps de ce prince, et
alors du roi Jacques II, son frère et son successeur, et général de
leurs armées, engagea le roi à envoyer M. le maréchal de Lorges
complimenter le roi d'Angleterre Jacques II sur la victoire que le comte
de Feversham venait de remporter contre les rebelles, qui coûta la tête
sur un échafaud au duc de Monmouth, bâtard de Charles II, qui n'aspirait
à rien moins qu'à la\,: couronne d'Angleterre, dès lors l'objet des
désirs et des espérances du prince d'Orange qui l'avait poussé et aidé
pour s'en préparer les voies à lui-même, dès cette année-là 1685. En
1688, M. le maréchal de Larges, fait chevalier de l'ordre dans la grande
promotion du dernier jour de cette année, eut le commandement en chef de
Guyenne avec tous les appointements et l'autorité du gouverneur, jusqu'à
ce que M. le comte de Toulouse qui l'était fût en âge. Les appointements
lui demeurèrent jusqu'alors\,; mais à peine fut-il arrivé en Guyenne,
qu'il fut rappelé pour le commandement de l'armée du Rhin, où il arriva
comme Mayence venait de se rendre.

Le dessein de Louvois n'était pas de terminer en peu de temps la guerre
que son intérêt particulier venait de rallumer, ni d'en procurer
l'honneur à un général aussi peu à son gré que l'était M. le maréchal de
Lorges. Aussi fut-ce en vain que celui-ci ne cessa de représenter
l'impossibilité d'y parvenir par le côté de la Flandre, si coupé de
rivières et si hérissé de places, et la facilité et l'utilité des
progrès en portant le fort de la guerre de l'autre côté du Rhin, où les
princes de l'empire se lasseraient bientôt de leurs pertes, et les
alliés de voir les troupes du roi au milieu de l'Allemagne. Plus il
avait raison, moins était-il écouté. Louvois avait tellement persuadé le
roi de ne rien tenter en Allemagne, que ce même esprit régna après sa
mort\,; on a vu sur l'année 1693 ce qu'il s'y passa en présence de
Monseigneur, qui s'arrêta devant Heilbronn, après ses avantages que la
facilité de celui-là aurait comblés en ouvrant l'Allemagne. Tout ce que
le maréchal de Lorges employa fut inutile pour faire résoudre l'attaque
de ce poste, et le désespoir qu'il ne put cacher de se voir arrêté en si
beau chemin par l'avis de Beringhen, premier écuyer, et de
Saint-Pouange, qui accompagnaient ce prince avec la confiance du roi
auprès de lui. Ils n'osèrent se hasarder avec un général qui les aurait
menés trop loin à leur gré, et qui l'année précédente avait forcé par un
combat le prince Louis de Bade à repasser le Rhin, l'y avait suivi,
défait et pris l'administrateur de Würtemberg, pris deux mille chevaux
qui remontèrent sa cavalerie en partie, onze pièces de canon, Pfortzheim
et quelques autres places, et qui fit ensuite lever au landgrave de
Hesse le siège d'Eberbourg qu'il avait formé depuis dix jours, et tout
seul avec une armée plus faible que celle du prince Louis de Bade.

Ce général, qui pendant toute cette guerre commanda toujours l'armée
opposée à celle de M. le maréchal de Lorges, avait conçu pour lui tant
d'estime, qu'ayant pris un courrier de son armée avec les lettres dont
il était chargé pour la cour, il lui en renvoya un paquet après l'avoir
lu, et avait écrit dessus ces paroles si connues\,: \emph{Ne sutor ultra
crepidam}. M. le maréchal de Lorges, surpris au dernier point de cette
unique suscription, demanda au trompette s'il n'apportait rien autre,
qui lui répondit n'avoir charge que de lui remettre ce paquet en main
propre. À son ouverture il se trouva une lettre de La Fond, intendant de
son armée, qui devait tout ce qu'il était et avait à M. de Duras et à
lui, par laquelle il critiquait toute la campagne, donnait ses avis et
se prétendait bien meilleur général. Alors M. le maréchal de Lorges vit
la raison de la suscription, et remercia le prince Louis comme ce
service le méritait. Il manda La Fond qu'il traita comme il devait,
envoya sa lettre et les réflexions qu'elle méritait, et le fit révoquer
honteusement. Cette aventure n'empêcha pas depuis que les avis de La
Grange, successeur de La Fond, préférés aux raisons de M. le maréchal de
Lorges, n'aient coûté le dégât de la basse Alsace, et n'aient pensé
coûter pis, comme je l'ai raconté en son lieu, tant la plume a eu sous
le roi d'avantage sur l'épée, jusque dans son métier et malgré les
expériences.

J'aurais encore tant de grandes choses à dire de mon beau-père que ce
serait passer de trop loin les bornes d'une digression que je n'ai pu me
refuser. On n'a point connu une plus belle âme ni un cœur plus grand ni
meilleur que le sien, et cette vérité n'a point trouvé de
contradicteurs. Jamais un plus honnête homme, plus droit, plus égal,
plus uni, plus simple, plus aise de servir et d'obliger, et bien
rarement aucun qui le fût autant. D'ailleurs la vérité et la candeur
même, sans humeur, sans fiel, toujours prompt à pardonner, c'est encore
ce dont personne n'a douté. Avec une énonciation peu heureuse et un
esprit peu brillant et peu soucieux de l'être, c'était le plus grand
sens d'homme, et le plus droit qu'il fût possible, et qui, avec une
hauteur naturelle qui ne se faisait jamais sentir qu'à propos, mais que
nulle considération aussi n'en pouvait faire rien rabattre, dédaignait
les routes les plus utiles si elles n'étaient frayées par l'honneur le
plus délicat et la vertu la plus épurée. Avec la plus fine valeur et la
plus tranquille, ses vues étaient vastes, ses projets concertés et
démontrés\,; une facilité extrême à manier des troupes, l'art de prendre
ses sûretés partout, sans jamais les fatiguer, le choix exquis des
postes, et toute la prévoyance et la combinaison de ses mouvements avec
ses subsistances. Jamais avec lui de gardes superflues, de marches
embarrassées ou inutiles, d'ordres confus. Il avait la science de se
savoir déployer avec justesse, et celle des précautions sans fatiguer
ses troupes, qui achevaient toujours sous lui leurs campagnes en bon
état. J'ai ouï dire merveilles, à ceux qui l'ont vu dans les actions, du
flegme sans lenteur dans ses dispositions, de la justesse de son coup
d'œil, et de sa diligence à se porter et à remédier à tout, et à
profiter de ce qui aurait échappé à d'autres généraux.

Plus jaloux de la gloire d'autrui que de la sienne, il la donnait tout
entière à qui la méritait, et sauvait les fautes avec une bonté
paternelle. Aussi était-il adoré, dans les armées, des troupes et des
officiers généraux et particuliers, dont la, confiance en lui était
parfaite par estime. Sa compagnie des gardes avait pour lui le même
amour. Mais ce qui est bien rare, c'est que la cour si jalouse, et où
chacun est si personnel, ne le chérissait pas moins, et qu'excepté M. de
Louvois, et encore sur le compte de M. de Turenne, il n'eut pas un
ennemi, et s'acquit l'estime universelle jusqu'à une sorte de
vénération. Rien n'était égal à sa tendresse et à sa douceur dans sa
famille, et au réciproque dont il jouissait. Il traita toujours en tout
ses neveux comme ses enfants\,: il avait beaucoup d'amis, et d'amis
véritables\,; il sentait tout le prix des gens et celui de l'amitié,
parce que personne n'en était plus capable et n'avait un meilleur
discernement que lui\,; au reste, grand ennemi des fripons, leur fléau
sans ménagement, et l'homme qui, avec le plus de simplicité et de
modestie, conservait le plus de dignité et s'attirait le plus de
considération et de respect. Le roi même, qui l'aimait, le ménageait\,;
il lui disait sans détour toutes les vérités que ses emplois
l'obligeaient à ne lui point dissimuler, et il en était cru par
l'opinion générale de sa vérité. Avec le respect qu'il devait au roi, il
était hardi à rompre pour les malheureux ou pour la justice des glaces
qui auraient fait peur aux plus favorisés, et plus d'une fois il a forcé
le roi à se rendre, même contre son goût. Dans sa pauvreté, et depuis à
la tête des armées, son désintéressement fut sans pareil, et les
sauvegardes dont, au moins en pays ennemi et qui les demande, les
généraux croient pouvoir profiter, jamais il n'en souilla ses mains\,:
il avait, disait-il, appris cette leçon de M. de Turenne.

Tous les Bouillon lui étaient singulièrement chers à cause de leur
oncle, et, jusqu'au colonel général de la cavalerie\footnote{Le colonel
  général de la cavalerie légère était Frédéric Maurice de La Tour,
  comte d'Auvergne, fils du duc de Bouillon et neveu de Turenne. Cette
  phrase a été altérée par les anciens éditeurs qui ont cru devoir
  ajouter le mot \emph{régiment}. Voici le texte qu'ils ont substitué à
  celui du manuscrit\,: «\,Tous les Bouillon lui étaient singulièrement
  chers à cause de son oncle, et jusqu'au régiment\,; colonel général de
  la cavalerie, il l'avait tant qu'il pouvait dans son armée,\,» etc. On
  a supposé que c'était le maréchal de Lorges qui était colonel général
  de la cavalerie et qu'il avait dans son armée un prétendu régiment de
  Bouillon, dont ne parle pas Saint-Simon.}\,; il l'avait tant qu'il
pouvait dans son armée, et lui témoignait toutes sortes de
prédilections. Partout il vivait non seulement avec toute sorte de
magnificence, mais avec splendeur, sans intéresser en rien sa modestie
et sa simplicité naturelle\,; aussi jamais homme si aimable dans le
commerce, si égal, si sûr, si aise d'y mettre tout le monde, ni plus
honnêtement gai\,; aussi jamais homme si tendrement, si généralement, si
amèrement ni si longuement regretté.

\hypertarget{chapitre-iii.}{%
\chapter{CHAPITRE III.}\label{chapitre-iii.}}

1702

~

{\textsc{Mort de la duchesse de Gesvres.}} {\textsc{- Trianon.}}
{\textsc{- Retour de Fontainebleau.}} {\textsc{- Mort du comte de
Noailles.}} {\textsc{- Succès des alliés en Flandre.}} {\textsc{-
Marlborough pris et ignoramment relâché.}} {\textsc{- Vendôme court la
même fortune.}} {\textsc{- Prince d'Harcourt salue enfin le roi.}}
{\textsc{- Sa vie et son caractère, et de sa femme.}} {\textsc{- Retour
brillant du maréchal de Villeroy après une dure captivité\,; sa lourde
et vaine méprise\,; est déclaré général de l'armée en Flandre.}}
{\textsc{- Mort du chevalier de Lorraine.}} {\textsc{- Retour et
opération du comte d'Estrées.}} {\textsc{- Comte d'Albert, Pertuis et
Conflans sortent de prison.}} {\textsc{- Charmois et du Héron chassés de
Ratisbonne et de Pologne.}} {\textsc{- Catinat retiré ne sert plus.}}
{\textsc{- Mgr le duc de Bourgogne entre dans tous les conseils.}}
{\textsc{- Ubilla assis au conseil.}} {\textsc{- Régiments des gardes
espagnole et wallonne.}} {\textsc{- Orry et sa fortune.}} {\textsc{-
Marsin de retour.}} {\textsc{- Dispute entre le chancelier et les
évêques pour le privilège de leurs ouvrages doctrinaux.}} {\textsc{-
Chamilly de retour de Danemark\,; sa fâcheuse méprise\,; celle de
d'Avaux.}} {\textsc{- Mort du cardinal Cantelmi\,; du duc d'Albemarle\,;
de Champflour, évêque de la Rochelle\,; de Brillac, premier président du
parlement de Bretagne.}} {\textsc{- Mariage du duc de Lorges avec la
troisième fille de Chamillart.}} {\textsc{- Mon intime liaison avec
Chamillart, qui me demande instamment mon amitié}}

~

La duchesse de Gesvres mourut dans le même temps, séparée d'un mari
fléau de toute sa famille, et qui lui avait mangé des millions. Son nom
était du Val. Elle était fille unique de Fontenay-Mareuil, ambassadeur
de France à Rome, du temps de l'entreprise du duc de Guise à Naples.
C'était une espèce de fée, grande et maigre, qui marchait comme ces
grands oiseaux qu'on appelle des demoiselles de Numidie. Elle venait
quelquefois à la cour\,; et avec du singulier et l'air de la famine où
son mari l'avait réduite, elle avait beaucoup de vertu, d'esprit, et de
la dignité. Je me souviens qu'un été que le roi s'était mis à aller fort
souvent les soirs à Trianon, et qu'une fois pour toutes il avait permis
à toute la cour de l'y suivre, hommes et femmes, il y avait une grande
collation pour les princesses ses filles, qui y menaient leurs amies, et
où les autres femmes allaient aussi quand elles voulaient. Il prit en
gré un jour à la duchesse de Gesvres d'aller à Trianon et d'y faire
collation. Son âge, sa rareté à la cour, son accoutrement et sa figure
excitèrent ces princesses à se moquer tout bas d'elle avec leurs
favorites. Elle s'en aperçut, et, sans s'en embarrasser, leur donna leur
fait si sec et si serré, qu'elle les fit taire et leur fit baisser les
yeux. Ce ne fut pas tout\,: après la collation elle s'expliqua si
librement mais si plaisamment sur leur compte, que la peur leur en prit
au point qu'elles lui firent faire des excuses, et tout franchement
demander quartier. M\textsuperscript{me} de Gesvres voulut bien le leur
accorder, mais leur fit dire que ce n'était qu'à condition qu'elles
apprendraient à vivre. Oncques depuis elles n'osèrent la regarder entre
deux yeux. Rien n'était si magnifique que ces soirées de Trianon. Tous
les parterres changeaient tous les jours de compartiments de fleurs, et
j'ai vu le roi et toute la cour les quitter à force de tubéreuses, dont
l'odeur embaumait l'air, mais était si forte par leur quantité, que
personne ne put tenir dans le jardin, quoique très vaste et en terrasse
sur un bras du canal.

Le roi revint de Fontainebleau le 26 octobre et coucha à Villeroy, où il
parut prendre part comme à sa propre maison et parla fort du maréchal de
Villeroy avec beaucoup d'amitié. Il apprit en arrivant à Versailles la
mort du second fils du duc de Noailles, d'un coup de mousquet dans la
tête, se promenant près Strasbourg, au bord du Rhin, qui lui fut tiré de
l'autre côté à balle perdue, et qui était dans le régiment de son frère.
Il sut en même temps que la citadelle de Liège avait été emportée
d'assaut, le gouverneur et la garnison prisonniers\,; que la Chartreuse,
que nous tenions bien fortifiée, ne tarda pas à suivre, et que son armée
fort affaiblie par les détachements pour le Rhin se retirait derrière
les lignes, hors d'état de tenir la campagne, qui finit de la sorte. M.
de Marlborough, en séparant la sienne, se mit sur la Meuse avec M.
d'Obdam, lieutenant général des Hollandais, et M. de Galde-Mersheim, un
des députés des États généraux à l'armée des alliés. Chemin faisant, un
parti de Gueldres vint sur le bord de l'eau, et, à coups de fusil, les
obligea d'aborder. La capture était belle, mais le sot partisan se
contenta du passeport qu'avait le député, qui fit passer Marlborough
pour son écuyer et Obdam pour son secrétaire, et les laissa aller. M. de
Vendôme ne l'avait pas échappé moins belle avant l'arrivée du roi
d'Espagne. Il s'était mis dans une cassine un peu éloignée de son camp,
couverte d'un petit naviglio. On eut beau lui représenter qu'il n'y
était pas en sûreté\,; tout ce qu'on put obtenir fut qu'il ajouterait
une vingtaine de grenadiers à sa garde\,; il était temps. La nuit même
un détachement, des ennemis vint pour l'enlever, et, sans les
grenadiers, qui tinrent ferme et donnèrent le temps à ce qui était le
plus à portée d'accourir au bruit des coups de fusil, il était pris. Sa
campagne finit aussi au commencement de novembre. Il décampa enfin le
premier de Luzzara, et le prince Eugène, qui n'inquiéta point sa
retraite, en décampa aussi le lendemain, et tous deux prirent leurs
quartiers d'hiver et les avantages qu'ils purent.

Le prince d'Harcourt eut enfin permission de faire la révérence au roi,
au bout de dix-sept ans qu'il ne s'était présenté devant lui. Il avait
suivi le roi en toutes ses conquêtes des Pays-Bas et de la
Franche-Comté, mais il était demeuré peu à la cour depuis son, voyage
d'Espagne, où on a vu, ci-devant, que lui et sa femme avaient conduit la
fille de Monsieur au roi Charles II, son époux. Le prince d'Harcourt se
mit au service des Vénitiens, se distingua en Morée, et ne revint qu'à
la paix de cette république avec les Turcs. C'était un grand homme, bien
fait, qui, avec l'air noble et de l'esprit, avait tout à fait celui d'un
comédien de campagne. Grand menteur, grand libertin d'esprit et de
corps, grand dépensier en tout, grand escroc avec effronterie, et d'une
crapule obscure qui l'anéantit toute sa vie. Après avoir longtemps
voltigé après son retour, et ne pouvant vivre avec sa femme, en quoi il
n'avait pas grand tort, ni s'accommoder de la cour ni de Paris, il se
fixa à Lyon avec du vin, des maîtresses du coin des rues, une compagnie
à l'avenant, une meute, et un jeu pour soutenir, sa dépense et vivre aux
dépens des dupes, des sots et des fils de gros marchands qu'il attirait
dans ses filets. Il y tirait toute la considération que lui pouvait
donner là le maréchal de Villeroy par rapport à M. le Grand, et il y
passa de la sorte grand nombre d'années, sans imaginer qu'il y eût en ce
monde une autre ville ni un autre pays que Lyon. À la fin il s'en lassa
et revint à Paris. Le roi, qui le méprisait, le laissait faire, mais ne
voulut pas le voir\,; et ce ne fut qu'au bout de deux mois d'instances,
et de pardons {[}demandés{]} pour lui {[}de la part{]} de tous les
Lorrains\footnote{Le texte du manuscrit a été exactement reproduit. Les
  anciens éditeurs ont modifié ainsi la phrase\,: «\,Ce ne fut qu'au
  bout de deux mois d'instances et de pardons \emph{de tous ses
  larcins}.\,» Saint-Simon, toujours mal disposé pour les Lorrains, n'a
  pas manqué de rappeler que Harcourt était de leur maison et que ce fut
  à leurs instantes sollicitations qu'il dut son retour.}, qu'il lui
permit enfin en ce temps-ci de le venir saluer.

Sa femme, qui était de tous les voyages, favorite de
M\textsuperscript{me} de Maintenon, par la forte et sale raison qu'on en
a vue ailleurs, échoua pour lui sur Marly, où tous les maris allaient de
droit, et sans être nommés dès que leurs femmes l'étaient. Elle
s'abstint d'y aller, espérant que, pour continuer à l'y avoir,
M\textsuperscript{me} de Maintenon obtiendrait la grâce entière. Elle
s'y trompa\,; M\textsuperscript{me} de Maintenon, qui se faisait un
devoir de la protéger en tout, ne laissait pas d'en être souvent
importunée, et de s'en passer fort bien. La peur qu'elle ne s'en passât
tout à fait la fit bientôt retourner seule à Marly\,; et le roi tint bon
à n'y jamais admettre le prince d'Harcourt\,; cela le ralentit sur la
cour\,; mais il retourna peu en province et se cantonna enfin en
Lorraine.

Cette princesse d'Harcourt fut une sorte de personnage qu'il est bon de
faire connaître, pour faire connaître plus particulièrement une cour qui
ne laissait pas d'en recevoir de pareils. Elle avait été fort belle et
galante\,; quoiqu'elle ne fût pas vieille, les grâces et la beauté
s'étaient tournées en gratte-cul. C'était alors une grande et grosse
créature, fort allante, couleur de soupe au lait, avec de grosses et
vilaines lippes, et des cheveux de filasse toujours sortants et
traînants comme tout son habillement. Sale, malpropre, toujours
intriguant, prétendant, entreprenant, toujours querellant et toujours
basse comme l'herbe, ou sur l'arc-en-ciel, selon ceux à qui elle avait
affaire\,; c'était une furie blonde, et de plus une harpie\,; elle en
avait l'effronterie, la méchanceté, la fourbe et la violence\,; elle en
avait l'avarice et l'avidité\,; elle en avait encore la gourmandise et
la promptitude à s'en soulager, et mettait au désespoir ceux chez qui
elle allait dîner, parce qu'elle ne se faisait faute de ses commodités
au sortir de table, qu'assez souvent elle n'avait pas loisir de gagner,
et salissait le chemin d'une effroyable traînée, qui l'ont mainte fois
fait donner au diable par les gens de M\textsuperscript{me} du Maine et
de M. le Grand. Elle ne s'en embarrassait pas le moins du monde,
troussait ses jupes et allait son chemin, puis revenait disant qu'elle
s'était trouvée mal\,: on y était accoutumé.

Elle faisait des affaires à toutes mains, et courait autant pour cent
francs que pour cent mille\,; les contrôleurs généraux ne s'en
défaisaient pas aisément\,; et, tant qu'elle pouvait, trompait les gens
d'affaires pour en tirer davantage. Sa hardiesse à voler au jeu était
inconcevable, et cela ouvertement. On l'y surprenait, elle chantait
pouille et empochait\,; et comme il n'en était jamais autre chose, on la
regardait comme une harengère avec qui on ne voulait pas se commettre,
et cela en plein salon de Marly, au lansquenet, en présence de Mgr et de
M\textsuperscript{me} la duchesse de Bourgogne. À d'autres jeux, comme
l'hombre, etc., on l'évitait, mais cela ne se pouvait pas toujours\,; et
comme elle y volait aussi tant qu'elle pouvait, elle ne manquait jamais
de dire à la fin des parties qu'elle donnait ce qui pouvait n'avoir pas
été de bon jeu et demandait aussi qu'on le lui donnât, et s'en assurait
sans qu'on lui répondît. C'est qu'elle était grande dévote de profession
et comptait de mettre ainsi sa conscience en sûreté, parce que,
ajoutait-elle, dans le jeu il y a toujours quelque méprise. Elle allait
à toutes les dévotions et communiait incessamment, fort ordinairement
après avoir joué jusqu'à quatre heures du matin.

Un jour de grande fête à Fontainebleau, que le maréchal de Villeroy
était en quartier, elle alla voir la maréchale de Villeroy entre vêpres
et le salut. De malice, la maréchale lui proposa de jouer, pour lui
faire manquer le salut. L'autre s'en défendit, et dit enfin que
M\textsuperscript{me} de Maintenon y devait aller. La maréchale insiste,
et dit que cela était plaisant, comme si M\textsuperscript{me} de
Maintenon pouvait voir et remarquer tout ce qui serait ou ne serait pas
à la chapelle. Les voilà au jeu. Au sortir du salut,
M\textsuperscript{me} de Maintenon, qui presque jamais n'allait nulle
part, s'avise d'aller voir la maréchale de Villeroy, devant
l'appartement de qui elle passait au pied de son degré. On ouvre la
porte et on l'annonce\,; voilà un coup de foudre pour la princesse
d'Harcourt. «\,Je suis perdue, s'écria-t-elle de toute sa force, car
elle ne pouvait se retenir\,; elle me va voir jouant, au lieu d'être au
salut,\,» laisse tomber ses cartes, et soi-même dans son fauteuil tout
éperdue. La maréchale riait de tout son cœur d'une aventure si complète.
M\textsuperscript{me} de Maintenon entre lentement, et les trouve en cet
état avec cinq ou six personnes. La maréchale de Villeroy, qui avait
infiniment d'esprit, lui dit qu'avec l'honneur qu'elle lui faisait, elle
causait un grand désordre\,; et lui montre la princesse d'Harcourt en
désarroi. M\textsuperscript{me} de Maintenon sourit avec une majestueuse
bonté, et s'adressant à la princesse d'Harcourt\,: «\,Est-ce comme cela,
lui dit-elle, madame, que vous allez au salut aujourd'hui\,?» Là-dessus
la princesse d'Harcourt sort en furie de son espèce de pâmoison\,; dit
que voilà des tours qu'on lui fait, qu'apparemment M\textsuperscript{me}
la maréchale de Villeroy se doutait bien de la visite de
M\textsuperscript{me} de Maintenon, et que c'est pour cela qu'elle l'a
persécutée de jouer, pour lui faire manquer le salut. «\,Persécutée\,!
répondit la maréchale, j'ai cru ne pouvoir vous mieux recevoir qu'en
vous proposant un jeu\,; il est vrai que vous avez été un moment en
peine de n'être point vue au salut, mais le goût l'a emporté. Voilà,
madame, s'adressant à M\textsuperscript{me} de Maintenon, tout mon
crime,\,» et de rire tous, plus fort qu'auparavant.
M\textsuperscript{me} de Maintenon, pour faire cesser la querelle,
voulut qu'elles continuassent de jouer\,; la princesse d'Harcourt,
grommelant toujours, et toujours éperdue, ne savait ce qu'elle faisait,
et la furie redoublait de ses fautes. Enfin, ce fut une farce qui
divertit toute la cour plusieurs jours, car cette belle princesse était
également crainte, haïe et méprisée.

Mgr {[}le duc{]} et M\textsuperscript{me} la duchesse de Bourgogne lui
faisaient des espiègleries continuelles. Ils firent mettre un jour des
pétards tout du long de l'allée qui, du château de Marly, va à la
perspective, où elle logeait. Elle craignait horriblement tout. On
attira deux porteurs pour se présenter à la porter lorsqu'elle voulut
s'en aller. Comme elle fut vers le milieu de l'allée, tout le salon à la
porte pour voir le spectacle\,; les pétards commencèrent à jouer, elle à
crier miséricorde, et les porteurs à la mettre à terre et à s'enfuir.
Elle se débattait dans cette chaise, de rage à la renverser, et criait
comme un démon. La compagnie accourut pour s'en donner le plaisir de
plus près, et l'entendre chanter pouille à tout ce qui s'en approchait,
à commencer par Mgr {[}le duc{]} et M\textsuperscript{me} la duchesse de
Bourgogne. Une autre fois ce prince lui accommoda un pétard sous son
siège, dans le salon où elle jouait au piquet. Comme il y allait mettre
le feu, quelque âme charitable l'avisa que ce pétard l'estropierait, et
l'empêcha.

Quelquefois ils lui faisaient entrer une vingtaine de Suisses avec des
tambours dans sa chambre, qui l'éveillaient dans son premier somme avec
ce tintamarre. Une autre fois, et ces scènes étaient toujours à Marly,
on attendit fort tard qu'elle fût couchée et endormie. Elle logeait ce
voyage-là dans le château, assez près du capitaine des gardes en
quartier qui était lors M. le maréchal de Lorges. Il avait fort neigé et
il gelait\,; M\textsuperscript{me} la duchesse de Bourgogne et sa suite
prirent de la neige sur la terrasse qui est autour du haut du salon, et
de plain-pied à ces logements hauts, et, pour sien mieux fournir,
éveillèrent les gens du maréchal, qui ne les laissèrent pas manquer de
pelotes\,; puis, avec un passe-partout et des bougies, se glissent
doucement dans la chambre de la princesse d'Harcourt, et, tirant tout
d'un coup les rideaux, l'accablent de pelotes de neige. Cette sale
créature au lit, éveillée en sursaut, froissée et noyée de neige sur les
oreilles et partout, échevelée, criant à pleine tête, et remuant comme
une anguille, sans savoir où se fourrer, fut un spectacle qui les
divertit plus d'une demi-heure, en sorte que la nymphe nageait dans son
lit, d'où l'eau découlant de partout noyait toute la chambre. Il y avait
de quoi la faire crever. Le lendemain elle bouda\,; on s'en moqua d'elle
encore mieux.

Ces bouderies lui arrivaient quelquefois, ou quand les pièces étaient
trop fortes, ou quand M. le Grand l'avait malmenée. Il trouvait avec
raison qu'une personne qui portait le nom de Lorraine ne se devait pas
mettre sur ce pied de bouffonne\,; et comme il était brutal, il lui
disait quelquefois en pleine table les dernières horreurs, et la
princesse d'Harcourt se mettait à pleurer, puis rageait et boudait.
M\textsuperscript{me} la duchesse de Bourgogne faisait alors semblant de
bouder aussi, et s'en divertissait. L'autre n'y tenait pas longtemps,
elle venait ramper aux reproches, qu'elle n'avait plus de bontés pour
elle, et en venait jusqu'à pleurer, demander pardon d'avoir boudé, et
prier qu'on ne cessât plus de s'amuser avec elle. Quand on l'avait bien
fait craqueter, M\textsuperscript{me} la duchesse de Bourgogne se
laissait toucher\,; c'était pour lui faire pis qu'auparavant\,; tout
était bon de M\textsuperscript{me} la duchesse de Bourgogne auprès du
roi et de M\textsuperscript{me} de Maintenon, et la princesse d'Harcourt
n'avait point de ressource\,; elle n'osait même se prendre à aucunes de
celles qui aidaient à la tourmenter, mais d'ailleurs il n'eût pas fait
bon la fâcher.

Elle payait mal ou point ses gens, qui un beau jour de concert
l'arrêtèrent sur le pont Neuf. Le cocher descendit et les laquais, qui
lui vinrent dire mots nouveaux à sa portière. Son écuyer et sa femme de
chambre l'ouvrirent, et tous ensemble s'en allèrent et la laissèrent
devenir ce qu'elle pourrait. Elle se mit à haranguer ce qui s'était
amassé là de canaille, et fut trop heureuse de trouver un cocher de
louage, qui monta sur son siège et la mena chez elle. Une autre fois,
M\textsuperscript{me} de Saint-Simon, revenant dans sa chaise de la
messe aux Récollets, à Versailles, rencontra la princesse d'Harcourt à
pied dans la rue, seule, en grand habit, tenant sa queue dans ses bras.
M\textsuperscript{me} de Saint-Simon arrêta, et lui offrit secours\,:
c'est que tous ses gens l'avaient abandonnée, et lui avaient fait le
second tome du pont Neuf, et pendant leur désertion dans la rue, ceux
qui étaient restés chez elle s'en étaient allés\,; elle les battait, et
était forte et violente, et changeait de domestique tous les jours.

Elle prit, entre autres, une femme de chambre forte et robuste, à qui,
dès les premières journées, elle distribua force tapes et soufflets. La
femme de chambre ne dit mot, et comme il ne lui était rien dû, n'étant
entrée que depuis cinq ou six jours, elle donna le mot aux autres, de
qui elle avait su l'air de la maison, et un matin qu'elle était seule
dans la chambre de la princesse d'Harcourt, et qu'elle avait envoyé son
paquet dehors, elle ferme la porte en dedans sans qu'elle s'en
aperçût\,; répond à se faire battre, comme elle l'avait déjà été, et au
premier soufflet, saute sur la princesse d'Harcourt, lui donne cent
soufflets et autant de coups de poing et de pied, la terrasse, la
meurtrit depuis les pieds jusqu'à la tête, et quand elle l'a bien battue
à son aise et à son plaisir, la laisse à terre toute déchirée, et tout
échevelée, hurlant à pleine tête, ouvre la porte, la ferme dehors à
double tour, gagne le degré, et sort de la maison.

C'était tous les jours des combats et des aventures nouvelles. Ses
voisines à Marly disaient qu'elles ne pouvaient dormir au tapage de
toutes les nuits, et je me souviens qu'après une de ces scènes tout le
monde allait voir la chambre de la duchesse de Villeroy et celle de
M\textsuperscript{me} d'Espinoy, qui avaient mis leur lit tout au
milieu, et qui contaient leurs veilles à tout le monde. Telle était
cette favorite de M\textsuperscript{me} de Maintenon, si insolente et si
insupportable à tout le monde, et qui avec cela, pour ce qui la
regardait, avait toute faveur et préférence, et qui, en affaires de
finances et en fils de famille et autres gens qu'elle a ruinés, avait
gagné des trésors et se faisait craindre à la cour et ménager jusque par
les princesses et les ministres. Reprenons le sérieux.

C'était à la reine d'Angleterre à qui le maréchal de Villeroy était
redevable de sa liberté sans rançon et de la permission enfin de n'être
pas conduit à son retour par l'armée du prince Eugène. M. de Modène,
frère de la reine d'Angleterre, et fort bien avec l'empereur, l'avait
obtenu\,; il ne se peut rien ajouter aux étranges traitements que les
Allemands se plurent de faire essuyer au maréchal et pendant sa prison,
et par les chemins, et à Gratz, capitale de Styrie, où ils le
confinèrent. La populace accabla sa maison de pierres à la nouvelle du
combat de Luzzara. Ils lui firent accroire qu'ils y avaient eu une
pleine victoire, et que nous y avions perdu une infinité de gens de
marque qu'ils lui nommèrent. Ils eurent la cruauté de le laisser un mois
dans le doute sur son fils. Il voulut aussi prendre de grands airs à
Gratz, qui ne lui réussirent pas. Le chemin de son retour fut par Venise
et par Milan, où il s'arrêta avec le cardinal d'Estrées, et il y vit le
roi d'Espagne, il passa par l'armée d'Italie qu'il avait commandée, et
arriva à Versailles le 14 novembre.

Rien n'est égal à la manière dont le roi le reçut et le traita, d'abord
chez M\textsuperscript{me} de Maintenon, puis en public. Cette faveur
alla jusqu'à lui parler d'affaires d'État, et à lui en faire communiquer
quelques dépêches par Torcy. Le chevalier de Lorraine, son ami intime
dès leur jeunesse, et ami de galanteries, d'intrigues, d'affaires, et
d'alliance proche par M, le Grand, et qui avait infiniment d'esprit et
de connaissance du roi et de la cour, lui conseilla d'abdiquer le
commandement des armées, où il n'était pas heureux, et de suivre ce
rayon de faveur si singulier pour essayer d'entrer dans le conseil. Le
chevalier de Lorraine, homme de grandes vues, n'aurait pas été fâché
sans doute d'y avoir un ami de peu de lumières, accoutumé à n'avoir
point de secret pour lui et à s'en laisser conduire en beaucoup de
choses. Il fit tout ce qu'il put pour le persuader qu'établi aussi
complètement qu'il était, ce serait mettre un comble solide à sa
fortune, auquel nul autre portant épée n'était parvenu de ce règne, que
le duc de Beauvilliers. Le maréchal en convint, il lui avoua même qu'à
ce qui se passait du roi à lui, il pouvait se flatter que d'être admis
au conseil ne serait pas une grâce difficile\,; mais il soutint que
quitter le commandement des armées sur les malheurs qui lui étaient
arrivés, ce serait se déshonorer.

Un homme de peu d'esprit et de sens, et qui se croit beaucoup de l'un et
de l'autre, s'entête aisément. Jamais le chevalier de Lorraine ne put le
tirer de ce faux raisonnement. Il ne mit guère à se repentir de n'avoir
pas suivi un conseil si salutaire. Il fut peu de jours après déclaré
général de l'armée de Flandre\,; mais le chevalier de Lorraine n'en vit
pas le triste succès. Il avait eu une légère attaque d'apoplexie pendant
Fontainebleau. Il n'en avait pas quitté sa vie ordinaire. Jouant à
l'hombre dans son appartement du Palais-Royal, après son dîner, le 7
décembre, il lui en prit une seconde, et perdit en même temps
connaissance\,; il en mourut vingt-quatre heures après, sans que la
connaissance lui fût revenue, n'ayant pas encore soixante ans. Il était
lieutenant général, et avait servi sous le roi à toutes ses conquêtes.
Monsieur lui avait donné les abbayes de Saint-Benoît-sur-Loire,
Saint-Père en Vallée à Chartres, de la Trinité de Tiron et de Saint-Jean
des Vignes à Soissons. Il les garda toute sa vie\,; et outre ce qu'il
avait tiré de Monsieur, qui était immense, il avait de grosses pensions
du roi, et souvent des gratifications très considérables. Peu de gens le
regrettèrent, excepté M\textsuperscript{lle} de Lislebonne qu'on croyait
qu'il avait épousée secrètement depuis longtemps. J'ai assez parlé
ailleurs de ces personnages, pour n'avoir rien à y ajouter.

Le comte d'Estrées arriva de Toulon et s'arrêta à Essonne, où toute sa
famille l'alla trouver. Ce fut, au retour, force plaisanteries à sa
femme\,; il fut rapporté à peine à Paris, où peu de jours après,
c'est-à-dire le 23 novembre, on lui fit une grande opération qu'on
n'expliqua point, mais qu'on prétendit qui l'empêcherait d'avoir des
enfants. Son beau-frère, le duc de Guiche, obtint en même temps pour une
confiscation de vingt mille livres de rente sur les biens des Hollandais
en Poitou. Lui et sa femme, qui étaient mal dans leurs affaires, étaient
continuellement à l'affût d'en faire, et les contrôleurs généraux
avaient ordre de ne leur en refuser aucune possible, ni à la maréchale
de Noailles. Il est incroyable tout ce qu'ils en firent.

Le roi permit aussi en même temps au comte d'Albert de sortir de la
Conciergerie, où il était depuis deux ans, quoique le parlement l'eût
absous du duel dont il était accusé\,; mais il demeura cassé. Pertuis,
en prison aussi depuis neuf ans, et le marquis de Conflans aussi, pour
s'être aussi battus, en sortirent de même, mais sans rentrer dans le
service.

Chamois, envoyé du roi à Ratisbonne, en avait été chassé fort
brusquement, il y avait trois mois. Du Héron, envoyé du roi en Pologne,
fut traité de même en ce temps-ci\,; et Boneu, envoyé du roi près du roi
de Suède, passant pays sur la foi de son caractère, fut enlevé par les
Polonais. On arrêta à Paris tous ceux de cette nation et tous les Saxons
qui s'y trouvèrent\,; et, pour s'assurer mieux de la Lorraine, on occupa
Nancy, au cuisant regret de M. et de M\textsuperscript{me} de Lorraine,
qui s'en allèrent pour toujours à Lunéville d'où ils ne sont plus
revenus à Nancy. Le maréchal Catinat, qui ne venait presque point à la
cour, et des moments, eut une audience du roi dans son cabinet, à
l'issue de son lever, courte et honnête, et de la part du maréchal fort
froide et réservée, après laquelle on sut qu'il ne servirait plus.

Le lundi 4 décembre, au sortir du conseil de dépêches, où était Mgr le
duc de Bourgogne, le roi lui dit qu'il lui donnait l'entrée du conseil
des finances et même du conseil d'État, qu'il comptait qu'il y
écouterait et s'y formerait quelque temps sans opiner, et qu'après cela
il serait bien aise qu'il entrât dans tout. Ce prince s'y attendait
d'autant moins, que Monseigneur n'y était entré que beaucoup plus tard,
et fut fort touché de cet honneur. M\textsuperscript{me} de Maintenon,
par amitié pour M\textsuperscript{me} la duchesse de Bourgogne, y eut
grand'part, ainsi que le témoignage que rendit le duc de Beauvilliers de
la maturité et de l'application de ce jeune prince.
M\textsuperscript{me} la duchesse de Bourgogne parut transportée de
joie, et M. de Beauvilliers en fut ravi.

Parlant des conseils, il arriva un notable changement au cérémonial de
celui d'Espagne. Les conseillers d'État, c'est-à-dire les ministres à
notre façon de parler, y sont assis devant le roi, mais le secrétaire
des dépêches universelles qui y rapporte toutes les affaires y est
toujours debout au bas bout de la table ou à son choix à genoux sur un
carreau. Je ne sais si par similitude cela déplut à nos secrétaires
d'État, qui pourtant ne se sont jamais assis du vivant du roi au conseil
des dépêches en présence des ministres assis, qui ne sont jamais entrés
dans les autres conseils que lorsqu'ils ont été ministres, et qui, bien
que ministres, sont demeurés debout en celui des dépêches, ou si le roi
le fit de son mouvement en considération des services qu'Ubilla,
secrétaire des dépêches universelles, avait rendus si essentiellement
lors du testament du roi Charles II\,; quoi qu'il en soit, ce fut à la
recommandation du roi que le roi d'Espagne, en arrivant à Madrid avec le
cardinal d'Estrées, qui entra dans le conseil, y fit asseoir Rivas au
bout de la table. Cette grâce fit quelque rumeur, comme font les
nouveautés dans un pays qui les abhorre, mais elle passa, et Rivas eut
un titre de Castille, et s'appela le marquis de Rivas\,; mais ces titres
ne donnent rien ou comme rien. Une autre nouveauté fit bien plus de
fracas. Le roi d'Espagne, sous prétexte des gardes que la reine son
épouse avait pris sur la fin de sa régence à propos de ces bruits dont
elle s'était effrayée la nuit auprès de son appartement, déclara qu'il
voulait avoir deux régiments des gardes sur le modèle entièrement, pour
le nombre et le service, de ceux de France\,; le premier, d'Espagnols,
et le second, de Flamands ou Wallons que M\textsuperscript{me} des
Ursins fit donner au duc d'Havrech, dont elle avait connu la mère à
Paris, qui était demeurée fort de ses amies. Ils furent levés, formés et
entrèrent en service fort promptement. Le marquis de Custanaga,
gouverneur des Pays-Bas sous Charles II, et qui depuis était demeuré en
considération eu Espagne, et s'était fort bien conduit à l'avènement de
Philippe V, eut le régiment des gardes espagnoles, mais il mourut avant
qu'il fût en état de servir.

Orry fut en même temps renvoyé en Espagne. C'était une manière de
sourdaud de beaucoup d'esprit, de la lie du peuple, et qui avilit fait
toutes sortes de métiers pour vivre, puis pour gagner. D'abord rat de
cave, puis homme d'affaires de la duchesse de Portsmouth qui le trouva
en friponnerie et le chassa. Retourné à son premier métier, il s'y fit
connaître des gros financiers, qui lui donnèrent diverses commissions
dont il s'acquitta à leur gré, et qui le firent percer jusqu'à
Chamillart. On eut envie de savoir plus distinctement ce que c'était que
la consistance et la gestion des finances d'Espagne\,; on n'y voulut
envoyer qu'un homme obscur, qui n'effarouchât point ceux qui en étaient
chargés, et qui eût pourtant assez d'insinuation pour s'introduire, et
de lumière pour voir et en rendre bon compte. Orry fut proposé et
choisi. Il était donc revenu depuis peu d'Espagne pour rendre compte de
ce qu'il y avait appris. M\textsuperscript{me} des Ursins qui, à l'appui
de la régence de la reine dont elle avait saisi les bonnes grâces au
dernier point, avait dès lors projeté de la faire entrer dans toutes les
affaires, et de les gouverner, elle, par ce moyen. Orry lui fit sa
cour\,; son esprit lui plut, elle le trouva obséquieux pour elle, et
d'humeur à entreprendre sous ses auspices. C'était pour elle un moyen de
mettre utilement le nez dans les finances que de l'y pousser\,; ils
lièrent de valet à maîtresse, et en apporta ici les plus fortes
recommandations. Chamillart, ravi qu'on se fût bien trouvé de son choix,
l'appuya ici de toute sa faveur, et le fit renvoyer avec des commissions
qui le firent compter. Nous le verrons devenir assez rapidement un
principal personnage.

En ce même temps, Marsin, que le roi d'Espagne avait mené jusqu'à
Perpignan, arriva à Versailles au lever du roi, qui l'entretint dans son
cabinet, et le soir deux heures chez M\textsuperscript{me} de
Maintenon\,; il fut reçu à merveille\,: aussi n'avait-il rien oublié
pour se concilier tout ce qui le pouvait servir. Desgranges, maître des
cérémonies, avait été au débarquement du roi d'Espagne à Marseille et
l'avait accompagné jusqu'à la frontière de Catalogne pour le faire
servir et sa suite de tout ce qu'il pouvait être nécessaire, et empêcher
les cérémonies et les réceptions, dont il ne voulut aucune, et qui
l'auraient fort importuné.

Il y avait quelque temps qu'il se couvait une querelle entre M. le
chancelier et les évêques, lorsqu'une nouvelle dispute avec M. de
Chartres la fit éclater tout à la fin de cette année. Les évêques, en
possession de faire imprimer leurs mandements ordinaires pour la
conduite et les besoins de leurs diocèses, les livres d'église, quelques
catéchismes courts, à l'usage des enfants, sans permission et de leur
propre autorité, voulurent profiter du double zèle du roi contre le
jansénisme et le quiétisme, et se donner peu à peu l'autorité de
l'impression pour des livres de doctrine plus étendus sans avoir besoin
de permission ni de privilège. Le chancelier ne s'accommoda pas de ces
prétentions, ils se tiraillèrent quelque temps là-dessus\,: les évêques
alléguant qu'étant juges de la foi, ils ne pouvaient être revus ni
corrigés de personne dans leurs ouvragea de doctrine ni par conséquent
avoir besoin de permission pour les faire imprimer\,: le chancelier
maintenant son ancien droit, et que, sans prétendre s'en arroger aucun
sur la doctrine, c'était à lui à empêcher que, sous ce prétexte, les
disputes s'échauffassent jusqu'à troubler l'État\,; qu'il ne se glissât
des sentiments qui, n'étant que particuliers, ne feraient que les
aigrir\,; que la domination anciennement usurpée par les évêques, et
sagement réduite à des bornes tolérables, ne vint à se reproduire\,;
enfin à veiller qu'il ne se glissât rien dans ces ouvrages de contraire
aux libertés de l'Église gallicane.

Cette fermentation dura jusqu'à ce que M. de Meaux, et M. de Chartres
vinrent à y prendre une part personnelle pour leurs ouvrages prêts à
être publiés contre M. Simon, savant inquiet, auteur d'une foule
d'ouvrages ecclésiastiques, entre autres une traduction du Nouveau
Testament avec des remarques littérales et critiques que M. le cardinal
de Noailles et M. de Meaux condamnèrent par des instructions pastorales.
Il se rebéqua par des remontrances. M. de Meaux et M. de Chartres
écrivirent contre lui\,; et ce furent ces ouvrages qu'ils prétendirent
soustraire à l'inspection et à l'autorité du chancelier, qui fit l'éclat
couvé depuis assez longtemps. Avec cet appui les évêques haussèrent le
ton, et prétendirent que c'était à eux, chacun dans son diocèse, à
donner la permission d'imprimer les livres sur la religion, et non à
d'autres à les examiner ni à en permettre ou défendre l'impression.
L'affaire s'échauffa. M\textsuperscript{me} de Maintenon, de longue main
assez peu contente du chancelier pour avoir été ravie de s'en défaire
aux finances, et à la marine par les sceaux, gouvernée d'ailleurs tout à
fait par M. de Chartres, et raccommodée avec M. de Meaux par l'affaire
de M. de Cambrai, se déclara pour eux contre lui. Le roi, tout obsédé
qu'il était par une partialité si puissante et par les jésuites, qui
poussaient le P. de La Chaise contre le chancelier, qu'ils regardaient
comme leur ennemi parce qu'il aimait les règles et qu'il était exact et
délicat sur toutes les matières de Rome, et n'oubliaient rien pour lui
donner auprès du roi l'odieux vernis de jansénisme\,; le roi, dis-je, ne
laissait pas d'être embarrassé. Le chancelier lui montrait la nouveauté
de ces prétentions, et les prodigieux abus qui s'en pourraient faire dès
que tout livre de religion dépendrait uniquement des évêques\,; le
danger que l'ambition de ceux qui tourneraient leurs vues du côté de
Rome pouvait rendre très redoutable, et celui de tout tirer comme
autrefois à la religion, pour dominer indépendamment sur tout. Le roi
craignit donc de juger une question qu'il eût tranchée d'un mot, mais
qui aurait fâché les jésuites et mis M\textsuperscript{me} de Maintenon
de mauvaise humeur. Il pria donc les parties de tâcher de s'accommoder à
l'amiable, et il espéra qu'en les laissant à elles-mêmes, de guerre
lasse enfin, elles prendraient ce parti dont il les pressait toujours.
En effet toutes deux désespérant d'une décision du roi, par conséquent
d'emporter tout ce qu'elles prétendaient, prêtèrent l'oreille à un
accommodement, dont le cardinal de Noailles, et MM. de Meaux et de
Chartres se mêlèrent uniquement pour leur parti.

Les évêques avaient peut-être étendu leurs prétentions au delà de leurs
espérances pour tirer davantage, et le chancelier, peiné de fatiguer le
roi, et d'en voir retomber le dégoût sur soi, par l'adresse des jésuites
et le manège de M\textsuperscript{me} de Maintenon, prit aussi son parti
de finir la querelle en y laissant le moins qu'il pourrait du sien. Il
fut donc enfin convenu que les évêques abandonneraient la prétention
aussi nouvelle que monstrueuse d'avoir l'autorité privative à toute
autre de, permettre l'impression des livres concernant la religion, mais
qu'ils les pourront censurer, ce qui ne leur était pas contesté, et
qu'ils pourront faire imprimer sans permission les livres de religion
dont ils seront les auteurs, article qui fit après une queue. Qu'à
l'égard de leurs rituels, la matière des mariages sera soumise à
l'examen et à l'autorité du chancelier par rapport à l'État. En
particulier sur les ouvrages contre M. Simon, qu'il y serait changé
quelque chose que le chancelier n'approuvait pas.

L'affaire finit ainsi\,; mais le venin demeura dans le cœur\,; les
jésuites ni les évêques, par des vues différentes, ni
M\textsuperscript{me} de Maintenon, à cause de son directeur, ne purent
se consoler d'avoir manqué un si beau coup, ni le chancelier de leur
voir emporter des choses si nouvelles et si dangereuses. C'est ce qui
produisit depuis une lutte entre eux sur cet article des livres de
religion que les évêques voudraient faire. Ils prétendirent que cette
expression enveloppait toute matière de doctrine. Le chancelier
maintenait qu'elle se bornait à ce qu'on appelle livres de liturgie,
missels, rituels et autres semblables\,; de décision il n'y en eut
point\,; mais le chancelier, qui n'avait rien à perdre du côté des
jésuites ni à regagner de celui de M\textsuperscript{me} de Maintenon,
et qui était maître de la librairie, en vint à bout par les menus, et
tint ferme à ne rien laisser imprimer que sous l'examen et l'autorité
ordinaire.

M. de Meaux vieillissait, il aimait la paix, il n'était point ennemi du
chancelier. M. de Chartres, noyé dans Saint-Cyr, et toujours occupé dans
l'intérieur du roi et de M\textsuperscript{me} de Maintenon, et dans la
confidence entière de leur mariage, ne fit plus guère rien au dehors\,;
et des autres évêques, il n'y en avait point, ou bien peu, qui par leurs
ouvrages fussent pour entretenir la dispute\,; mais de cette affaire le
chancelier demeura essentiellement mal avec M\textsuperscript{me} de
Maintenon qui, peu à peu, avec les jésuites l'éreintèrent auprès du roi,
sans toutefois lui en pouvoir liter ni l'estime ni un certain goût
naturel qu'il avait toujours eu pour lui, et que le dégoût de ce
refroidissement empêcha le chancelier, aisé à dépiter, de cultiver et de
ramener comme il lui aurait été aisé de faire pour peu qu'il, en eût
voulu prendre la peine, ainsi que cela parut depuis en plusieurs
occasions qui se retrouveront dans la suite.

Chamilly, revenant de son ambassade de Danemark, salua le roi à la fin
de cette année, et ne fut pas bien reçu\,: il était fils d'un homme très
distingué à la guerre, et qui, s'il eût vécu, aurait été maréchal de
France en 1675, et à qui le roi destinait de loin une compagnie de ses
gardes, et neveu de Chamilly que nous allons bientôt voir maréchal de
France. Chamilly dont je parle était un très grand et très gros homme
qui, avec beaucoup d'esprit, de grâce et de facilité à parler, et
beaucoup de toutes sortes de lectures, se croyait de tout cela le triple
de ce qu'il en avait, et le laissait sentir. Il se rendit odieux au roi
de Danemark et à ses ministres par ses grands airs et ses hauteurs, et
des protections qu'il entreprit contre eux dans leur propre cour et
jusque contre l'autorité du roi de Danemark\,; mais ce qui le perdit
dans l'esprit du roi fut la méprise d'un dessus de lettre à Torcy et à
Barbezieux\,; ce dernier, qui se croyait de ses amis, ouvrit la lettre
écrite à Torcy, y vit un portrait de soi et une espèce de parallèle si
fâcheux, qu'il le perdit auprès du roi si radicalement, qu'après la mort
de Barbezieux même, l'impression ne s'en put jamais effacer. Pareille
aventure était arrivée à d'Avaux avec les deux mêmes, leur écrivant
d'Irlande où il était auprès du roi d'Angleterre, dont il eut toutes les
peines du monde à se relever. Il ne s'en releva même jamais
parfaitement, mais il n'en fut pas perdu comme l'autre, parce qu'il
n'était pas homme de guerre, et que Croissy à qui il avait écrit, et
Torcy depuis, le soutinrent et le firent renvoyer en d'autres
ambassades. On ne saurait croire le nombre et le mal de pareilles
méprises.

En cette même fin d'année, trois bagatelles qui devinrent trois époques
qui se retrouveront\,: la mort du cardinal Cantelmi, archevêque de
Naples, frère du duc de Popoli\,; {[}de{]} Brillac, conseiller au
parlement de Paris, fait premier président du parlement de Bretagne, et
surtout {[}de{]} Champflour, nommé à l'évêché de la Rochelle. Une autre
mort, qui ne vaut pas la peine d'être comptée, arrivée en même temps,
fut celle du duc d'Albemarle, bâtard du roi d'Angleterre Jacques II, en
Languedoc, où il était allé tâcher de se guérir. Sa naissance, si au
goût du roi, l'avait fait, tout jeune, lieutenant général des armées
navales. M. et M\textsuperscript{me} du Maine en faisaient comme de leur
frère, et toutefois l'avaient marié à la fille de Lussan, premier
gentilhomme de la chambre de M. le Prince, et de M\textsuperscript{me}
de Lussan, dame d'honneur de M\textsuperscript{me} la Princesse, qui
n'avait rien, et n'en eut pas d'enfants.

L'année finit par le mariage de mon beau-frère avec la troisième fille
de Chamillart\,; dès l'été précédent, il en avait été parlé dans le
monde, en sorte que je demandai à M\textsuperscript{me} la maréchale de
Lorges ce qu'il convenait que je répondisse aux questions qu'on me
faisait là-dessus\,; elle m'assura qu'il n'y avait rien de fondé en ces
bruits, sur quoi je crus pouvoir et devoir lui parler avec franchise
d'un mariage si peu touchant par l'alliance et les entours, si peu
réparé par le bien, si peu encore par les espérances, avec un gendre tel
que La Feuillade, dont Chamillart était affolé, et tout de suite
j'ajoutai qu'une fille du duc d'Harcourt serait bien plus convenable par
la naissance, par l'état brillant d'Harcourt, par l'âge fort supérieur à
ses enfants qu'aurait ce gendre, susceptible en tout des prémices de sa
faveur. Cela ne fut point goûté, et j'en demeurai là. M. de Lauzun, qui
sur la prochaine opération de M. le maréchal de Lorges n'avait pu éviter
de se rapprocher par degrés, et qu'on vit avec surprise emmener chez lui
la maréchale de Lorges, après ce qui s'était passé de si éclatant, et la
garder chez lui les premiers jours de notre perte commune, voulut en
tirer parti. Il compta se faire un mérite auprès du tout-puissant
ministre de presser le mariage de sa fille, et que, devenant son
beau-frère, cette alliance lui ouvrirait la porte du cœur et de l'esprit
de Chamillart, et le remettrait auprès du roi dans sa première faveur.
Il n'eut pas peine à persuader la maréchale qui en mourait d'envie, ni
le jeune homme à qui il fit accroire que tout par là deviendrait or
entre ses mains.

Tout se fit et se conclut sans que M\textsuperscript{me} de Saint-Simon
ni moi en sussions rien, que par le monde. J'en parlai à la maréchale
qui m'avoua l'affaire seulement fort avancée\,; je ne pus m'empêcher de
lui dire encore mon sentiment. J'ajoutai que, quant à moi, rien ne me
convenait davantage, mais que, par plusieurs raisons, je craignais fort
qu'elle et son fils ne s'en repentissent. Alors elle me parla plus
ouvertement, et je vis si bien que c'était chose faite que je crus en
devoir faire compliment à Chamillart dès le lendemain. Ce qui me pressa
là-dessus fut le souvenir d'un avis que, dès l'été que j'en avais parlé
à la maréchale sur les bruits qui couraient, M\textsuperscript{me} de
Noailles m'avait averti de prendre garde à ne pas montrer de répugnance
pour ce mariage, parce que les Chamillart en étaient avertis, et qu'il
n'en serait autre chose. J'allai donc voir Chamillart que je ne
connaissais que comme on connaît les gens en place, et à qui je n'avais
jamais parlé que lorsque, très rarement, j'avais eu affaire à lui\,: il
quitta pour moi les directeurs des finances avec qui il travaillait. La
réception fut des plus gracieuses. Je me bornais aux compliments,
lorsque ce ministre, avec qui je n'avais pas la plus légère liaison, se
mit à me raconter les détails du mariage, et à me faire ses plaintes des
procédés qu'il avait eus à essuyer de M\textsuperscript{me} la maréchale
de Lorges\,; que ce mariage, fait dès l'été, avait traîné jusqu'alors
par toutes sortes d'entortillements, et m'en dit tant, que plein de mon
côté je ne pus m'empêcher de lui répondre avec la même franchise. Il
m'apprit qu'une pension de vingt mille livres, que le duc de Quintin
avait obtenue à la mort de son père, était uniquement en faveur du
mariage, et il me montra une lettre de la maréchale qu'il avait lue au
roi, dont les termes me firent rougir. Je pense qu'il n'y a point
d'exemple d'une première conversation si pleine de confiance réciproque,
mais prévenue par celle de Chamillart, entre deux hommes aussi peu
connus l'un à l'autre, et d'âges et d'emplois si différents. La surprise
en doit être plus grande, quand on verra, comme je le raconterai
bientôt, que le ministre était plus qu'informé de mon éloignement de ce
mariage, et combien la maréchale de Noailles m'avait fidèlement averti.
Il produisit encore bien de la tracasserie sur l'intérêt entre ma
belle-mère, et moi, qui, non contente de ce que j'avais bien voulu
faire, ne cessa de tenter plus, à force de propositions captieuses, qui
aboutirent enfin à n'accepter ni renoncer à la communauté, et à ne rien
faire de tout ce à quoi les lois obligent les veuves, en quoi, les
procédés de sa part furent encore, s'il se peut, plus étranges que le
fond. Ce détail domestique pourra paraître étranger ici, mais on verra
par la suite qu'il y est nécessaire.

Le mercredi, 13 décembre, nous allâmes à l'Étang, où l'évêque de Senlis
maria mon beau-frère à sa nièce, dont la dot ne fut que de cent mille
écus, comme celle de sa sœur la duchesse de La Feuillade, et de même
logés et nourris partout, ce qui me procura l'usage de l'appartement que
M. le maréchal de Lorges avait dans le château de Versailles. La noce
fut nombreuse et magnifique\,; rien n'égalait la joie du ministre et de
sa famille\,; rien n'approcha des empressements de M. de Lauzun, rien ne
fut pareil à ceux de Chamillart pour M\textsuperscript{me} de
Saint-Simon et pour moi, de sa femme, de ses filles et jusque de ses
amis particuliers qu'il avait conviés. Si j'avais été surpris de la
franchise avec laquelle il m'avait parlé la première fois, je le fus
encore davantage de la façon dont il me demanda mon amitié. La plus
grande politesse et l'énergie se disputèrent en ses expressions, et je
vis la sincérité du désir y dominer. Je fus embarrassé\,; il s'en
aperçut. J'en usai avec lui comme en pareil cas j'avais fait avec le
chancelier. Je lui avouai naturellement mon intimité avec le père, ma
liaison avec le fils, celle de M\textsuperscript{me} de Saint-Simon et
de M\textsuperscript{me} de Pontchartrain, cousines germaines, mais plus
étroitement unies que deux véritables sœurs, et je lui dis que, si à
cette condition il désirait mon amitié, je la lui donnerais de tout mon
cœur. Cette franchise le toucha. Il me dit qu'elle augmentait son
empressement d'obtenir mon amitié, nous nous la promîmes, et nous nous
la sommes toujours tendrement et fidèlement tenue dans tous les temps
jusqu'à la mort. Il était outrément brouillé avec le chancelier et avec
son fils, et eux avec lui. C'était à qui pis se ferait. Je crus donc, au
sortir de l'Étang, leur devoir dire ce qui serait passé entre Chamillart
et moi\,; le chancelier me reçut comme avait fait M. de Beauvilliers en
pareil cas sur lui\,; sa femme et sa belle-fille de même, son fils
autant bien qu'il put être en lui. Ils eurent tous de part et d'autre
cette considération pour moi, et toujours soutenue, qu'en ma présence
quand il y avait quelqu'un, jamais ils ne parlèrent les uns des autres.
Pour en particulier avec moi, ils ne s'en contraignirent pas tant. Ils
se comptaient en sûreté avec moi, et ils ne s'y trompèrent jamais\,; je
devins donc de la sorte ami intime de Chamillart\,; je l'étais déjà des
ducs de Beauvilliers et de Chevreuse et du chancelier, et aussi bien
avec Pontchartrain qu'il était possible. Cela m'initia dans bien des
choses importantes, et me donna un air de considération à la cour fort
différent de ceux de mon âge.

Chamillart ne fut pas longtemps sans me donner des preuves d'amitié.
Sans que j'y pensasse, il voulut me raccommoder avec le roi\,; quoiqu'il
n'y pût réussir, je ne sentis pas moins cette tentative. Un jour que
j'en parlais à sa femme, elle prit un air de plus de confiance encore
qu'à l'ordinaire, et me dit qu'elle était ravie que je fusse plus
content d'eux que je ne l'avais cru, et sur ce que je lui parus
n'entendre point ce langage, elle me dit qu'ils savaient bien que je ne
voulais point du tout que mon beau-frère épousât leur fille, mais
qu'elle m'avouait qu'elle était fort curieuse de savoir pourquoi. Dans
ma surprise, je tournai court et je lui dis qu'il était vrai\,; et que
puisqu'elle en voulait savoir la raison, je la lui dirais avec la même
franchise. Il n'était pourtant pas à propos de l'avoir entière là-dessus
avec elle. Je lui dis que j'avais toujours pensé, sur les mariages,
qu'il ne fallait jamais prendre plus fort que soi, surtout des
ministres, si rarement traitables et raisonnables, pour n'être point
écrasé par ce qu'on a pris pour se soutenir et s'avancer\,; qu'un
mariage égal engageait chaque côté à mettre également du sien, et
faisait plus justement espérer l'union des familles\,; que, pour cette
raison, je n'avais pas goûté leur mariage, et que j'avais proposé celui
d'une fille du duc d'Harcourt par les raisons que j'ai ci-devant
rapportées, et je me rabattis à l'assurer que si je les avais connus
alors tels que je les connaissais maintenant, j'aurais pressé leur
mariage, bien loin d'en dégoûter.

La franchise de ma réponse, et le peu qu'il avait fallu pour l'attirer,
plut tant à M\textsuperscript{me} Chamillart, qu'elle me répondit qu'il
la fallait payer par la sienne. Elle m'apprit que, dès l'hiver
précédent, le mariage s'était traité pour M\textsuperscript{me} de La
Feuillade\,; que, ne s'étant pu faire, et M\textsuperscript{me} de La
Feuillade mariée, M\textsuperscript{me} la maréchale de Lorges avait
tout tenté pour leur dernière fille, par M. de Chamilly et par Robert,
après qu'elle fut partie avec son mari pour la Rochelle\,; enfin par
elle-même\,; qu'il était comme fait lorsque la maréchale me répondit
l'été dernier qu'il n'y avait pas le moindre fondement, qui fut
l'occasion où je lui parlai contre ce mariage et pour celui de
M\textsuperscript{lle} d'Harcourt\,; qu'aussitôt après la maréchale alla
à l'Étang sous un autre prétexte, et qu'en ce voyage, que
M\textsuperscript{me} Chamillart me rappela par des circonstances,
traitant avec elle le mariage, la maréchale lui avait dit que j'y étais
entièrement opposé, et voulais celui de M\textsuperscript{lle}
d'Harcourt. Je laisse les réflexions sur ce trait et sur ses suites,
mais je ne l'ai pas voulu omettre pour montrer combien M. et
M\textsuperscript{me} Chamillart étaient de bonnes gens d'en user après,
cela comme ils firent avec moi, et d'en faire toutes les avances. Cela
aussi scella entièrement notre amitié et notre liaison intime.

Ce mariage eut le sort que j'avais prédit à la maréchale il fut de fer
pour eux et d'or pour moi, non pas en finance, par l'horreur que nous
avons toujours eue, M\textsuperscript{me} de Saint-Simon et moi, de ce
qu'on appelle à la cour faire des affaires, et à quoi tant de gens du
premier ordre se sont enrichis, mais par le plaisir de la confiance de
Chamillart, des services que je fus à portée de rendre à mes amis, et
d'en tirer pour moi, et dans les suites assez promptes, par la
satisfaction de ma curiosité sur les choses de la cour et de l'État les
plus importantes, qui me mettait au fait journalier de tout. Je gardai
ce secret à M\textsuperscript{me} Chamillart excepté pour son mari, avec
qui je me répandis, et lui avec moi, et pour M\textsuperscript{me} de
Saint-Simon qui en fut informée. Il suffit de dire que le mariage alla
tout de travers entre le mari et la femme tant qu'il dura\,; que mon
beau-frère acheva de se perdre en quittant le service aussitôt après ses
noces, sans que l'offre d'être fait brigadier hors de rang le pût
retenir, et que M\textsuperscript{me} de Saint-Simon et moi fûmes
toujours les dépositaires des douleurs de Chamillart et de tout ce
triste domestique. M\textsuperscript{me} la maréchale de Lorges n'avait
acquis ni leur estime ni leur amitié\,; elle prit le parti d'une grande
retraite. C'était bien fait pour l'autre monde, et ne fut guère moins
bien pour celui-ci\,; il faut dire à sa louange qu'à la fin elle rentra
en elle-même\,; et que sa vie fut austère, pénitente, pleine de bonnes
œuvres et parfaitement retirée. Je fus bien des années à revenir pour
elle, cela se retrouvera en son lieu. Je le répète, j'aurais passé sous
silence ce détail triste et peu intéressant, si je ne l'avais jugé tout
à fait nécessaire à montrer l'origine et le fondement de l'intimité qui
se verra dans la suite entre Chamillart et moi, et qui m'a mis à portée
de savoir et de faire fort au delà de mon âge et de mon apparente
situation, tandis que j'y étais de l'autre partie opposée\footnote{Cette
  phrase a été modifiée par les précédents éditeurs qui l'ont
  probablement trouvée trop obscure. Il est cependant facile de
  comprendre ce que Saint-Simon veut dire, qu'il était dans la
  confidence des deux partis opposés, de Chamillart et du chancelier de
  Pontchartrain.}, je veux dire le chancelier et son fils, et par M. de
Beauvilliers mal avec eux, mais fort ami de Chamillart. Les filles de
celui-ci, avec qui j'étais aussi en toute confiance, me mettaient au
fait de mille bagatelles de femme, souvent plus importantes
qu'elles-mêmes ne croyaient, et qui m'ouvraient les yeux, et une
infinité de combinaisons considérables, jointes à ce que j'apprenais par
les dames du palais, mes amies, et par la duchesse de Villeroy avec qui
j'étais étroitement lié, ainsi qu'avec la maréchale sa belle-mère, que
j'eus le plaisir de raccommoder intimement, et de voir durer leur union
jusqu'à leur mort, après avoir été longues années on ne saurait plus mal
ensemble. J'étais aussi très bien avec le duc de Villeroy et en grande
et la plus familière société avec eux\,; mais je ne pus m'accoutumer aux
grands airs du maréchal\,: je trouvais qu'il pompait l'air de partout où
il était, et qu'il en faisait une machine pneumatique. Je ne m'en
cachais ni à sa femme, ni à son fils, ni à sa belle-fille, qui en
riaient et qui ne purent jamais m'y apprivoiser.

Pour ne plus revenir à un triste sujet, je dirai ici d'avance que mon
beau-frère prit {[}le nom de duc de Lorges{]} peu après son mariage,
pour faire porter le nom de Lorges, si illustré par son père, à son
duché de Quintin\,; et qu'il porta depuis le nom de duc de Lorges.

\hypertarget{chapitre-iv.}{%
\chapter{CHAPITRE IV.}\label{chapitre-iv.}}

1703

~

{\textsc{Année 1703.}} {\textsc{- Marsin, chevalier de l'ordre.}}
{\textsc{- Marlborough duc d'Angleterre, etc.}} {\textsc{- Mariage de
Marillac avec une sœur du duc de Beauvilliers.}} {\textsc{- Mariage du
duc de Gesvres avec M\textsuperscript{lle} de La Chénelaye.}} {\textsc{-
Rétablissement de M. le duc d'Orléans dans l'ordre de succession à la
couronne d'Espagne, où il envoie l'abbé Dubois.}} {\textsc{- Promotion
de dix maréchaux de France\,; leur fortune et leur caractère.}}
{\textsc{- Chamilly.}} {\textsc{- Estrées.}} {\textsc{-
Châteaurenauld.}} {\textsc{- Vauban.}} {\textsc{- Rosen.}} {\textsc{-
Huxelles.}} {\textsc{- Tessé.}} {\textsc{- Montrevel.}} {\textsc{-
Tallard.}} {\textsc{- Harcourt.}}

~

Le premier jour de cette année 1703 fut celui de la déclaration que fit
le roi au chapitre de l'ordre, de la distinction sans exemple qu'il fit,
comme je l'ai déjà dit ailleurs d'avance, en faveur du cardinal
Portocarrero, qu'il nomma à la première place vacante de cardinal dans
l'ordre, et toutes quatre étaient alors remplies, et de lui permettre de
porter l'ordre en attendant, dont il lui envoya une croix de diamants de
plus de cinquante mille écus\,; grâce à laquelle il fut extrêmement
sensible. Marsin reçut au même chapitre la récompense de son ambassade
et du mérite qu'il s'était fait du refus de la Toison d'or et de la
grandesse, il fut seul nommé chevalier de l'ordre, et reçu seul à la
Chandeleur suivante. En même temps, le comte de Marlborough fut fait duc
en Angleterre avec cinq mille livres sterling de pension, qui est une
somme prodigieuse.

M. de Beauvilliers maria sa sœur du second lit au fils unique de
Marillac, conseiller d'État, qui était colonel et brigadier
d'infanterie, fort estimé dans les troupes, quoique encore fort jeune,
et qui devait être fort riche, étant unique. Il était de mes amis dès
notre jeunesse, et je puis dire qu'il avait tout ce qu'il fallait pour
se faire aimer, pour réussir à la guerre, et pour plaire à la famille où
on voulait bien le recevoir. Le duc de Saint-Aignan, veuf d'une Servien,
mère du duc de Beauvilliers, avait fait la folie d'épouser, dix-huit
mois après, une créature de la lie du peuple, qui, après avoir eu
longtemps le soin des chiens de sa femme, était montée à l'état de sa
femme de chambre. Il mourut six ans après, parfaitement ruiné, et laissa
deux garçons et une fille de ce beau mariage. La mère avait de l'esprit
et de la vertu. Le roi même, qui aimait M. de Saint-Aignan, l'avait
pressé plus d'une fois de lui faire prendre son tabouret\,; elle n'y
voulut jamais consentir, et se borna à plaire et à avoir soin de M. de
Saint-Aignan dans l'intérieur de sa maison sans vouloir se produire,
mais portant la housse et le manteau ducal. Sa conduite gagna la vertu
de M. et de M\textsuperscript{me} de Beauvilliers, qui, à la mort de M.
de Saint-Aignan, prirent soin d'elle et de leurs enfants comme des
leurs, avec qui ils furent élevés et avec la même amitié\,: ce trait,
soutenu en tout et toute leur vie, n'en est pas un des moindres traits.
Le mariage se fit à petit bruit à Vaucresson, petite maison de campagne
que le duc avait achetée à portée de Versailles et de Marly, où il se
retirait le plus souvent que ses emplois le lui pouvaient permettre.

Le vieux duc de Gesvres, à quatre-vingts ans, se remaria peu de jours
après à M\textsuperscript{lle} de La Chénelaye, du nom de Romillé, belle
et bien faite et riche, que l'ambition d'un tabouret y fit consentir. Le
roi l'en détourna tant qu'il put lorsqu'il lui en vint parler, mais le
bonhomme ne sachant faire pis à son fils, à qui ce mariage fit grand
tort, n'en put être dissuadé. Il voulut faire le gaillard au souper de
la noce, il en fut puni, et la jeune mariée encore plus\,: il fit
partout dans le lit, tellement qu'il en fallut passer une partie à les
torcher et à changer de tout. On peut juger des suites d'un tel mariage.
La belle en usa pourtant bien et en femme d'esprit\,: elle se rendit si
bien maîtresse de celui de son mari, qu'elle le raccommoda avec son
fils, lui fit signer une cession de ses biens pour qu'il ne se ruinât
pas davantage, et la démission de son duché avant l'année révolue\,: on
admira comment elle avait pu en venir à bout. Aussi, l'union entre elle
et le marquis de Gesvres a-t-elle été constante depuis, et s'est
continuée avec ses enfants, qui tous ont toujours eu une grande
considération pour elle\,; du reste, elle ne se contraignit pas
d'elle-même elle était riche.

M. le duc d'Orléans avait toujours sur le cœur d'avoir été oublié dans
le testament du roi d'Espagne, et Monsieur, fils d'Anne, fille et sœur
de Philippe III et de Philippe IV, rois d'Espagne, avait trouvé fort
mauvais de n'avoir pas été appelé au défaut des descendants du duc
d'Anjou. M. le duc d'Orléans en avait fort entretenu Louville au voyage
qu'il fit ici pour celui du roi d'Espagne en Italie. Maintenant que ce
prince en était de retour à Madrid, M. le duc d'Orléans voulut
travailler tout de bon à son rétablissement dans l'ordre de la
succession. Il avait envoyé l'abbé Dubois au passage du roi d'Espagne à
Montpellier pour y prendre des mesures avec Louville et y faire entrer
ce prince\,; et il y fut réglé que deux mois après son retour dans son
royaume, pendant lesquels les choses se prépareraient en faveur de M. le
duc d'Orléans, l'abbé Dubois irait à Madrid pour finir cette affaire,
que le roi aussi désirait, et qui eut en effet son exécution, quelques
mois ensuite, telle que M. le duc d'Orléans la pouvait désirer. C'est ce
même abbé Dubois dont il a été parlé à l'occasion du mariage de M. le
duc d'Orléans, et dont il n'y aura que trop à dire dans les suites.

Le dimanche 14 janvier, le roi fit dix maréchaux de France, qui, avec
les neuf qui l'étaient, firent dix-neuf\,: c'était pour n'en pas
manquer.

Les neuf étaient\,:

1675,

\begin{enumerate}
\def\labelenumi{\Roman{enumi}.}
\setcounter{enumi}{3999}
\tightlist
\item
  \#8203;. de Duras.
\end{enumerate}

1681,

Estrées père.

1693,

Choiseul.

---

Villeroy.

---

Joyeuse.

---

Boufflers.

---

Noailles.

---

Catinat.

1702,

Villars.

Les dix nouveaux furent\,:

\begin{enumerate}
\def\labelenumi{\Roman{enumi}.}
\setcounter{enumi}{3999}
\tightlist
\item
  \#8203;. de Chamilly, lieutenant général en 1678.
\end{enumerate}

Estrées fils\footnote{{[}Estrées{]} prit le nom de maréchal de Cœuvres
  pour le distinguer de son père. Rare singularité de l'être},

\begin{enumerate}
\def\labelenumi{\arabic{enumi}.}
\setcounter{enumi}{1683}
\tightlist
\item
\end{enumerate}

Châteaurenauld,

févr. 1688.

Vauban,

août 1688.

Rosen,

\begin{enumerate}
\def\labelenumi{\arabic{enumi}.}
\setcounter{enumi}{1687}
\tightlist
\item
\end{enumerate}

Huxelles,

\begin{enumerate}
\def\labelenumi{\arabic{enumi}.}
\setcounter{enumi}{1687}
\tightlist
\item
\end{enumerate}

Tessé,

\begin{enumerate}
\def\labelenumi{\arabic{enumi}.}
\setcounter{enumi}{1691}
\tightlist
\item
\end{enumerate}

Montrevel,

\begin{enumerate}
\def\labelenumi{\arabic{enumi}.}
\setcounter{enumi}{1692}
\tightlist
\item
\end{enumerate}

Tallard,

\begin{enumerate}
\def\labelenumi{\arabic{enumi}.}
\setcounter{enumi}{1692}
\tightlist
\item
\end{enumerate}

Harcourt,

\begin{enumerate}
\def\labelenumi{\arabic{enumi}.}
\setcounter{enumi}{1692}
\tightlist
\item
\end{enumerate}

Le roi n'en dit rien jusqu'après son dîner, au sortir de table\,; il
envoya chercher le duc d'Harcourt, Tallard, Rosen et Montrevel. Le
premier et le dernier se trouvèrent à Paris. Tallard arriva le premier
dans le cabinet du roi, qui lui dit qu'il le faisait maréchal de France.
Vint après Rosen, qui fut reçu avec la même grâce. Les deux autres
mandés à Paris vinrent sur-le-champ remercier\,; Chamillart dépêcha des
courriers aux autres qui étaient absents, et Pontchartrain un à
Châteaurenauld, en Espagne, et un au comte d'Estrées, malade à Paris\,:
il avait quarante-deux ans et six semaines, étant né le 30 novembre
1660. Il faut dire un mot de ces messieurs, dont plusieurs ont figuré
dans la suite.

Chamilly s'appelait Bouton, d'une race noble de Bourgogne, dont on en
voit servir avant 1400 avec des écuyers sous eux, et dès les premières
années de 1400, des chambellans des ducs de Bourgogne. Ils ont toujours
servi depuis, et aucun d'eux n'a porté robe\,: quelques-uns ont été
gouverneurs de Dijon. Le père et le frère aîné du maréchal s'attachèrent
à M. le Prince, le suivirent partout, en furent estimés\,; cet aîné,
depuis son retour de Flandre, se distingua tellement aux guerres de
Hollande, sous les yeux du roi, qu'il en acquit assez de part dans son
estime et dans sa confiance pour encourir la jalousie et de là la haine
de Louvois, malgré lequel pourtant il allait être maréchal de France
lorsqu'il mourut, et que le roi a dit depuis qu'il lui avait destiné la
première compagnie de ses gardes du corps qui viendrait à vaquer.

Sous ce frère, celui dont je parle, de six ans plus jeune, commença à se
distinguer. Il avait servi avec réputation en Portugal et en Candie. À
le voir et à l'entendre, on n'aurait jamais pu se persuader qu'il eût
inspiré un amour aussi démesuré que celui qui est l'âme de ces fameuses
\emph{Lettres portugaises}, ni qu'il eût écrit les réponses qu'on y voit
à cette religieuse. Entre plusieurs commandements qu'il eut pendant la
guerre de Hollande, le gouvernement de Grave l'illustra par cette
admirable défense de plus de quatre mois, qui coûta seize mille hommes
au prince d'Orange, dont il mérita les éloges, et à qui il ne se rendit
qu'avec la plus honorable composition, sur les ordres réitérés du roi.
Ce fameux siège l'avança en grades et en divers gouvernements, sans
cesser de servir, malgré la haine de Louvois qu'il avait héritée de son
frère, qui toutefois ne put empêcher que, lorsque le roi se saisit de
Strasbourg au printemps de 1685, il ne lui en donnât le gouvernement\,;
mais le ministre s'en vengea en y tenant le commandant en chef de
l'Alsace, dont le dégoût bannit presque toujours Chamilly de Strasbourg.
La même cause l'empêcha d'être du nombre de tant de militaires qui
furent chevaliers de l'ordre à la fin de 1688, et Barbezieux ne lui fut
pas plus favorable que son père. La femme de son successeur se trouva
amie de celle de Chamilly, qui était une personne singulièrement
accomplie, à qui Louvois même avait eu peine à résister. C'était une
vertu et une piété toujours égale dès sa première jeunesse, mais qui
n'était que pour elle\,; beaucoup d'esprit et du plus aimable et fait
exprès pour le monde, un tour, une aisance, une liberté qui ne prenait
jamais rien sur la bienséance, la modestie, la politesse, le
discernement, et avec cela un grand sens, beaucoup de gaieté, de la
noblesse et même de la magnificence, en sorte que, tout occupée de
bonnes œuvres, on ne l'aurait crue attentive qu'au monde et à ce qui y
avait rapport. Sa conversation et ses manières faisaient oublier sa
singulière laideur\,: l'union entre elle et son {[}mari{]} avait
toujours été la plus intime.

C'était un grand et gros homme, le meilleur homme du monde, le plus
brave et le plus plein d'honneur, mais si bête et si lourd, qu'on ne
comprenait pas qu'il pût avoir quelque talent pour la guerre. L'âge et
le chagrin l'avaient fort approché de l'imbécile. Ils étaient riches
chacun de leur côté, et sans enfants. Sa femme, pleine de vues, séchait
pour lui de douleur. Dans les divers commandements et gouvernements où
elle l'avait suivi, elle avait eu l'art de tout faire, de suppléer
jusqu'à ses fonctions, de laisser croire que c'était lui qui faisait
tout, jusqu'au détail domestique, et partout ils s'étaient fait aimer et
respecter, mais elle singulièrement. Par Chamillart, elle remit son mari
à flot, qui lui procura ce commandement de la Rochelle et des provinces
voisines qu'avait eu le maréchal d'Estrées, avant qu'il allât en
Bretagne, et le porta ainsi au bâton d'autant plus aisément, que le roi
avait toujours eu pour lui de l'estime et de l'amitié\,: sa promotion
trop retardée fut généralement applaudie.

Le comte d'Estrées fut heureux. Son père, qui s'était fort distingué à
la guerre et lieutenant général dès 1655, fut choisi pour passer au
service de mer, lorsque Colbert fit prendre au roi la résolution de
rétablir la marine en 1668. Il y acquit de la gloire dès sa première
campagne, qui fut en Amérique, au retour de laquelle il fut vice-amiral.
M. de Seignelay, ami du comte d'Estrées, contribua fort à lui faire
donner la survivance de cette charge en 1684, à l'âge de vingt-quatre
ans, mais à condition de passer un certain nombre d'années par les
degrés, et que son ancienneté de lieutenant général ne lui serait
comptée que du jour qu'il lui serait permis d'en servir. Seignelay,
maître de l'expédition, et ministre audacieux qui savait nuire et servir
mieux que personne, omit exprès cette dernière condition. Le comte
d'Estrées, servant à terre au siège de Barcelone, prise en 1697 par M.
de Vendôme, prétendit, sinon ne pas rouler avec les lieutenants généraux
comme vice-amiral ayant amené là une escadre, au moins être le premier
d'entre eux. Sur cette dispute, Pontchartrain, encore secrétaire d'État
de la marine et ami particulier de tous les Estrées, trancha la
difficulté en faisant remonter l'ancienneté du comte d'Estrées à la date
de sa survivance\,; il l'emporta sur la mémoire du roi, qui se souvenait
très bien de la condition qu'il avait commandée et qui se trouva omise,
et de cette façon cette ancienneté demeura fixée à l'année 1684.
Lorsqu'il fut question de faire ces maréchaux de France, Châteaurenauld,
l'autre vice-amiral qu'on voulut faire, se trouva moins ancien
lieutenant général et vice-amiral que le comte d'Estrées. Ce dernier
avait pour lui Pontchartrain père et fils, qui pour la marine voulaient
avoir deux bâtons\,; et mieux qu'eux alors, le groupe des Noailles, dont
la faveur était au plus haut point, la considération du maréchal et du
cardinal d'Estrées, celle des enfances de la comtesse d'Estrées, dont le
roi s'amusait beaucoup. Le sujet de plus n'avait contre lui qu'un âge
disproportionné de celui des autres candidats\,; il avait vu beaucoup
d'actions par terre et par mer, et commandé en chef en la plupart des
dernières avec succès, réputation et beaucoup de valeur\,; il entendait
bien la marine, était appliqué, avec de l'esprit et du savoir. Tout cela
ensemble, fondé sur le bonheur de sa survivance à vingt-quatre ans, et
du trait hardi de Seignelay, le fit huit ans après maréchal de France.

C'était un fort honnête homme, mais qui ayant été longtemps fort pauvre,
ne s'épargna pas à se faire riche du temps du fameux Law, dans la
dernière régence, et qui y réussit prodigieusement, mais pour vivre dans
une grande magnificence et fort désordonnée. Ce qu'il amassa de livres
rares et curieux, d'étoffes, de porcelaine, de diamants, de bijoux, de
curiosités précieuses de toutes les sortes, ne se peut nombrer, sans en
avoir jamais su user. Il avait cinquante-deux mille volumes, qui toute
sa vie restèrent en ballots presque tous à l'hôtel de Louvois, où
M\textsuperscript{me} de Courtenvaux, sa sœur, lui avait prêté où les
garder. Il en était de même de tout le reste. Ses gens lassés
d'emprunter tous les jours du linge pour de grands repas qu'il donnait,
le pressèrent tant un jour d'ouvrir des coffres qui en étaient pleins et
qu'il n'avait jamais ouverts depuis dix ans qu'il les avait fait venir
de Flandre et de Hollande, qu'il y consentit. Il y en avait une quantité
prodigieuse. On les ouvrit et on les trouva tous coupés à tous leurs
plis, en sorte que pour les avoir gardés si longtemps tout se trouva
perdu.

Il allait toujours brocantant. Il se souvint d'un buste de Jupiter Ammon
d'un marbre unique et de la première antiquité qu'il avait vu quelque
part autrefois, bien fâché de l'avoir manqué, et mit des gens en
campagne pour le rechercher. L'un d'eux lui demanda ce qu'il lui
donnerait pour le lui faire avoir, il lui promit mille écus. L'autre se
mit à rire, et lui promit de le lui livrer pour rien, ni pour achat ni
pour sa peine, et lui apprit qu'il était dans son magasin, où
sur-le-champ il le mena et le lui montra. On ne tarirait point sur les
contes à en rapporter, ni sur ses distractions.

Avec de la capacité, du savoir et de l'esprit, c'était un esprit confus.
On ne le débrouillait point quand il rapportait une affaire. Je me
souviens qu'un jour au conseil de régence, M. le comte de Toulouse qui,
avec bien moins d'esprit, était la justesse, la précision et la clarté
même, et auprès duquel j'étais toujours assis par mon rang, me dit en
nous mettant à la table que le maréchal d'Estrées allait rapporter une
affaire du conseil de marine qui était importante, mais où je
n'entendrais rien à son rapport, et qu'il me priait qu'il me la pût
expliquer tout bas, comme cela se faisait à l'oreille pendant que le
maréchal rapportait\,; j'entendis assez l'affaire pour être de l'avis du
confite de Toulouse, mais non pas assez distinctement pour bien parler
dessus, de manière que quand ce fut à moi à opiner qui parlais toujours
immédiatement avant le chancelier, et le comte de Toulouse immédiatement
après, je souris et dis que j'étais de l'avis dont serait M. le comte de
Toulouse. Voilà la compagnie bien étonnée, et M. le duc d'Orléans à me
dire en riant que ce n'était pas opiner. Je lui en dis la raison que je
viens d'expliquer et conclus à ce que j'avais déjà fait, ou que la voix
de M. le comte de Toulouse fût comptée pour deux, et l'affaire passa
ainsi. La Vrillière disait de lui que c'était une bouteille d'encre,
qui, renversée, tantôt ne donnait rien, tantôt filait menu, tantôt
laissait tomber de gros bourbillons, et cela était vrai de sa manière de
rapporter et d'opiner. Il était avec cela fort bon homme, doux et poli
dans le commerce, et de bonne compagnie\,; mais bien glorieux et aisé à
égarer, grand courtisan quoique non corrompu. Il faut achever de lui
donner quelques moindres traits.

Il aimait fort Nanteuil, et y avait dépensé follement à un potager. Il y
menait souvent du monde, mais ni portes ni fenêtres qui tinssent. Il fit
boiser toute sa maison. Sa boiserie prête à poser tout entière, on
l'amena et on la mit en pile tout plein une grande salle. Il y a bien
vingt-cinq ans, elle y est encore, et le pont d'entrée, il y en a autant
que personne n'a osé y passer qu'à pied. Il s'impatienta d'ouïr toujours
vanter ces veaux de Royaumont que M. le Grand y faisait nourrir d'œufs
avec leurs coquilles et de lait, dont il donnait des quartiers au roi,
et qui étaient excellents. Il en voulut faire engraisser un à Nanteuil
de même. On le fit, et quand il fut bien gras on le lui manda. Lui
compta qu'en continuant à le nourrir, il engraisserait bien davantage.
Cela dura ainsi plus de deux ans, et toujours en veufs et en lait, dont
les comptes allèrent fort loin pour en faire enfin un taureau qui ne fut
bon qu'à en faire d'autres. Avec cela grand chimiste, grand ennemi des
médecins, il donnait de ses remèdes et y dépensait fort à les faire, et,
de la meilleure foi du monde, se traitait lui-même le premier. Il vécut
toujours fort bien avec sa femme, elle avec lui, chacun à leur manière.

Châteaurenauld, du nom de Rousselet, inconnu entièrement avant le
mariage de son bisaïeul avec une sœur du cardinal et du maréchal de
Retz, à l'arrivée obscure des Gondi en France, fut le plus heureux homme
de mer de son temps, où il gagna des combats et des batailles, et où il
exécuta force entreprises difficiles, et fit beaucoup de belles actions.
L'aventure de Vigo, racontée ailleurs, ne dut pas lui être imputée, mais
à l'opiniâtreté des Espagnols à qui il n'en put persuader le danger.
Elle eut pourtant besoin de toute la protection de Pontchartrain auprès
du roi. Ce secrétaire d'État s'était coiffé de Châteaurenauld, et il
était de plus bien aise de décorer la marine. La promotion de ce
vice-amiral fut fort applaudie\,; il y avait longtemps qu'il avait
mérité le bâton.

C'était un petit homme goussaut, blondasse, qui paraissait hébété, et
qui ne trompait guère. On ne comprenait pas à le voir qu'il eût pu
jamais être bon à rien. Il n'y avait pas moyen de lui parler, encore
moins de l'écouter, hors quelques récits d'actions de mer. D'ailleurs
bon homme et honnête homme. Depuis qu'il fut maréchal de France il
allait assez souvent à Marly, où quand il s'approchait de quelque
compagnie, chacun tournait à droite et à gauche.

Il était Breton, parent de la femme de Cavoye qui avait une maison
charmante à Lucienne tout auprès de Marly, où Cavoye allait souvent
dîner avec bonne compagnie et la plupart gens de faciende\footnote{Vieux
  mot synonyme de \emph{cabale}.}, et de manège, où tout se savait, où
il se brassait mille choses avec sûreté, parce que le roi aimait Cavoye,
et ne se défiait point de ce qui allait chez lui. C'était un monde
trayé, et ce qui était hors de ce cercle ne s'exposait pas à l'y
troubler. M. de Lauzun, trop craint pour être jamais de quelque chose et
qui le trouvait fort mauvais, voulut au moites se divertir aux dépens de
gens avec qui il n'avait point d'accès. Il se mit au commencement d'un,
long voyage de Marly à accoster Châteaurenauld, puis à lui dire que
comme son ami il voulait l'avertir que Cavoye et sa femme, qui se
faisaient honneur de lui appartenir, se plaignaient de ce qu'il ne les
voyait point, et qu'il n'allait jamais chez eux à Lucienne, où ils
avaient toujours bonne compagnie, que c'était des gens que le roi
aimait, qui étaient considérés, qu'il ne fallait point avoir contre soi,
quand on en pouvait aussi aisément faire ses amis, et qu'il lui
conseillait comme le sien d'aller à Lucienne et souvent et longtemps, et
de les laisser faire et dire\,; qu'il l'avertissait qu'ils avaient la
fantaisie de recevoir froidement et de faire tout ce qu'il fallait pour
persuader aux gens qu'ils ne leur faisaient pas plaisir d'aller chez
eux, mais que c'était un jargon et une marotte, que chacun avait ses
manières et sa fantaisie, que telle était la leur\,; mais qu'au fond ils
seraient outrés qu'on les en crût et qu'on s'y arrêtât, et que la preuve
en était au monde qui partout, et surtout à Lucienne abondait chez eux.
Le maréchal fut ravi de recevoir un avis si salutaire, se prit à se
disculper sur Cavoye, à remercier, et surtout à assurer M. de Lauzun
qu'il profiterait de ses bons conseils. Celui-ci lui fit entendre qu'il
ne fallait jamais faire semblant qu'il lui eût donné cet avis, et le
quitta bien résolu au secret et à s'établir promptement à Lucienne.

Il ne tarda pas à y aller. À son aspect, voilà tout en émoi, puis en
silence. Ce fut une bombe tombée au milieu de cet élixir de cour. On
crut en être quitte pour une courte visite\,; il y passa
l'après-dînée\,: ce fut une grande désolation. Deux jours après il
arrive pour dîner, ce fut bien pis\,; ils firent tout ce qu'ils purent
pour lui faire entendre qu'ils étaient là pour éviter le monde et
demeurer en particulier à d'autres\,! Châteaurenauld connaissait ce
langage, et se savait le meilleur gré du monde. Il y persévéra jusqu'au
soir, et les désespéra ainsi presque tous les jours, quelque clairement
que pussent s'expliquer des gens poussés à bout. Ce ne fut pas tout\,;
il se mit à ne bouger de chez eux dès qu'il était à Versailles, et les
infesta toujours depuis à Lucienne toutes les fois qu'il était de Marly.
Ce fut une lèpre dont Cavoye ne put jamais se purifier\,; il disait que
c'était un sort et s'en plaignait à tout le monde, et ses familiers
aussi, qui n'en étaient pas moins affligés que lui. Enfin longtemps
après ils découvrirent celui qui leur avait jeté ce sort. L'histoire en
fut au roi qui en pensa mourir de rire, et Cavoye et ses familiers de
désespoir.

Vauban s'appelait Leprêtre, petit gentilhomme de Bourgogne tout au plus,
mais peut-être le plus honnête homme et le plus vertueux de son siècle,
et avec la plus grande réputation du plus savant homme dans l'art des
sièges et de la fortification, le plus simple, le plus vrai et le plus
modeste. C'était un homme de médiocre taille, assez trapu, qui, avait
fort l'air de guerre, mais en même temps un extérieur rustre et grossier
pour ne pas dire brutal et féroce. Il n'était rien moins. Jamais homme
plus doux, plus compatissant, plus obligeant, mais respectueux, sans
nulle politesse, et le plus avare ménager de la vie des hommes, avec une
valeur qui prenait tout sur soi et donnait tout aux autres. Il est
inconcevable qu'avec tant de droiture et de franchise, incapable de se
prêter à rien de faux ni de mauvais, il ait pu gagner au point qu'il fit
l'amitié et la confiance de Louvois et du roi.

Ce prince s'était ouvert à lui un an auparavant de la volonté qu'il
avait de le faire maréchal de France. Vauban l'avait supplié de faire
réflexion que cette dignité n'était point faite pour un homme de son
état, qui ne pouvait jamais commander ses armées, et qui les jetterait
dans l'embarras si, faisant un siège, le général se trouvait moins
ancien maréchal de France que lui. Un refus si généreux, appuyé de
raisons que la seule vertu fournissait, augmenta encore le désir du roi
de la couronner.

Vauban avait fait cinquante-trois sièges en chef, dont une vingtaine en
présence du roi, qui crut se faire maréchal de France soi-même, et
honorer ses propres lauriers en donnant le bâton à Vauban. Il le reçut
avec la même modestie qu'il avait marqué de désintéressement. Tout
applaudit à ce comble d'honneur, où aucun autre de ce genre n'était
parvenu avant lui et n'est arrivé depuis. Je n'ajouterai rien ici sur
cet homme véritablement fameux, il se trouvera ailleurs occasion d'en
parler encore.

Rosen était de Livonie. M. le prince de Conti me conta qu'il avait eu la
curiosité de s'informer soigneusement de sa naissance, en son voyage de
Pologne, à des gens qui lui en auraient dit la vérité de quelque façon
qu'elle eût été. Il apprit d'eux qu'il était de très ancienne noblesse,
alliée à la meilleure de ces pays-là, et qui avait eu de tout temps des
emplois considérables, ce qui se rapporte aux certificats de la noblesse
de Livonie et du roi de Suède Charles XII que Rosen, dont il s'agit ici,
obtint, et dont celui du czar Pierre Ier, donné à Paris, confirme la
forme. Rosen s'enrôla tout jeune, et servit quelque temps simple
cavalier. Il fut pris avec d'autres en maraude et tira au billet. Le
maréchal ferrant de la compagnie où il était se trouva de sa chambrée.
Il survécut leurs autres camarades, et finit aux Invalides. Tous les ans
Rosen, même maréchal de France, l'envoyait quérir, lui donnait bien à
dîner et dînait avec lui\,; ils parlaient de leurs vieilles guerres, et
le renvoyait avec de l'argent assez considérablement. Outre cela, il
avait soin de s'en informer dans le reste de l'année, et de mettre ordre
qu'il eût de tout et fût à son aise. Rosen, devenu officier, {[}fut{]}
attiré et protégé en France par Rosen, son parent de même nom, qui avait
un régiment et mille chevaux sous le grand Gustave Adolphe, à la
bataille de Lutzen, puis sous le duc de Weimar, {[}qui{]} commanda en
chef pour le roi en Alsace, et mourut en 1667, ayant donné sa fille en
mariage à Rosen dont je parle.

C'était un grand homme sec, qui sentait son reître, et qui aurait fait
peur au coin d'un bois, avec une jambe arquée d'un coup de canon, ou
plutôt du vent du canon, qu'il amenait tout d'une pièce. Excellent
officier, de cavalerie, très bon même à mener une aile, mais à qui la
tête tournait en chef, et fort brutal à l'armée et partout ailleurs qu'à
table, où sans aucune ivrognerie il faisait une chère délicate, et
entretenait sa compagnie de faits de guerre qui instruisaient avec
plaisir. C'était un homme grossier à l'extérieur, mais délié au dernier
point, et qui connaissait à merveille à qui il avait affaire, avec de
l'esprit, du tour et de la grâce en ce qu'il disait du plus mauvais
français du monde qu'il affectait. Il connaissait le roi et son faible
et celui de la nation pour les étrangers\,; aussi reprochait-il à son
fils qu'il parlait si bien français qu'il ne serait jamais qu'un sot.
Rosen fut toujours bien avec les ministres et au gré de ses généraux,
par conséquent du roi, qui l'employa toujours avec distinction, et qui
pourvut souvent à sa subsistance. Châteaurenauld, Vauban et lui étaient
grands-croix de Saint-Louis, et il fut mestre de camp général à la mort
de Montclar, qu'il vendit à Montpéroux, lorsqu'il fut maréchal de
France. En tout c'était un homme qui avait voulu faire fortune, mais qui
en était digne et bon homme et honnête homme, avec la plus grande
valeur. Il m'avait pris en amitié pendant la campagne de 1693, qui avait
toujours continué depuis, et me prêtait tous les ans sa maison toute
meublée à Strasbourg. Nous lui verrons faire une fin tout à fait digne,
sage et chrétienne.

Huxelles, dont le nom était de Laye, et par adoption du Blé, du père du
trisaïeul de celui dont il s'agit ici. Malgré ce nombre de degrés, ce ne
fut que vers l'an 1500 que cette adoption fut faite par le grand-oncle
maternel de ce bisaïeul, dont la femme devint par l'événement héritière
de sa famille, à condition, comme il a été exécuté, de prendre le nom et
les armes de du Blé et de quitter celles de Laye. Avant cela, on ne
connaît pas trop ces de Laye. Il y avait plusieurs familles de ce nom.
Depuis ils ont eu une Baufremont et quelques bonnes alliances. Mais
avant d'aller plus loin, il faut expliquer celles dont notre marquis
d'Huxelles sut faire les échelons de sa fortune.

Son père et son grand-père, qui furent tués à la guerre, et son
bisaïeul, eurent le gouvernement de Châlons et cette petite lieutenance
générale de Bourgogne. Le grand-père épousa une Phélypeaux, par où notre
marquis d'Huxelles se trouva fort proche de Châteauneuf, secrétaire
d'État, et de Pontchartrain depuis chancelier, et du maréchal
d'Humières, c'est-à-dire que son père était cousin germain de
Châteauneuf, issu de germain de Pontchartrain, et germain du maréchal
d'Humières. La sœur du père du marquis d'Huxelles avait fort étrangement
épousé Beringhen, premier écuyer qui avait été premier valet de chambre,
dont le fils, premier écuyer aussi, et cousin germain de notre marquis
d'Huxelles, avait bien plus étrangement encore épousé une fille du duc
d'Aumont et de la sœur de M. de Louvois. L'intrigue ancienne de tout
cela mènerait trop loin. Il suffit de marquer la proximité des alliances
et d'ajouter que l'amitié de la vieille Beringhen pour son neveu, et
l'honneur que son mari tirait d'elle firent élever ce neveu avec leurs
enfants comme frères, que l'amitié a subsisté entre eux à ce même degré,
et que Beringhen, neveu de Louvois par une alliance si distinguée pour
tous les deux, entra dans sa plus étroite confiance et d'affaires et de
famille, fut après sa mort sur le même pied avec Barbezieux, et, tant
par là que par sa charge, fut une manière de personnage. Il protégea son
cousin d'Huxelles de toutes ses forces auprès de Louvois, puis de
Barbezieux, et l'a soutenu toute sa vie. Ce préambule était nécessaire
pour bien faire entendre ce qui suivra ici et ailleurs\,; ajoutons
seulement que le marquis de Créqui, fils du maréchal, avait épousé
l'autre fille du duc d'Aumont et de la sœur de Louvois, et que MM. de
Créqui vivaient fort unis avec M. d'Aumont, les Louvois et les
Beringhen. Revenons maintenant à notre marquis d'Huxelles.

Son père n'avait que dix ans quand il perdit le sien, et vingt lorsqu'il
perdit sa mère. C'était un homme d'ambition qui, trouvant Beringhen dans
la plus intime faveur de la reine régente qui le regardait comme son
martyr, l'avait, pour prémices de son autorité, rappelé des Pays-Bas où
il s'était enfui, et de valet l'avait fait premier écuyer. Huxelles crut
se donner un fort appui en l'honorant à bon marché du mariage de sa
sœur, duquel il était seul le maître, et ne s'y trompa pas. Il servit
avec réputation et distinction\,; il eut même le grade singulier de
capitaine général qui ne fut donné qu'à quatre ou cinq personnes en
divers temps, et qui commandait les lieutenants généraux, et il n'était
pas loin du bâton lorsqu'il fut tué, avant cinquante ans, devant
Gravelines, en 1658. Sa veuve, fille du président Bailleul, surintendant
des finances lors de leur mariage, était une femme galante, impérieuse,
de beaucoup d'esprit et de lecture, fort du grand monde, dominant sur
ses amis, se comptant pour tout, et les autres, ses plus proches même
pour fort peu, qui a su se conserver une considération, et une sorte de
tribunal chez elle jusqu'à sa dernière vieillesse, où la compagnie fut
longtemps bonne et frayée, et où le prix se distribuait aux gens et aux
choses. À son seul aspect, tout cela se voyait en elle. Son fils et elle
ne purent être longtemps d'accord, et ne l'ont été de leur vie. Il se
jeta aux Beringhen qui le reçurent comme leur enfant, il avait près de
vingt-cinq ans quand il la perdit. La plus intime liaison s'était
consolidée entre ses enfants et son neveu, et le vieux Beringhen, qui ne
s'était pas moins conservé d'autorité dans sa famille, que de
considération dans le monde et auprès du roi jusqu'à l'extrême
vieillesse, eut d'autant plus de soin de l'entretenir qu'il aimait ce
neveu comme son fils. Il ne mourut qu'en 1692, et dès 1677 il avait
marié son fils à M\textsuperscript{lle} d'Aumont.

Avec tous ces avantages Huxelles sut cheminer\,; il devint l'homme de M.
de Louvois à qui il rendait compte et qui le mena vite. Il lui fit
donner le commandement de malheureux camp de Maintenon pour l'approcher
du roi, dont les inutiles travaux ruinèrent l'infanterie, et où il
n'était pas permis de parler de malades, encore moins de morts. À
trente-cinq ans, n'étant que maréchal de camp, Louvois lui procura, le
commandement de l'Alsace sous Montclar, puis en chef, à sa mort au
commencement de 1690, et le fit résider à Strasbourg pour mortifier
Chamilly à qui le roi en venait de donner le gouvernement, et quatre ans
après le fit lieutenant général et chevalier de l'ordre à la fin de
1688. Il résida toujours à Strasbourg jusqu'en 1710, roi plutôt que
commandant d'Alsace, et servit, toutes les campagnes sur le Rhin, de
lieutenant général, mais avec beaucoup d'égards et de distinctions.

C'était un grand et assez gros homme, tout d'une venue, qui marchait
lentement et comme se traînant, un grand visage couperosé, mais assez
agréable, quoique de physionomie refrognée par de gros sourcils, sous
lesquels deux petits yeux vifs ne laissaient rien échapper à leurs
regards il ressemblait tout à fait à ces gros brutaux de marchands de
bœufs. Paresseux, voluptueux à l'excès en toutes sortes de commodités,
de chère exquise grande, journalière, en choix de compagnie, en
débauches grecques dont il ne prenait pas la peine de se cacher, et
accrochait de jeunes officiers qu'il adomestiquait, outre de jeunes
valets très bien faits, et cela sans voile, à l'armée et à Strasbourg\,;
glorieux jusqu'avec ses généraux et ses camarades, et ce qu'il y avait
de plus distingué, pour qui, par un air de paresse, il ne se levait pas
de son siège, allait peu chez le général, et ne montait presque jamais à
cheval pendant les campagnes\,; bas, souple, flatteur auprès des
ministres et des gens dont il croyait avoir à craindre ou à espérer,
dominant sur tout le reste sans nul ménagement, ce qui mêlait ses
compagnies et les esseulait assez souvent. Sa grosse tête sous une
grosse perruque, un silence rarement interrompu, et toujours en peu de
mots, quelques sourires à propos, un air d'autorité et de poids, qu'il
tirait plus de celui de son corps et de sa place que de lui-même\,; et
cette lourde tête offusquée d'une perruque vaste lui donnèrent la
réputation d'une bonne tête, qui toutefois était meilleure à peindre par
le Rembrandt pour une tête forte qu'à consulter. Timide de cœur et
d'esprit, faux, corrompu dans le cœur comme dans les mœurs, jaloux,
envieux, n'ayant que son but, sans contrainte des moyens pourvu qu'il
pût se conserver une écorce de probité et de vertu feinte, mais qui
laissait voir le jour à travers et qui cédait même au besoin
véritable\,; avec de l'esprit et quelque lecture, assez peu instruit et
rien moins qu'homme de guerre, sinon quelquefois dans le discours\,; en
tout genre le père des difficultés, sans jamais trouver de solution à
pas une\,; fin, délié, profondément caché, incapable d'amitié que
relative à soi, ni de servir personne, toujours occupé de ruses et de
cabales de courtisan, avec la simplicité la plus composée que j'aie vue
de ma vie, un grand chapeau clabaud toujours sur ses yeux, un habit gris
dont il coulait la pièce à fond, sans jamais d'or que les boutons, et
boutonné tout du long, sans vestige de cordon bleu, et son Saint-Esprit
bien caché sous sa perruque\,; toujours des voies obliques, jamais rien
de net, et se conservant partout des portes de derrière\,; esclave du
public et n'approuvant aucun particulier.

Jusqu'en 1710 il ne venait à Paris et à la cour que des moments, pour se
conserver les amis importants qu'il se savait ménager. À la fin il
s'ennuya de son Alsace, et, sans en quitter le commandement, moins
encore les appointements, car avec une grande dépense que sa vanité et
ses voluptés tiraient de lui, il était avare, il trouva le moyen de
venir demeurer à Paris pour travailler à sa fortune. Sous un masque
d'indifférence et de paresse, il brûlait d'envie d'être de quelque
chose, surtout d'être duc. Il se lia étroitement aux bâtards par le
premier président de Mesmes, esclave de M. et de M\textsuperscript{me}
du Maine, et le plus intime ami de Beringhen, par conséquent le sien.
Par M. du Maine, qui fut la dupe de sa capacité et des secours qu'il
pourrait trouver en lui, il eut quelques secrets accès auprès de
M\textsuperscript{me} de Maintenon. Il ne négligea pas le côté de
Monseigneur\,; Beringhen et sa femme étaient fort amis de la Choin\,;
ils lui vantèrent Huxelles, elle consentit à le voir.

Il devint son courtisan, jusqu'à la bassesse d'envoyer tous les jours de
la rue Neuve-Saint-Augustin, où il logeait, auprès du petit
Saint-Antoine, où elle demeurait, des têtes de lapins à sa chienne. Par
elle il fut approché de Monseigneur, il eut avec lui des entretiens
secrets à Meudon\,; et ce prince, à qui il n'en fallait pas tant pour
l'éblouir, prit une estime pour lui jusqu'à le croire propre à tout, et
à s'en expliquer autant qu'il le pouvait oser. Dès qu'il fut mort, la
pauvre chienne fut oubliée, plus de têtes de lapins\,; la maîtresse le
fut aussi. Elle avait pu la sottise de compter sur son amitié\,;
surprise et blessée d'un abandon si subit, elle lui en fit revenir
quelque chose. Lui-même fit le surpris\,; il ne pouvait comprendre sur
quoi ces plaintes étaient fondées. Il dit effrontément qu'il ne la
connaissait presque pas, et qu'il ne l'était de Monseigneur que par son
nom, ainsi qu'il ne savait pas ce qu'elle voulait dire. De cette sorte
finit ce commerce avec la cause de la faveur, et elle n'en a pas ouï
parler depuis.

En voilà assez pour le présent sur un homme dont j'ai déjà parlé
ailleurs, et que nous verrons toujours le même figurer en plus d'une
sorte, et se déshonorer enfin de plus d'une façon. Nous aurons donc
aussi occasion d'en parler plus d'une fois encore. Il suffira de dire
ici que la tête lui pensa tourner de ne voir point de succès de tant de
menées, et qu'il y avait plusieurs mois qu'il était enfermé chez lui
dans une farouche et menaçante mélancolie, ne voyant presque et qu'à
peine Beringhen, lorsque l'espérance d'aller traiter la paix raffermit
son cerveau déjà fort égaré.

Tessé dont j'ai eu occasion de parler plus d'une fois. Sa mère était
sœur du père du marquis de Lavardin, ambassadeur à Rome, excommunié par
Innocent XI pour les franchises, chevalier de l'ordre, etc., duquel par
l'événement il a beaucoup hérité\,; le frère cadet de son père était le
comte de Froulay, grand maréchal des logis de la maison du roi,
chevalier de l'ordre en 1661, mort en 1671, grand-père de Froulay,
ambassadeur à Venise, de l'évêque du Mans, et du bailli de Froulay,
ambassadeur de son ordre en France. Une autre alliance fut plus utile à
la fortune de Tessé. La mère de son père était Escoubleau, sœur du père
de Sourdis, chevalier, de l'ordre en 1688, puis commandant en Guyenne,
duquel j'ai parlé, ami intime de Saint-Pouange, au fils duquel il donna
enfin sa fille unique, et créature de Louvois, auprès duquel il
produisit Tessé encore tout jeune\,: c'était un grand homme, bien fait,
d'une figure noble, et d'un visage agréable\,; doux, liant, poli,
flatteur, voulant plaire à tout le monde, surtout à la faveur et aux
ministres. Il devint bientôt comme Huxelles, mais dans un genre
différent, l'homme à tout faire de Louvois, et celui qui, de partout,
l'informait de toutes choses. Aussi en fut-il promptement et roidement
récompensé\,: il acheta pour rien la charge nulle de colonel général des
carabins\footnote{Les carabins étaient un corps de cavalerie légère
  souvent cité sous Henri IV et Louis XIII\,; il fut supprimé par Louis
  XIV.} qui le porta, pour la supprimer, à celle de mestre de camp
général des dragons, qui fut créée pour lui dès 1684, étant à peine
brigadier, et il venait d'être fait maréchal de camp en 1688, quand
Louvois le fit faire chevalier de l'ordre. Trois ans après, il eut le
meilleur gouvernement de Flandre qui est Ypres, et, en 1692, il fut tout
à la fois lieutenant général et colonel général des dragons.

C'était un Manceau, digne de son pays\,: fin, adroit, ingrat à
merveille, fourbe et artificieux de même. On en a vu ci-devant un
étrange échantillon avec Catinat, auquel il dut le comble de sa fortune,
pour s'élever sur ses ruines. Il avait le jargon des femmes, assez celui
du courtisan, tout à fait l'air du seigneur et du grand monde, sans
pourtant dépenser\,; au fond, ignorant à la guerre qu'il n'avait jamais
faite, par un hasard d'avoir été partout et de s'être toujours trouvé à
côté des actions et de presque tous les sièges. Avec un air de modestie,
hardi à se faire valoir et à insinuer tout ce qui lui était utile,
toujours au mieux avec tout ce qui fut en crédit, ou dans le ministère,
surtout avec les puissants valets. Sa douceur et son accortise le firent
aimer, sa fadeur et le tuf, qui se trouvait bientôt pour peu qu'il fût
recherché, le firent mépriser. Conteur quelquefois assez amusant,
bientôt après plat et ennuyeux, et toujours plein de vues et de manèges,
il sut profiter de ses bassesses auprès du maréchal de Villeroy, de
Vendôme, de Vaudemont, et par ses souplesses auprès de Chamillart, de
Torcy, de Pontchartrain, de Desmarets, surtout auprès de
M\textsuperscript{me} de Maintenon, chez qui Chamillart d'un côté, et
M\textsuperscript{me} la duchesse de Bourgogne de l'autre, l'initièrent.
Il sut tirer un merveilleux parti du mariage de cette princesse qu'il
avait conclu, et de toute la privance que la tendresse du roi et de
M\textsuperscript{me} de Maintenon lui avait donnée avec eux\,; elle se
piqua d'aimer et de servir Tessé, comme ayant été l'ouvrier de son
bonheur\,; elle sentit qu'en cela même elle plaisait au roi, à
M\textsuperscript{me} de Maintenon, à Mgr le duc de Bourgogne, et Tessé
en sut bien profiter. Elle ne laissait pas d'être quelquefois peinée et
même embarrassée des pauvretés qui lui échappaient souvent, et de
l'avouer à quelques-unes de ses dames du palais. L'esprit n'était pas
son fort\,; un grand usage du monde y suppléait et une fortune toujours
riante, et ce qu'il avait d'esprit tout tourné à l'adresse, la ruse et
les souterrains, et tout fait pour la cour. Il se retrouvera en plus
d'un endroit dans la suite.

Montrevel primait de loin cette promotion par la naissance. Il se
pouvait dire aussi que, jointe à une brillante valeur et à une figure
devenue courte et goussaude, mais qui avait enchanté les dames, elle
suppléait en lui à toute autre qualité. Le roi qui se prenait fort aux
figures (et celle de Tessé ne lui fut pas inutile) et qui avait toujours
du faible pour la galanterie, s'était fort prévenu pour Montrevel. La
même raison le lia avec le maréchal de Villeroy, qui fut toujours son
protecteur. C'était raison\,: jamais deux hommes si semblables, à la
différence du désintéressement du maréchal de Villeroy et du pillage de
Montrevel, né fort pauvre et grand dépensier, qui aurait dépouillé les
autels. Une veine de mécontentement du duc de Chevreuse résolut le roi à
le faire défaire de la compagnie des chevau-légers de sa garde en faveur
de Montrevel. Il lui en fit la confidence sous le plus entier secret.
Montrevel, enivré de sa fortune, ne se put contenir\,; il en fit
confidence à La Feuillade, son ami. Celui-ci, qui ne l'était que de la
fortune, et que sa haine pour Louvois avait lié avec Colbert, courut
l'avertir du danger de son gendre. Colbert en parla au roi, qui, moins
touché en faveur de Chevreuse que piqué contre Montrevel d'avoir manqué
au secret, rassura la charge à Chevreuse, et fut longtemps à faire
sentir son mécontentement à Montrevel. Mais le goût y était\,; sa sorte
de fatuité, qui pourtant était extrême, était toute faite pour le roi.
Les dames, les modes, un gros jeu, un langage qu'il s'était fait de
phrases comme en musique, mais tout à fait vides de sens et fort
ordinairement de raison, les grands airs, tout cela imposait aux sots,
et plaisait merveilleusement au roi, soutenu d'un service très assidu
dont toute l'âme n'était qu'ambition et valeur, sans qu'il ait su jamais
distinguer sa droite d'avec sa gauche, mais couvrant son ignorance
universelle d'une audace que la faveur, la mode et la naissance
protégeaient. Il fut commissaire général de la cavalerie avant Villars,
il eut le gouvernement de Mont-Royal, il commanda en chef dans les pays
de Liée et de Cologne, où il ne s'oublia pas. Sa probité ne passait pas
ses lèvres\,; son peu d'esprit découvrait ses bas manèges et sa
fausseté\,; valet, et souverainement glorieux, deux qualités fort
opposées, qui néanmoins se trouvent très ordinairement unies, et qu'il
avait toutes deux suprêmement. Tel qu'il était, le roi se complut à le
faire maréchal de France, et, n'osant lui confier d'armée, à le faire
subsister par des commandements de province qu'il pilla sans en être
mieux. Il, se retrouvera plus d'une fois dans ces Mémoires. Rien de plus
ridicule que sa fin.

Tallard était tout un autre homme. Harcourt et lui se pouvaient seuls
disputer d'esprit, de finesse, d'industrie, de manège et d'intrigue, de
désir d'être, d'envie de plaire, et de charmes dans le commerce de la
vie et dans le commandement. L'application, la suite, beaucoup de
talents étaient en eux les mêmes, l'aisance dans le travail, et tous
deux jamais un pas sans vue, en apparence même le plus indifférent\,;
l'ambition pareille, et le même peu d'égards aux moyens\,; tous deux,
doux, polis, affables, accessibles en tous temps, et capables de servir
quand il n'y allait de guère, et de peu de dépense de crédit\,; tous
deux les meilleurs intendants d'armée et les meilleurs munitionnaires\,;
tous deux se jouant du détail\,; tous deux adorés de leurs généraux et
depuis qu'ils le furent adorés aussi des officiers généraux et
particuliers et des troupes, sans abandonner la discipline\,; tous deux
arrivés par le service continuel d'été et d'hiver et enfin par les
ambassades, Harcourt plus haut avec M\textsuperscript{me} de Maintenon
en croupe, Tallard plus souple\,; tous deux avec la même {[}habileté{]}
et la même sorte d'ambition\,; et le dernier porté par le maréchal de
Villeroy, et à la fin par les Soubise. Une alliance, point extrêmement
proche, commença et soutint sa fortune dans un temps où les parents se
piquaient de le sentir. La mère de Tallard était fille d'une sœur du
premier maréchal de Villeroy remariée depuis à Courcelles, sous le nom
duquel elle fit tant de bruit en son temps par ses galanteries. Elle
mourut en 1688, et le maréchal son frère en 1685. La mère de Tallard
était fort du grand monde. Tallard, nourri dans l'intime liaison des
Villeroy et courtisan du second maréchal, s'initia dans toutes les
bonnes compagnies de la cour.

C'était un homme de médiocre taille avec des yeux un peu jaloux, pleins
de feu et d'esprit, mais qui ne voyaient goutte\,; maigre, hâve, qui
représentait l'ambition, l'envie et l'avarice\,; beaucoup d'esprit et de
grâces dans l'esprit, mais sans cesse battu du diable par son ambition,
ses vues, ses menées, ses détours, et qui ne pensait et ne respirait
autre chose. J'en ai parlé ailleurs, et j'aurai lieu d'en parler plus
d'une fois encore. Il suffira de dire ici, que qui que ce soit ne se
fiait en lui, et que tout le monde se plaisait en sa compagnie.

Harcourt, j'en ai beaucoup parlé en divers endroits, et j'aurai occasion
d'en parler bien encore. Je pense en avoir assez dit pour le faire
connaître. C'était un beau et vaste génie d'homme, un esprit charmant,
mais une ambition sans bornes, une avarice sordide, et quand il pouvait
prendre le montant, une hauteur, un mépris des autres, une domination
insupportable\,; tous les dehors de la vertu, tous les langages, mais,
au fond, rien ne lui coûtait pour arriver à ses fins\,; toutefois plus
honnêtement corrompu qu'Huxelles et même que Tallard et Tessé\,; le plus
adroit de tous les hommes, en ménagements et en souterrains, et à se
concilier l'estime et les vœux publics sous une écorce d'indifférence\,;
de simplicité, d'amour de sa campagne et des soins domestiques, et de
faire peu ou point de cas de tout le reste. Il sut captiver Louvois,
être ami de Barbezieux et s'en faire respecter, plus encore de
Chamillart jusqu'à ce qu'il trouvât son bon à le culbuter, et de
Desmarets, fort bien avec Monseigneur et la Choin, et avec eux tous sur
un pied de seigneur et de grande estime. On a vu pourquoi et comment il
était si bien avec M\textsuperscript{me} de Maintenon. Cela même
l'écarta des ducs de Chevreuse et de. Beauvilliers et de Mgr le duc de
Bourgogne même, sans rien perdre du côté de M\textsuperscript{me} la
duchesse de Bourgogne. Il savait tout allier et se rallier, jusqu'aux
bâtards, quoique ami de toute sa vie de M. de Luxembourg, de M. le duc
et de M. le prince de Conti. Il était assez supérieur à lui-même pour
sentir ce qui lui manquait du côté de la guerre, quoiqu'il en eût des
parties, mais les grandes il n'y atteignait pas\,; aussi, fort
dissemblable en tout au maréchal de Villeroy, tourna-t-il court vers le
conseil dès qu'il espéra y pouvoir entrer.

Aucun seigneur n'eut le monde et la cour si généralement pour lui, aucun
n'était plus tourné à y faire le premier personnage, peu ou point de
plus capables de le soutenir\,; avec cela beaucoup de hauteur et
d'avarice, qui toutefois ne sont pas des qualités attirantes. Pour la
première il la savait ménager\,; mais l'autre se montrait à découvert
jusque par la singulière frugalité de sa table à la cour, où fort peu de
gens étaient reçus, et qu'il avait avancée à onze heures le matin, pour
en bannir mieux la compagnie. Il mêlait avec grâce un air de guerre à un
air de cour, d'une façon tout à fait noble et naturelle. Il était gros,
point grand, et d'une laideur particulière, et qui surprenait, mais avec
des yeux si vifs et un regard si perçant, si haut et pourtant doux, et
toute une physionomie qui pétillait tellement d'esprit et de grâce, qu'à
peine le trouvait-on laid. Il s'était démis une hanche d'une chute qu'il
fit du rempart de Luxembourg en bas, où il commandait alors, qui ne fut
jamais bien remise et qui le fit demeurer fort boiteux et fort
vilainement, parce que c'était en arrière\,; naturellement gai, et
aimant à s'amuser.

Il prenait autant de tabac que le maréchal d'Huxelles, mais non pas si
salement que lui, dont l'habit et la cravate en étaient toujours
couverts. Le roi haïssait fort le tabac. Harcourt s'aperçut, en lui
parlant souvent, que son tabac lui faisait peine\,; il craignit que
cette répugnance n'éloignât ses desseins et ses espérances. Il quitta le
tabac tout d'un coup\,; on attribua à cela les apoplexies qu'il eut dans
la suite, et qui lui causèrent une terrible fin de vie. Les médecins lui
en firent reprendre l'usage pour rappeler les humeurs à leur ancien
cours, et les détourner de celui qu'elles avaient pris, mais il était
trop tard\,; l'interruption avait été trop longue, et le retour au tabac
ne lui servit de rien. Je me suis étendu sur ces dix maréchaux de
France\,; le mérite de quelques-uns m'y a convié, mais plus encore la
nécessité de faire connaître des personnages qu'on verra beaucoup
figurer en plus d'une façon, comme les maréchaux d'Estrées, d'Huxelles,
de Tessé, de Tallard et d'Harcourt. Reprenons maintenant le courant.

\hypertarget{chapitre-v.}{%
\chapter{CHAPITRE V.}\label{chapitre-v.}}

1703

~

{\textsc{Comte d'Évreux colonel général de la cavalerie\,; son
caractère.}} {\textsc{- Mariage de Beaumanoir avec une fille du duc de
Noailles.}} {\textsc{- Généraux des armées.}} {\textsc{- Ridicules de
Villars sur sa femme.}} {\textsc{- Fanatiques\,; Montrevel en
Languedoc.}} {\textsc{- Encouragements aux officiers.}} {\textsc{-
Gouvernement d'Aire à Marsin, à vendre cent mille livres au maréchal de
Villeroy.}} {\textsc{- Harcourt capitaine des gardes du corps.}}
{\textsc{- Électeur de Bavière déclaré pour la France et l'Espagne.}}
{\textsc{- Kehl pris par Villars.}} {\textsc{- Générosité de Vauban.}}
{\textsc{- Barbezières pris déguisé, sa ruse heureuse.}} {\textsc{-
Grand prieur en Italie sous son frère.}} {\textsc{- Duc de Guiche et
Hautefeuille colonel général et mestre de camp général des dragons.}}
{\textsc{- Comte de Verue commissaire général de la cavalerie.}}
{\textsc{- Bachelier.}} {\textsc{- Trois cent mille livres de brevet de
retenue à M. de La Rochefoucauld.}} {\textsc{- Mort et héritage de la
vieille Toisy.}} {\textsc{- M\textsuperscript{me} Guyon en liberté, mais
exilée en Touraine.}} {\textsc{- Procès sur la coadjutorerie de Cluni,
gagné par l'abbé d'Auvergne.}} {\textsc{- Vertamont plus que mortifié.}}
{\textsc{- Fanatiques\,; raison de ce nom.}} {\textsc{- Bâville\,; son
caractère\,; sa puissance en Languedoc.}} {\textsc{- Ressources secrètes
des fanatiques\,; triste situation du Languedoc.}} {\textsc{- Bals à
Marly.}}

~

Les Bouillon, uniquement attentifs à leur maison, et toujours et en
toutes sortes de temps et de conjonctures, firent en ce temps-ci une
grande affaire pour elle, malgré la profonde disgrâce du cardinal de
Bouillon. Le comte d'Auvergne avait eu la charge de colonel général de
la cavalerie à la mort de M. de Turenne, dans laquelle M. de Louvois,
ennemi de M. de Turenne et de tout ce qui lui appartenait, lui avait
tant qu'il avait vécu donné tous les dégoûts imaginables, et Barbezieux
après lui. Le roi, piqué d'avoir longtemps inutilement travaillé à
l'engager de la vendre à M. du Maine, qu'il en consola enfin par mettre
les carabiniers en corps sous sa charge, avait continué à maltraiter le
comte d'Auvergne dans ses fonctions, et à le traiter médiocrement bien
d'ailleurs. C'était une manière de bœuf ou de sanglier fort glorieux et
fort court d'esprit\,; toujours occupé et toujours embarrassé de son
rang, et pourtant fort à la cour et dans le monde. D'ailleurs honnête
homme, fort brave homme, et officier jusqu'à un certain point\,; il
était fort ancien lieutenant général, il avait bien et longtemps servi.
Lui et M. de Soubise, quoique se voulant donner pour princes, avaient
été mortifiés de n'être point maréchaux de France, et tous deux ne
servaient plus.

Le comte d'Auvergne, par les tristes aventures de ses deux fils laïques,
n'en avait plus que deux, l'un et l'autre dans l'Église\,; des trois
fils de M. de Bouillon, les deux aînés étaient fort mal avec le roi\,:
restait le comte d'Évreux, dont la figure et le jargon plaisaient aux
dames. Avec un esprit médiocre, il savait tout faire valoir, et n'était
pas moins occupé de sa maison que tous ses parents. Il en tirait fort
peu, il n'avait qu'un nouveau et méchant petit régiment d'infanterie, il
était assidu à la guerre et à la cour. Il savait se faire aimer. On
était touché de le voir si mal à son aise, si reculé, si éloigné d'une
meilleure fortune. Il s'attacha au comte de Toulouse\,: cela plut au
roi, de qui il tira quelquefois quelque argent pour lui aider à faire
ses campagnes. Le comte de Toulouse prit de l'amitié pour lui, il en
profita. Le roi fut bien aise d'acquérir à ce fils un ami considérable,
et de lui en procurer d'autres par un coup de crédit, et cela valut au
comte d'Évreux la charge de son oncle, qui par sa persévérance à la
garder la conserva ainsi dans sa maison. Il la vendit six cent mille
livres comme à un étranger\,: il était mal dans ses affaires. La somme
parut monstrueuse pour un cadet qui n'avait rien, et pour un effet de
vingt mille livres de rente. Le cardinal de Bouillon lui donna cent
mille francs\,; M. le comte de Toulouse, qui lui avait fait donner
l'agrément, s'intéressa pour lui faire trouver de l'argent, et il
consomma promptement son affaire. Le roi voulut qu'il servît quelque
temps de brigadier de cavalerie, avant que de faire aucune fonction de
colonel général\,; ce temps-là même fut encore abrégé par la même
protection qui lui avait valu la charge. Il n'avait que vingt-cinq ans,
n'avait servi que dans l'infanterie. Le roi était piqué contre le
cardinal de Bouillon, contre le comte d'Auvergne, contre la fraîche
désertion de son fils, contre le, chevalier de Bouillon, de propos fort
impertinents qu'il avait tenus\,; et malgré tant de raisons, il fit en
faveur du comte de Toulouse la faveur la plus signalée au comte
d'Évreux, tandis qu'aucun des quatre fils de France n'aurait pas osé lui
demander la moindre grâce pour personne, et que s'ils l'avaient hasardé,
outre le refus certain, celui pour qui ils se seraient intéressés aurait
été perdu sans ressources.

La cour venait de voir un mariage fait sous d'étranges auspices,
auxquels aussi le succès répondit promptement ce fut du marquis de
Beaumanoir avec une fille du duc de Noailles. Lavardin, son père, avait
épousé en premières noces une fille du duc de Luynes, dont une fille
unique mariée à La Châtre. Il s'était remarié à une sœur du duc et du
cardinal de Noailles, dont il fut encore veuf, et en laissa un fils
unique, seul reste de son illustre nom, et deux fille et aucun des trois
établis. En mourant il défendit à son fils d'épouser une Noailles sous
peine de sa malédiction, et conjura le cardinal de Noailles, à qui il le
recommanda, de ne le pas souffrir. Je ne sais quel mécontentement il
avait eu d'eux, mais il comprit que son fils étant riche, et ayant
besoin de protection pour entrer dans le monde, pour avoir un régiment
et surtout pour obtenir la lieutenance générale de Bretagne, sur
laquelle il n'avait que cent cinquante mille livres de brevet de
retenue, les Noailles à l'affût des bons partis tâcheraient bien de ne
pas manquer celui-là, qui s'y livrerait volontiers pour trouver ces
avantages, et c'est ce qui l'engagea à y mettre tout l'obstacle que
l'autorité paternelle, la religion et la confiance forcée en son
beau-frère, pour le piquer d'honneur, lui purent suggérer\,; mais
Lavardin eut le sort des rois, dont les volontés sont après leur mort
autant méprisées que redoutées de leur vivant.

Il mourut en août 1710. Les Noailles empêchèrent que le roi disposât de
la charge, quoique fort demandée, et laissèrent croître le petit garçon
qui n'avait que seize ans à la mort de son père, et aucun parent proche
en état de s'opposer à leurs volontés. Ils en prirent soin comme en
étant eux-mêmes les plus proches\,; ils le gagnèrent, ils effacèrent ou
affaiblirent dans son esprit la défense et l'imprécation que son père
lui avait prononcée à la mort\,; ils lui montrèrent un régiment, la
charge de son père, les cieux ouverts à la cour en épousant une de leurs
files. Le jeune homme ne connaissait qu'eux, il se laissa aller, le
mariage se conclut et s'exécuta moyennant la charge\,: on fut surpris
avec raison de la mollesse du cardinal de Noailles. Ceux qui comme moi
savaient avec qu'elle résistance il avait soutenu toutes les, attaques
qui lui avaient été portées lors de l'affaire de M. de Cambrai, et que
lui seul avait empêché le roi de chasser le duc de Beauvilliers et de
donner ses places du conseil au duc de Noailles, son frère, ne purent
comprendre sa complaisance pour sa famille en une occasion qui demandait
toute sa fermeté\,; mais les saints ne font pas toujours des actions
vertueuses, ils sont hommes, et ils le montrent quelquefois. Le cardinal
de Noailles put dire sur cette occasion et sur quelque autre qui se
retrouvera en son temps, mais qui furent épurées par de longues
souffrances, ce que Paul III Farnèse dit avec plus de raison et dans la
plus juste amertume de son cœur en mourant\,: \emph{Si mei non fuissent
dominati, tunc immaculatus essem et emundarer a delicto maximo}. Ce
mariage ne dura pas un an. Le jeune Beaumanoir fut tué à la fin de la
campagne, à la bataille de Spire\,; finit son nom et sa maison, laissa
ses deux sœurs héritières, et sa charge en proie aux Noailles, qui en
marièrent une autre fille à Châteaurenauld, fils de celui que nous
venons de voir faire maréchal de France, et qui eut la lieutenance
générale de Bretagne.

Les dispositions ne tardèrent pas à être faites pour les armées\,; il
n'y eut pas à toucher à celle d'Italie, où le duc de Vendôme était
demeuré\,; le maréchal de Villeroy passait presque tout l'hiver à
Bruxelles, et eut avec le maréchal de Boufflers l'armée de Flandre\,; le
maréchal de Tallard une sur la Moselle, et le maréchal de Villars, resté
à Strasbourg, celle d'Allemagne.

Il y avait tait venir sa femme, dont il était également amoureux et
jaloux, à qui il avait donné pour duègne une de ses sœurs, qui ne la
perdit guère de vue nulle part nombre d'années, et qui se trouvait mieux
là qu'à mourir de faim dans sa province, avec Vogué son mari, où elle ne
retourna plus. Les ridicules furent grands et les précautions pas
toujours heureuses.

Montrevel fut envoyé en Languedoc, où les religionnaires commençaient à
donner de l'inquiétude. Leur nombre et les rigueurs de Bâville,
intendant moins que roi de la province\footnote{Cette phrase, qui
  pourrait présenter quelque obscurité, est expliquée par plusieurs
  autres passages où Saint-Simon dit que Bâville était plutôt roi
  qu'intendant du Languedoc.}, les avait encouragés. Plusieurs avaient
pris les armes et fait de cruelles exécutions sur les curés et sur
d'autres prêtres. Les protestants étrangers attisèrent et soutinrent
sourdement ce feu qui pensa devenir un embrasement funeste. Broglio, qui
y commandait en chef, mais il se peut dire sous Bâville, son beau-frère,
y demeura quelque temps sous le nouveau maréchal. On y envoya quelques
troupes avec un nommé Julien qu'on avait débauché du service de Savoie,
et qui avait bien fait du mal pendant la dernière guère, en brave
aventurier qui connaissait le pays.

Le roi répandit pour cent cinquante mille livres en petites pensions
dans les corps, et releva l'émulation pour l'ordre de Saint-Louis, en le
conférant à Mgr le duc de Bourgogne, non seul et en particulier, comme
il avait fait à Monseigneur seul, mais en public, et à la tête d'un
nombre d'officiers qu'il fit en même temps chevaliers de Saint-Louis. Il
donna peu après le gouvernement d'Aire à vendre à Marsin, vacant par la
mort du chevalier de Tessé, frère du maréchal, mort l'été précédent à
Mantoue, où il commandait\,; et cent mille francs au maréchal de
Villeroy pour faire son équipage\,; puis, disposa enfin de la charge de
capitaine des gardes de mon beau-père en faveur du maréchal d'Harcourt,
qui, de tous les candidats, était le moins en état de l'exercer, et
celui de tous aussi qui la désirait le moins ardemment. Il était sans
cela fort rapproché du roi, mais M\textsuperscript{me} de Maintenon, sa
protectrice, qui n'avait pas moins de désir que lui-même de le voir dans
le conseil, jugea que l'assiduité nécessaire et les détails de cette
charge seraient une ressource pour l'y conduire.

En conséquence du traité que Puységur, de qui j'ai eu souvent occasion
de parler, avait fait, dès la Flandre, avec l'électeur de Bavière, ce
prince était retourné dans ses États préparer à l'empereur une guerre
fâcheuse, à l'ombre d'une neutralité suspecte. On avait grand besoin
d'une pareille diversion\,; l'électeur enfin venait de lever le masque,
nonobstant la déclaration de la diète de Ratisbonne, que la guerre de la
succession d'Espagne était guerre d'empire. Il fallait soutenir
l'électeur, et lui fournir un puissant secours, suivant l'engagement
réciproque. Villars, plus occupé de sa femme que d'exécuter les ordres
dont il était chargé, passa enfin le Rhin au commencement de février,
après force délais, et fut remplacé au deçà par Tallard, fortifié d'un
gros détachement de Flandre. L'électeur cependant faisait force petites
conquêtes en attendant qu'il se fût formé une armée impériale pour
s'opposer à lui. Cependant Villars assiégea le fort de Kehl, qui se
rendit le 9 mars\,; on y perdit fort peu de monde, et la défense fut
molle. Trois, mille hommes environ qui en sortirent furent conduits à
Philippsbourg. On y trouva vingt-six milliers de poudre\,; mais les
paysans tuèrent une infinité de maraudeurs. Vauban avait proposé au roi
de l'envoyer à Kehl, qui trouva que cela serait au-dessous de la dignité
où il venait de l'élever\,; et quoique Vauban insistât avec toute la
reconnaissance, la modestie et la bonne volonté possibles, le roi ne
voulut pas le lui permettre\,; et peu de jours après il l'en récompensa
par des entrées moindres que celles des brevets, mais plus grandes que
celles de la chambre.

Barbezières, envoyé de l'armée d'Italie conférer avec l'électeur de
Bavière sur divers projets, et qui était un excellent officier général,
fort hasardeux, avec de l'esprit, et fort avant dans la confidence du
duc de Vendôme, fut pris déguisé en paysan près du lac de Constance,
passant pays à pied, et fut conduit à Inspruck, jeté dans un cachot,
puis gardé à vue. Ne sachant comment donner de ses nouvelles, et
craignant d'être pendu comme un espion, il fit le malade, et demanda un
capucin à qui il tira bien fort la barbe pour voir si ce n'était point
un moine supposé. Quand il s'en fut assuré, il essaya de le toucher et
de l'engager à faire avertir M. de Vendôme de l'état misérable et
périlleux où il se trouvait. Le capucin se trouva charitable, et il le
fit sans perdre de temps. Aussitôt M. de Vendôme manda au comte de
Staremberg, qui commandait l'armée impériale en l'absence du prince
Eugène, qu'il ferait au commandant et à toute la garnison de Vercelli
les mêmes traitements qu'on ferait à Barbezières, qu'ils savaient bien
être lieutenant général des armées du roi\,: peut-être cela lui
sauva-t-il la vie\,; mais la prison fut longue et extrêmement dure,
surtout d'être jour et nuit gardé à vue, pour un homme aussi vif et
aussi pétulant que l'était Barbezières, qu'ils renvoyèrent à la fin.
Parlant d'Italie, M. du Maine obtint avec grand'peine que le grand
prieur allât servir sous son frère en Italie où son ancienneté le
faisait premier lieutenant général.

Tessé, devenu maréchal de France, ne se souciait plus de sa charge de
colonel général des dragons. Il la vendit quatre cent quatre-vingt mille
livres au duc de Guiche, qui en était mestre de camp général, et se
défit de cette dernière charge à Hautefeuille. Par même raison, Villars
fit aussi de l'argent de la sienne de commissaire général de la
cavalerie, et en eut gros du comte de Verrue que sa triste situation
avait banni depuis longtemps de son pays, et qui se voulut lier tout de
bon au service de France.

M. de La Rochefoucauld obtint en même temps la survivance de la charge
de premier valet de garde-robe du roi, qu'avait Bachelier, pour son
fils. Il aimait extrêmement le père qui avait été son laquais, et que de
là il avait poussé à cette fortune. Il faut dire aussi que ce Bachelier
était un des plus honnêtes hommes qu'on pût voir, le plus modeste, le
plus respectueux, le plus reconnaissant pour son maître. Il avait
conservé un crédit sur lui dont ses amis et le plus souvent encore ses
enfants avaient besoin. M. de La Rochefoucauld aimait bien mieux ses
valets que ses enfants, et ruinait ces derniers pour eux. Bachelier, se
comporta toujours avec tant de droiture et d'attachement entre le père
et les enfants, qu'ils l'aimaient presque autant que le père\,; j'ai ouï
M. de La Rocheguyon, et le duc de Villeroy, son ami intime, et son
beau-frère en faire de grandes louanges, et quoique Bachelier fût devenu
riche, jamais on n'a soupçonné sa probité. Son fils ne vaut pas moins.
Il acheta de Bloin, après la mort du roi, sa charge de premier valet de
chambre, et il y a apparence qu'après le premier ministre auquel il a pu
résister, malgré la toute-puissance de ce cardinal, il figurera beaucoup
dans l'intérieur des cabinets. Bientôt après M. de La Rochefoucauld eut
trois cent mille livres de brevet de retenue sur ses charges, M. de La
Rocheguyon, son fils, en avait les survivances depuis longtemps\,: ce
fut donc à ses dépens, à quoi il fut obligé de consentir.

La vieille Toisy, dont j'ai parlé à l'occasion du mariage de la comtesse
d'Estrées, dont elle avait fourni la plus grande partie de la dot,
mourut fort vieille, s'étant toujours conservé son tribunal chez elle et
tout son air d'autorité à force d'esprit. Elle n'avait point d'enfants,
et toute bourgeoise qu'elle était, elle n'estima pas ses parents dignes
d'hériter d'elle. Elle avait donné en mariage à la duchesse de Guiche et
à la comtesse d'Estrées. Les Noailles, qui sentaient la succession
bonne, lui avaient toujours fait soigneusement leur cour\,; ce ne fut
pas en vain\,: elle donna presque tout ce qu'elle avait à la duchesse de
Noailles, et fit une amitié de quarante mille livres au cardinal
d'Estrées, son bon ami, pour qu'en revenant d'Espagne, il trouvât à
acheter quelque petite maison pour aller prendre l'air autour de Paris.

Un personnage du même sexe, plus rare et plus célèbre, obtint en ce
temps-ci sa liberté. Les amis de M\textsuperscript{me} Guyon, toujours
attentivement fidèles, en furent redevables à la charité toujours
compatissante du cardinal de Noailles qui la fit sortir de la Bastille
où elle était depuis plusieurs années sans voir personne, et lui obtint
la permission de se retirer en Touraine. Ce ne fut pas la dernière
époque de l'illustre béate, mais la liberté lui fut toujours depuis
conservée. Le cardinal de Noailles n'en recueillit rien moins que la
reconnaissance de tout ce petit troupeau.

Le cardinal de Bouillon n'était pas en repos dans son exil. Les moines
de Cluni en avaient voulu profiter. Il leur avait arraché la
coadjutorerie pour son neveu plutôt qu'il ne l'avait obtenue. Ils
n'avaient osé résister au nom du roi et à la présence du cardinal allant
à Rome dans la faveur où il était pour lors\,; mais ils s'étaient ménagé
des moyens à la pouvoir contester un jour. Il y avait eu du bruit et des
oppositions étouffées par autorité\,; les moines étaient fort affligés
de se voir toujours hors de mains régulières\,; ils étaient encore plus
outrés de se voir passer des cardinaux à un abbé, qui n'avait pas le
même privilège que le sacré collège se donne, de pouvoir tout posséder
et régir. Ils ne virent donc pas plutôt le cardinal en disgrâce qu'ils
attaquèrent la coadjutorerie au grand conseil, et donnèrent bien à
courir aux Bouillon. Outre les raisons du procès, le meilleur moyen des
moines était de persuader aux juges que le roi, mécontent de leur abbé,
y prenait part pour eux, tellement que les Bouillon voulurent se parer
de leurs proches, faire effort de crédit et faire comprendre par cette
assistance ouverte que le roi demeurait neutre entre eux. Je ne pus
refuser d'aller avec eux à l'entrée des juges, et les solliciter avec le
duc d'Albret et l'abbé d'Auvergne, et de dire à chacun bien
affirmativement que le roi n'y prenait aucune part. Ces sollicitations
durèrent ainsi que les entrées des juges, où la compagnie était assez
nombreuse\,; enfin le 30 mars l'abbé d'Auvergne gagna en plein, tout
d'une voix. Ils me surent un gré infini d'avoir toujours été avec eux
partout, dont plusieurs s'étaient très souvent dispensés. Je les
retrouvai après bien à point dans une autre affaire où ils me servirent
très utilement, et avec la dernière chaleur. On est fort, quand on se
soutient dans les familles et les parentés, et on est toujours la dupe
et la proie de s'abandonner, c'est ce qui se voit et se sent tous les
jours avec un dommage irréparable. L'arrêt signé, l'abbé d'Auvergne fut
bien étonné de ne le pas trouver tel que tous les juges l'avaient dit,
en les allant remercier. Il s'en plaignit à Vertamont, premier
président\,; la dispute fut forte. Les Bouillon crièrent, menacèrent de
se plaindre au roi et au grand conseil. Les juges s'émurent, il fallut
leur porter l'arrêt, ils le réformèrent aux hauts cris de Vertamont à
qui pour l'honneur de la présidence on laissa dans l'arrêt quelque chose
de ce qui n'y avait pas été prononcé.

Montrevel ne trouva pas les \emph{fanatiques} si aisés à réduire qu'il
avait cru. On leur avait donné ce nom, parce que chaque troupe
considérable de ces protestants révoltés avait avec eux quelque prétendu
prophète ou prophétesse, qui, d'intelligence avec les chefs, faisaient
les inspirés et menaient ces gens-là où ils voulaient, avec une
confiance, une obéissance et une furie inconcevable.

Le Languedoc gémissait depuis longues années sous la tyrannie de
l'intendant Bâville, qui, après avoir culbuté le cardinal Bonzi, comme
on le dira en son lieu, tira toute l'autorité à lui, et qui, pour que
rien ne lui en pût échapper, fit donner le commandement des armées dans
toute la province à son beau-frère Broglio, qui n'avait pas servi depuis
la malheureuse campagne de Consarbrück du maréchal de Créqui, où il
était maréchal de camp. Par ce moyen, le commandement et toute
considération des lieutenants généraux de la province tombèrent, et tout
fut réuni à Bâville, devant qui son beau-frère, d'ailleurs très
incapable, ne fut qu'un petit garçon. Bâville était un beau génie, un
esprit supérieur, très éclairé, très actif, très laborieux. C'était un
homme rusé, artificieux, implacable, qui savait, aussi parfaitement
servir ses amis et se faire des créatures\,; un esprit surtout de
domination qui brisait toute résistance, et à qui rien ne coûtait, parce
qu'il n'était arrêté par rien sur les moyens. Il avait fort augmenté le
produit de la province\,; l'invention de la capitation l'avait beaucoup
fait valoir. Ce génie vaste, lumineux, impérieux était redouté des
ministres, qui ne le laissaient pas approcher de la cour, et qui, pour
le retenir en Languedoc, lui laissaient toute puissance, dont il abusait
sans ménagement.

Je ne sais si Broglio et lui se voulurent faire valoir du côté des
armes, mais ils inquiétèrent fort les non ou mauvais convertis, qui à la
fin s'attroupèrent. On sut après que Genève d'une part, le duc de Savoie
d'autre, leur fournirent des armes et des vivres dans le dernier
secret\,; l'une, des prédicants, l'autre, quelques gens de tête et de
main, et de l'argent\,; tellement qu'on fût très longtemps dans la
surprise de les voir en apparence dénués de tout, et néanmoins se
soutenir et entreprendre.

On eut grande obligation à ce fanatisme qui s'empara d'eux, et qui
bientôt leur fit commettre les derniers excès en sacrilège, en meurtres
et en supplices sur les prêtres et les moines. S'ils s'en étaient tenus
à ne maltraiter personne que suivant les lois de la guerre, à demander
seulement liberté de conscience et soulagement des impôts, force
catholiques qui par crainte, par compassion ou par espérance que ces
troubles forceraient à quelques diminutions de subsides, auraient
persévéré et peut-être levé le masque sous leur protection, et en
auraient entraîné le grand nombre.

Ils avaient des cantons entiers, et presque quelques villes de leur
intelligence, comme Nîmes, Uzès, etc., et force gentilshommes distingués
et accrédités dans le pays qui les recevaient clandestinement dans leurs
châteaux, qui les avertissaient de tout, et à qui ils s'adressaient avec
sûreté, qui eux-mêmes pour la plupart avaient leurs ordres et leurs
secours de Genève ou de Turin. Les Cévennes et les pays voisins pleins
de montagnes et de déserts étaient une merveilleuse retraite pour ces
sortes de gens, d'où ils faisaient leurs courses. Broglio, qui y voulut
faire le capitaine, y fut traité et s'y conduisit en intendant. Ni
troupes, ni artillerie, ni vivres, ni armes nulle part, en sorte que
Montrevel fut obligé de demander de toutes ces choses, en attendant
lesquelles les fanatiques désolaient toujours la province, en recevant
aussi de temps en temps quelques petites pertes de la part de Julien.
Broglio, qui n'entendait rien qu'à dominer sous l'ombre de Bâville, fut
rappelé, et eut l'impudence de répandre que c'était avec parole d'être
fait chevalier de l'ordre. On envoya trois ou quatre lieutenants
généraux ou maréchaux de camp à Montrevel avec vingt bataillons et de
l'artillerie, dont il sut très médiocrement s'aider. On pendit quelques
chefs qui furent pris en divers petits combats ou surprises. Ils se
trouvèrent tous de la lie du peuple\,; et leur parti n'en fut ni effrayé
ni ralenti.

Tant d'occupations étrangères et domestiques n'empêchèrent pas le roi de
s'amuser à des bals à Marly.

\hypertarget{chapitre-vi.}{%
\chapter{CHAPITRE VI.}\label{chapitre-vi.}}

1703

~

{\textsc{Honteux délais de Villars de passer en Bavière\,; jaloux de sa
femme, refuse de la mener avec lui\,; joint enfin l'électeur.}}
{\textsc{- Mort de la comtesse Dalmont à Saint-Germain.}} {\textsc{-
Mort du baron d'Hautefeuille, ambassadeur de Naples.}} {\textsc{- Mort
de Bechameil\,; sa fortune et son caractère.}} {\textsc{- Prince
d'Auvergne pendu en Grève en effigie.}} {\textsc{- Défection du duc
Molez.}} {\textsc{- Duc de Bourgogne déclaré pour l'armée sur le Rhin,
avec Tallard sous lui et Marsin près de lui.}} {\textsc{- Duchesse de
Ventadour quitte Madame\,; ses vues.}} {\textsc{- Duchesse de Brancas
dame d'honneur de Madame pour son pain\,; son caractère et ses
malheurs.}} {\textsc{- Mort de Félix\,: Maréchal premier chirurgien du
roi en sa place\,; son caractère.}} {\textsc{- Curieux fait d'un voyage
de Maréchal à Port-Royal des Champs.}} {\textsc{- Comtesse de
Grammont\,; son caractère\,; sa courte disgrâce\,; le roi lui donne
Pontali.}} {\textsc{- Mort d'Aubigné.}} {\textsc{- Aversion du roi pour
le deuil.}} {\textsc{- Maladie du comte d'Ayen, singulièrement visité.}}
{\textsc{- Papiers du P. Quesnel pris et lui arrêté, qui s'échappe.}}
{\textsc{- Disgrâce de l'archevêque de Reims et son raccommodement.}}
{\textsc{- Mort de Gourville\,; son mariage secret et sa sage
disposition.}} {\textsc{- Bonn rendu par d'Alègre.}} {\textsc{- Combat
d'Eckeren.}} {\textsc{- Toison d'or à Boufflers.}} {\textsc{- Bedmar
conseiller d'État en Espagne.}} {\textsc{- Trois cent mille livres de
brevet de retenue, outre trois cent mille autres, à Chamillart.}}
{\textsc{- Walstein, ambassadeur de l'empereur en Portugal,
prisonnier.}} {\textsc{- Succès de mer.}}

~

Kehl pris, et les comtes Schick et Stirum à la tête des troupes
impériales pour contenir l'électeur de Bavière, il devenait fort pressé
de faire passer une armée à son secours\,; Villars et la sienne y
étaient destinés. Il était revenu à Strasbourg après sa conquête\,; il
fut difficile de l'en faire sortir\,; il ne pouvait s'éloigner de sa
femme. Le prince Louis rassemblait des troupes, et se retranchait aux
passages des montagnes. Le maréchal lui envoya demander un passeport
pour sa femme\,; il en fut refusé, et il s'en vengea depuis honteusement
en brillant et ravageant les terres de ce prince lorsqu'il y passa en
allant en Bavière. Le roi, à qui il demanda permission de se faire
accompagner par sa femme, ne se montra pas plus galant que le prince
Louis, tellement que Villars en furie ne songea qu'à différer.
L'approvisionnement, les recrues, l'arrivée des officiers, mille détails
dont il sut profiter, furent ses prétextes. Cinquante bataillons et
quatre-vingts escadrons, avec force officiers généraux, destinés à
passer avec lui, se morfondirent longtemps, peu touchés des charmes de
la maréchale. Le comte d'Albert, que le roi ne voulut jamais rétablir,
non pas même le laisser colonel réformé, eut permission d'aller chercher
fortune en Bavière, au service de l'électeur et alla avec Monasterol,
son envoyé ici, joindre ses troupes pour passer avec elles.

À la fin Villars, poussé à bout d'ordres pressants, et ne pouvant plus
trouver d'excuses, sous les yeux de tant de témoins, passa le Rhin, et
se mit sérieusement en marche. Il poussa devant lui Blainville avec une
vingtaine de bataillons, qui emporta le château d'Haslach, où cent
quatre-vingts hommes demeurèrent prisonniers dans la vallée de la
Quinche, à trois lieues de Gegenbach, où était le prince Louis, qui, par
toutes les lenteurs du maréchal, était sur le point d'être joint par
vingt bataillons que lui envoyaient les Hollandais. Ces retranchements,
examinés et tournés, furent trouvés de digestion trop dure\,; il fallut
prendre des détours on réussit, et Villars, capitaine de vaisseau, qui
avait eu permission de faire la campagne auprès du maréchal son frère,
arriva le 6 mai après dîner à Versailles, dans le temps que le roi
travaillait avec Chamillart dans son cabinet, qui l'y fit entrer
d'abord. Il apportait la nouvelle que l'armée avait surmonté tous les
obstacles et les défilés\,; qu'on avait attaqué le château d'Hornberg, à
côté de Wolfach, et que trois ou quatre mille hommes qui étaient
derrière Hornberg s'étaient retirés précipitamment\,; qu'ils avaient
perdu trois cents hommes, et nous une trentaine\,; qu'on n'avait pas
voulu s'amuser à les poursuivre\,; que l'armée était le 2 campée à
Saint-Georges, entrée sur trois colonnes dans la plaine\,; qu'elle
n'était plus qu'à trois lieues de Rothweil et Villingen\,; qu'on
n'entendait point parler du prince Louis depuis qu'on l'avait tournoyé
et laissé à côté\,; qu'enfin la jonction avec l'électeur était désormais
sûre et certaine. Il ajouta des détails sur les vivres, les convois et,
l'artillerie, qui furent satisfaisants\,; et que Saint-Maurice et
Clérembault, lieutenants généraux, étaient demeurés avec quatre
bataillons et vingt-trois escadrons à Offenbourg, où le maréchal de
Tallard venait d'arriver.

Villars ne voulut point attaquer Villingen, qu'il laissa sur la gauche,
pour ne point retarder sa marche. Il détacha le 4, de Donausching,
d'Aubusson, mestre de camp de cavalerie, avec cinq cents chevaux, pour
aller porter de ses nouvelles à M. de Bavière. Ce prince avait aussi
envoyé cinq cent chevaux au-devant du maréchal. Les détachements se
rencontrèrent, se reconnurent, et ce fut grande joie des deux côtés.
Villars avait avec lui cinquante bons bataillons et soixante escadrons,
avec pouvoir de faire des brigadiers et de donner amnistie aux
déserteurs voulant revenir. Enfin le maréchal de Villars vit, le 12 mai,
l'électeur de Bavière, qui pleura de joie en l'embrassant, et le combla
en son particulier de tout ce qui se peut de plus flatteur, et témoigna
une grande reconnaissance pour le roi. Il lui fit voir ses troupes et
faire trois salves de canon et de mousqueterie, jetant le premier son
chapeau en l'air et criant\,: \emph{Vive le roi\,!} ce qui fut imité par
toute son armée. Deux jours après, l'électeur vint dîner chez le
maréchal, et voir une trentaine de nos bataillons, qui le reçurent avec
de grands cris de \emph{Vive le roi et monsieur l'électeur\,!} Il les
trouva parfaitement belles. Contentons-nous de les avoir mis ensemble
pour le présent, et allons voir ce qui se passa ailleurs.

La reine d'Angleterre, fort incommodée d'une glande au sein, dont elle
guérit à la longue par un régime très sévère, eut une nouvelle
affliction\,: elle perdit la comtesse Dalmont, Italienne et
Montécuculli, qu'elle avait amenée et mariée en Angleterre, qui ne
l'avait jamais quittée, et pour qui elle avait eu la plus grande amitié
et la plus grande confiance toute sa vie. C'était une grande femme, très
bien faite et de beaucoup d'esprit, dont notre cour s'accommodait
extrêmement. La reine l'aimait tant, qu'elle lui avait fait donner un
tabouret de grâce, comme je crois l'avoir déjà remarqué ailleurs.

Le bailli d'Hautefeuille, ambassadeur de Malte, mourut en même temps.
C'était un vieillard qui avait fort servi et avec valeur, qui ne
ressemblait pas mal à un spectre, et qui avait usurpé et conservé
quelque familiarité avec le roi, qui lui marqua toujours de la bonté. Il
était farci d'abbayes et de commanderies, de vaisselle et de beaux
meubles, surtout de beaucoup de beaux tableaux, fort riche et fort
avare. Se sentant fort mal, et voulant recevoir ses sacrements, il
envoya lui-même chercher le receveur de l'ordre et quelques chevaliers,
à qui il fit livrer et emporter ses meubles, ses tableaux, sa vaisselle,
et tout ce qui se trouva chez lui, pour que l'ordre ne fût frustré de
rien après lui.

Bechameil le suivit immédiatement, assez vieux aussi. Il était père de
la femme de Desmarets, qui venait de revenir sur l'eau, et qui ne tarda
guère à y voguer en plein, et de la femme de Cossé, qui devint duc de
Brissac, comme je l'ai expliqué en son lieu. Bechameil avait été fort
dans les affaires, mais avec bonne réputation, autant qu'en peuvent
conserver des financiers qui s'enrichissent. Il avait succédé à
Boisfranc, beau-père du marquis de Gesvres, dans la surintendance de la
maison de Monsieur, quand ce dernier en fut chassé. Bechameil s'y fit
aimer, estimer et considérer. Il était fort lié avec le marquis d'Effiat
et le chevalier de Lorraine, et par ce dernier avec le maréchal de
Villeroy. C'était un homme d'esprit et fort à sa place, qui faisait une
chère délicate et choisie en mets et en compagnie, et qui voyait chez
lui la meilleure de la ville et la plus distinguée de la cour. Son goût
était exquis en tableaux, en pierreries, en meubles, en bâtiments, en
jardins, et c'est lui qui a fait tout ce qu'il y a de plus beau à
Saint-Cloud. Le roi, qui le traitait bien, le consultait souvent sur ses
bâtiments et sur ses jardins, et le menait quelquefois à Marly. Sans
Mansart, qui en prit beaucoup d'inquiétude, le roi lui aurait marqué
plus de confiance et de bonté. Son fils, qui portait le nom de Nointel,
fut intendant en Bretagne et fort honnête homme, que Monsieur fit faire
conseiller d'État. Bechameil fit de prodigieuses dépenses à faire des
beautés en cette terre en Beauvoisis. Le comte de Fiesque fit sur son
entrée en ce lieu la plus plaisante chanson du monde, dont le refrain
est\,: \emph{Vive le roi et Bechameil son favori, son favori\,!} dont le
roi pensa mourir de rire, et le pauvre Bechameil de dépit.

Il était bien fait et de bonne mine, et croyait avoir de l'air du duc de
Grammont. Le comte de Grammont le voyant se promener aux Tuileries\,:
«\,Voulez-vous parier, dit-il à sa compagnie, que je vais donner un coup
de pied au cul à Bechameil, et qu'il m'en saura le meilleur gré du
monde\,?» En effet, il l'exécuta en plein. Bechameil bien étonné se
retourne, et le comte de Grammont à lui faire de grandes excuses sur ce
qu'il l'a pris pour son neveu. Bechameil fut charmé, et les deux
compagnies encore davantage. Louville, peu après son retour absolu
d'Espagne, épousa une fille de son fils, qui se trouva une personne très
vertueuse et d'une très aimable vertu.

Le samedi 28 avril, le prince d'Auvergne fut pendu en effigie en Grève,
à Paris, en vertu d'un arrêt du parlement, sur sa désertion aux ennemis,
dont j'ai parlé en son temps\,; et le tableau avec son inscription y
demeura près de deux fois vingt-quatre heures.

Le duc Molez, Napolitain d'assez peu de chose, ambassadeur d'Espagne,
c'est-à-dire de Charles II, à Vienne, et qui y était demeuré sans
caractère et sans mission depuis la mort de son maître jusqu'à la
déclaration de la guerre, qu'il fut arrêté, déclara en ce temps-ci qu'il
ne l'avait été que de son consentement\,; qu'il avait été toujours dans
le parti de l'empereur, publia un manifeste sur sa conduite, et fut
récompensé d'une des premières charges dans la maison de l'archiduc, où
il ne fit jamais aucune figure.

Le maréchal de Villeroy partit pour la Flandre, où le maréchal de
Boufflers l'attendait\,; le maréchal d'Estrées pour son commandement de
Bretagne, et le maréchal de Cœuvres, son fils, pour Toulon, préparer
tout en attendant M. le comte de Toulouse\,; et Mgr le duc de Bourgogne,
au lieu de sa première destination en Flandre, fut déclaré pour
l'Allemagne, où le maréchal de Tallard était avec une armée, et Marsin
choisi pour être auprès de la personne de ce prince.

La duchesse de Ventadour, voyant la maréchale de La Motte, sa mère,
vieillir, et M\textsuperscript{me} la duchesse de Bourgogne donner des
espérances d'avoir bientôt des enfants, jugea qu'il était temps de
quitter Madame, pour s'ôter le prétexte de la considération de cette
princesse, et s'aplanir la voie à la survivance de gouvernante des
enfants de France. Son ancien ami, le maréchal de Villeroy, était
parvenu à la mettre bien dans l'esprit de M\textsuperscript{me} de
Maintenon, auprès de laquelle elle avait les grâces de la ressemblance
qui la touchait le plus, c'est-à-dire celles des aventures galantes
plâtrées après de dévotion.

Madame qui l'aimait fort, et qu'elle avait bien servie à la mort de
Monsieur, entra dans ses vues, et chercha quelque duchesse sans pain et
brouillée avec son mari, comme était la duchesse de Ventadour, quand
elle fit l'étrange planche d'entrer à elle, au scandale public, à
l'étonnement du roi, qui eut peine à l'accorder aux instances de
Monsieur, et qui voulut savoir si sa famille y consentait.

Madame fut quelque temps à trouver cette misérable duchesse. À la fin,
la duchesse de Brancas se présenta, et fut acceptée avec une grande
joie. Elle était sœur de la princesse d'Harcourt, et lui était
parfaitement dissemblable\,: c'était une femme de peu d'esprit, sans
toutefois manquer de sens et de conduite, très vertueuse et très
véritablement dévote dans tous les temps de sa vie, et la plus
complètement malheureuse. Elle et son mari étaient enfants des deux
frères, lesquels étaient fils du premier duc de Villars, frère de
l'amiral, et d'une sœur de la belle et fameuse Gabrielle, et du premier
maréchal duc d'Estrées. Le duc de Brancas avait perdu son père et sa
mère à seize ans, qui n'avaient jamais figuré. Son oncle, le comte de
Brancas, a voit fort paru à la cour et dans le monde, et parmi la
meilleure, la plus galante et la plus spirituelle compagnie de son
temps, et fort bien avec le roi et les reines. Nous avons vu en son lieu
qu'il fut encore mieux avec M\textsuperscript{me} Scarron, depuis la
fameuse M\textsuperscript{me} de Maintenon, qui s'en souvint toute sa
vie. Le comte de Brancas est encore célèbre par ses prodigieuses
distractions, que La Bruyère a immortalisées dans ses \emph{Caractères}.
Il l'est encore par la singularité de sa retraite à Paris, au dehors des
Carmélites, qu'il exhortait à la grille depuis qu'il fut dans la
dévotion, qui ne l'empêchait pas devoir toujours bonne compagnie et de
conserver du crédit à la cour. Il avait marié l'aînée de ses deux filles
au prince d'Harcourt. N'ayant pas grand'chose à donner à l'autre, il
jeta les yeux sur son neveu, qui était assez pauvre et encore plus
abandonné, n'ayant que cet oncle qui en pût prendre soin. Il était plus
jeune de plusieurs années que sa cousine\,; son oncle, partie par
amitié, partie par autorité, l'engagea à l'épouser, et lui en fit même
parler par le roi. À dix-sept ans, et sans parents à qui avoir recours,
il n'en faut pas tant pour paqueter un homme. Il se maria malgré lui en
1680, avec cent mille livres que le roi donna à sa femme, et fort peu de
son beau-père qu'il perdit six mois après, et avec lui tout le frein qui
pouvait le retenir.

C'était un homme pétillant d'esprit, mais de cet esprit de saillie, de
plaisanterie, de légèreté et de bons mots, sans la moindre solidité,
sans aucun sens, sans aucune conduite, qui se jeta dans la crapule et
dans les plus infâmes débauches, où il se ruina dans une continuelle et
profonde obscurité. Sa femme devint l'objet des regrets d'un mauvais
mariage fait contre son goût et contre son gré, dont elle n'était pas
cause\,; elle passa sa vie le plus souvent sans pain et sans habits, et
souvent encore parmi les plus fâcheux traitements, que sa vertu, sa
douceur et sa patience ne purent adoucir. Heureusement pour elle, elle
trouva des amies qui la secoururent, et sans la maréchale de Chamilly,
elle serait morte souvent de toutes sortes de besoins. Elle persuada
enfin une séparation au duc de Brancas, qui, pour y parvenir solidement
et de complot fait, battit sa femme et la chassa à coups de pied devant
M\textsuperscript{me} de Chamilly, d'autres témoins et tous les valets,
qui l'emmena chez elle, où elle la garda longtemps. De pain, elle en eut
comme point par la séparation, parce qu'il ne se trouva pas où en
prendre. Elle en était là depuis plusieurs années quand, pour son pain,
elle se mit à Madame et encore chargée d'enfants, dont son mari se
mettait fort peu en peine. Madame, qui s'en trouvait fort honorée, la
traita jusqu'à sa mort avec beaucoup d'égards et de distinctions, et
elle se fit aimer et considérer à la cour par sa douceur et sa vertu.

Félix, premier chirurgien du roi, mourut vers ce temps-là, laissant un
fils qui n'avait point voulu tâter de sa profession. Fagon, premier
médecin du roi, qui avait toute sa confiance et celle de
M\textsuperscript{me} de Maintenon sur leur santé, mit en cette place
Maréchal, chirurgien de la Charité, à Paris, le premier de tous en
réputation et en habileté, et qui lui avait fait très heureusement
l'opération de, la taille. Outre sa capacité dans son métier, c'était un
homme qui, avec fort peu d'esprit, avait très bon sens, connaissait bien
ses gens, était plein d'honneur, d'équité, de probité, et d'aversion
pour le contraire\,; droit, franc et vrai, et fort libre à le montrer,
bon homme et rondement homme de bien, et fort capable de servir, et par
équité ou par amitié, de se commettre très librement à rompre des glaces
auprès du roi, quand il se fut bien initié (et on l'était bientôt dans
ces sortes d'emplois familiers auprès de lui). On verra dans la suite
que ce n'est pas sans raison que je m'étends sur cette espèce de
personnage des cabinets intérieurs, que sa faveur laissa toujours doux,
respectueux, et quoique avec quelque grossièreté, tout à fait en sa
place. Mon père, et moi après lui, avons logé toute notre vie auprès de
la Charité. Ce voisinage avait fait Maréchal le chirurgien de notre
maison\,; il nous était tout à fait attaché, et il le demeura dans sa
fortune.

Je me souviens qu'il nous conta, à M\textsuperscript{me} de Saint-Simon
et à moi, une aventure qui lui arriva, et qui mérite d'être rapportée.
Moins d'un an depuis qu'il fut premier chirurgien, et déjà en
familiarité et en faveur, mais voyant, comme il a toujours fait, tous
les malades de toute espèce qui avaient besoin de sa main dans
Versailles et autour, il fut prié par le chirurgien de Port-Royal des
Champs d'y aller voir une religieuse à qui il croyait devoir couper la
jambe. Maréchal s'y engagea pour le lendemain. Ce même lendemain, on lui
proposa, au sortir du lever du roi, d'aller à une opération qu'on devait
faire\,; il s'en excusa sur l'engagement qu'il avait pris pour
Port-Royal. À ce nom, quelqu'un de la Faculté le tira à part, et lui
demanda s'il savait bien ce qu'il faisait d'aller à Port-Royal.
Maréchal, tout uni, et fort ignorant de toutes les affaires qui, sous ce
nom, avaient fait tant de bruit, fut surpris de la question, et encore
plus quand on lui dit qu'il ne jouait pas à moins qu'à se faire
chasser\,; il ne pouvait comprendre que le roi trouvât mauvais qu'il
allât voir si on y couperait ou non la jambe à une religieuse. Par
composition, il promit de le dire au roi avant d'y aller. En effet, il
se trouva au retour du roi de sa messe, et comme ce n'était pas une
heure où il eût accoutumé de se présenter, le roi, surpris, lui demanda
ce qu'il voulait. Maréchal lui raconta avec simplicité ce qui l'amenait,
et la surprise où il en était lui-même. À ce nom de Port-Royal, le roi
se redressa comme il avait accoutumé aux choses qui lui déplaisaient, et
demeura deux ou trois \emph{Pater} sans répondre, sérieux et
réfléchissant, puis dit à Maréchal\,: «\,Je veux bien que vous y alliez,
mais à condition que vous y alliez tout à l'heure pour avoir du temps
devant vous\,; que, sous prétexte de curiosité, vous voyiez toute la
maison, et les religieuses au chœur et partout où vous les pourrez
voir\,; que vous les fassiez causer, et que vous examiniez bien tout de
très près, et que ce soir vous m'en rendiez compte.\,» Maréchal, encore
plus étonné, fit son voyage, vit tout, et ne manqua à rien de tout ce
qui lui était prescrit. Il fut attendu avec impatience\,; le roi le
demanda plusieurs fois, et le tint à son arrivée près d'une heure en
questions et en récits. Maréchal fit un éloge continuel de Port-Royal\,;
il dit au roi que le premier mot qui lui fut dit fut pour lui demander
des nouvelles de la santé du roi, et à plusieurs reprises\,; qu'il n'y
avait lieu où on priât tant pour lui, dont il avait été témoin aux
offices du chœur. Il admira la charité, la patience et la pénitence
qu'il y avait remarquées\,; il ajouta qu'il n'avait jamais été en aucune
maison dont la piété et la Sainteté lui eût fait autant d'impression. La
fin de ce compte fut un soupir du roi, qui dit que c'étaient des Saintes
qu'on avait trop poussées, dont on n'avait pas assez ménagé l'ignorance
des faits et l'entêtement, et à l'égard desquelles on avait été beaucoup
trop loin. Voilà le sens droit et naturel, produit par un récit sans
fard, d'un homme neuf et neutre, qui dit ce qu'il a vu\,; et dont le roi
ne se pouvait défier, et qui eut par là toute liberté de parler\,; mais
le roi, vendu à la contrepartie, ne donnait d'accès qu'à elle\,; aussi
cette impression fortuite du vrai fut-elle bientôt anéantie. Il ne s'en
souvint plus quelques années après, lorsque le P. Tellier lui lit
détruire jusqu'aux pierres et aux fondements matériels de Port-Royal, et
y passer partout la charrue.

Félix avait eu pour sa vie une petite maison dans le pare de Versailles,
au bout du canal où aboutissaient toutes les eaux. Il l'avait rendue
fort jolie. Le roi la donna à la comtesse de Grammont. Les étranges
Mémoires du comte de Grammont, écrits par lui-même, apprennent qu'elle
était Hamilton, et comment il l'épousa en Angleterre. Elle avait été
belle et bien faite\,; elle en avait conservé de grands restes et la
plus haute mine. On ne pouvait avoir plus d'esprit, et, malgré sa
hauteur, plus d'agrément, plus de politesse, plus de choix. Elle l'avait
orné, elle avait été dame du palais de la reine, avait passé sa vie dans
la meilleure compagnie de la cour, et toujours très bien avec le roi,
qui goûtait son esprit, et qu'elle avait accoutumé à ses manières libres
dans les particuliers de ses maîtresses. C'était une femme qui avait eu
ses galanteries, mais qui n'avait pas laissé de se respecter, et qui,
ayant bec et ongles, l'était fort à la cour, et jusque par les
ministres, qu'elle cultivait même très peu.

M\textsuperscript{me} de Maintenon, qui la craignait, n'avait pu
l'écarter\,; le roi s'amusait fort avec elle. Elle sentait l'aversion et
la jalousie de M\textsuperscript{me} de Maintenon\,: elle l'avait vue
sortir de terre, et surpasser rapidement les plus hauts cèdres\,; jamais
elle n'avait pu se résoudre à lui faire sa cour. Elle était née de
parents catholiques, qui l'avaient mise toute jeune à Port-Royal, où
elle avait été élevée. Il lui en était resté un germe qui la rappela à
une solide dévotion avant même que l'âge, le monde ni le miroir la
pussent faire penser à changer de conduite. Avec la piété, instruite
comme elle l'avait été, l'amour de celles à qui elle devait son
éducation, et qu'elle avait admirées dans tous les temps de sa vie, prit
en elle le dessus de la politique. Ce fut par où M\textsuperscript{me}
de Maintenon espéra éloigner le roi d'elle. Elle y échoua toujours avec
un extrême dépit\,: la comtesse s'en tirait avec tant d'esprit et de
grâces, souvent avec tant de liberté\,; que les reproches du roi se
tournaient à rien, et qu'elle n'en était que mieux et plus familière
avec lui, jusqu'à hasarder quelquefois quelques regards altiers à
M\textsuperscript{me} de Maintenon, et quelques plaisanteries salées
jusqu'à l'amertume. Trop enhardie par une longue habitude de succès,
elle osa s'enfermer à Port-Royal toute une octave de la Fête-Dieu. Son
absence fit un vide qui importuna le roi et qui donna beau jeu à
M\textsuperscript{me} de Maintenon sur la découverte. Le roi en dit son
avis au comte de Grammont fort aigrement, et le chargea de le rendre à
sa femme. Il en fallut venir aux excuses et aux pardons, qui furent mal
reçus. Elle fut renvoyée à Paris, et on alla à Marly sans elle. Elle y
écrivit au roi par son mari sur la fin du voyage\,; mais on ne la put
jamais résoudre à écrire à M\textsuperscript{me} de Maintenon, ni à lui
faire dire la moindre chose. La lettre demeura, sans réponse et parut
sans succès. Peu de jours après le retour à Versailles, le roi lui fit
dire par son mari d'y venir\,: il la vit dans son cabinet par les
derrières, et quoique très expressément elle tînt ferme sur Port-Royal,
ils se raccommodèrent à condition de n'y plus faire de ces disparates,
comme lui dit le roi, et d'avoir pour lui cette complaisance. Elle
n'alla point chez limé de Maintenon, qu'elle ne vit qu'avec le roi,
comme elle avait accoutumé, et fut mieux avec lui que jamais.

Cela s'était passé l'année précédente. Le présent des Moulineaux, cette
petite maison revenue à la disposition du roi par la mort de Félix,
qu'elle appela Pontali, fit du bruit, et marqua combien elle était bien
avec le roi. Ce lieu devint à la mode. M\textsuperscript{me} la duchesse
de Bourgogne, les princesses l'y allèrent voir, et assez souvent. N'y
était pas reçu qui voulait, et le dépit que M\textsuperscript{me} de
Maintenon en avait, mais qu'elle n'osait montrer, ne fut capable de
retenir que bien peu de ses plus attachées, qui même sur les propos du
roi à elles dans l'intérieur, et sur l'exemple de ses filles, n'osèrent
s'en dispenser tout à fait\,; et le roi, jaloux de montrer qu'il n'était
pas gouverné, suivait en cela d'autant plus volontiers son goût pour la
comtesse de Grammont, qui, avec toute la cour, ne s'en haussa ni baissa.

M\textsuperscript{me} de Maintenon se consola de cette petite peine par
la délivrance d'une bien plus grande\,: ce fut celle de son frère, qui
mourut aux eaux de Vichy, toujours gardé à vue par ce Madot, prêtre de
Saint-Sulpice, qui en fut, bientôt après, récompensé d'un bon évêché. Je
ne dirai rien ici de ce M. d'Aubigné, parce que j'en ai parlé
suffisamment ailleurs.

Le roi, qui haïssait tout ce qui était lugubre, ne voulut pas que
M\textsuperscript{me} de Maintenon drapât, comme on faisait encore alors
pour les frères et les sœurs, non pas même que ses valets de chambre ni
ses femmes, fussent vêtus de noir, et elle-même en porta un deuil fort
léger et fort court. Il ne vaqua par cette mort qu'un collier de
l'ordre, et le gouvernement de Berry, dont le comte d'Ayen, son gendre,
avait la survivance.

Ce gendre était tombé dans une langueur où les médecins ne purent rien
connaître, et qui, sans maladie autre qu'une grande douleur au creux de
l'estomac, le réduisit à l'extrémité. Il ne fut pas question de songer à
faire la campagne. Il passa l'été au coin du feu, enveloppé comme dans
le plus rigoureux hiver. M\textsuperscript{me} de Maintenon l'allait
voir souvent, et ce qui parut de bien extraordinaire,
M\textsuperscript{me} la duchesse de Bourgogne y passait des
après-dînées, et quelquefois sans elle. Soit fantaisie de malade, soit
raisons domestiques, il se lassa d'être dans l'appartement de son père
et de sa mère, où lui et sa femme étaient très commodément logés, et si
vaste que cela s'appelait la rue de Noailles, et tenait toute la moitié
du haut de la galerie de l'aile neuve. Il fit demander à l'archevêque de
Reims son logement à emprunter, qui était à l'autre extrémité du
château. Il n'en avait point d'autre, et la demande était d'autant plus
incivile que l'archevêque étant lors au plus mal avec le roi, et le
comte d'Ayen n'étant pas le maître de lui céder celui que M. le duc de
Berry avait quitté depuis quelque temps, sous celui du duc de Noailles,
où il s'était mis, c'était déloger tout à fait l'archevêque. J'avance ce
délogement pour ne pas séparer le raccommodement de l'archevêque de
Reims de trop loin de sa disgrâce, et rapporter de suite l'une et
l'autre. Ce sont de ces curiosités de cour dont les époques ne sont pas
importantes dans leur exactitude, lorsque les matières portent à ne s'y
pas arrêter, pourvu qu'on ait celle de les remarquer. Voici donc la
cause de la disgrâce de l'archevêque de Reims, dont la source arriva la
veille de la Pentecôte de cette année.

Le fameux Arnauld était mort à quatre-vingt-deux ans, à Bruxelles, en
1694. Le P. Quesnel, toujours connu sous ce nom pour avoir été longtemps
dans l'Oratoire, avait succédé à ce grand chef de parti. Il se tenait
caché comme son maître, en butte aux puissances remuées par tous les
ressorts des jésuites et de leurs créatures. Également possesseurs de la
conscience du roi et du roi d'Espagne, ils jugèrent la conjoncture
favorable pour tâcher de se saisir, par leur concours, de la personne du
P. Quesnel et de tous ses papiers. Il fut vendu, découvert et arrêté à
Bruxelles la veille de la Pentecôte de cette année. J'en laisse le
curieux détail aux annalistes jansénistes. Il me suffira ici de dire
qu'il se sauva en perçant une maison voisine, et gagna la Hollande à
travers mille dangers\,; mais ses papiers furent pris, où il se trouva
force marchandise dont le parti moliniste sut grandement profiter. On y
trouva des chiffres, quantité de noms avec la clef, et beaucoup de
lettres et de commerces. Un bénédictin de l'abbaye d'Auvillé, en
Champagne, s'y trouva fort mêlé, qui avait déjà eu des affaires sur la
doctrine. On résolut de l'arrêter, et de faire saisir tout ce qui se
trouverait d'écrits dans ce monastère. Le moine se sauva, et pas un
papier dans sa cellule\,; mais on fut dédommagé par l'ample moisson
qu'on fit dans celle du sous-prieur, qui en était farcie, Tout fut
apporté à Paris et bien examiné. Il s'y trouva une étroite
correspondance entre le P. Quesnel et ce religieux, et une fort grande
aussi par son canal entre le même P. Quesnel et M. de Reims. Le pis fut
qu'on y trouva aussi les brouillons de la main du moine d'un livre
imprimé depuis peu en hollande, qui confondait fort la monarchie avec la
tyrannie, et qui sentait fort le républicain, tout à fait dans les
sentiments dont le fameux Richer, si odieux à Rome et aux jésuites,
s'était solennellement rétracté depuis, mais qu'il avait imprimés durant
les fureurs de la Ligue. Ce moine d'Auvillé fut donc avéré d'être
l'auteur de ce livre qui venait de paraître contre-la monarchie. Il n'en
fallut pas davantage pour faire soupçonner au moins le P. Quesnel d'être
du même avis, et M. de Reims d'être au moins le confident de l'ouvrage,
s'il n'était pas dans les mêmes sentiments. On peut juger de tout
l'usage que les jésuites, ses ennemis, et qu'il avait toujours
maltraités impunément, surent faire d'un si grand avantage. Le roi entra
dans une grande indignation. La famille de l'archevêque, tout à fait
tombée de crédit et de considération depuis que le ministère en était
sorti, et ses amis, furent alarmés. Ils en donnèrent avis à
l'archevêque, qui était à Reims, et que la frayeur y retint au lieu de
venir essayer de se justifier. Son séjour dans une telle conjoncture fut
un autre sujet de triomphe et de mauvais offices contre lui, qui à la
fin le forcèrent au retour. Il obtint avec peine une audience du roi\,:
elle fut fâcheuse\,; il en sortit plus mal encore avec lui qu'il n'y
était entré, et sa disgrâce très marquée dura jusqu'à ce hasard
longtemps après, que je viens de raconter du comte d'Ayen.

L'archevêque savait trop bien la cour pour ne pas saisir cette occasion
favorable. Il comprit dans l'instant que M\textsuperscript{me} de
Maintenon, plus contente alors de sa nièce qu'elle ne l'avait été,
raffolée du comte d'Ayen malade, et plus qu'importunée de la duchesse de
Noailles, dont elle n'aimait pas la personne, et moins encore les vues
et les demandes continuelles pour une vaste famille, fatiguée même du
duc de Noailles, serait ravie d'être en retraite à son aise et loin
d'eux, chez le comte et la comtesse d'Ayen, dans son appartement, qui
était séparé de ceux du père et du fils de tout le château. Il répondit
donc en envoyant ses clefs avec toute la politesse d'un rustre en
disgrâce, et protesta que quand il n'irait pas dans son diocèse, il ne
rentrerait point dans son appartement. Dès le même jour il en fit ôter
tous les meubles sans y rien laisser, et s'en alla loger dans sa maison
à la ville. Le lendemain, le roi rencontrant l'archevêque sur son
passage, alla droit à lui, le remercia le plus obligeamment du monde,
lui dit qu'il n'était pas juste qu'il fût délogé, lui ordonna d'aller
voir l'appartement que M. le duc de Berry avait quitté, qui avait été
prêté au comte d'Ayen\,; de voir s'il s'en pourrait accommoder, d'y
ordonner tous les changements et tous les agréments qui lui plairaient,
et ajouta que, contre ce qu'il avait établi depuis quelque temps, il ne
voulait pas qu'il lui en coûtât rien, et qu'il ordonnerait aux bâtiments
de tout exécuter sous ses ordres. M. de Reims, comblé bien au-dessus de
ses espérances, profita de cet heureux moment. Il obtint une audience du
roi, qui lui fut aussi favorable que la dernière avait été affligeante.
Elle fut longue, détaillée\,; le roi lui rendit ses bonnes grâces
premières, et il promit aussi au roi les siennes pour les jésuites, sans
que le roi l'eût exigé. Il fit accommoder aux dépens du roi, qui lui en
demanda souvent des nouvelles, ce logement de M. le duc de Berry, qui,
un peu moins grand que le sien qu'il quittait, était de plain-pied à la
galerie haute de l'aile neuve et aux appartements du roi, et un des
beaux qui ont vue sur les jardins, au lieu que le sien était au haut du
château à l'opposite, et qu'il n'avait rien à y perdre pour le voisinage
de la surintendance, où son père et son neveu étaient morts, qui était
occupée par Chamillart et sa famille, successeur de leur charge\,; et
voilà comment, dans les cours, des riens raccommodent souvent les
affaires les plus désespérées\,; mais ces hasards heureux y sont pour
bien peu de gens.

Gourville mourut en ce temps-ci, à quatre-vingt-quatre ou cinq ans, dans
l'hôtel de Condé, où il avait été le maître toute sa vie. Il avait été
laquais de M. de La Rochefoucauld, père du grand veneur, qui, lui
trouvant de l'esprit, et étant de ses terres de Poitou, en voulut faire
quelque chose. Il s'en trouva si bien pour ses affaires domestiques et
pour ses menées aussi, à quoi il était fort propre, qu'il s'en servit
pour les intrigues les plus considérables de ces temps-là. Elles
{[}le{]} firent bientôt connaître à M. le Prince, à qui M. de La
Rochefoucauld le donna, et qui demeura toujours depuis dans la maison de
Condé. Les Mémoires qu'il a laissés, et ceux de tous ces temps de
troubles, de la minorité du roi jusqu'à son mariage, et au retour de M.
le Prince par la paix des Pyrénées, l'ont assez fait connaître pour que
je n'aie rien à y ajouter. Gourville, par son esprit, son grand sens,
les amis considérables qu'il s'était faits, était devenu un
personnage\,; l'intimité des ministres l'y maintint, celle de M. Fouquet
l'enrichit à l'excès. L'autorité qu'il acquit et qu'il se conserva à
l'hôtel de Condé, où il était plus maître de tout que les deux princes
de Condé, qui eurent en lui toute leur confiance, tout cela ensemble le
soutint toujours dans une véritable considération. Il n'oublia pas en
aucun temps qu'il devait tout à M. de La Rochefoucauld, ni ce qu'il
avait été en sa jeunesse\,; et quoique naturellement assez brutal, il ne
se méconnut jamais, quoique mêlé toute sa vie avec la plus illustre
compagnie. Le roi même le traitait toujours avec distinction. Ce qui est
prodigieux, il avait secrètement épousé une des trois saurs de M. de La
Rochefoucauld. Il était continuellement chez elle à l'hôtel de La
Rochefoucauld, mais toujours, et avec elle-même, en ancien domestique de
la maison. M. de La Rochefoucauld et toute sa famille le savaient, et
presque tout le monde, mais à les voir on ne s'en serait jamais aperçu.
Les trois sœurs filles et celle-là, qui avait beaucoup d'esprit et
passant pour telle, logeaient ensemble dans un coin séparé de l'hôtel de
La Rochefoucauld, et Gourville à l'hôtel de Condé. C'était un fort grand
et gros homme, qui avait été bien fait, et qui conserva sa bonne mine,
une santé parfaite, sa tête entière jusqu'à la fin. Il avait peu de
domestiques, bien choisis. Lorsqu'il se vit fort vieux, il les fit tous
venir un matin dans sa chambre\,; là il leur déclara qu'il était fort
content d'eux, mais qu'ils ne s'attendissent pas un d'eux qu'il leur
laissât quoi que ce fait par testament, mais qu'il leur promettait
d'augmenter à chacun ses gages tous les ans d'un quart, et de plus,
s'ils le servaient bien et avec affection\,; que c'était à eux à avoir
bien soin de lui, et à prier Dieu de, le leur conserver longtemps\,;
que, par ce moyen, ils auraient de lui, s'il vivait encore plusieurs
années, plus qu'ils n'en auraient pu espérer par testament. Il leur tint
exactement parole. Il n'avait point d'enfants, mais des neveux et des
nièces qu'on ne voyait point, hors un neveu, qui même se produisit peu,
qui furent ses héritiers, et qui sont demeurés dans l'obscurité.

En Flandre, les Hollandais perdirent le comte d'Athlone de maladie, qui
commandait leurs troupes en chef. Ils mirent en sa place Obdam, frère
d'Overkerke, bâtard des princes d'Orange, qui avait été dans la faveur
et l'intime confidence du roi Guillaume, duquel il était grand écuyer.
Les ennemis firent le siège de Bonn, que d'Alègre leur rendit, le 17
mai, après trois semaines de siège. Ils avaient grande envie de faire
celui d'Anvers. Cohorn, leur Vauban, força nos lignes en trois endroits
avec sept ou huit mille hommes, et entra dans le pays de Waës, ayant, à
une lieue d'Anvers, Obdam avec vingt-huit bataillons, et la commodité de
nos lignes forcées pour leur servir de circonvallation pour ce siège. Le
maréchal de Boufflers, sur ces nouvelles, quitta le maréchal de Villeroy
sur le Demer, et marcha avec trente escadrons et trente compagnies de
dragons vers le corps du marquis de Bedmar, avec lequel il attaqua, le
samedi dernier juin, les vingt-cinq bataillons et les vingt-neuf
escadrons qu'avait Obdam près du village d'Eckeren, à trois heures après
midi, deux heures avant l'arrivée de, son infanterie, dans la crainte
que les ennemis se retirassent. Le combat, fort vif et fort heureux pour
le maréchal, dura jusqu'à la nuit, qui empêcha la défaite entière de ces
troupes Hollandaises. Elles y perdirent quatre mille hommes, huit cents
prisonniers, quatre cents chariots, cinquante charrettes d'artillerie,
presque tout leur canon, quatre gros mortiers et quarante petits. La
comtesse de Tilly, qui était venue dîner avec son mari assez mal à
propos, y fut aussi prise. Nos troupes y eurent près de deux mille tués
ou blessés, et n'y perdirent de marque que le comte de Brias, neveu du
dernier archevêque de Cambrai, colonel d'un régiment wallon, que je
connaissais fort. Obdam prit une cocarde blanche et se retira avec ce
qu'il put à Breda\,: le reste s'embarqua à Lillo\footnote{Forteresse du
  royaume de Hollande située sur l'Escaut. Les précédents éditeurs ont
  remplacé Lillo par Lille.}. On intercepta une lettre qu'il écrivait de
Breda au duc de Marlborough, par laquelle il lui mandait que, n'ayant
plus d'armée, il allait à la Haye rendre compte aux États généraux de
son malheur, et se plaignait fort de Cohorn. Le reste de la campagne se
passa en campements et en subsistances\,; les ennemis prirent Huy et la
garnison prisonnière de guerre tout à la fin d'août. Il ne se fit plus
rien de part ni d'autre. Cette victoire d'Eckeren fut si agréable au roi
et au roi d'Espagne, que le maréchal de Boufflers en eut la Toison d'or,
et le marquis de Bedmar le brevet de conseiller d'État qui est le comble
de la fortune en Espagne, et ce que nous appelons ici ministre d'État.
Chamillart profita de la bonne humeur\,; il avait cent mille écus de
brevet de retenue sur sa charge de secrétaire d'État, qu'il avait payés
aux héritiers de Barbezieux\,; il en eut encore autant de plus.

Coetlogon, avec cinq vaisseaux, prit le 22 juin, vers la rivière de
Lisbonne, cinq vaisseaux Hollandais, après un grand combat et fort
opiniâtre, qui dura jusqu'à la nuit. Ces vaisseaux Hollandais
escortaient cent voiles marchandes qui eurent le temps de se sauver. Le
comte de Walstein, ambassadeur de l'empereur à Lisbonne, fut pris sur un
des vaisseaux de guerre avec un envoyé de l'électeur de Mayence, qui
s'en retournaient en Allemagne. Walstein fut amené à Vincennes, et
quelque temps après envoyé à Bourges, où il demeura assez longtemps avec
Saint-Olon, gentilhomme ordinaire, chargé de prendre garde à sa
conduite. Saint-Paul Hécourt, avec quatre vaisseaux, prit et coula à
fond quatre vaisseaux de guerre Hollandais au nord d'Écosse, qui
escortaient la pêche du hareng, dont il brûla cent soixante bateaux. Un
des vaisseaux coula aussi à fond\,: cela se passa à la fin de juin. Dans
cette même campagne, Saint-Paul eut un autre avantage aussi
considérable, et de la même espèce, vers le Nord.

\hypertarget{chapitre-vii.}{%
\chapter{CHAPITRE VII.}\label{chapitre-vii.}}

1703

~

{\textsc{Cardinal Bonzi\,; son extraction, son caractère, sa fortune, sa
mort.}} {\textsc{- Mort du duc de La Ferté.}} {\textsc{- P. de La Ferté
jésuite.}} {\textsc{- Maréchal de Joyeuse gouverneur des Évêchés.}}
{\textsc{- Bailli de Noailles ambassadeur de Malte.}} {\textsc{- M. de
Roye lieutenant général des galères.}} {\textsc{- Comte de Toulouse à
Toulon.}} {\textsc{- Duc de Bourgogne sur le Rhin.}} {\textsc{- Villars
fait demander par l'électeur de Bavière d'être duc\,; est refusé\,;
remplit ses coffres.}} {\textsc{- Villars échoue encore à faire venir sa
femme le trouver\,; se brouille avec l'électeur.}} {\textsc{- Vues et
conduite pernicieuse de Villars.}} {\textsc{- Projet insensé du Tyrol.}}
{\textsc{- Le roi amusé par Vendôme.}} {\textsc{- Legal bat à
Minderkingen le général Latour\,; est fait lieutenant général.}}
{\textsc{- Triste succès du projet du Tyrol.}} {\textsc{- Conduite de
Vaudémont.}} {\textsc{- Duquesne brûle les magasins d'Aquilée.}}
{\textsc{- Naissance du duc de Chartres\,; sa pension.}} {\textsc{- Duc
d'Orléans tire du roi plus d'un million par an.}} {\textsc{- Règlement
sur l'artillerie.}} {\textsc{- Trésor inutilement cherché à Meudon.}}
{\textsc{- Président de Mesmes prévôt et grand maître des cérémonies de
l'ordre.}}

~

Le cardinal Bonzi mourut à Montpellier vers la mi-juillet de cette
année, à soixante-treize ans. Il était archevêque de Narbonne, et avait
cinq abbayes, et commandeur de l'ordre. Ainsi le cardinal Portocarrero
eut cette place qui lui avait, été assurée d'avance, avec la permission,
en attendant, de porter le cordon bleu. Ces Bonzi sont des premières
familles de Florence\,; ils ont eu souvent les premières charges de
cette république et des alliances directes avec les Médicis. Ce fut un
Bonzi, évêque de Terracine, qui fit le funeste mariage de Catherine de
Médicis, qui en amena en France, avec les Strozzi, les Gondi et d'autres
Italiens. Un Bonzi eut l'évêché de Béziers du cardinal Strozzi, son
oncle, qui a été possédé par six Bonzi, d'oncle à neveu, dont deux ont
été cardinaux. Le second Bonzi, évêque de Béziers, fit le triste mariage
de Marie de Médicis. Sa parenté avec elle engagea Henri IV à le faire
grand aumônier de la reine, c'est-à-dire à ériger cette charge pour lui,
l'unique qui, chez les reines, ait le titre de grand. C'était un homme
de grand mérite, et qui avait habilement traité beaucoup d'affaires
dehors et dedans, et qui eut la nomination de France au chapeau que Paul
V lui donna en 1611. Pierre Bonzi, dont il s'agit ici, élevé auprès de
l'évêque de Béziers, son oncle, auquel il succéda, plut de bonne heure
au cardinal Mazarin. Ces Bonzi n'ont été heureux en mariages que pour
eux-mêmes. Il fit celui du grand-duc avec une fille de Gaston, qu'il
conduisit à Florence, d'où il fut ambassadeur à Venise, de là en
Pologne, pour empêcher le roi Casimir d'abdiquer. Il en rapporta la
nomination de Pologne au cardinalat. Après son départ, Casimir abdiqua.
Bonzi fut renvoyé en Pologne, où il rompit les mesures des Impériaux, et
fit élire Michel Wiesnowieski. À son retour, il eut l'archevêché de
Toulouse, et alla ambassadeur en Espagne. Bientôt après, il eut
l'archevêché de Narbonne, le chapeau, que Clément X lui donna en 1672,
et fut grand aumônier de la reine. Il se trouva aux conclaves d'Innocent
XI, Alexandre VIII et Innocent XII, et partout il brilla et réussit.

C'était un petit homme trapu, qui avait eu un très beau visage, à qui
l'âge en avait laissé de grands restes, avec les plus beaux yeux noirs,
les plus parlants, les plus perçants, les plus lumineux, et le plus
agréable regard, le plus noble et le plus spirituel que j'aie jamais vu
à personne\,; beaucoup d'esprit, de douceur, de politesse, de grâces, de
bonté, de magnificence, avec un air uni et des manières charmantes\,;
supérieur à sa dignité, toujours à ses affaires, toujours prêt à
obliger\,; beaucoup d'adresse, de finesse, de souplesse, sans
friponnerie, sans mensonge et sans bassesse\,; beaucoup de grâces et de
facilité à parler. Son commerce, à ce que j'ai ouï dire à tout ce qui a
vécu avec lui, était délicieux, sa conversation jamais recherchée et
toujours charmante\,; familier avec dignité, toujours ouvert, jamais
enflé de ses emplois ni de sa faveur. Avec ces qualités et un
discernement fort juste, il n'est pas surprenant qu'il se soit fait
aimer à la cour et dans les pays étrangers.

Sa place de Narbonne le rendait le maître des affaires du Languedoc\,;
il le fut encore plus par y être adoré et y avoir gagné la confiance des
premiers et des trois ordres, que par son siège. Fleury, receveur des
décimes du diocèse de Lodève, s'insinua dans le domestique du cardinal,
parvint jusqu'à lui, et à lui oser présenter son fils, qui plut
tellement à cette éminence italienne, qu'il en prit soin, et fit, ce
qu'on pourrait bien affirmativement dire, sa fortune, si elle n'avait
pris plaisir d'en insulter la France en l'en établissant roi absolu, et
unique et public, et dans un âge où les autres radotent quand ils font
tant que d'y parvenir.

Bonzi jouit longtemps d'une faveur à la cour et d'une puissance en
Languedoc, qui, établie premièrement sur les cœurs, n'était contredite
de personne. M. de Verneuil, gouverneur, n'y existait pas\,; M. du
Maine, en bas âge, puis en jeunesse, qui lui succéda, ne s'en mêla pas
davantage. Bâville, intendant du Languedoc, y voulait régner, et ne
savait comment supplanter une autorité si établie, lorsque, bien averti
de la cour d'un accès de dévotion qui diminua depuis, mais qui dans sa
ferveur portait le roi à des réformes d'autrui, lui fit revenir, par des
voies de conscience, des choses qui le blessèrent sur la conduite du
cardinal Bonzi. Les Lamoignon, de tout temps livrés aux jésuites,
réciproquement disposaient d'eux\,; et ces pères n'ont jamais aimé des
prélats assez grands pour n'avoir pas besoin d'eux, et dont étant
néanmoins ménagés et bien traités comme ils l'étaient de Bonzi, se
trouvaient en posture de les faire compter avec eux si d'aventure il
leur en prenait envie.

Le bon cardinal, quoique en âge où les passions sont ordinairement
amorties, était éperdument amoureux d'une M\textsuperscript{me} de
Gange, belle-sœur de celle dont la vertu et l'horrible catastrophe a
fait tant de bruit. Les Soubise ne sont pas si rares qu'on le croit\,:
Cet amour était fort utile au mari\,; il ne voulut donc jamais rien
voir, et profitait grandement de ce que toute la province voyait, et
qu'il avait bien résolu de ne voir jamais, quoique sous ses yeux. Le
scandale était en effet très réel, et sans l'affection générale que
toute la province portait au cardinal, cela aurait fait beaucoup plus de
bruit. Bâville l'excita tant qu'il put\,: il procura au cardinal des
avis fâcheux de la part du roi, puis des lettres du P. de La Chaise par
son ordre, enfin quelque chose de plus par Châteauneuf, secrétaire
d'État de la province. Bonzi alla à la cour, espérant tout de sa
présence\,: il y fut trompé\,; il trouva le roi bien instruit, qui lui
parla fort franchement, et qui, par son expérience, ne se paya point de
l'aveuglement volontaire du mari. Bonzi, rappelé à Montpellier pour les
états, ne put se contenir. Il avait découvert que le coup lui était
porté par Bâville. Il le trouva plus hardi et plus ferme dans le cours
des affaires qu'il n'avait encore osé se montrer\,; il fit des parties
contre le cardinal, qui s'attira des dégoûts sur ce qu'il ne changeait
point de conduite avec sa belle. Il était accusé de ne lui rien refuser,
et comme il disposait dans les états, et hors leur tenue, de beaucoup de
choses pécuniaires et de bien des emplois de toutes les sortes,
M\textsuperscript{me} de Gange était accusée de s'y enrichir, et il y en
avait bien quelque chose. Cette espèce de déprédation fut grossie à la
cour par Bâville, dont le but était d'ôter au cardinal tout ce qu'il
pourrait de dispositions, de grâces à faire et d'autorité, d'y entrer en
part d'abord comme par un concert nécessaire contre l'abus, et de s'en
emparer dans la suite. Il n'en fallut pas davantage pour les brouiller.
Bâville fit valoir le service du roi et le bien de la province
intéressés dans l'abus que le cardinal faisait d'une autorité que sa
maîtresse tournait toute à la sienne et à un honteux profit. Peu à peu
cette autorité, toujours butée et mise en compromis, s'affaiblit en l'un
et crût en l'autre. L'intérêt, qui souvent est préféré à tout autre
sentiment, fit des créatures à Bâville, qui commença à se montrer utile
ami et dangereux ennemi. Cette lutte dura ainsi quelques années, Bâville
croissant toujours aux dépens du cardinal, malgré ses voyages à la cour.
Enfin le cardinal eut l'affront et la douleur de voir arriver une lettre
de -cachet à M\textsuperscript{me} de Gange, qui l'exilait fort loin.
Son cœur et sa réputation en souffrirent également. De cette époque, son
crédit et son autorité tombèrent entièrement, et Bâville devint le
maître, qui sut bien le faire sentir au cardinal et à tout ce qui lui
demeura attaché.

Porté par terre, il espéra se relever par le mariage de Castries, fils
de sa sœur et gouverneur de Montpellier, avec une fille du feu
maréchal-duc de Vivonne, frère de M\textsuperscript{me} de Montespan,
qui n'avait rien vaillant qu'une naissance et des alliances qui
faisaient grand honneur aux Castries, et la protection du duc du Maine,
qui la promit tout entière à l'oncle et au neveu, mais l'accorda à son
ordinaire quand le mariage fut fait, en 1693, qui fut son ouvrage. Il
redonna pourtant par l'opinion quelque vie au cardinal et quelque mesure
à Bâville, qui n'en fut pas longtemps la dupe. Le cardinal, qui se la
vit de l'appui qu'il avait espéré, tomba peu à peu en vapeurs qui
dégénérèrent en épilepsie, et qui lui attaquèrent la tête. La tristesse
l'accabla, la mémoire se confondit, les accès redoublèrent. Le dernier
voyage qu'il fit à la cour, ce n'était plus lui en rien\,; il était même
singulièrement rapetissé, et quelque part qu'il allât, même chez le roi,
il était toujours suivi par son médecin et son confesseur, qui passait
pour un aumônier. Il mourut bientôt après son retour en Languedoc,
consommé par Bâville, devenu tyran de la province.

Le duc de La Ferté mourut aussi cet été d'hydropisie, à quarante-sept
ans. Sa valeur l'avait avancé de bonne heure\,; il avait toujours servi,
il était devenu très bon officier général et faisait espérer qu'il ne
serait pas moins bon à la tête d'une armée que le maréchal son père. Il
avait beaucoup d'esprit, ou plutôt d'imagination ou de saillies, gai,
plaisant, excellent convive\,; mais le vin et la crapule le perdirent
après en avoir bien tué à table. Le roi, qui avait du goût pour lui, fit
tout ce qu'il put pour le corriger de ses débauches\,; il lui en parla
souvent dans son cabinet, tantôt avec amitié, tantôt avec sévérité. Il
lui manquait peu, en 1688, de l'âge nécessaire pour être chevalier de
l'ordre, et le roi lui fit dire qu'il l'eût dispensé s'il avait voulu
profiter de ses avis. Il était incorrigible, et même, les dernières
campagnes qu'il fit, peu capable de servir par une continuelle ivresse.
Il avait passé sa vie brouillé et séparé de sa femme, fille de la
maréchale de La Mothe, dont il n'eut que deux filles.

On ne savait ce qu'était devenu son frère, le chevalier de La Ferté,
qu'on a cru péri et dont on n'a jamais ouï parler, qui était un étrange
garnement\,: son autre frère, séduit enfant par les jésuites, se l'était
fait malgré son père qui, le rencontrant jeune novice sur le pont Neuf
avec le sac de quête sur le dos, comme faisaient encore alors les jeunes
jésuites. Je fit courre par ses valets, dont il se sauva à grand'peine.
Il avait aussi beaucoup d'esprit et devint célèbre prédicateur\,; mais
il aimait la bonne chère et la bonne compagnie et n'était pas fait pour
être religieux. Il mécontenta les jésuites qui à la fin le reléguèrent à
la Flèche, où il mourut longtemps après son frère, non, je pense, sans
regretter ses vœux qui l'exclurent de succéder à la dignité de son frère
qui demeura éteinte trente-huit ans après son érection. Le gouvernement
de Metz, Verdun, {[}de Toul{]} et de leurs évêchés, vacant par cette
mort, fut donné au maréchal de Joyeuse.

Le bailli de Noailles, frère du duc et du cardinal de Noailles, succéda
au bailli d'Hautefeuille à l'ambassade de la religion\footnote{La
  \emph{religion} signifie ici l'ordre de Malte.} en France. Il était
lieutenant général des galères de France, qu'il vendit au marquis de
Roye, capitaine de vaisseau, lors à la mer, qui avait épousé la fille
unique de Ducasse. Pontchartrain, mari de sa sœur, en fit le marché, et
en eut l'agrément pour lui en son absence, ce qui le fit tout d'un coup
lieutenant général des armées navales.

M. le comte de Toulouse était parti pour Toulon, et Mgr le duc de
Bourgogne pour aller prendre le commandement de l'armée du maréchal de
Tallard sur le Rhin, où le prince Louis de Bade et les autres généraux
en chef de l'empereur, occupés à la tête de divers corps à s'opposer aux
progrès déjà faits de l'électeur de Bavière, et à ceux qu'il en
craignait bien plus depuis que Villars l'avait joint, n'étaient pas en
état de s'opposer beaucoup aux projets du maréchal de Tallard, qui fut
assez longtemps à observer le prince Louis et à subsister, tandis que
l'empire tremblait dans son centre, par les avantages que l'électeur
avait remportés sur les Impériaux, et que la diète de Ratisbonne ne s'y
continuait que sous ses auspices. L'électeur comptait bien de profiter
de la jonction des François, et il n'y eut complaisance qu'il n'eût pour
leur général. Celui-ci, dont l'audace {[}était{]} excitée par son bâton,
et par la faveur où il se croyait, et la gloire d'autrui qu'il avait
revêtue par la bataille de Friedlingen, s'oublia jusqu'à croire pouvoir
atteindre tout, et ne se trompa pas dans la suite, mais le moment n'en
était pas arrivé. Il profita du besoin que l'électeur de Bavière avait
de son concours pour le forcer à demander au roi de le faire duc. La
proposition parut telle qu'elle était, et fut refusée à plat.

Alors, Villars, n'espérant plus rien de l'électeur, songea à remplir ses
coffres. Il mit dans tous les pays où ses partis purent atteindre des
sauvegardes et des contributions, qui n'épargnèrent pas même les pays de
l'électeur dont il fit peu de part à la caisse militaire, et se fit à
lui des millions. Des millions ne sont pas ici un terme en l'air pour
exprimer de grandes sommes, je dis des millions très réels. Ce pillage
déplut extrêmement à l'électeur\,; mais ce qui l'outra, fut l'opposition
qu'il trouva en Villars à tout, ce qu'il lui proposa de projets et
mouvements de guerre. Villars voulait s'enrichir, et rejetait tout ce
qui pouvait resserrer ses contributions et, ses sauvegardes par
l'éloignement de son armée, et par des entreprises faciles et utiles,
mais qui, le, tenant près de l'ennemi, le mettaient hors de portée de ce
gain immense.

D'autre part, loin de craindre de se brouiller avec l'électeur, c'était
tout son but, depuis qu'il avait échoué à une dernière tentative de
faire venir sa femme le trouver. Le roi, à force d'importunité, y avait
consenti\,; là-dessus Villars avait demandé un passeport pour elle au
prince Louis de Bade, qui, piqué du ravage de ses terres, sur son
premier refus, renvoya à Villars la lettre qu'il en avait reçue tout
ouverte, sans lui faire un seul mot de réponse. La jalousie le
poignardait\,; à quelque prix que ce fût il voulait aller rejoindre sa
femme. Ni les succès sur le Danube, ni le concert avec l'électeur
n'étaient pas propres à avancer son dessein\,; il réduisit donc ce
prince à ne pouvoir demeurer avec lui, ni à espérer de rien exécuter en
Allemagne.

Cette étrange situation lui fit concevoir le dessein, pour ne pas
demeurer inutile spectateur des trésors que Villars amassait, de se
rendre maître du Tyrol. Villars, ravi de se délivrer de lui et de ses
troupes, pour avoir ses coudées plus franches et qu'on se prît moins à
lui d'une si fatale inaction dans le cœur de l'empire, admira et
confirma ce projet qu'il avait peut-être fait naître. La difficulté du
passage des Alpes gardées et retranchées partout, ni celle des
subsistances qui pouvait faire périr l'électeur et ses troupes comme il
en fut au moment, ne parurent rien à Villars. Pour mieux faire goûter au
roi un projet si insensé, il lui proposa celui d'une communication avec
l'électeur par Trente, qui affranchirait des dépenses, des difficultés
et des dangers de porter par l'Allemagne des recrues, des secours et les
besoins aux troupes françaises en Bavière, du moment que par Trente et
le Tyrol la communication serait ouverte en tout temps de l'armée
d'Italie, jusqu'en Bavière, par où on aurait le choix de faire les
grands et certains efforts en Allemagne par des détachements d'Italie,
ou en Italie par ceux de l'Allemagne. Rien toutefois n'était si
palpablement insensé.

Par la jonction de Villars on était au comble des désirs qu'on avait
formés\,: toute l'Allemagne tremblait\,; les forces ennemies étonnées,
moindres que les nôtres\,; un pays neuf, ouvert, point de ces places à
tenir plusieurs mois comme sur le Rhin et en Flandre\,; la confusion
portée en Allemagne, et les princes de l'empire jetés par leur ruine,
ainsi que les villes impériales, dans le repentir de leur complaisance
pour l'empereur et dans la nécessité de s'en retirer\,; l'empereur, dans
la dernière inquiétude des succès des mécontents de Hongrie, grossis,
organisés, maîtres de la haute Hongrie, et dont les contributions
s'étendaient jusque autour de Presbourg. Quels autres succès pouvaient
être comparables à ceux qu'on avait lieu de se promettre dans le cœur de
l'Allemagne, et pour les plus sûrs avantages, et pour forcer l'empereur
d'entendre à une paix qui conservât la monarchie d'Espagne à celui qui
déjà y régnait\,! En quittant ce certain pour le projet du Tyrol, outre
les difficultés d'y atteindre et de s'y maintenir avec les seules forces
de l'électeur, dont l'armée française aurait toujours le pays électoral
à garder et ce qu'il y venait d'ajouter, quel chemin le détachement de
l'armée d'Italien aurait-il point à faire, avec les difficultés des
subsistances, des rivières à passer, des lacs à tourner, des montagnes
et des défilés bien gardés à franchir\,? Combien de temps, à bien
employer ailleurs en Allemagne et en Italie, perdu à faire ce long et
fâcheux trajet des deux côtés jusqu'à Trente, et cependant quel temps de
respirer et d'entreprendre donné aux ennemis sur le Pô et sur le Danube,
et pour achever la folie, dans un temps où on commençait à se défier du
duc de Savoie\,! Mais il était arrêté dans les décrets de la Providence
que l'aveuglement qui mit l'État si près du précipice devait commencer
ici.

La communication des nouvelles de Bavière n'était pas facile\,; aucun
officier général n'osait se commettre à écrire ce qu'ils voyaient tous
et dont ils gémissaient\,; tout se discutait et se décidait pour la
guerre entre le roi et Chamillart uniquement, et presque toujours en
présence de M\textsuperscript{me} de Maintenon. On a vu ce qu'elle était
à Villars\,; elle voulait qu'il fût un héros. Chamillart n'avait garde
d'oser penser autrement\,; son apprentissage dans les projets de guerre
était nouveau. Le roi, qui se piquait d'y être maître, se complaisait en
un ministre novice qu'il comptait former et à qui les grandes opérations
ne pourraient être attribuées. Friedlingen, la jonction, plus que tout
cela, M\textsuperscript{me} de Maintenon l'avait ébloui sur Villars. Ils
voyaient l'électeur aussi ardent que lui au projet du Tyrol\,; le moyen
de ne les en pas croire sans réflexion, sans avisement des motifs, sans
contradicteur\,? La carte blanche leur fut donc laissée, et les ordres
en conséquence envoyés en Italie pour l'exécution de la jonction par
Trente. Vendôme amusait le roi de bicoques emportées, de succès de trois
cents ou quatre cents hommes, de projets qui ne s'exécutaient pas. Ses
courriers étaient continuels, qui ne satisfaisaient que le roi, par le
mérite de sa naissance et les soins attentifs de M. du Maine, et par lui
de M\textsuperscript{me} de Maintenon, qui lui avaient dévoué
Chamillart.

Vendôme, qui aimait à faire du bruit, fut ravi de se trouver chargé de
percer jusqu'à Trente. C'était un homme qui ne doutait de rien, quoique
souvent arrêté, qui soutenait ses fautes avec une audace que sa faveur
augmentait, et qui ne convenait jamais d'aucune méprise\,; il fit donc
un très gros détachement avec lequel il se mit en chemin de Trente,
laissant M. de Vaudemont à la tête de l'armée.

Pendant le voyage de l'électeur en Tyrol, les Impériaux rassemblèrent
leurs troupes et tinrent toujours le maréchal de Villars de fort près.
Lui cependant projeta de surprendre le général La Tour, campé avec cinq
mille chevaux près de la petite ville de Minderkingen qui a un pont sur
le Danube, à six lieues d'Ulm, où Legal était allé avec douze escadrons,
sous prétexte de garantir cette dernière ville des courses des ennemis
qui en empêchaient le commerce et les marchés. Il eut ordre de marcher
sans bruit, à huit heures du soir. Du Héron le joignit avec six
escadrons de dragons\,; il prit en croupe sept cents hommes
d'infanterie, et cinq cents chevaux le joignirent en chemin avec
Fonboisart. Quoiqu'ils eussent marché sans bruit toute la nuit, un parti
de hussards les découvrit, tellement qu'ils trouvèrent le général en
bataille dans une belle prairie devant son camp, et son bagage ayant
passé le Danube. Ils avaient quinze cents chevaux plus que Legal, et le
débordaient des deux côtés, aussi attaquèrent-ils les premiers par une
grande décharge. Il ne leur fut répondu que l'épée à la main. L'affaire
fut disputée et notre gauche avait ployé. Le peu d'infanterie qu'avait
Legal marcha, la baïonnette au bout du fusil, et arrêta en plaine la
cavalerie qui avait poussé cette gauche qui se rallia, et alors la
victoire ne balança plus. Ils se jetèrent dans Minderkingen, où la
quantité de gens tués sur le pont les empêcha d'être poursuivis dans la
ville, parce qu'ils eurent le temps de hausser le pont-levis\,; quatre
de leurs escadrons furent renversés dans le Danube\,; ils perdirent
`environ quinze cents hommes tués, peu de prisonniers, tant
l'acharnement fut grand, et sept étendards. Du Héron, dont ce fut grand
dommage, y fut tué avec cinquante officiers et quatre ou cinq cents
hommes. Legal se retira le lendemain, il, août, en bon ordre, craignant
quelques gros détachements du prince Louis de Bade. Cette action, qui
fut belle, fit grand plaisir au roi, qui en fit compliment à la femme de
Legal, qu'il rencontra dans la galerie, venant de la messe, et fit son
mari lieutenant général.

La course vers Trente eut le succès qu'on en devait attendre. L'électeur
et M. de Vendôme furent, chacun de leur côté, arrêtés à chaque pas. Ce
ne furent que pas retranchés dans les montagnes, châteaux escarpés et
bicoques très fâcheuses à prendre, à chacun desquels M. de Vendôme se
paradait et amusait le roi, tantôt d'un courrier, tantôt d'un officier
pour apporter ces grandes nouvelles. Il ne put jamais recevoir qu'une
seule fois des nouvelles de l'électeur. On s'épanouissait déjà de ses
succès comme d'une communication sûre et établie, lorsque l'électeur,
qui était maître d'Inspruck où il avait fait chanter le \emph{Te Deum},
auquel, par une étrange singularité, la mûre de l'impératrice et
l'évêque d'Augsbourg, frère de l'impératrice, qui y avaient été pris,
assistèrent\,; l'électeur, dis-je, avancé vers Brixen, trouva toute la
milice et toute la noblesse du pays en armes, tellement que, craignant
de manquer de tout et de trouver sa communication avec son pays coupée,
il s'en retourna tout court. Il était temps\,: le pain manqua\,; nul
moyen d'en avoir du pays, où tout leur courait sus, et les défilés déjà
assez occupés pour se remercier de n'avoir pas différé de vingt-quatre
heures\,; encore y perdit-on assez de monde et même autour de
l'électeur. Il rejoignit le maréchal de Villars avec ses troupes
diminuées et horriblement fatiguées d'une course dont il ne tira pour
tout fruit que la perte de tout le temps qu'il y employa et qui eût pu
l'être bien utilement en Allemagne\,; mais on a vu à qui en fut la
faute. M. de Vendôme eut au moins le plaisir de bombarder Trente, à qui
il ne fit pas grand mal. Il revint comme il put. Staremberg tourmenta
fort ce retour, sur lequel il sut gagner trois marches, faire perdre
force monde en détail à son ennemi et pousser à bout ses troupes de
fatigues. Vaudemont, qui cependant avait fait battre Murcé avec un gros
détachement d'une manière plus que grossière, était à San-Benedetto,
faisant fort le malade pressé d'aller aux eaux. Sa conduite, toujours
soutenue, rendra toujours difficile à croire qu'il ne fût pas dans la
bouteille, et qu'il ne fût pressé de se mettre à quartier de ce qui
allait arriver. Dès que le duc de Vendôme fut à San-Benedetto, il en
partit pour s'aller mettre à l'abri de tous événements. L'aveuglement
sur lui fut tel, qu'il eut sur-le-champ qu'il le demanda le régiment
d'Espinchal, tué à ce détachement de Murcé, pour le prince d'Elbœuf,
neveu de sa femme.

M. de Vendôme manda au roi une belle et singulière action de
Duquesne-Monier, qui commandait les vaisseaux du roi dans le golfe de
Venise. Il sut que les Impériaux avaient de grands magasins dans
Aquilée, qui est à sept lieues dans les terres. Il s'embarqua sur des
chaloupes avec cent vingt soldats, remonta la petite rivière qui vient
d'Aquilée, et qui est si étroite qu'il y avait des endroits où il ne
pouvait passer qu'une chaloupe à la fois. Il trouva deux forts sur son
passage, mit pied à terre avec ses gens, les emporta, et au dernier,
Beaucaire, capitaine de frégate, qui commandait les cent vingt soldats,
poursuivit ceux du fort jusque dans Aquilée qu'il pilla, brûla les
magasins malgré deux cents hommes de troupes réglées et beaucoup de
milices qui étaient là, ne perdit presque personne et revint trouver
Duquesne qui l'attendait vis-à-vis du dernier fort qu'il avait pris.
Cela arriva vers la fin de juillet.

Le samedi 4 août, le roi étant à Marly, M\textsuperscript{me} la
duchesse d'Orléans accoucha d'un prince à Versailles\,; M. le duc
d'Orléans vint demander au roi la permission de lui faire porter le nom
de duc de Chartres, et l'honneur d'être son parrain. Le roi lui
répondit\,: «\,Ne me demandez-vous que cela\,?» M. le duc d'Orléans dit
que les gens de sa maison le pressaient de demander autre chose, mais
qu'il y aurait dans ce temps-ci de l'indiscrétion. «\,Je préviendrai
donc votre demande, répliqua le roi, et je donne à votre fils la pension
de premier prince du sang de cent cinquante mille livres.\,» Cela
faisait un million cinquante mille francs à M. le duc d'Orléans,
savoir\,: six cent cinquante mille livres de sa pension, cent mille
livres pour l'intérêt de la dot de M\textsuperscript{me} la duchesse
d'Orléans, cent cinquante mille livres de sa pension et cent cinquante
mille livres de celle de M. le duc de Chartres âgé de deux jours, sans
compter les pensions de Madame.

Le roi fit, quelques jours après, un règlement sur l'artillerie, dont il
vendit les charges\,: c'était un objet de cinq millions. Il en laissa
quelques-uns à la disposition de M. du Maine, grand maître de
l'artillerie, augmenta ses appointements de vingt mille livres et lui
donna cent mille écus. Le besoin d'argent qui fit faire cette affaire à
plusieurs autres, fit prêter l'oreille à un invalide qui prétendit avoir
travaillé autrefois à faire à Meudon une cache pour un gros trésor, du
temps de M. de Louvois. Il y fouilla donc et longtemps et en plusieurs
endroits, maintenant toujours qu'il la trouverait. On en fut pour la
dépense de raccommoder ce qu'il avait gâté, et pour la honte d'avoir
sérieusement ajouté foi à cela.

M. d'Avaux vendit en ce temps-ci au président de Mesmes son neveu, sa
charge de prévôt et grand maître des cérémonies de l'ordre, avec
permission de continuer à porter le cordon bleu. D'Avaux l'avait eue, en
1684, du président de Mesmes son frère, qui lui-même avait obtenu la
même permission de continuer à porter l'ordre, et ce président de Mesmes
l'avait eue en 1671 lors de la déroute de La Bazinière, son beau-père,
fameux financier, puis trésorier de l'épargne, qui fut longtemps en
prison, puis revint sur l'eau, mais sans emploi, et à qui il ne fut pas
permis de porter l'ordre, depuis qu'il eut donné sa charge à son gendre,
lors de son malheur. J'ai parlé plus d'une fois de ces ventes de charges
de l'ordre, et, emporté par d'autres matières, je ne me suis pas étendu
sur celle-là, qui ne laisse pas d'avoir sa curiosité, par cela même
qu'on voit arriver tous les jours cette multiplication de cordons bleus
par la transmission de ces charges. Une fois pour toutes il est à propos
de l'expliquer. J'irais trop loin si j'entreprenais de traiter ici ce
qui regarde l'ordre du Saint-Esprit, la digression serait longue et
déplacée. Je me renfermerai aux charges, puisque l'occasion en a été
manquée plus haut, et qu'elle se présente ici naturellement.

\hypertarget{chapitre-viii.}{%
\chapter{CHAPITRE VIII.}\label{chapitre-viii.}}

1703

~

{\textsc{Digression sur les charges de l'ordre.}} {\textsc{- Grand
aumônier\,; pourquoi sans preuves.}} {\textsc{- Amyot privé de sa charge
de grand aumônier.}} {\textsc{- Grands officiers des grands ordres n'en
portent point de marques comme ceux du Saint-Esprit.}} {\textsc{-
Différence des grands officiers d'avec les chevaliers, et des grands
officiers entre eux, et de l'abus du titre de commandeurs\,; d'où
venus.}} {\textsc{- Origine des honneurs du Louvre et de la singulière
distinction du chancelier de l'ordre.}} {\textsc{- Distinction unique de
l'archevêque de Rouen, frère bâtard d'Henri IV.}} {\textsc{- Vétérans de
l'ordre et leurs abus\,; comment introduits.}} {\textsc{- Origine de la
première fortune solide de MM. de Villeroy.}} {\textsc{- Râpés de
l'ordre.}} {\textsc{- Collier de l'ordre aux armes des grands
officiers.}} {\textsc{- Abus des couronnes.}} {\textsc{- Abus des grands
officiers de l'ordre représentés en statues sur leurs tombeaux avec le
collier et le manteau de l'ordre, sans nulle différence d'un
chevalier.}} {\textsc{- Plaisante question d'une bonne femme.}}
{\textsc{- Méprise des Suédois et leur instruction sur le cordon bleu de
d'Avaux, nuisible à son ambassade.}}

~

Henri III, en créant l'ordre du Saint-Esprit, y établit en même temps
cinq charges\,: celle de grand aumônier de l'ordre, qu'il unit dès lors
à celle de grand aumônier de France, et sans preuves. Ce fut pour
gratifier M. Amyot, évêque d'Auxerre, qui avait été son précepteur et
des rois ses frères, et que Charles IX lit grand aumônier. Il était
aussi porté par les Guise, et se livra depuis à la Ligue avec tant
d'ingratitude que, quelque débonnaire que fût Henri IV, une des
premières marques qu'il donna de son autorité fut de le priver de la
charge de grand aumônier de France à la fin de 1591, et de la donner au
célèbre Renaud de Beaune, archevêque de Bourges alors, puis, de Sens\,;
en conséquence de quoi M. Amyot fut en même temps privé de porter
l'ordre, et M. de Beaune le reçut le dernier jour de cette année dans
l'église de Mantes des mains du maréchal de Biron père, qui fit en même
temps son fils chevalier du Saint-Esprit par commission d'Henri IV, qui
n'était pas encore catholique.

Les quatre autres charges furent\,: chancelier, garde des sceaux et
surintendant des deniers de l'ordre en une seule et même charge, qui a
été quelquefois, quoique rarement, partagée\,; prévôt et grand maître
des cérémonies en une seule charge, qui n'a jamais souffert de
division\,; grand trésorier, et greffer. Henri III fit ces charges en
faveur de ses ministres, ou plutôt des Guise, qui se les voulurent
dévouer de plus en plus, les lui firent établir en leur faveur d'une
manière sans exemple, dans les deux autres grands ordres, la Jarretière
et la Toison, et même l'Éléphant, dont les officiers, qui sont des
ministres, des évêques et des personnes au moins aussi considérables
dans leurs cours, depuis l'institution de ces ordres jusque aujourd'hui,
que l'ont été et le sont nos grands officiers de l'ordre, ne portent
aucunes marques de la Toison et de l'Éléphant (et ceux de la Jarretière
une marque entièrement différente en tout de celle des chevaliers), au
lieu que les grands officiers de celui du Saint-Esprit eurent par leur
institution les mêmes marques sur leur personne, hors les jours de
cérémonie de l'ordre, que les chevaliers du Saint-Esprit. Je dis les
grands officiers, parce qu'Henri III en créa en même temps de petits,
tels que le héraut, l'huissier, etc., tout différents des grands
officiers, et qui, pour marque de leurs charges, n'ont porté jusqu'à la
dernière régence qu'une petite croix du Saint-Esprit, attachée d'un
petit ruban bleu céleste à leur boutonnière. Ces mêmes petits officiers
se trouvent aussi dans les autres trois grands ordres cités ci-dessus, à
la différence de leurs grands officiers.

Cette introduction de similitude entière de porter ordinairement l'ordre
du Saint-Esprit entre les chevaliers et les grands officiers, fut
d'autant plus aisée à établir, qu'excepté les magistrats, tout le monde
était alors en pourpoint et en manteau, dont la couleur et la simplicité
seule distinguait les gens les uns d'avec les autres, et que le cordon
bleu se portait au cou\,; mais avec toute cette parité journalière entre
les chevaliers et les grands officiers, ceux-ci étaient fort distingués
des chevaliers les jours de cérémonie, comme ils le sont encore, en ce
qu'ils n'ont point de collier, et ils le sont encore entre eux quatre
par la différence de leurs grands manteaux. Celui du chancelier est en
tout et partout semblable à. celui des chevaliers. Le prévôt et grand
maître des cérémonies n'a point le collier de l'ordre brodé autour du
sien ni de son mantelet, mais du reste il est pareil à ceux des
chevaliers. Ceux du grand trésorier et du greffier ont les flammes de la
broderie de leurs manteaux et mantelets considérablement plus
clairsemées et un peu moins larges, et entre ces deux derniers manteaux
il y a encore quelque petite différence, à l'avantage du grand trésorier
sur le greffier. Les grands officiers eurent encore cette ressemblance
avec les chevaliers, qu'Henri III, qui avait compté donner à son nouvel
ordre les bénéfices en commande, comme en ont ceux d'Espagne, en destina
aussi aux grands officiers pour appointements de leurs charges. Cette
destination rendit dès lors commune aux chevaliers et aux grands
officiers cette dénomination de commandeurs, dont le fonds n'ayant pas
eu lieu d'abord par les désordres de la Ligue, ni depuis, cette
dénomination de commandeur est demeurée propre aux huit cardinaux et
prélats de l'ordre. Les grands officiers ont continué de l'affecter,
qui, pour s'assimiler tant qu'ils peuvent aux chevaliers, la leur
donnent, quoique aucun d'eux ne la veuille, et ne se donne que la
qualité de chevalier des ordres du roi, tandis que les grands officiers
sont très jaloux de la prendre, quoiqu'elle soit demeurée vaine pour
tous, puisque aucun n'a de commanderie, et que les grands officiers sont
suffisamment désignés par le titre de leurs charges, sans y joindre le
vain et inutile titre de commandeur.

On verra, outre cette similitude, l'usage particulier dont ils se le
sont rendu. Outre les distinctions susdites des charges entre elles, les
deux premières font les mêmes preuves que les chevaliers. Le chancelier
de Cheverny, qui l'était de l'ordre de Saint-Michel après les cardinaux
de Bourbon et de Lorraine, le fut de celui du Saint-Esprit à son
institution, auquel celui de Saint-Michel fut uni. Son nom était
Hurault\,: il était garde des sceaux dès 1578, lorsque le chancelier
Birague fut fait cardinal, et chancelier à sa mort en 1585. Il l'avait
été du duc d'Anjou, l'avait suivi en Pologne, était attaché à Catherine
de Médicis, et tellement aux Guise qu'il perdit les sceaux et fut exilé,
ainsi que M. de Villeroy, etc., lorsqu'en 1588, après les Barricades de
Paris, Henri III eut pris la tardive résolution de se défaire des Guise.
C'était un personnage en toutes façons, à qui Henri IV rendit les sceaux
dès 1590. Sa mère était sœur du père de Renaud de Beaune, dont je viens
de parler et qui donna l'absolution à Henri IV à Saint-Denis et le reçut
dans l'Église catholique. Son fils aîné était gendre, dès le
commencement de 1588, de Chabot, comte de Charny, grand écuyer de
France, et par conséquent beau-frère du duc d'Elbœuf. Son autre fils
était gendre de M\textsuperscript{me} de Sourdis, si importante alors,
et tante de la trop fameuse Gabrielle d'Estrées, sur l'esprit de
laquelle elle avait un grand ascendant. Un troisième avait cinq grosses
abbayes avec l'évêché de Chartres, et fut après premier aumônier de
Marie de Médicis. Les filles de ce chancelier étaient mariées dès ayant
l'institution de l'ordre\,: l'aînée au marquis de Nesle-Laval, puis au
brave Givry d'Anglure\,; la deuxième, en 1592, au marquis de Royan La
Trémoille\,; la dernière au marquis d'Alluye-Escoubleau, puis au marquis
d'Aumont. Avec ces alliances, quoique fort nouvelles pour ce chancelier
et la figure personnelle qu'il faisait, il se prétendit homme à faire
des preuves, et véritablement il ne faut pas se lever de grand matin
pour faire celles de l'ordre du Saint-Esprit, autre distinction des
autres grands ordres où il ne faut pas de preuves, parce que les
instituteurs ont cru, sur l'exemple qu'ils en donnaient, que tous ceux
qui y seraient admis dans la suite seraient d'une naissance trop
grandement connue pour qu'on pût leur en demander. Cheverny donc voulut
faire des preuves, comme les chevaliers, et cette nécessité de preuves,
ou pour mieux dire cette distinction, est demeurée à cette charge de
l'ordre. Quoique chancelier de France, il prit sa place aux cérémonies,
de l'ordre comme en étant chancelier, c'est-à-dire après le dernier
chevalier et avec une distance entre-deux, s'y trouva toujours et n'en
fit jamais difficulté. Mais je pense que l'office de la couronne dont il
était revêtu lui procura, et par lui à ses successeurs chanceliers de
l'ordre, la distinction sur les trois autres charges de parler assis et
couvert aux chapitres de l'ordre, où le prévôt, le grand trésorier et le
greffier sont debout et découverts, et de manger au réfectoire du roi à
la dernière place des chevaliers, mais comme eux\,; tandis que les trois
autres charges mangent dans le même temps dans une autre pièce avec les
petits officiers de l'ordre.

C'est aussi cette différence que les ministres, accrédités, revêtus dans
la suite de ces trois autres charges, n'ont pu supporter, qui par leur
crédit a fait tenir les chapitres débout, découverts et sans rang
pêle-mêle, et qui a banni l'usage du repas du roi avec les chevaliers.
Cette même raison de l'office de chancelier de France donna force à
cette autre, que les papiers de l'ordre étant chez le chancelier de
l'ordre, de tenir toutes les commissions pour les affaires de l'ordre
chez le chancelier de l'ordre, de quelque dignité et qualité que soient
les commandeurs et chevaliers commissaires, cardinaux, ducs et princes
de maison souveraine, car les princes du sang seuls ne le sont jamais.
Sur cet exemple, la même chose s'est continuée chez les chanceliers de
l'ordre toujours depuis, et à l'appui de cette raison des papiers, les
grands trésoriers de l'ordre ont obtenu le même avantage que les
commissions de l'ordre se tiennent aussi chez eux.

Quoique ces charges de l'ordre fussent destinées à la décoration des
ministres, celle de prévôt et de grand maître des cérémonies de l'ordre
fut donnée à M. de Rhodes, qui eut le choix de la prendre ou d'être
chevalier de l'ordre. Le goût d'Henri III pour les cérémonies décida M.
de Rhodes, du nom de Pot, et d'une grande naissance. Un Pot avait été
chevalier de la Toison d'or à l'institution de cet ordre et reçu à la
première promotion qu'en fit Philippe le Bon. C'est ce même M. de Rhodes
pour qui fut faite la charge de grand maître des cérémonies de France.
Il voulut, en seigneur qu'il était, faire les mêmes preuves que les
chevaliers, et cela est demeuré à cette charge comme à celle de
chancelier de l'ordre.

Ce qu'on appelle les honneurs du Louvre était inconnu avant le
connétable Anne duc de Montmorency, et réservé aux seuls fils et filles
de France qui montaient et descendaient de cheval ou de coche, comme on
disait alors, et qui étaient même peu en usage aux plus grandes dames,
dans la cour du logis du roi. Ce fut ce célèbre Anne qui, décoré de ses
services, de ses dignités et de sa faveur, entra un beau jour, à cheval,
dans la cour du logis du roi et y monta ensuite, et se maintint dans cet
usage. Quelque temps après, son émule, M. de Guise, hasarda d'en faire
autant. Les uns après les autres ce qu'il y eut de plus distingué imita
par émulation, et la tolérance de l'entreprise étendit peu à peu cet
honneur aux personnes à qui il est maintenant réservé. Les officiers de
la couronne y arrivèrent aussi, tellement que le chancelier de Cheverny
en jouissait comme chancelier de France.

À sa mort, en 1599 l'archevêque de Rouen fut chancelier de l'ordre. Il
était bâtard du roi de Navarre et de M\textsuperscript{lle} du
Rouhet\footnote{Louise de La Beraudière, demoiselle du Rouhet ou Rouët.},
par conséquent frère bâtard d'Henri IV. Ce prince, qui l'aimait
extrêmement, fit tout ce qu'il put pour le faire cardinal, quoique
beaucoup de bâtards, non seulement de papes, mais de particuliers, et
depuis, du temps d'Henri IV même, M. Sérafin, bâtard du chancelier
Olivier, fut cardinal le premier de la dernière promotion de Clément
VIII, en 1604, qui fut la mémé du cardinal du Perron. Il s'appelait
Sérafin Olivier, mais il ne s'appelait que M. Sérafin, avait été
auditeur de rote pour la France, dont il devint doyen et eut après le
titre de patriarche d'Alexandrie. Clément VIII, ayant tenu bon à refuser
le chapeau à Henri IV pour l'archevêque de Rouen, fit en sa faveur une
chose bien plus extraordinaire et sans aucun exemple devant ni depuis ce
fut de lui donner, par une bulle du mois de juin 1597, tous les honneurs
des cardinaux\,: rang, habit, distinctions, privilèges, en sorte
qu'excepté le nom, le chapeau (qui ne se prend qu'à Rome où il ne fut
point), les conclaves et les consistoires, il eut en tout et partout le
même extérieur des cardinaux avec la calotte et le bonnet rouges. On
peut juger qu'avec ces distinctions il eut aussi celle des honneurs du
Louvre. Deux ans après avoir rougi de la sorte, c'est-à-dire en 1599, il
fut chancelier de l'ordre par la mort du chancelier de Cheverny. Il en
fit toutes les fonctions sans difficulté comme avait fait son
prédécesseur. En 1606 Henri IV s'avisa que cette charge était au-dessous
de ce frère décoré de tout ce qu'ont les cardinaux, quoiqu'il fût dans
ce même état deux ans avant qu'elle lui fût donnée (ce n'est pas ici le
lieu de s'écarter sur les bâtards). Henri IV le déclara donc l'un des
prélats associés à l'ordre et donna sa charge de chancelier à
L'Aubépine, père du garde des sceaux de Châteauneuf, de l'évêque
d'Orléans qui fut commandeur de l'ordre en 1619, et du père de ma mère.
Il avait été ambassadeur en Angleterre et était ministre d'État,
beau-frère du premier maréchal de La Châtré, et de M. de Villeroy, le
célèbre secrétaire d'État. Ses filles avaient, épousé MM. de
Saint-Chamond et de Vaucelas, ambassadeur en Espagne, et tous deux
chevaliers de l'ordre, et son père était celui qui avait mis les
secrétaires d'État hors de page, signé le premier\,: \emph{le roi}, et
qui fut en si grande et longue considération sous Henri II, François II
et Charles IX. Établi de la sorte, il obtint une singularité pour sa
charge de chancelier de l'ordre, qui subsiste encore aujourd'hui, qui
est d'entrer en carrosse dans la cour du logis du roi en son absence,
même la reine y étant, ce que n'ont pas les chevaliers de l'ordre, ni
aucun autre, que longtemps depuis le chevalier d'honneur et les dames
d'honneur, et d'atours de la reine.

Ces grands officiers de l'ordre n'étaient pas compris dans le nombre de
cent, dont l'ordre du Saint-Esprit est composé, et lés statuts premiers
et originaux les en excluent. Les Guise qui les firent changer par deux
différentes fois, toujours à leur avantage, à mesure que leur puissance
augmenta, et qui voulurent toujours favoriser les ministres pour les
mieux sceller dans leur dépendance, pour leurs vues sur les projets de
la Ligue qui de jour en jour les approchaient du succès de leur dessein
sur la couronne, les firent comprendre dans le nombre de cent. Outre ce
motif de les assimiler de plus en plus aux chevaliers de l'ordre, ils
eurent encore celui de diminuer le nombre de grâces que Henri III
s'était proposé de pouvoir faire. C'est ce qui porta les Guise à faire
comprendre en même temps dans le nombre des cent les huit cardinaux ou
prélats et les chevaliers étrangers non régnicoles, qui n'y étaient pas
d'abord compris, ce qui était treize places de chevaliers au roi, sans
dompter les incertaines des chevaliers étrangers non régnicoles. Il est
resté jusqu'à présent une trace de cette innovation, en ce que ces
derniers ne sont point payés des mille écus de pension comme tous les
autres chevaliers du Saint-Esprit, régnicoles, et que les Guise qui
firent après coup fixer un âge à leur avantage pour tous les chevaliers
de l'ordre, qui ne l'était point par les premiers statuts, comme il ne
l'est point encore dans aucun autre ordre de l'Europe, n'en firent point
fixer aux charges de l'ordre.

Les deux charges de grands officiers de l'ordre, de grand trésorier et
de greffier, qui ne font point de preuves, furent données, la première à
M. de Villeroy, secrétaire d'État, l'autre à M. de Verderonne, lors en
pays étranger pour les affaires du roi. Il était L'Aubépine, cousin
germain de la femme de M. de Villeroy, et de son frère M. de L'Aubépine,
que nous venons de voir troisième chancelier de l'ordre. M. de
Verderonne était gendre de M. de Rhodes, qui fut en même temps premier
prévôt et grand maître des cérémonies de l'ordre. M. de Villeroy n'a pas
besoin d'être expliqué. C'est à lui et à ce Verderonne, son cousin
germain, qu'a commencé l'abus de ce qu'on appelle vétérans, qui a donné
lieu à un autre plus grand, connu sous le ridicule nom de râpés de
l'ordre, qui est ce que je me suis proposé d'expliquer ici.

M. de Villeroy maria son fils, M. d'Alincourt, en février 1588, à la
fille unique de M. de Mandelot, chevalier de l'ordre de 1582, et
gouverneur de Lyon, Lyonnais et Beaujolais. La Ligue, dont ils étaient
tous deux des plus avant et des membres les plus affidés, et chacun en
leur genre des plus utiles et des plus considérés, fit cette alliance et
arracha de la faiblesse d'Henri III la survivance de cet important
gouvernement, en faveur du mariage que M. d'Alincourt eut en titre, en
novembre de la même année, par la mort de Mandelot, son beau-père. Ce
fut, pour le dire en passant, ce qui fit la première grande fortune des
Villeroy, comme, je le dirai pour la curiosité ci-après. M. de Villeroy
fut chassé en septembre 1588, après les Barricades de Paris, avec les
autres ministres créatures des Guise, lorsque Henri III eut enfin pris
la résolution de se défaire de ces, tyrans avant qu'ils eussent achevé
d'usurper sa couronne. En perdant osa charge de secrétaire d'État, il
perdit sa charge de l'ordre, et le cordon bleu par conséquent. Ses
propres Mémoires, et tous ceux de ce temps, montrent son dévouement aux,
Guise et à la Ligue, et en même temps quand il en désespéra, avec quel
art il sut se retourner et persuader Henri IV qu'il lui avait rendu de
grands services. Sa grande capacité, son expérience, l'important
gouvernement de son fils, tant de personnages considérables à qui il
tenait, tout contribua à persuader à Henri IV, si facile pour ses
ennemis, de lui rendre sa charge et sa place dans le conseil, où il crut
s'en servir utilement, et dans lesquelles ce prince le conserva toute sa
vie avec une grande considération. Sa charge de l'ordre était donnée à
Rusé de Beaulieu, avec celle de secrétaire d'État, à qui Henri IV,
venant à la couronne, les confirma toutes deux. Villeroy eut la charge
de secrétaire d'État qui vaqua en 1594, et comme Henri IV était content
de Rusé de Beaulieu, qui avait eu les charges de M. de Villeroy, il ne
voulut pas lui ôter celle de l'ordre pour la rendre à Villeroy comme il
lui avait laissé celle de secrétaire d'État du même\,; mais en remettant
Villeroy dans sa confiance et dans son conseil il lui permit verbalement
de reprendre je cordon bleu, quoiqu'il n'eut plus de charge, et ce fut
le premier exemple d'un cordon bleu sans charge. Quelque nouvelle que
fût cette grâce, il en obtint une bien plus étrange. Ce fut de faire
Alincourt, son fils, chevalier du Saint-Esprit, le dernier de la
promotion qu'Henri IV fit le 5 janvier 1597, dans l'église de l'abbaye
de Saint-Ouen de Rouen, et pour comble n'ayant que trente ans. Avec un
tel crédit, on fait aisément la planche de porter l'ordre sans charge.

Achevons maintenant la curiosité qui fit la solide fortune des Villeroy
avant la grandeur où ils sont, depuis parvenus. Le secrétaire d'État fit
donner à son petit-fils, de fort bonne heure, la survivance du
gouvernement de son fils. Ce gouvernement éblouit M. de Lesdiguières,
gouverneur de Dauphiné et qui commandait en roi dans cette province, en
Provence et dans quelques pays voisins. Il voulut augmenter sa
considération et sa puissance par se rendre le maître du gouvernement de
Lyon, en s'attachant les Villeroy par le lien le plus indissoluble. Il
proposa ses vues à M. de Créqui, son gendre, qui rejeta bien loin
l'alliance des Villeroy. Le bonhomme, secrétaire d'État, vivait encore.
Après une autre éclipse, essuyée sous le gouvernement de la reine mère
et du maréchal d'Ancre, leur ruine l'avait rétabli aussi bien que
jamais. Mais cette faveur ni l'établissement de Lyon ne pouvaient tenter
Créqui d'une alliance si inégale. Il avait marié sa fille aînée au
marquis de Rosny, fils aîné du célèbre Maximilien, premier duc de Sully,
qui survivait à sa disgrâce, et qui avait toujours traité M. de Villeroy
avec hauteur, qui, de soi côté, l'avait toujours regardé aussi comme son
ennemi. C'était de tous points donner à ce gendre un étrange beau-frère.
Mais Lesdiguières était absolu dans sa famille. Il voulut si fermement
ce mariage de sa petite-fille avec le fils d'Alincourt, qu'il fallut
bien que Créqui y consentît. Le vieux secrétaire d'État eut la joie de
voir arriver jette grandeur dans sa famille. Qu'eût-il dit s'il eût pu
savoir le torrent d'autres dont elle fut suivie\,? Ce mariage se fit en
juillet 1617, et le secrétaire d'État mourut à Rouen\,; à
soixante-quatorze ans, au mois de novembre suivant, pendant l'assemblée
des notables. Par l'événement, tous les grands biens de Créqui et de
Lesdiguières sont tombés au fils de ce mariage, maréchal de France comme
son père, etc., et duc et pair après lui.

M. de Verderonne garda sa charge de greffier jusqu'en 1608, que M. de
Sceaux, Potier\,; secrétaire d'État, en fut pourvu, et Verderonne eut
permission de continuer à porter l'ordre. On a vu par ses entours qu'il
n'était pas sans crédit, et qu'il eut pour lui l'exemple de Villeroy son
cousin, si considéré alors et en termes bien moins favorables.

Les exemples ont en France de grandes suites. Sur ces deux-là M. de
Rhodes, vendit sa charge de prévôt et grand maître des cérémonies de
l'ordre à M. de La Ville aux Clercs-Loménie, secrétaire d'État en
1619\,; il eut permission de continuer à porter l'ordre\,; mais, en,
faveur de la naissance dont il était, il lui fut expédié un brevet
portant promesse d'être fait chevalier de l'ordre à la première
promotion, et, en attendant, de porter l'ordre. Il était plus que
naturel que cette promesse lui fût gardée\,; néanmoins, il ne fut point
de la nombreuse promotion qui fut faite le dernier jour de cette, année,
et il fut tué en 1622 devant Montpellier sans avoir été même nommé.

M. de Puysieux, secrétaire d'État, fils du chancelier de Sillery et
gendre de M. de Villeroy, secrétaire d'État, tous deux en vie et en
crédit, et lui personnellement aussi, entre ses deux disgrâces, vendit
sa charge de grand trésorier de l'ordre à M. Morand, trésorier de
l'épargne, et sur l'exemple de M. de Rhodes, quelque disproportion qu'il
y eût entre un Plot et un Brûlart, il eut le même brevet de promesse
d'être fait chevalier de l'ordre à là première promotion, et de
permission, de continuer en attendant, à porter l'ordre.

Cette dernière planche faite, M. d'Avaux, ce célèbre ambassadeur,
surintendant des finances, vendit sa charge de greffier de l'ordre en
1643 à M. de Bonelles, qui, malgré l'alliance qu'il fit de Charlotte de
Prie, sœur année de la maréchale de La Mothe, ne fut jamais que
conseiller d'honneur au parlement, et n'aurait pas cru que son
petit-fils deviendrait chevalier de l'ordre. M. d'Avaux eut le brevet de
promesse et de permission pareil à celui, qu'avait obtenu M. de
Puysieux.

Enfin la charge de chancelier et de garde des, sceaux de l'ordre ayant
été séparée en deux, pendant la prison du garde des sceaux de France de
Châteauneuf, en 1633 les sceaux de l'ordre furent donnés à M. de.
Bullion, surintendant des finances et président à mortier au parlement
de Paris. Il les vendit en 1636, à M. le premier président Le Jay, et il
eut un brevet pareil aux précédents.

Ces deux charges ayant été réunies, en 1645, en rendant les sceaux de
l'ordre à M. de Châteauneuf, il la vendit entière, peu de mois après, à
La Rivière, évêque-duc de Langres, ce favori de Gaston, si connu dans
tous les Mémoires de la minorité de Louis XIV et les commencements de sa
majorité. Comme M. de Châteauneuf avait des abbayes, quoiqu'il ne fût
point dans les ordres, le brevet qu'il eut, pareil aux autres, porta,
avec la permission de continuer à porter l'ordre, promesse de la
première des quatre places de prélat qui viendrait à vaquer dans l'ordre
qu'il n'a jamais eue, non plus qu'aucun des vendeurs de charges, qui,
presque tous jusqu'à aujourd'hui, ont eut de pareils brevets, et n'ont
jamais été chevaliers de l'ordre. Outre le ridicule, général de ces
brevets, ils en ont un particulier qui échappe et qu'il est curieux
d'exposer ici.

On a vu ci-dessus que le chancelier de l'ordre, entre les distinctions,
qu'il a par-dessus les autres grands officiers ou laïques, à celle
d'avoir le grand manteau de l'ordre semblable en tout à ceux des
chevaliers, et avec le collier de l'ordre brodé tout autour comme eux\,;
il n'a même de différence d'eux que le dernier rang après tous et avec
les trois autres officiers, et de n'avoir point le collier d'or massif
émaillé. De cette privation du collier, le statut en fait comme une
excuse, disant que le chancelier n'a point de collier parce qu'il est
censé être personne, de robe longue, et c'est toutefois à cette personne
de rote longue, et par cela même exclue du collier, qui n'est propre
qu'à ceux de la noblesse et dont la profession et les armes, que ce
collier est promis en vendant sa charge, et aux autres grands officiers
en se défaisant des leurs, tous de robe ou de plume, par ce brevet
illusoire qui n'a eu d'exécution dans aucun, dont aucun n'a espéré
l'accomplissement, et qu'aucun roi n'a jamais imaginé d'effectuer. Je me
contente de marquer le premier de chacune de ces quatre charges qui l'a
obtenu. Il suffit de dire que depuis cet exemple de vendre et d'obtenir
ces brevets que je viens d'exposer, l'usage en a été continuel parmi
tous ces grands officiers de l'ordre, et que, ce brevet n'a été refusé à
pas un, excepté peut-être à quatre ou cinq tombés en disgrâce, et à qui,
en leur ôtant leurs charges de l'ordre, il n'a pas été permis de
continuer à le porter. Jusque-là que pendant la dernière régence, Crosat
et Montargis, très riches financiers, ayant obtenu permission d'acheter
les charges de grand trésorier et de greffier de la succession du frère
aîné du garde des sceaux Chauvelin, et du président Lamoignon, ont
obtenu les mêmes brevets de promesse d'être faits chevalier de l'ordre à
la première promotion, et de continuer à le porter en attendant\,; en
même temps qu'aux approches du sacre du roi, ils eurent commandement de
vendre leurs charges, l'un à M. Dodun, contrôleur général des finances,
l'autre à M. de Maurepas, secrétaire d'État, par l'indécence qu'on
trouva à voir faire à ces deux financiers les fonctions de ces charges,
lorsque, le lendemain du sacre, le roi recevrait l'ordre des mains de
l'archevêque-duc de Reims.

Voilà donc un étrange abus tourné en règle par l'habitude ancienne et
non interrompue\,; il n'en est pas demeuré là. Il a donné naissance à un
autre encore plus étrange et plus ridicule\,; celui qu'on vient
d'expliquer est connu sous le nom de vétérans, celui qui va l'être sous
celui de râpés. Le premier nom est pris des officiers de justice qui,
ayant exercé leurs charges vingt ans, prennent, en les vendant, des
lettres de vétérance qu'on ne leur refuse pas, pour continuer à jouir,
leur vie durant, des honneurs et séances attachés à ces charges. Mais
ceux de l'ordre ont de tout temps gardé la plupart leurs charges peu
d'années, et à force de les garder peu, ont donné ouverture aux râpés.

Ce sobriquet ou ce nom est pris de l'eau qu'on passe sur le marc du
raisin, après qu'il a été pressé, et tout le jus ou le moût tiré, qui
est le vin\,; cette eau fermente sur ce marc, et y prend une couleur et
une impression de petit vin ou de piquette, et cela s'appelle un
\emph{râpé} de vin.

On va voir que la comparaison est juste et le nom bien appliqué. Voici
la belle invention qui a été trouvée par les grands officiers de
l'ordre. Pierre, par exemple, a une charge de l'ordre depuis quelques
années\,; il la vend à Paul, et obtient le brevet ordinaire. Jean se
trouve en place, et veut se parer de l'ordre sans bourse délier. Avec
l'agrément du roi, et le marché fait et déclaré avec Paul, Jean se met
entre Pierre et lui, fait un achat simulé de la charge de Pierre, et y
est reçu par le roi. Quelques semaines après il donne sa démission, fait
une vente simulée à Paul, et obtient le brevet accoutumé, et Paul est
reçu dans la charge. Avec cette invention on a vu pendant la dernière
régence, jusqu'à seize officiers vétérans ou râpés de l'ordre vivant
tous en même temps.

Le premier exemple fut le moins grossier de tous. Bonelles vendit
effectivement la charge de greffier de l'ordre à Novion, président à
mortier, qui fut depuis premier président\,: ce fut en 1656\,; il la
garda quelques mois et la vendit en 1657 à Jeannin de Castille. Le
second exemple se traita plus rondement. Barbezieux eut à la mort de
Louvois, son père, sa charge de chancelier de l'ordre. Boucherat,
chancelier de France, en fut simultanément pourvu d'abord, et huit jours
après qu'il eut été reçu, il fit semblant de se démettre comme il avait
fait semblant d'acheter, et Barbezieux fut reçu. Depuis cet exemple tout
franc, tous les autres n'ont pas eu plus de couverture dans les huit ou
douze qui l'ont suivi jusqu'à présent.

Ces vétérans et ces râpés prennent tous sans difficulté la qualité de
commandeurs des ordres du roi, sans mention même de la charge qui la
leur a donnée, mais qui, à la vérité, n'a pu la leur laisser, non plus
que le brevet de promesse et de permission qu'ils obtiennent, la leur
conférer. À la vérité, ni vétérans ni râpés ne font nombre dans les
cent, dont l'ordre est composé.

À tant d'abus qui ne croirait qu'il n'y en a pas au moins davantage\,?
Mais ce n'est pas tout. De ce que le chancelier de l'ordre a le collier
brodé autour de son grand manteau comme les chevaliers, il a quitté le
cordon bleu qu'il portait autour de ses armes, comme les cardinaux et
les prélats de l'ordre, et quoiqu'il n'ait point le collier d'or massif,
émaillé comme les chevaliers de l'ordre, il l'a mis partout à ses armes.
Cet exemple n'a pas tardé à être suivi par les autres grands officiers,
quoique le collier ne soit pas brodé autour de leurs manteaux, et que
tout leur manque jusqu'à ce vain prétexte. Je ne puis dater cet abus
avec la même assurance et la même précision que je viens de faire les
précédents. De ceux-là, l'origine s'en voit, mais de celui qui a dépendu
de la volonté de l'entreprise plus ou moins tardive, et d'une exécution
domestique faite par un peintre ou par un graveur sur des armes, ce sont
des dates qui ne se peuvent retrouver.

Qui pourrait dire maintenant qui a commencé l'usurpation des
couronnes\,? Il n'est si petit compagnon qui n'en porte une, et les
ducales sont tombées à la plus nouvelle robe. Il est pourtant vrai que
cet abus n'a pas cinquante ans, et qu'un peu auparavant nul homme de
robe ne portait aucune sorte de couronne. Il en existe encore un
témoignage évident. Les armes de M. Séguier alors chancelier, et non
encore duc à brevet, sont en relief des deux côtés du grand autel de
l'église des Carmes-Déchaussés, dont le couvent est à Paris, rue de
Vaugirard\,; toutes les marques de chancelier y sont, manteau sans armes
au revers, masses, mortier, et point de couronne. Tout ce que je puis
dire, c'est qu'étant allé voir M\textsuperscript{me} la maréchale de
Villeroy à Villeroy, de Fontainebleau peu avant sa mort, c'est-à-dire
vers 1706 ou 1707, j'ai vu les armes de Villeroy en pierre avec le
cordon autour, et la croix comme le portent les prélats de l'ordre et
sans collier. Je les ai vues de même dans une église de Paris, je ne me
souviens plus laquelle assez fermement pour la citer. J'ai vu aussi une
chapelle de sépulture des L'Aubépine aux Jacobins de la rue
Saint-Jacques, leurs armes plusieurs fois répétées sans collier, et
entourées du cordon, et la dernière année de la vie du maréchal de
Berwick, tué devant Philippsbourg en 1734, je l'allai voir à Fitz-James,
d'où je m'allai promener un matin à Verderonne qui en est près, où je
vis sur plusieurs portes les armes de L'Aubépine, en pierre, entourées
du cordon avec la croix sans collier.

Mais voici le comble, ce sont les grands officiers de l'ordre, peints et
en sculpture, vêtus avec le manteau de chevalier de l'ordre, et avec le
collier de l'ordre par-dessus comme l'ont les chevaliers. Châteauneuf,
secrétaire d'État, fit faire à Rome le tombeau et la statue de son père
La Vrillière, à genoux dessus, de grandeur naturelle dans cet équipage
complet. C'est même un très beau morceau que j'ai vu sur leur sépulture
à Châteauneuf-sur-Loire. Qui que ce soit à l'inspection ne se peut
douter que ce bonhomme La Vrillière n'ait été que prévôt et grand maître
des cérémonies de l'ordre. Il n'y a nulle différence quelle que ce soit
d'un chevalier du Saint-Esprit. On voit dans Paris et dans la paroisse
de Saint-Eustache la statue au naturel de M. Colbert, grand trésorier de
l'ordre, avec le manteau et le collier des chevaliers\,; il n'est
personne qui puisse ne le pas prendre pour un chevalier du
Saint-Esprit\,; il y en a peut-être d'autres exemples que j'ignore.

Ces abus me font souvenir de ce que me conta la maréchale de Chamilly,
quelque temps après que son mari fut chevalier de l'ordre. Il entendait
la messe, et portait l'ordre par-dessus, comme il était rare alors
qu'aucun le portât par-dessous. Une bonne femme du peuple, qui était
derrière ses laquais, en tira un par la manche, et le pria de lui dire
si ce cordon bleu là était un véritable chevalier de l'ordre. Le laquais
fut si surpris de la question de la part d'une femme qu'il ne jugeait
pas avec raison savoir cette différence, qu'il le conta à son maître au
sortir de la messe.

Les Suédois y furent attrapés à M. d'Avaux, dont on vient de voir le
marché de sa charge à son neveu\,; et lui firent toutes sortes
d'honneurs. Quelque temps après ils surent que c'était un conseiller
d'État de robe qui avait une charge de l'ordre. Ils cessèrent de le
considérer et de le traiter comme ils avaient fait jusque-là, et cette
fâcheuse découverte nuisit fort au succès de son ambassade.

\hypertarget{chapitre-ix.}{%
\chapter{CHAPITRE IX.}\label{chapitre-ix.}}

1703

~

{\textsc{Siège et prise de Brisach par Mgr le duc de Bourgogne, qui
revient à la cour.}} {\textsc{- Le Portugal se joint aux alliés.}}
{\textsc{- Infidélité du duc de Savoie.}} {\textsc{- Changement entier
en Espagne\,; vues de la princesse des Ursins\,; routes qui la
conduisent à régner en Espagne.}} {\textsc{- Princesse des Ursins
s'empare de la reine d'Espagne.}} {\textsc{- Caractère de la reine
d'Espagne.}} {\textsc{- Princesse des Ursins gagne les deux rois.}}
{\textsc{- Caractère de Philippe V.}} {\textsc{- Junte ou despacho
devenu ridicule.}} {\textsc{- Discrédit des deux cardinaux et leur
conduite.}} {\textsc{- Personnage d'Harcourt.}} {\textsc{- Artifice de
retraite en Italie demandée par la princesse des Ursins.}} {\textsc{-
Louville écarté.}} {\textsc{- Aubigny\,; son énorme progrès et sa
licence.}} {\textsc{- Retraite des cardinaux.}} {\textsc{- Chute du
despacho.}} {\textsc{- Louville a ordre de revenir tout à fait.}}
{\textsc{- Abbé d'Estrées ambassadeur de France.}} {\textsc{- Princesse
des Ursins règne pleinement avec Orry sous elle et Aubigny par elle.}}
{\textsc{- Valouse et sa fortune.}} {\textsc{- La Roche à
l'estampille.}} {\textsc{- Peu de Français demeurent à Madrid.}}
{\textsc{- Chute de Rivas.}}

~

Mgr le duc de Bourgogne, après plusieurs camps, avait passé le Rhin. Le
maréchal de Vauban partit de Paris en cadence, le joignit peu après, et
le 15 août Brisach fut investi. Marsin avait paru le matin du même jour
devant Fribourg. Le gouverneur, se comptant investi, brûla ses
faubourgs, et celui de Brisach lui envoya quatre cents hommes de sa
garnison et soixante canonniers. Tous deux en furent les dupes, et
Brisach se trouva investi le soir. Il tint jusqu'au 6 septembre, et
Denonville, fils d'un des sous-gouverneurs des trois princes, en apporta
la nouvelle, et Mimeur la capitulation. La garnison, qui était de quatre
mille hommes, était encore de trois mille cinq cents qui sortirent par
la brèche avec les honneurs de la guerre, et furent conduits à
Rhinfels\,; la défense fut médiocre. Mgr le duc de Bourgogne s'acquit
beaucoup d'honneur par son application, son assiduité aux travaux, avec
une valeur simple et naturelle qui n'affecte rien et qui va partout où
il convient, et où il y a à voir, à ordonner, à apprendre, et qui ne
s'aperçait pas du danger. Marsin qui prenait jour de lieutenant général,
mais que le roi avait attaché à sa personne pour cette campagne, lui
faisait souvent là-dessus des représentations inutiles. La libéralité,
le soin des blessés, l'affabilité et sa mesure suivant l'état des
personnes et leur mérite, lui acquirent les cœurs de toute l'armée. Il
la quitta à regret sur les ordres réitérés du roi, pour retourner en
poste à la cour, où il arriva le 22 septembre à Fontainebleau. On
s'était bien gardé de lui laisser entrevoir que la campagne n'était pas
finie. Le projet du maréchal de Tallard aurait été embarrassé de sa
personne depuis que l'exemple du roi a borné ces premières têtes de
l'État à des sièges et à des campements exempts des hasards des
batailles.

Le Portugal nous avait manqué, ou plutôt nous avions manqué au Portugal,
avec qui on ne put exécuter ce qu'on lui avait promis de forces navales
pour le mettre à couvert de celles des Anglais. Le duc de Cadaval, le
plus grand seigneur et le plus accrédité du conseil du roi de Portugal,
l'avait fait conclure. L'exécution en était d'autant plus essentielle,
qu'il était clair que les Portugais ne pouvaient point se défendre par
leurs propres forces d'ouvrir leurs ports aux flottes ennemies. Il ne
l'était pas moins que l'Espagne ne pouvait être attaquée que par le côté
du Portugal, et que l'archiduc ne pouvait mettre pied à terre ailleurs
pour y porter la guerre. Rien n'était donc plus principal que de garder
contre lui cette unique avenue, de conserver le continent de l'Espagne
en paix en gardant bien ses ports et ses côtes, et de s'épargner une
guerre ruineuse et dangereuse en ce pays-là, tandis qu'on en avait
partout ailleurs à soutenir. Les alliés avaient le plus puissant intérêt
à s'ouvrir une diversion si avantageuse, qui de plus donnerait par mer
une jalousie et une contrainte continuelle, dès qu'ils pourraient faire
hiverner leur flotte dans le port de Lisbonne, et avoir la liberté dans
tous les autres du Portugal. Aussi ne perdirent-ils pas de temps à
prévenir l'obstacle que nous y pouvions mettre, et par la lenteur ou
l'impuissance d'accomplir à temps notre traité, ils forcèrent le roi de
Portugal à en signer un avec eux, qui pensa plus d'une fois dans la
suite coûter la couronne à Philippe V.

Presque en même temps on s'aperçut de l'infidélité du duc de Savoie.
Phélypeaux, ambassadeur du roi auprès de lui, qui avait le nez fin, en
avertit longtemps sans qu'on voulût le croire. Les traités, la double
alliance, les anciens mécontentements sur le dédommagement du
Montferrat, la ferme opinion de Vaudemont qui se gardait bien de mander
ce qu'il en pensait, la duperie et la confiance si ordinaire de Vendôme,
tout cela rassurait\,; M\textsuperscript{me} de Maintenon ne pouvait
croire coupable le père de M\textsuperscript{me} la duchesse de
Bourgogne\,; Chamillart, séduit par les deux généraux, était de plus
entraîné par elle, et le roi ne voyait que par leurs yeux. À la fin mais
trop tard, ils s'ouvrirent\,: mais avant de raconter le périlleux remède
auquel, pour avoir trop attendu à croire, on fut forcé d'avoir recours,
il faut voir l'entier changement de scène qui arriva en Espagne, et y
reprendre les choses de plus haut.

Si on se souvient de ce que j'ai dit (t. III, p.~217 et suiv.) de la
princesse des Ursins, lorsqu'elle fut choisie pour être camarera-mayor
de la reine d'Espagne à son mariage, et depuis lors de l'apparente
régence de cette princesse, pendant le voyage du roi son mari en Italie,
on verra que M\textsuperscript{me} des Ursins voulait régner\,; elle n'y
pouvait atteindre qu'en donnant à la reine le goût des affaires et le
désir d'y dominer, et se servir du tempérament de Philippe V et des
grâces de son épouse pour un partage du sceptre qui, en laissant
l'extérieur au roi, en ferait passer la puissance à la reine,
c'est-à-dire à elle-même, qui la gouvernerait, et par elle le roi et sa
monarchie. Un si grand projet avait un besoin indispensable d'être
appuyé du roi, qui dans ces commencements surtout ne gouvernait pas
moins la cour d'Espagne que la sienne propre, avec l'entière influence
sur les affaires. Dans ce vaste dessein, conçu dès qu'elle eut joint et
reconnu le roi et la reine, elle acheva de gagner son esprit qu'elle
avait ménagé pendant le voyage de Provence à Barcelone, par lui faire
peur des dames espagnoles, à quoi ne lui servit pas peu l'incartade des
dames du palais au souper du jour du mariage et celle de la reine qui la
suivit. Elle crut n'avoir de ressource qu'en M\textsuperscript{me} des
Ursins, elle s'y livra tout entière.

Cette princesse n'avait pas été moins soigneusement élevée que
M\textsuperscript{me} la duchesse de Bourgogne, ni moins bien instruite.
Elle se trouva née avec de l'esprit et dans cette première jeunesse avec
un bon esprit sage, ferme, suivi, capable de conseil et de contrainte,
et qui, déployé et plus formé dans les suites, montra une constance et
un courage que la douceur et les grâces naturelles de ce même esprit
relevèrent infiniment. À tout ce que j'en ai ouï dire en France, et
surtout en Espagne, elle avait tout ce qu'il fallait pour être adorée.
Aussi en devint-elle la divinité. L'affection des Espagnols, qui seule
et plus d'une fois a conservé la couronne à Philippe V, fut en la plus
grande partie due à cette reine dont ils sont encore idolâtres, dont ils
ne se souviennent encore qu'avec larmes, je dis seigneurs, dames,
militaire, peuple, et où, après tant d'années qu'ils l'ont perdue, ils
ne se peuvent encore consoler.

Un esprit de cette trempe, manié d'abord par un autre esprit tel
qu'était celui de la princesse des Ursins, et sans témoins et à toute
heure, était pour aller bien loin, comme il fit. Le voyage de Barcelone
à Saragosse et de Saragosse à Madrid lui donna un grand loisir
d'insinuation et d'instruction imperceptible\,; et la tenue des états
d'Aragon, où, pour la forme, les affaires passaient par la reine qui les
tenait, instruisit la camarera-mayor elle-même et la mit à portée
d'inspirer l'amour de l'autorité et du gouvernement à la reine, et de
reconnaître peu à peu ce qu'elle en pouvait espérer de ce côté-là.
Arrivé à Madrid, les mêmes moyens se présentèrent par la régence de la
reine avec plus d'étendue qu'à Saragosse. Elle y eut toute l'occasion
qu'elle voulut d'y connaître et d'y sonder l'esprit, les vues, les
intérêts, la capacité de ceux qui formaient la junte, et de tâter,
autant qu'elle put, tout ce qui était ou pouvait devenir personnage. La
bienséance ne voulait pas que la reine fût seule avec tous les hommes
qui étaient de la junte. M\textsuperscript{me} des Ursins l'y accompagna
donc nécessairement et par ce moyen prit nécessairement aussi
connaissance de toutes les affaires. Déjà conduisant la reine, qui avait
mis en elle toute l'affection et la confiance d'une jeune personne qui
ne connaissait qu'elle, qui en dépendait entièrement pour sa conduite
particulière et pour ses amusements, et qui y trouvait toutes les
grâces, la douceur, la complaisance, et la ressource possible,
M\textsuperscript{me} des Ursins la rendit assidue à la junte pour y
être assidue elle-même, et sut fort bien user du respect des Espagnols
pour leur princesse et de ce commencement d'affection qui naissait déjà
en eux pour elle, pour lui faire porter les affaires même hors de la
junte, qui n'étaient pas de nature à y passer avant qu'avoir été
examinées par les deux ou trois têtes principales, telles que le
cardinal Portocarrero, Arias et Ubilla, à qui je donnerai désormais le
nom de marquis de Rivas, du titre de Castille que le roi d'Espagne lui
conféra. Il était l'âme de tout, comme secrétaire de la dépêche
universelle, et comme ayant été du secret et principal acteur du
testament qu'il avait dressé en faveur de Philippe V.

On peut croire que la princesse des Ursins n'avait pas négligé de faire
soigneusement sa cour à la nôtre, et d'y rendre tous les ordinaires un
compte exact de tout ce qui regardait la reine, jusqu'aux plus petits
détails, et de la faire valoir le plus qu'il lui était possible. Ces
comptes s'adressaient à M\textsuperscript{me} de Maintenon, et passaient
au roi par elle\,; en même temps elle n'était pas moins attentive à
informer de même le roi d'Espagne en Italie, et à former la reine à lui
écrire, et à M\textsuperscript{me} la duchesse de Bourgogne sa sœur. Les
louanges que la princesse des Ursins donnait par ses lettres à la reine
tombèrent peu à peu fort naturellement sur les affaires\,; et comme elle
était témoin de ce qui s'y passait, peu à peu aussi elle s'étendit sur
les affaires mêmes, et accoutuma ainsi les deux rois à l'en voir
instruite par la nécessité d'accompagner la reine, sans leur donner de
soupçon d'ambition et de s'en vouloir mêler. Ancrée insensiblement de la
sorte, et sûre à peu près de l'Espagne si la France la voulait soutenir,
elle flatta M\textsuperscript{me} de Maintenon par degrés, pour ne
s'avancer qu'avec justesse, et parvint à la persuader que son crédit ne
serait que le sien, que si on lui laissait quelque autorité dans les
affaires, elle n'en userait que pour la croire et lui obéir
aveuglément\,; que par elle à Madrid, elle à Versailles régnerait en
Espagne, plus absolument encore qu'elle ne faisait en France,
puisqu'elle n'aurait besoin d'aucun détour, mais seulement de
commander\,; enfin, qu'elle ne pourrait atteindre ce degré de puissance
que par la sienne, qui n'aurait et ne pouvait espérer d'autre appui, au
lieu que les ambassadeurs se gouverneraient par le ministère de France,
lesquels les uns et les autres agiraient directement du roi au ministère
d'Espagne, et indépendamment d'elle, qui ignorerait même la plupart des
choses, et ne serait au fil de rien, ni en état d'influer en rien que
par des contours longs et incertains, sur les choses seulement qu'elle
apprendrait du roi même.

M\textsuperscript{me} de Maintenon, dont la passion était de savoir
tout, de se mêler de tout, et de gouverner tout, se trouva enchantée par
la sirène. Cette voie de gouverner l'Espagne sans moyens de ministres
lui parut un coup de partie. Elle l'embrassa avec avidité, sans
comprendre qu'elle ne gouvernerait qu'en apparence, et ferait gouverner
M\textsuperscript{me} des Ursins en effet, puisqu'elle ne pourrait rien
savoir que par elle, ni rien voir que du côté qu'elle lui présenterait.
De là cette union si intime entre ces deux si importantes femmes, de là
cette autorité sans bornes de M\textsuperscript{me} des Ursins, de là la
chute de tous ceux qui avaient mis Philippe V sur le trône, et de tous
ceux dont les conseils l'y pouvaient maintenir, et le néant de nos
ministres sur l'Espagne, et de nos ambassadeurs en Espagne, dont aucun
ne s'y put soutenir qu'en s'abandonnant sans réserve à la princesse des
Ursins. Telle fut son adresse, et telle la faiblesse du roi, qui aima
mieux gouverner son petit-fils par la reine, que de le conduire
directement par ses volontés et ses conseils en se servant du canal
naturel de ses ministres.

Ce grand pas fait et l'alliance intime et secrète conclue entre ces deux
femmes pour gouverner l'Espagne, il fallut faire tomber le roi d'Espagne
dans les mêmes filets\,; la nature y avait pourvu, et un art alors
nécessaire avait achevé. Ce prince, cadet d'un aîné vif, violent,
impétueux, plein d'esprit, mais d'humeur terrible et de volonté outrée,
je le dis d'autant plus librement, qu'on verra dans la suite le triomphe
de sa vertu, ce cadet, dis-je, avait été élevé dans une dépendance, une
soumission nécessaire à bien établir, pour éviter les troubles et
assurer la tranquillité de la famille royale. Jusqu'au moment du
testament de Charles II, on n'avait pu regarder le duc d'Anjou, que
comme un sujet pour toute sa vie, qui plus il était grand par sa
naissance, plus il était à craindre sous un frère roi tel que je viens
de le représenter, et qui, par conséquent, ne pouvait être trop abaissé
par l'éducation, et duit à toute patience et dépendance. La suprême loi,
qui est la raison d'État, demandait cette préférence pour la sûreté et
le bonheur du royaume sur le personnel de ce prince cadet. Son esprit et
tout ce qui en dépend fut donc raccourci et rabattu par cette sorte
d'éducation indispensable, qui, tombant sur un naturel doux et
tranquille, ne l'accoutuma pas à penser ni à produire, mais à se laisser
conduire facilement quoique la justesse fût restée pour choisir le
meilleur de ce qui lui était présenté, et s'expliquer même en bons
termes quand la lenteur, pour ne pas dire la paresse d'esprit, ne
l'empêchait pas de parler. La grande piété qui lui avait été
soigneusement inspirée, et qu'il a toujours conservée, ne trouvant pas
en lui l'habitude de juger et de discerner, le rabattit et le raccourcit
encore, tellement, qu'avec du sens, de l'esprit, et une expression
lente, mais juste et en bons termes, ce fut un prince fait exprès pour
se laisser enfermer et gouverner.

À tant de dispositions si favorables aux desseins de la princesse des
Ursins, il s'y en joignit une autre tout à fait singulière, née du
concours de la piété avec le tempérament. Ce prince en eut un si fort et
si abondant, qu'il en fut incommodé jusqu'au danger pendant son voyage
en Italie. Tout s'enfla prodigieusement\,; la cause de l'enflure ne
trouvant point d'issue par des vaisseaux forts aussi, et peu accoutumés
à céder d'eux-mêmes à la nature, reflua dans le sang. Cela causa des
vapeurs considérables. Enfin cela hâta son retour, et il n'eut de
soulagement qu'après avoir retrouvé la reine. De là on peut juger
combien il l'aima, combien il s'attacha à elle et combien elle sut s'en
prévaloir, déjà initiée aux affaires et conduite par son habile et
ambitieuse gouvernante. Ainsi la présence du roi à Madrid n'exclut point
la reine des secrets ni de l'administration. Elle ne présidait plus à la
junte, mais rien ne s'y délibérait à son insu. La confiance et
l'affection de cette princesse pour la camarera-mayor passa bientôt par
elle au roi, qui ne cherchait qu'à lui plaire. Bientôt la junte devint
une représentation\,; tout se portait en particulier au roi,
ordinairement devant la reine, qui ne décidait rien sur-le-champ, et qui
prenait son parti entre elle et la princesse des Ursins\,; cette
conduite ne fut point contredite par notre cour. Les cardinaux d'Estrées
et Portocarrero eurent beau s'en plaindre et s'y appuyer de nos
ministres, M\textsuperscript{me} de Maintenon se moquait d'eux et le roi
croyait d'une profonde politique d'accréditer la reine de plus en plus,
dans la pensée que l'intérêt personnel de M\textsuperscript{me} de
Maintenon lui inspirait, et dans laquelle elle l'affermissait sans cesse
de gouverner le roi son petit-fils par la reine plus sûrement que par
tout autre canal.

Les anciennes et si intimes liaisons de M\textsuperscript{me} des Ursins
avec les deux cardinaux sur lesquels notre cour avait si principalement
compté cédèrent au désir et à la possibilité de gouverner seule,
indépendamment d'eux, et sûre du roi d'Espagne par la reine elle
n'hésita plus à leur montrer son pouvoir. Cette conduite produisit des
froideurs et des raccommodements\,; trop faible pour les chasser, mais
résolue à s'en défaire à force de dégoûts, elle ne les leur ménagea
qu'autant qu'elle se le crut nécessaire. Elle essaya d'abord de désunir
les deux cardinaux pour les détruire l'un par l'autre. Portocarrero, tel
que je l'ai dépeint et fier du grand personnage qu'il avait fait au
testament de Charles II, et depuis sa mort, portait avec la dernière
impatience le partage d'autorité avec l'homme du roi de France élevé à
la pourpre comme lui. Estrées, vif, ardent, bouillant, haut à la main,
accoutumé aux grandes affaires et à décider, n'était guère moins
impatient que l'autre de n'être pas le maître. Ces bourrasques
dégoûtèrent tellement le cardinal espagnol qu'il voulut quitter la
junte. M\textsuperscript{me} des Ursins trouva qu'il n'en était pas
encore temps, et qu'il serait trop dangereux de délivrer le cardinal
français de ce compagnon. Pour le retenir elle s'avisa de flatter sa
vanité par un expédient tout à fait ridicule. Castanaga, autrefois
gouverneur des Pays-Bas, venait de mourir. Il avait le régiment des
gardes. On avait cru faire passer cette nouveauté d'un régiment des
gardes plus doucement, en le donnant à un homme si distingué. On le
proposa au cardinal Portocarrero, prêtre, archevêque, primat, cardinal
ex-régent\,; il l'accepta, on se moqua de lui. Je ne sais si le cardinal
d'Estrées en prit occasion de se raccommoder avec lui contre la
camarera-mayor, mais enfin ils reconnurent qu'elle les jouait, et ils
s'unirent pour se maintenir contre elle.

Harcourt, dans l'intime liaison de M\textsuperscript{me} de Maintenon,
l'avait extrêmement portée à s'emparer, autant qu'elle le pourrait, des
affaires d'Espagne, et par elle s'était extrêmement lié avec
M\textsuperscript{me} des Ursins, quoique de Paris à Madrid. Ils
s'étaient reconnus réciproquement nécessaires, elle pour avoir des
lumières et des instructions sur la cour et les affaires d'Espagne, où
elle était toute nouvelle encore, et pour avoir un canal et un appui
auprès de M\textsuperscript{me} de Maintenon contre les ambassadeurs du
roi et ses ministres\,; Harcourt, qui visait toujours au ministère, qui
avait manqué son coup, qui, porté par sa protectrice, espérait d'y
revenir, qui n'avait aucune autre voie pour y réussir que de se
conserver des occasions continuelles de parler des affaires et de la
cour d'Espagne, et d'être écouté et consulté sur ces matières. Cela lui
était ôté dès qu'elles passeraient par le canal naturel des ambassadeurs
et des ministres du roi. Torcy, avec qui il avait rompu, était celui
qui, par son département, en avait le détail, et qui faisait et recevait
les dépêches des deux rois et voyait même celles qui étaient de leur
main. Par là, impossibilité qu'Harcourt pût se mêler de rien, ni même
pénétrer ce qui se passait, sans dépayser des gens si nécessairement nés
et initiés dans ces affaires privativement à tous autres. Son intérêt,
celui de M\textsuperscript{me} de Maintenon, celui de
M\textsuperscript{me} des Ursins était en cela le même\,; ce fut aussi
ce qui forma, puis affermit leur union intime, antérieure déjà entre
M\textsuperscript{me} de Maintenon et Harcourt, et ce qui les roidit à
soutenir M\textsuperscript{me} des Ursins pour ôter le secret et la
confiance des affaires d'Espagne aux ambassadeurs et aux ministres et ne
leur en laisser que le gros et les expéditions indispensables.

Sûre de cette position, M\textsuperscript{me} des Ursins leva le masque
contre le cardinal et l'abbé d'Estrées, après avoir jeté ce régiment des
gardes au cardinal Portocarrero, qui bien que réuni à eux n'osa d'abord
après crier si haut qu'eux. Cette guerre déclarée fit un grand éclat.
C'est ce que la camarera-mayor voulait, qui, se sentant si bien appuyée,
demanda hautement la permission de se retirer en Italie, bien sûre de
n'être pas prise au mot, et de faire tout retomber sur les Estrées qui
ne pourraient demeurer avec elle, et de s'en délivrer par cet
artificieux moyen. Il ne réussit pourtant pas sans combat.

Les ministres qui sentaient que tout leur échappait en Espagne, si
M\textsuperscript{me} des Ursins y demeurait la maîtresse, soutinrent
les Estrées tant qu'il leur fut possible, et M\textsuperscript{me} de
Maintenon d'autre part à remontrer au roi le désespoir où on jetterait
la reine, en laissant retirer M\textsuperscript{me} des Ursins\,; qu'il
était meilleur et plus sûr de gouverner le roi d'Espagne par la reine
qui, quoi qu'on pût faire, serait toujours maîtresse de son cœur, et par
là de son esprit lent et timide, laquelle elle-même serait conduite par
M\textsuperscript{me} des Ursins si sensée, si bien intentionnée, qui
déjà avait si parfaitement formé la reine\,; que la facilité de voir le
roi à tous moments, et avec toute liberté, à quoi un ambassadeur ne
pouvait prétendre, était une grande commodité pour toutes sortes
d'affaires, que l'insinuation et le choix des temps ferait toujours
passer comme on voudrait d'ici. À ces raisons, M\textsuperscript{me} de
Maintenon, bien instruite par Harcourt et par son propre usage, ajouta
celles de la défiance si fortes en notre cour. Ils persuadèrent au roi
que M\textsuperscript{me} des Ursins, associée en tout à l'ambassadeur
de France, formerait un aide et un éclaircissement mutuel, que l'un par
l'autre l'empêcheraient de tomber dans la dépendance des lumières et de
la volonté de l'un des deux, et le mettraient en état de décider de tout
sans prévention en connaissance de cause, et d'être obéi en Espagne,
promptement et sûrement, sur tous les partis qui seraient pris à
Versailles. Ce spécieux hameçon fut avalé avec facilité, et le roi ne
voulut point ouïr parler de retraite en Italie, ni même que
M\textsuperscript{me} des Ursins cessât d'avoir toute la part aux
affaires qu'elle avait accoutumé d'y prendre. Ainsi entraves à
l'ambassadeur de France, entraves à nos ministres, entraves même à ceux
d'Espagne, mystère de tout ce qu'on voulut et à quiconque on en voulut
faire, dégoût complet aux Estrées qui s'étaient flattés de chasser
M\textsuperscript{me} des Ursins, et qui se voyaient supplantés par
elle, matières continuelles à délibérations secrètes de
M\textsuperscript{me} de Maintenon avec le roi, où Harcourt ne se
laissait pas oublier, et qui sacrifia à M\textsuperscript{me} des Ursins
toutes ses liaisons avec le cardinal Portocarrero, et tout ce qu'il en
avait pu tirer, qui instruisirent la nouvelle amie d'une infinité de
choses importantes.

Cette trame, ourdie dans les plus obscurs réduits de
M\textsuperscript{me} de Maintenon, fut longtemps ignorée de nos
ministres\,; ils ne se réveillèrent tout à fait qu'aux cris redoublés
des Estrées, lorsqu'il n'en fut plus temps. Ils avaient compté sur la
protection de M\textsuperscript{me} de Maintenon, si favorable au
maréchal de Cœuvres et à eux tous jusqu'alors, par le crédit des
Noailles. Leur indolence les empêcha d'éveiller un intérêt plus
pressant, et plus personnel que celui de toutes les alliances et de
toutes les amitiés. Cependant le cardinal Portocarrero, leurré de ce
régiment des gardes, était rentré dans la junte où le cardinal d'Estrées
était demeuré, avec lequel il s'était réuni comme je l'ai déjà dit.
Rivas seul y travaillait avec eux, tellement que déjà
M\textsuperscript{me} des Ursins s'y était défaite de peu d'autres qui
en étaient et qui en étaient sortis sur la querelle et l'éclat du
cardinal Portocarrero. Elle s'était bien gardée de les y laisser
rappeler. C'était autant d'élagué en attendant de se défaire des deux
cardinaux et de Rivas même pour demeurer pleinement maîtresse.

Louville, jusqu'au retour d'Italie, modérateur du roi et de la monarchie
d'Espagne, le seul confident de son cœur, et le distributeur des grâces,
se vit, tout en arrivant avec le roi, écarté. Son esprit, son courage,
sa vivacité, sa vigilance, l'agrément et la gaieté dont il amusait le
roi, l'habitude dès l'enfance, l'autorité qu'il avait acquise sur lui,
la confiance intime dans laquelle il était avec nos ministres, celle où
il était entré par leur ordre et par le conseil de tous ses amis d'ici
avec le cardinal et l'abbé d'Estrées si prévenus en sa faveur par la
grandesse dont le maréchal de Cœuvres lui était uniquement redevable,
tout cela le rendait trop redoutable à M\textsuperscript{me} des Ursins
pour ne s'en pas défaire. Elle avait bien instruit la reine avant le
retour du roi, et l'avait irritée sur le fauteuil de M. de Savoie.
Harcourt, qui avait vu de près tout le terrain que sa maladie avait fait
gagner à Louville dans les affaires et à qui il était si principal que
la camarera-mayor ne fût pas contre-balancée par quelqu'un d'aussi
accoutumé à manier l'esprit du roi d'Espagne, si instruit et si peu
capable de se laisser ni gagner ni intimider, le perdit auprès de
M\textsuperscript{me} de Maintenon, comme un homme fort capable, encore
plus hardi, et dévoué sans réserve au duc de Beauvilliers et à Torcy
qu'elle rie pouvait souffrir. Louville donc, arrivant à Madrid avec le
roi, trouvant une reine dans le palais qui en excluait tous les hommes,
y perdit son logement et bientôt toutes ses privances. La reine retint
presque toujours le roi dans son appartement, souvent dans celui de la
camarera-mayor qui y était contigu. Là, tout se traitait en cachette des
ministres de l'une et de l'autre cour. Rien ne se réglait au despacho
sur-le-champ, nom qui depuis le retour du roi succéda à celui de junte,
et qui était la même chose, et où la reine n'assistait plus. Le roi, qui
sans elle n'avait garde de se déterminer sur quoi que ce fût et qui
assistait très exactement au despacho, en emportait tous les mémoires
chez la reine ou chez M\textsuperscript{me} des Ursins. Orry, dont on a
vu l'union intime avec elle, et qui avait les finances et le commerce,
s'y trouvait en quart avec eux\,; et là se prenaient toutes les
résolutions que le roi reportait toutes faites le lendemain au despacho,
ou quand bon lui semblait, c'est-à-dire quand Orry et
M\textsuperscript{me} des Ursins avaient eu le temps de prendre leurs
délibérations.

Dans la suite, un cinquième fut souvent admis à ce conseil étroit,
l'unique où se réglaient toutes choses, ce cinquième était bien couplé
avec Orry. Il s'appelait d'Aubigny, fils de\ldots.\footnote{Le nom est
  en blanc dans le manuscrit.} procureur au Châtelet de Paris. C'était
un beau et grand drôle, très bien fait et très découplé de corps et
d'esprit, qui était depuis longues années à la princesse des Ursins sur
le pied et sous le nom d'écuyer, et sur laquelle il avait le pouvoir
qu'ont ceux qui suppléent à l'insuffisance des maris. Louville, à qui la
camarera-mayor voulut parler une après-dînée avec le duc de Medina-Celi,
et voulant les voir sans être interrompue, entra, suivie d'eux, dans une
pièce reculée de son appartement. D'Aubigny y écrivait, qui, ne voyant
entrer que sa maîtresse, se mit à jurer et à lui demander si elle ne le
laisserait jamais une heure en repos, en lui donnant des noms les plus
libres et les plus étranges, avec une impétuosité si brusque, que tout
fut dit avant que M\textsuperscript{me} des Ursins pût lui montrer qui
la suivait. Tous quatre demeurèrent confondus\,; d'Aubigny à s'enfuir\,;
le duc de Louville à considérer la chambre pour laisser quelques moments
à la camarera-mayor pour se remettre, et les prendre eux-mêmes. Le rare
est qu'après cela il n'y parut pas, et qu'ils se mirent à conférer comme
s'il ne fût rien arrivé. Bientôt après, ce compagnon qui n'était qu'un
avec Orry, qui le gorgea de biens dans les suites, fut logé au palais
comme un homme sans conséquence par son état, mais où\,? dans
l'appartement de l'infante Marie-Thérèse, depuis épouse de Louis XIV, et
cet appartement paraissant trop petit pour ce seigneur, on y augmenta
quelques pièces contiguës\,; ce ne fut pas sans murmures d'une nouveauté
si étrange, mais il fallut bien la supporter. Grands et autres, tout
fléchit le genou devant ce favori.

À la fin le cardinal d'Estrées, continuellement aux prises avec
M\textsuperscript{me} des Ursins, et continuellement battu, ne put
supporter davantage un séjour en Espagne si inutile à tout bien et si
honteux pour lui\,; il demanda instamment son rappel. Tout ce que purent
les ministres, et même les Noailles qui s'en mêlèrent pour lors, fut
d'obtenir que l'abbé d'Estrées demeurerait avec le caractère
d'ambassadeur. Quoique cela même ne fût pas agréable à la princesse des
Ursins, M\textsuperscript{me} de Maintenon entra dans ce tempérament
pour ne pas se montrer si partiale, et parce qu'en effet cet abbé, après
la déroute des deux cardinaux, n'était pas pour empêcher que tout ne
passât par M\textsuperscript{me} des Ursins, conséquemment par elle,
sans ambassadeurs ni ministres. Je dis la déroute des deux cardinaux,
parce que Portocarrero, voyant son confrère prêt à partir, quitta le
despacho et les affaires où il n'était plus rien après la figure qu'il
avait faite, et dit qu'à son âge il avait besoin de repos et de ne
s'occuper plus que de son salut et de son diocèse. Il ne trouva pas le
moindre obstacle à sa retraite. Don Manuel Arias, gouverneur du conseil
de Castille, qui sentit combien ce changement influait sur son ministère
et portait sur sa considération, imita Portocarrero, et se prépara à se
retirer en son archevêché de Séville, pour y attendre en repos la
pourpre romaine, à laquelle le roi d'Espagne l'avait nommé.

Louville eut ordre de revenir en même temps que le cardinal d'Estrées en
reçut la permission. Le roi d'Espagne en eut quelque légère peine,
quoiqu'il ne le vît plus en particulier. Il lui donna le gouvernement de
Courtrai, qu'il perdit quelque temps après par la guerre, et une grosse
pension qui ne fut pas longtemps payée. Mais il eut aussi environ cent
mille francs qu'il rapporta, et dont il accommoda ses affaires. Il eut
le bon esprit de n'en rien perdre de sa gaieté, d'oublier tout ce qu'il
avait été en Espagne, de vivre avec ses amis, dont il avait beaucoup et
de considérables, et de s'occuper de ses affaires et de se bâtir très
agréablement à Louville.

Ainsi M\textsuperscript{me} des Ursins et Orry, maîtres de tout sans
contradiction de personne, prirent le plus grand vol d'autorité et de
puissance en Espagne qu'on eût vu depuis le duc de Lerme et le comte-duc
d'Olivarès, et ne se servirent de Rivas que comme d'un secrétaire, en
attendant de le chasser comme ils avaient éloigné tous ceux qui avaient
eu le plus de part au testament de Charles II. Le peu de Français qui
étaient au roi d'Espagne furent rappelés en même temps, excepté quatre
ou cinq qui, de bonne heure, s'étaient attachés à la princesse des
Ursins, et qui n'avaient jamais été à portée de se mêler de rien, ni de
lui donner aucun ombrage. Tels furent Valouse qui était ici écuyer du
duc d'Anjou, et qui fit dans les suites une fortune en Espagne jusqu'à
devenir premier écuyer du roi et chevalier de la Toison d'or. Il y est
mort longues années après, toujours bien avec le roi et avec tout le
monde, et toujours fort en garde de se mêler de rien. Quelques bas
valets intérieurs restèrent aussi avec La Roche qui eut l'estampille,
incapable de faire rien qui pût déplaire à M\textsuperscript{me} des
Ursins, et Hersent qui eut l'emploi de \emph{guardaropa}. Le despacho
était déjà tombé en ridicule sur les fins des deux cardinaux. Pour le
rendre tel et fatiguer ces vieillards, M\textsuperscript{me} des Ursins
le fit tenir à dix heures du soir. Après leur retraite, ce ne fut plus
la peine de s'en contraindre, puisque Rivas y était demeuré seul\,; mais
l'étendue de sa charge importunait la camarera-mayor, qui, résolue à
s'en défaire, ne s'en voulait défaire qu'estropié, pour n'avoir pas à
lui donner de successeur entier. Elle détacha donc de sa charge, qui
embrassait tous les départements, excepté les finances et le commerce
qu'Orry faisait sans titre mais sans supérieur, le département de la
guerre et celui des affaires étrangères, qu'elle donna au marquis de
Canales, connu dans ses ambassades sous Charles II, par le nom de don
Gaspar Coloma. On peut juger ce qui resta au pauvre Rivas, dépouillé des
affaires étrangères, des finances et de la guerre. Ce ne fut qu'un
prélude : bientôt après Rivas fut tout à fait remercié. Il survécut à
ses places et à sa fortune dans une obscurité qui ne finit qu'avec sa
vie, qui dura encore pour le moins vingt-cinq ans, pendant lesquels il
eut le plaisir de voir la chute de son ennemie et force grands
changements.

\hypertarget{chapitre-x.}{%
\chapter{CHAPITRE X.}\label{chapitre-x.}}

1703

~

{\textsc{Desmarets enfin présenté au roi.}} {\textsc{- Voyage de
Fontainebleau.}} {\textsc{- Desmarets directeur des finances, et Rouillé
conseiller d'État surnuméraire.}} {\textsc{- Cour de Saint-Germain à
Fontainebleau.}} {\textsc{- Mort du duc de Lesdiguières\,; son
caractère.}} {\textsc{- Canaples duc de Lesdiguières.}} {\textsc{- Mort
de Saint-Évremond\,; sa disgrâce, sa cause.}} {\textsc{- Barbezières
relâché.}} {\textsc{- L'archiduc déclaré roi d'Espagne, sous le nom de
Charles III, par l'empereur.}} {\textsc{- Prince Eugène président du
conseil de guerre de l'empereur.}} {\textsc{- Ragotzi.}} {\textsc{-
Bataille d'Hochstedt gagnée sur les Impériaux.}} {\textsc{- Grand
Seigneur déposé.}} {\textsc{- Rupture avec le duc de Savoie\,; ses
troupes auxiliaires arrêtées et désarmées.}} {\textsc{- Traitement des
ambassadeurs à Turin et en France.}} {\textsc{- Usage de les faire
garder par un gentilhomme ordinaire.}} {\textsc{- Phélypeaux.}}
{\textsc{- Tessé en Dauphiné.}} {\textsc{- Siège de Landau.}} {\textsc{-
Villars ouvertement brouillé avec l'électeur de Bavière.}} {\textsc{-
Origine de l'intimité de Chamillart avec les Matignon.}} {\textsc{-
Famille des Matignon.}} {\textsc{- Coigny\,; son nom, sa fortune.}}
{\textsc{- Coigny refuse de passer en Bavière et {[}perd{]} par là, sans
le savoir, le bâton de maréchal.}} {\textsc{- Marsin passe en Bavière
malgré lui, et est fait maréchal de France.}} {\textsc{- Retour en
France de Villars bien muni.}} {\textsc{- Augsbourg pris par
l'électeur.}} {\textsc{- Armées du Danube et de Flandre en quartiers
d'hiver.}} {\textsc{- Maréchal de Villeroy reste à Bruxelles.}}
{\textsc{- Retour de Fontainebleau par Villeroy et Sceaux.}} {\textsc{-
M\textsuperscript{me} de Mailly se fait préférer pour le carrosse aux
dames titrées, comme dame d'atours.}} {\textsc{- Disgrâce, retour,
faveur et élévation de la marquise de Senecey.}} {\textsc{- Duchesses
ôtaient le service de la chemise et de la sale à la dame d'honneur de la
reine, et la préférence du carrosse.}} {\textsc{- Surintendante\,;
invention et occasion de cette charge.}}

~

Le mercredi 19 septembre, le roi alla coucher à Sceaux, et le lendemain
à Fontainebleau. Il y avait longtemps que les ducs de Chevreuse et de
Beauvilliers cherchaient à tirer Desmarets du triste état où il
languissait depuis la mort de M. Colbert, frère de sa mère. Si on se
souvient de ce que j'ai dit de lui (t. II, p.~406 et suiv.), on trouvera
que je n'ai pas besoin d'en rien répéter ici ni ailleurs. Dès lors
Chamillart avait eu permission de se servir de ses lumières à ressasser
les financiers, mais rien au delà. La surcharge des ministères de la
guerre et des finances avait forcé Chamillart, comme on l'a vu en son
temps, à se faire soulager par l'érection de deux directeurs des
finances par-dessus les intendants. Desmarets, porté par ses deux
cousins, continuait à aider le contrôleur général, mais sourdement et
obscurément, et comme à l'insu du roi, encore qu'il l'eût permis, mais à
cette condition. Cet état déplaisait fort aux deux ducs et à Torcy, qui
ne l'avaient procuré que comme un chausse-pied, pour pouvoir reparaître
et rentrer enfin en grâce, et en quelque place dans les finances,
Chamillart, ami intime de MM. de Chevreuse et de Beauvilliers, et
d'ailleurs le meilleur homme du monde et le plus compatissant au malheur
d'autrui, tenta enfin que ce que faisait Desmarets sous lui se fît
publiquement et par un ordre connu du roi. Il fut rabroué, mais à force
de ne se pas rebuter et de représenter à M\textsuperscript{me} de
Maintenon la nécessité des affaires, il l'obtint.

Ce pas fait, il fut question d'un autre. On voulut que Desmarets fût
présenté au roi. Après quelque intervalle, Chamillart se hasarda de le
demander. Ce fut bien pis que l'autre fois. Le roi se fâcha, dit que
c'était un voleur, de l'aveu de Colbert mourant, son propre oncle, qu'il
avait chassé sur son témoignage même avec éclat, et que c'était encore
trop qu'il eût permis de s'en servir dans un emploi, où, si on lui
laissait le moindre crédit, il ne se déferait pas d'un vice si utile.
Chamillart n'eut qu'à se taire. Néanmoins, encouragé par le dernier
succès, et pressé de temps en temps par les deux ducs, il eut encore
recours à M\textsuperscript{me} de Maintenon, à qui il représenta
l'indécence de se servir publiquement d'un homme en disgrâce, que le roi
ne voulait point voir, le dégoût extrême que cette situation répandait
sur le travail de Desmarets, et le discrédit qui en était la suite, qui
portait directement sur les affaires qu'il lui renvoyait. Il vanta sa
capacité, le soulagement qu'il en recevait, l'utilité qui en revenait
aux finances, et sut si bien faire auprès d'elle que le roi consentit
enfin, mais comme à regret, qu'il lui fût présenté. Chamillart le fit
donc entrer dans le cabinet du roi, à l'issue d'un conseil tenu
l'après-dînée du jour que Sa Majesté partit pour aller coucher à Sceaux,
et de là à Fontainebleau. On ne put rien de plus froid que la réception
que lui fit le roi\,; il y avait vingt ans qu'il ne l'avait vu.
Chamillart, embarrassé d'un éloignement si marqué contre la manière
toute gracieuse dont le roi recevait toujours ceux qu'il voulait bien
revoir après les disgrâces, n'osa passer plus loin. Desmarets demeura
sans titre, mais travailla avec plus de considération, et fut employé en
plus d'affaires qui allèrent sans milieu du contrôleur général à lui, et
de lui au contrôleur général. Mais on vit bientôt qu'il n'est que de
revenir, et que ce grand pas fait, tout vient ensuite et bientôt.

Un mois après, Beauvilliers, Chevreuse et Chamillart unis firent si
bien, que Rouillé fut fait conseiller d'État surnuméraire, en attendant
la première place qui vaquerait, et remit à Desmarets sa place de
directeur des finances en lui remboursant les huit cent mille livres
qu'il avait financées pour cette charge, dont les appointements étaient
de quatre-vingt mille livres de rente, sans ce qu'il s'y pouvait gagner
d'ailleurs. Armenonville, qui était l'autre, ne revit pas reparaître
sans peine ce nouvel astre sur l'horizon soutenu des grâces de la
nouveauté de Chamillart et des deux ducs. Il sentit ce qui en pouvoir
arriver, mais il fut sage et courtisan. Il était de mes amis et
Desmarets très anciennement, comme je l'ai dit ailleurs. La jalousie,
quoique discrète, fit naître dans leurs fonctions plus d'une difficulté
entre eux. Ils savaient la portée où j'étais avec Chamillart leur commun
maître\,; ils venaient à moi me conter leurs douleurs, et je les
remettais souvent bien ensemble, quelquefois même sans aller jusqu'à
Chamillart. La fortune se joua bien ensuite de tous trois, et ne s'est
guère plus moquée des hommes que parce qu'elle a fait enfin du fils de
Desmarets un chevalier de l'ordre, un maréchal de France.

La cour de Saint-Germain vint, le 3 octobre, à Fontainebleau et s'en
retourna le 16. Le roi y donna à Lavienne la survivance de sa charge de
premier valet de chambre à Chancenay son fils. J'ai fait connaître
Lavienne ailleurs. On y apprit la mort du duc de Lesdiguières, gendre du
maréchal de Duras, sans enfants. Une assez courte maladie l'emporta à
Modène. Il s'était extrêmement distingué et fait aimer et estimer en
Italie. Le roi le regretta fort. Il était brigadier, et pour aller
rapidement à tout par sa valeur et son application. Ce fut une véritable
perte pour sa famille et pour celle où il était entré. C'était un homme
doux, modeste, gai, mais qui se sentait fort et qui n'avait pas plus
d'esprit qu'il n'en fallait pour plaire et réussir à notre cour. Fort
honnête homme et fort magnifique, il vivait très bien avec sa femme, qui
en fut fort affligée. Le vieux Canaples se sut bon gré alors de n'avoir
jamais voulu renoncer à cette succession qui le fit duc de Lesdiguières.

On sut aussi, presque en même temps, la mort de Saint-Évremond, si connu
par son esprit, par ses ouvrages et par son constant amour pour
M\textsuperscript{me} Mazarin, qui acheva de le fixer en Angleterre
jusqu'à l'extrême vieillesse dans laquelle il y finit ses jours. Sa
disgrâce, moins connue que lui, est une curiosité qui peut trouver place
ici. La sienne l'avait conduit aux Pyrénées. Il était ami particulier du
maréchal de Créqui\,; il lui en écrivit une lettre de détails qui lui
développa les replis du cœur du cardinal Mazarin, et qui ne fit pas une
comparaison avantageuse de la conduite et de la capacité de ce premier
ministre avec celles du premier ministre espagnol. L'esprit et les
grâces qui sont répandus dans cette lettre en rendent encore les
raisonnements plus forts et plus piquants. Don Louis de Haro lui en eût
fait sa fortune, mais les deux premiers ministres l'ignorèrent jusqu'à
leur mort. Le maréchal de Créqui et M\textsuperscript{me} du
Plessis-Bellière, les deux plus intimes amis de M. Fouquet, furent
arrêtés en même temps que lui et leurs papiers saisis. Le maréchal, qui
ne l'était pas encore, en fut quitte pour un court exil, que le besoin
qu'on eut de lui pour commander une armée accourcit, et lui valut le
bâton de maréchal de France. M\textsuperscript{me} du Plessis-Bellière
n'en fut pas quitte à si bon marché. Parmi ses papiers, on en trouva du
maréchal de Créqui, et parmi ceux-là cette lettre qu'il n'avait pu se
résoudre à brûler, et qui a été depuis imprimée avec les ouvrages de
Saint-Évremond. Les ministres à qui elle fut portée craignirent un si
judicieux censeur. M. Colbert se para de reconnaissance pour son ancien
maître, M. Le Tellier le seconda. Ils piquèrent le roi sur sa jalousie
du gouvernement, et sur ses sentiments d'estime et d'amitié pour la
mémoire encore récente de son premier ministre. Il entra en colère et
fit chercher Saint-Évremond partout, qui, averti à temps par ses amis,
se cacha si bien qu'on ne put le trouver. Las enfin d'errer de lieu en
lieu et de ne trouver de sûreté nulle part, il se sauva en Angleterre où
il fut bientôt recherché par tout ce qu'il y avait de plus considérable
en esprit, en naissance et en places. Il employa longtemps tous ses amis
pour obtenir son pardon\,; la permission de revenir en France lui fut
constamment refusée. Elle lui fut offerte vingt ou vingt-cinq ans après,
lorsqu'il n'y songeait plus. Il avait eu le temps de se naturaliser à
Londres\,; il était fou de M\textsuperscript{me} Mazarin, il ne se
souciait plus de sa patrie\,; il ne jugea pas à propos de changer de
vie, de société, de climat, à soixante-douze ans. Il y vécut encore une
vingtaine d'années en philosophe et y mourut de même avec sa tête
entière et une grande santé, et recherché jusqu'à la fin comme il
l'avait été toute sa vie.

On apprit aussi à Fontainebleau qu'enfin Barbezières avait été mis en
liberté et qu'il allait être conduit, de Gratz où il était, à l'armée du
comte de Staremberg, pour de là passer en celle de M. de Vendôme.

Des nouvelles plus importantes furent\,: l'archiduc déclaré roi
d'Espagne par l'empereur, qui ne fit plus mystère de l'envoyer
incessamment attaquer l'Espagne par le Portugal. Il avait fait depuis
quelque temps un grand changement à sa cour. Le comte de Mansfeld, dont
la cour de Vienne s'était servie pendant son ambassade en Espagne pour
empoisonner la reine, première femme de Charles II, par le ministère de
la comtesse de Soissons, en avait été récompensé, à son retour, de la
présidence du conseil de guerre. Je ne sais ce qu'il commit dans cette
grande place, mais il fut disgracié et relégué, et sa présidence donnée
au prince Eugène, qui la joignit au commandement des armées de
l'empereur et de l'empire, et se trouva ainsi au comble de tout ce qu'il
pouvait prétendre. Cela arriva à la fin de juillet. Eugène a voit été
retenu à Vienne plus tard qu'il n'aurait voulu, par l'inquiétude qu'on y
prenait des mouvements de Hongrie, où le prince Ragotzi s'était déclaré
le chef des mécontents. Son grand-père et son bisaïeul avaient été
princes de Transylvanie. Sa mère avait épousé en secondes noces le
fameux comte Tekeli. Elle était fille du comte Serini, qui eut la tête
coupée avec Frangipani et Nadasti en 1671 à Neustadt, pour avoir voulu
se saisir de la personne de l'empereur Léopold, et s'être mis à la tête
d'une grande révolte en Hongrie. F. Léopold, prince Ragotzi, son fils,
soupçonné de vouloir remuer, avait été arrêté et mis en prison à
Neustadt, en avril 1701, d'où il trouva le moyen de se sauver déguisé en
dragon, en novembre suivant, ayant gagné le capitaine de sa garde et
fait enivrer les soldats. Il se retira en Pologne, d'où il vint joindre
le comte Berzini, l'un des chefs des mécontents en Hongrie. Tous lui
déférèrent la qualité de chef\,; ses troupes grossirent, prirent ou
s'emparèrent de force châteaux et petites villes, et causaient un grand
trouble dont Vienne commençait fort à s'alarmer.

En ce même temps, le 28 septembre, on eut nouvelle par un courrier
d'Usson, d'une bataille gagnée près d'Hochstedt sur les Impériaux
commandés par le comte de Stirum, qui avait soixante-quatre escadrons et
quatorze mille hommes de pied. D'Usson commandait un corps séparé de
vingt-huit escadrons, et de seize bataillons dans des retranchements\,;
il eut ordre d'en sortir, le 19 au soir, pour être en état d'attaquer le
20 au matin les Impériaux par un côté, tandis que l'électeur de Bavière
les attaquerait par un autre. Ce prince devait avertir de son arrivée
par trois coups de canon, et d'Usson lui répondre de même. Mais ce
dernier, arrivé trop tôt, joint par Cheyladet avec quelques troupes, fut
aperçu des Impériaux, qui, le croyant seul, vinrent sur lui et
poussèrent la brigade de Vivans jusque dans le village d'Hochstedt. Peny
la soutint avec la brigade de Bourbonnais, et ils s'y défendirent avec
grande valeur. D'Usson qui avait vu les ennemis couler cependant vers
ses retranchements, s'y porta assez à temps pour les obliger à se
retirer, et entendant en même temps redoubler très considérablement le
feu du côté d'Hochstedt, il se douta que c'était l'électeur et le
maréchal de Villars qui arrivaient, et y porta diligemment ses troupes.
Il ne se trompait pas\,; il joignit la tête de leurs troupes qui, avec
ce renfort, défirent les ennemis qui se retirèrent fort précipitamment.
L'électeur les poursuivit deux lieues durant, et son infanterie, qui
pénétra dans un bois où ils s'étaient retirés, sur le chemin de
Nordlingen, en fit un grand carnage. Quatre mille hommes des leurs
demeurèrent sur la place, on leur en prit autant, beaucoup d'étendards,
de drapeaux et de timbales, trente-trois pièces de canon, leurs bateaux
et leurs pontons, et tous leurs équipages. Enfin une victoire complète
qui ne coûta guère que mille hommes. Villars envoya le chevalier de
Tresmane qui arriva vingt-quatre heures après le courrier d'Usson, qui
plus en détail rapporta à peu près les mêmes choses. Il assura qu'on ne
croyait pas que l'armée battue pût se rassembler du reste de la
campagne, et que l'électeur allait marcher au prince Louis de Bade qui
était sous Augsbourg avec vingt mille hommes.

Le changement qui arriva en Turquie ne soulagea pas l'empereur. Les
janissaires, d'accord avec les spahis, entrèrent tumultueusement dans le
sérail à Andrinople, où était leur empereur Mustapha, le déposèrent,
mirent sur le trône son frère Achmet, âgé de sept ans, chassèrent le
grand vizir, et en firent un autre qui aimait fort la guerre, que ces
séditieux voulaient absolument, tuèrent le mufti fuyant vers l'Asie, et,
ce qui est incroyable d'un tel particulier, mais qui fut mandé par notre
ambassadeur comme une chose certaine, on lui trouva quarante millions.
Ce mouvement qui tendait à une rupture de la Porte avec l'empereur et
les autres puissances chrétiennes, donna du courage aux mécontents de
Hongrie, et réchauffa beaucoup le parti de Ragotzi, contre lequel il
fallut augmenter de troupes, à la tête desquelles le prince Eugène se
mit, au lieu de retourner en Italie comme il l'avait jusque-là espéré de
jour en jour.

Après s'être longtemps endormi sur les mauvais desseins du duc de
Savoie, malgré tous les avis de Phélypeaux, ambassadeur du roi à Turin,
on ouvrit enfin les yeux, et on ne put douter qu'il n'eût des ministres
de l'empereur cachés dans sa cour, avec lesquels il traitait. Le roi
témoigna par deux fois à l'ambassadeur de Savoie ses justes soupçons.
Soit que ce ministre fût de concert avec son maître, ou qu'il agît de
bonne foi, il répondit toutes les deux fois sur sa tête de la fidélité
du duc à ses traités avec les deux couronnes. L'éloignement de M. de
Vendôme et de ce qu'il avait mené à Trente retarda les résolutions à
prendre. Vaudemont, qui sentait qu'incontinent nous serions prévenus, ou
nous préviendrions M. de Savoie, avait quitté San-Benedetto et l'armée
qu'il commandait, sans attendre quelques jours de plus M. de Vendôme,
qui arrivait et s'en était allé aux eaux, comme je crois l'avoir déjà
marqué. Vendôme de retour avec ses troupes, fort harassées par la
vigilance de l'ennemi dans toute cette longue traversée, il fut question
de prendre des mesures contre les perfides intentions d'un allié qui
s'était laissé débaucher. On fut quelque temps à les résoudre, puis à
les arranger, et elles le furent avec tant de secret et de justesse,
qu'en un même instant toutes les troupes auxiliaires de Savoie furent
désarmées et arrêtées par notre armée. Il devait y avoir cinq mille
hommes, mais il en avait peu à peu fait déserter la moitié, et on
s'assura de même de ce qu'il y en avait dans les hôpitaux.

Le courrier qui apporta la nouvelle de cette expédition arriva le 5
octobre à Fontainebleau. Torcy fut l'après-dînée chez l'ambassadeur de
Savoie. On peut juger de l'éclat de cette action par toute l'Europe,
qu'on ne rendit publique à la cour que deux jours après. Le lendemain,
l'ambassadeur, de qui Torcy avait pris la parole qu'il ne sortirait
point du royaume, par rapport à la sûreté de Phélypeaux, reçut un
courrier de son maître, qui lui mandait qu'il allait assembler son
conseil sur la nouvelle qu'il recevait de l'arrêt de ses troupes. Il lit
prendre en même temps à Chambéry deux mille cinq cents fusils, qu'on
envoyait à l'armée d'Italie, et arrêter tous les courriers de France, et
tous les Français qui se trouvèrent partout dans ses États. En même
temps Vaudemont, qui ne voulait qu'éviter l'embarras du spectacle de
quelque part qu'il vînt, ne fut que peu de jours aux eaux, où apprenant
la bombé crevée et de notre part, dépêcha un courrier au roi, pour lui
mander qu'à cette nouvelle il quittait tout, et s'en allait trouver
Vendôme à Pavie, et retourner de là à son armée, qui était sur la
Secchia. On en fut encore la dupe, et ce double artifice lui réussit
fort bien malgré toutes les assurances qu'il n'avait cessé de donner de
la fidélité certaine du duc de Savoie. Bientôt après il en renvoya un
autre pour témoigner son zèle, par lequel il manda que M. de Savoie
faisait toutes les démarches d'un prince qui se prépare à la guerre. On
le savait bien sans lui. Cependant Montendre apporta la défaite par M.
de Vendôme, le 28 octobre, de deux mille chevaux que Staremberg envoyait
à M. de Savoie, où il n'y eut que vingt hommes de tués de notre parti.
Sur l'avis que Phélypeaux et l'ambassadeur d'Espagne à Turin étaient
fort resserrés, sans aucune communication entre eux ni avec personne, et
un corps de garde posé devant leurs maisons, du Libois, gentilhomme
ordinaire, eut ordre de se rendre chez l'ambassadeur de Savoie, d'y
loger et de l'accompagner partout. Cet usage en cas de rupture est
ordinaire, même à l'égard des nonces. Ce sont d'honnêtes espions et à
découvert, à qui la chambre de l'ambassadeur ne peut être fermée pour
voir et rendre compte de tout ce qu'il fait et se passe chez lui,
mangeant avec lui, et ne le quittant presque point de vue. Quelque
incommode, pour ne pas dire insupportable, que soit une telle compagnie,
Phélypeaux n'en fut pas quitte à si bon marché. C'était un homme
d'infiniment d'esprit et de lecture, éloquent naturellement et avec
grâce, la parole fort à la main\,; extrêmement haut et piquant, qui
essuya des barbaries étranges, qui souffrit toutes sortes de manquements
et d'extrémités jusque dans sa nourriture, et qui fut menacé plus d'une
fois du cachot et de la tête. Il ne se déconcerta jamais, et désola M.
de Savoie par sa fermeté, son égalité et la hauteur de ses réponses, de
ses mépris, de ses railleries. Ce qu'il a écrit en forme de relation de
cette espèce de prison est un morceau également curieux, instructif et
amusant. Tessé partit de Fontainebleau pour aller commander en Dauphiné,
entrer en Savoie, et commencer ce surcroît de guerre.

Cependant Tallard avait formé le siège de Landau. L'armée du comte de
Stirum était détruite par la bataille d'Hochstedt. Celle du prince
Louis, mal payée et délabrée, observait de loin l'électeur, et il n'y
avait rien au deçà du Rhin qui pût mettre obstacle à l'entreprise.
Marsin, fit l'investiture, et la tranchée fut ouverte le 18 octobre. Il
eût été heureux que la mésintelligence n'eût pas troublé tout ce qu'il
se pouvait faire sur le Danube, et au delà, où il n'y avait plus
d'armées en état de s'opposer à rien de ce que l'électeur eût voulu
entreprendre. Il était en état de porter la guerre dans les pays
héréditaires et de profiter du dénuement de l'empereur, qui de Vienne,
voyait le fer et les feux que Ragotzi portait dans son voisinage. Mais
une guerre intestine tourmentait plus l'électeur que ses prospérités ne
lui donnaient de joie. Villars, continuant à suivre ses projets pour sa
fortune particulière, ne cessait de traverser ce prince en tout, de lui
refuser ses secours pour toutes entreprises qui ne cadraient pas avec
les siennes pour s'enrichir, et de le rendre suspect au roi d'abandonner
ses intérêts. Les choses en vinrent au point que Villars cessa d'aller
chez l'électeur, hors pour des raisons très rares et indispensables, et
d'en user avec lui par ses défiances affectées et ses hauteurs à ne
pouvoir plus être supporté. En cette situation, l'électeur assembla chez
lui les principaux officiers de l'armée, et en leur présence interpella
Villars de lui déclarer s'il agissait avec lui comme il faisait par
ordre du roi ou de soi-même\,; le maréchal n'eut pas le mot à répondre,
et cette démarche, qui mit les choses au net, acheva aussi de le rendre
fort odieux. Il l'était déjà par ses incroyables rapines et par toute sa
conduite avec les troupes, tandis que l'électeur était adoré de tous. De
part et d'autre les courriers marchèrent. Villars, ses coffres remplis
et sa femme absente, ne désirait rien plus que de sortir d'une si triste
situation\,; et l'électeur demandait formellement d'être délivré d'un
homme qui lui manquait à tout avec audace, qui barrait ses projets les
plus certains, et qui tête levée ne semblait être venu en son pays que
pour le mettre à la plus forte contribution à son profit particulier. Le
roi enfin, voyant combien il y avait peu d'apparence de laisser plus
longtemps ces deux hommes ensemble, se détermina à leur donner
satisfaction en les séparant, et à faire maréchal de France celui qu'il
enverrait à la place de Villars, aucun de ceux qui l'étaient déjà n'y
paraissant propre. C'en était moins la raison que le prétexte.

Chamillart, avant sa dernière grande fortune, l'avait commencée par
l'intendance de Rouen que son père avait aussi eue. Ils y étaient
devenus amis intimes des Matignon, au point que le comte de Matignon,
père, longues années depuis, du duc de Valentinois, lui quitta pour rien
la mouvance d'une terre qu'il avait relevant de Thorigny, ce qui
enrichit depuis Matignon sous son ministère, fit son frère maréchal de
France et son fils duc et pair et gendre de M. de Monaco dans les
suites. Les Matignon avaient marié leurs sœurs comme ils avaient pu. Ils
étaient cinq frères et force filles, dont ils cloîtrèrent la plupart, et
firent deux frères d'Église\,: l'un évêque de Lisieux après son oncle
paternel\,; l'autre de Condom, fort homme de bien, mais rien au delà.
L'aîné n'eut que deux filles dont il donna l'aînée à son frère, l'autre
à Seignelay, qui se remaria au comte de Marsan, et le dernier frère,
qu'on appelait Gacé, nous le verrons maréchal de France. Les deux sœurs,
l'une jolie et bien faite, épousa un du Breuil, gentilhomme breton, qui
portait le nom de Nevet, dont elle ne laissa point d'enfants\,; l'autre
Coigny, père du maréchal d'aujourd'hui.

Coigny était fils d'un de ces petits juges de basse Normandie, qui
s'appelait Guillot, et qui, fils d'un manant, avait pris une de ces
petites charges pour se délivrer de la taille après s'être fort enrichi.
L'épée avait achevé de le décrasser. Il regarda comme sa fortune
d'épouser la sœur des Matignon pour rien, et avec de belles terres, le
gouvernement et le bailliage de Caen qu'il acheta, se fit tout un autre
homme. Il se trouva bon officier et devint lieutenant général. Son union
avec ses beaux-frères était intime, il les regardait avec grand respect,
et eux l'aimaient fort et leur sœur qui logeait chez eux et qui était
une femme de mérite. Coigny, fatigué de son nom de Guillot, et qui avait
acheté en basse Normandie la belle terre de Franquetot, vit par hasard
éteindre toute cette maison, ancienne, riche et bien alliée. Cela lui
donna envie d'en prendre le nom, et la facilité de l'obtenir, personne
n'en étant plus en droit de s'y opposer. Il obtint donc des lettres
patentes pour changer son nom de Guillot en celui de Franquetot qu'il
fit enregistrer au parlement de Rouen, et consacra ainsi ce changement à
la postérité la plus reculée. Mais on craint moins les fureteurs de
registres que le gros du monde qui se met à rire de Guillot, tandis
qu'il prend les Franquetot pour bons\,; parce que les véritables
l'étaient, et qu'il ignore si on s'est enté dessus avec du parchemin et
de la cire. Coigny donc, devenu Franquetot et dans les premiers grades
militaires, partagea avec les Matignon, ses beaux-frères, la faveur de
Chamillart. Il était lors en Flandre, où le ministre de la guerre lui
procurait de petits corps séparés. C'était lui qu'il voulait glisser en
la place de Villars, et par là le faire maréchal de France. Il lui manda
donc sa destination, et comme le bâton ne devait être déclaré qu'en
Bavière, même à celui qui lui était destiné, Chamillart n'osa lui en
révéler le secret\,; mais, à ce que m'a dit lui-même ce ministre dans
l'amertume de son cœur, il lui mit tellement le doigt sur la lettre,
que, hors lui déclarer la chose, il ne pouvait s'en expliquer avec lui
plus clairement. Coigny, qui était fort court, n'entendit rien à ce
langage. Il se trouvait bien où il était. D'aller en Bavière lui parut
la Chine\,; il refusa absolument, et mit son protecteur au désespoir, et
lui-même peu après, quand il sut ce qui lui était destiné.

On se tourna à Marsin, auquel arriva un courrier devant Landau, chargé
d'un paquet pour lui, qui en enfermait un autre. Par celui qu'il ouvrit,
il lui était ordonné de quitter le siège tout aussitôt, et de prendre le
chemin qui lui était marqué pour se rendre en Bavière, où seulement et
non plus tôt il devait ouvrir l'autre paquet. En le tâtant il reconnut
qu'il y avait un sceau, et comprit que c'était le bâton de maréchal de
France. La merveille fut que cela ne le tenta point. Il se sentit blessé
de ne l'obtenir que par besoin de lui après la promotion des autres, et
fut effrayé du poids dont on voulait le charger. Il renvoya donc le
courrier avec des excuses et le paquet, qu'il ne devait ouvrir qu'en
Bavière, tel qu'on le lui avait envoyé. Le roi persista et lui redépêcha
aussitôt les mêmes ordres avec le même paquet, pour ne l'ouvrir qu'en
Bavière. Il fallut obéir. Il partit et rencontra Villars en Suisse,
chargé de l'argent de ses contributions personnelles et de l'exécration
publique. L'électeur dit à qui le voulut entendre qu'il emportait deux
millions comptant de son pays, sans ce qu'il avait tiré du pays ennemi,
à quoi avait tendu tout son projet militaire qui lui avait énormément
rendu. Les troupes et les officiers généraux ne l'en dédirent point. Il
offrit de l'argent avant partir à qui en voudrait emprunter, pour s'en
décharger d'autant\,; mais la haine prévalut, qui que ce soit n'en
voulut prendre pour la malice de lui laisser ses coffres pleins, qu'il
amena à bon port en France. L'escorte qui l'avait amené ramena Marsin
chargé de cent mille pistoles pour l'électeur\,; il passa avec lui
beaucoup d'argent pour la paye et les besoins de nos officiers et de nos
troupes, et beaucoup d'autres choses nécessaires pour lesquelles on
profita de l'occasion. En joignant l'électeur, il lui rendit le repos,
et la joie à toute l'armée. Il ouvrit son paquet et y trouva ses ordres,
ses instructions et son bâton, comme il s'en était douté. Le roi le
déclara maréchal de France, quand il le crut arrivé. Il fut parfaitement
d'accord en tout avec l'électeur, et au gré des troupes et des officiers
généraux, et très éloigné de brigandages. Peu après son arrivée, ils
firent le siège d'Augsbourg qu'ils prirent en peu de jours, et mirent
après les troupes dans les quartiers, qui avaient grand besoin de repos.
Le maréchal de Villeroy, à qui les ennemis avaient pris Limbourg, sépara
aussi la sienne. Il prit la place du maréchal de Boufflers à Bruxelles,
pour commander tout l'hiver sur toutes ces frontières, et Boufflers
revint à la cour.

Elle partit de Fontainebleau le 25 octobre, retournant à Versailles par
Villeroy et par Sceaux. Le roi avait dans son carrosse
M\textsuperscript{me} la duchesse de Bourgogne, Madame,
M\textsuperscript{me} la duchesse d'Orléans, la duchesse du Lude et
M\textsuperscript{me} de Mailly, qui l'emporta sur la maréchale de
Cœuvres, grande d'Espagne. Pour expliquer comment se passa cette
préférence, il faut reprendre les choses d'un peu loin. La place de dame
d'honneur a presque toujours été remplie dans tous les temps par de
grandes dames, quelquefois par des femmes de princes du sang, comme on
le voit dans Brantôme. La dernière connétable de Montmorency la fut
aussi, et elle était aussi duchesse de Montmorency. Depuis
M\textsuperscript{me} de Senecey et la comtesse de Fleix, sa fille, en
survivance, qui furent dames d'honneur de la dernière reine mère,
qu'elles survécurent toutes deux, on n'a plus vu de dames d'honneur de
reine que duchesses. Ces deux-là le devinrent, quoique veuves en
1663\footnote{Voy. t. Ier, p.~449, la réception des ducs et pairs à la
  séance du 15 décembre 1663.}. Randan fut érigé pour elles deux
conjointement et pour M. de Foix, fils aîné de la comtesse de Fleix, à
qui, par mort sans enfant, le dernier duc de Foix succéda comme ayant
été appelé par les lettres, en qui cette illustre et heureuse maison de
Grailly, dite de Foix, s'éteignit avec son duché-pairie.

La marquise de Senecey, dame d'honneur de la reine mère et intimement
dans sa confidence, fut chassée lors de l'éclat du Val-de-Grâce, où le
chancelier Séguier eut ordre d'aller fouiller la reine jusque dans sa
gorge, et dont, en homme d'esprit et adroit, il s'acquitta sans
reproches du roi, ni rien perdre dans les bonnes grâces du cardinal de
Richelieu, mais de manière qu'il en mérita celles de la reine, qui de sa
vie n'oublia ce service. Il était question d'intelligence fort
criminelle avec l'Espagne. Il se trouva d'ailleurs assez de choses pour
que la fameuse duchesse de Chevreuse se sauvât hors du royaume, et que
Beringhen, premier valet de chambre du roi, s'enfuît à Bruxelles, ce qui
fit depuis son incroyable fortune. De cette affaire,
M\textsuperscript{me} de Senecey fut exilée à Randan, et pas un d'eux ne
revint qu'à la mort de Louis XIII. Aussitôt après, la reine, devenue
régente, les rappela, chassa M\textsuperscript{me} de Brassac, tante
paternelle de M. de Montausier, duc et pair si longtemps après, rendit à
M\textsuperscript{me} de Senecey sa charge de dame d'honneur, que
M\textsuperscript{me} de Brassac avait eue, et en donna en même temps la
survivance à la comtesse de Fleix pour l'exercer conjointement avec la
marquise de Senecey, sa mère, qui rentrèrent dès ce moment dans la plus
grande faveur et la plus haute considération, qui a toujours duré égale
jusqu'à la mort de la reine. Lorsque le rang des Bouillon se fut établi
et que celui de Rohan commença à poindre, ces deux dames obtinrent un
tabouret de grâce. Une assemblée de noblesse protégée par Gaston,
lieutenant général de l'État, fit ôter ces rangs sans titres et ces
tabourets de grâce, qui furent rendus après les troubles de la
régence\,; et lors de cette monstrueuse promotion de quatorze érections
de duchés-pairies en 1663, celle de Randan en fut une, comme je viens de
le dire, en faveur de la mère, de la fille et du petit-fils.

Jusqu'au retour de M\textsuperscript{me} de Senecey, aucune dame
d'honneur de la reine n'avait disputé la préférence du carrosse à aucune
duchesse, ni même l'honneur de donner la chemise à la reine et de lui
présenter la \emph{sale}, qui était déféré sans difficulté à la plus
ancienne duchesse qui se trouvait présente quand il n'y avait point de
princesse du sang. La \emph{sale} est une espèce de soucoupe de vermeil
sur laquelle les boites, étuis, montres et l'éventail de la reine lui
étaient présentés couverts d'un taffetas brodé, qui se lève en la lui
présentant. Il y a toute apparence que M\textsuperscript{me}s de Senecey
et de Fleix se prévalurent, à leur retour, et de la faveur de la reine
et de celle de la comtesse d'Harcourt et de la duchesse de Chevreuse
auprès d'elle, qui la tournèrent entièrement pour la maison de Lorraine
contre les ducs, pour se mettre en possession de présenter toujours la
sale et donner la chemise, sous prétexte de ne donner point de
préférence aux duchesses ni aux princesses lorraines, qui pourtant ne
faisaient que commencer à le disputer par la faveur des deux que je
viens de nommer. Pour le carrosse, M\textsuperscript{me}s de Senecey et
de Fleix n'y entreprirent rien, parce qu'apparemment que, ne s'agissant
pas là de fonctions, elles n'y purent trouver de prétexte. Il vint
depuis au mariage du roi. La maréchale de Guébriant, nommée dame
d'honneur et point duchesse, mourut en allant trouver la cour en
Guyenne, et ne vit jamais la reine. M\textsuperscript{me} de Navailles,
dont le mari était duc à brevet, qui avait tellement été attaché au
cardinal Mazarin, dont il commandait les chevau-légers, qu'il avait été
son correspondant intime et son homme de la plus grande confiance
pendant ses deux absences hors du royaume, fut nommée à la place de la
maréchale de Guébriant. Elle était en Gascogne dans les terres de son
mari, qui ne songeait à rien moins, et qui n'eut que le temps d'arriver
pour le mariage. Le cardinal Mazarin, qui fit tout pour que le comte de
Soissons ne se trouvât pas mal marié à sa nièce, venait d'inventer pour
elle la charge jusqu'alors inconnue de surintendante de la maison de la
reine, et pour conserver toute préférence à la reine mère avec laquelle
il avait toujours été si uni, à qui il devait tout, et que le roi
respectait si fort, il fit en même temps la princesse de Conti, son
autre nièce, surintendante de sa maison. Cette dernière, étant princesse
du sang, emportait beaucoup de choses par ce rang\,; mais sa piété,
l'extrême délicatesse de sa santé, son attachement à M. le prince de
Conti, presque toujours dans son gouvernement de Languedoc, ne lui
permettaient guère d'exercer cette charge. Elle était tout aux dépens de
celle de dame d'honneur prise sur le modèle du grand chambellan, avant
qu'il fût dépouillé par les premiers gentilshommes de la chambre.

La comtesse de Soissons, toujours à la cour, où elle donnait le ton par
sa faveur auprès du roi qui dans ces temps-là ne bougeait de chez elle,
faisait sa charge, et M\textsuperscript{me} de Navailles n'avait garde
de se commettre avec elle à cause du roi et du cardinal, son oncle, dont
le mari était la créature. La reine ne connaissait personne dans ces
commencements\,; à peine s'expliquait-elle en français. La comtesse de
Soissons montait dans son carrosse, et lui nommait les dames à appeler,
et les appelait pour la reine. Cet usage introduit fut suivi par la
duchesse de Navailles, lorsque la comtesse de Soissons ne s'y trouvait
pas. M\textsuperscript{me} de Montausier, duchesse à brevet, lui succéda
et en usa de même, et cet établissement a toujours continué, depuis
lequel il a valu la préférence aux dames d'honneur dans le carrosse sur
tout ce qui n'est point princesse du sang. Pour les dames d'atours
jamais pas une n'y avait songé, non pas même la comtesse de Béthune, si
longtemps dame d'atours de la reine, si fort et toujours sa favorite, et
si considérée par elle-même, par son beau-père et par son mari,
illustres par leurs charges et leurs négociations, et par le comte,
depuis duc de Saint-Aignan son frère, si bien alors avec le roi, en si
grande privance et premier gentilhomme de sa chambre. Jusqu'à
M\textsuperscript{me} de Mailly, il n'avait donc pas été question de
nulle prétention des dames d'atours. Celle-ci, fort glorieuse, nièce de
M\textsuperscript{me} de Maintenon, mariée de sa main, et parfaitement
bien alors avec elle, imagina cette préférence, la tortilla longtemps,
bouda, et, trouvant enfin sa belle contre un enfant comme la maréchale
de Cœuvres, dont le roi s'amusait comme telle (lequel n'aimait pas les
rangs, et M\textsuperscript{me} de Maintenon beaucoup moins qui avait
bien ses raisons pour cela), l'emporta, non par une décision que
M\textsuperscript{me} de Mailly ne put obtenir, mais par silence sur son
entreprise, qui en fut une approbation tacite dont elle sut se
prévaloir. Cela ne laissa pas de faire du bruit et de paraître
étrange\,; elle dit qu'elle n'imaginait pas disputer aux titrées, ni
avoir jamais que la dernière place\,; mais qu'elle était nécessaire dans
le carrosse, pour y porter et y donner à M\textsuperscript{me} la
duchesse de Bourgogne des coiffes et d'autres hardes légères à mettre
par-dessus tout, à cause des fluxions, à quoi elle était sujette. En
effet elle n'eut jamais que la dernière place, mais elle se conserva
dans la préférence que sa faveur lui fit embler.

\hypertarget{chapitre-xi.}{%
\chapter{CHAPITRE XI.}\label{chapitre-xi.}}

1703

~

{\textsc{L'archiduc en Hollande, non reconnu du pape.}} {\textsc{-
Marcilly à Lyon, dégradé à Vienne.}} {\textsc{- Bataille de Spire gagnée
sur les Impériaux.}} {\textsc{- Landau rendu à Tallard, qui met sont
armée en quartiers d'hiver.}} {\textsc{- Tessé à Chambéry\,; conduite de
Vaudémont\,; Tessé destiné à commander son armée.}} {\textsc{- Vendôme,
refusé du bâton, tente en vain de commander les maréchaux de France,
mais {[}il l'obtient pour{]} ses cadets de lieutenant général.}}
{\textsc{- La Feuillade en Dauphiné.}} {\textsc{- Retour du comte de
Toulouse et du maréchal de Cœuvres.}} {\textsc{- Retour de Villars.}}
{\textsc{- Retour de Tallard.}} {\textsc{- Retour du cardinal
d'Estrées.}} {\textsc{- Retour de Rouillé\,; son caractère.}} {\textsc{-
Berwick général en Espagne.}} {\textsc{- Puységur y va\,; son
caractère.}} {\textsc{- Troupes françaises en Espagne.}} {\textsc{-
Nouvelle junte en Espagne.}} {\textsc{- Caractère de l'abbé d'Estrées.}}
{\textsc{- Quatre compagnies et quatre capitaines des gardes du corps en
Espagne.}} {\textsc{- Duc d'Albe\,; son extraction\,; son caractère\,;
ambassade en France.}} {\textsc{- Sa première réception particulière et
de la duchesse sa femme.}} {\textsc{- Étrange singularité du duc d'Albe,
père de l'ambassadeur.}}

~

L'archiduc était arrivé en Hollande, reconnu par cette république,
l'Angleterre, le Portugal, Brandebourg, Savoie et Hanovre, comme roi
d'Espagne, sous le nom de Charles III, et bientôt après par presque
toutes les autres puissances de l'Europe. Le pape, à qui l'empereur
donna part de cette déclaration par une lettre, ayant su ce qu'elle
contenait, la renvoya à son ministre sans l'avoir ouverte. Landau se
défendait vigoureusement. La dégradation des armes prononcée contre
Marcilly, pour avoir rendu Brisach, par le conseil de guerre, et cet
officier en fuite et réfugié à Lyon, fut une vive leçon au gouverneur de
la place assiégée pour se bien défendre. Tout était en mouvement pour
son secours. Le prince aîné de Hesse, depuis roi de Suède, y menait
vingt-trois bataillons et trente escadrons des troupes du landgrave, son
père et de ce qui s'y était joint. Pracontal y marchait de Flandre avec
vingt et un bataillons, et vingt-quatre escadrons, et le comte de Roucy
fut détaché du siège avec deux mille chevaux et cinq cents hommes de
pied, pour garder les passages du Spirebach et empêcher la surprise, et
qui fut rappelé au camp dès qu'il parut des ennemis auxquels se
joignirent ce qu'il y avait de troupes palatines dans les lignes de
Stollhofen, et de celles qui voltigeaient en deçà du Rhin.

Sur ces nouvelles, Tallard résolut d'aller au-devant d'eux, et de ne les
point attendre dans ses lignes. Il remit la conduite du siège et de ce
qu'il y laissait de troupes au plus ancien lieutenant général, qui était
Laubanie, et sur lequel on pouvait sûrement se reposer\,; choisit
quarante-quatre escadrons et vingt bataillons dans son armée avec
lesquels il campa hors de ses lignes, dès le mercredi au soir, 14
novembre, et manda à Pracontal, arrivé à portée, de le joindre le
lendemain de bonne heure avec sa cavalerie seulement, si son infanterie
ne pouvait arriver, qui l'exécuta ainsi le jeudi 15, à la pointe du
jour. Ils trouvèrent le prince de Hesse qui commandait en chef entre la
petite Hollande et Spire, dont toute l'armée n'était pas tout à fait
encore en bataille. On ne tarda pas à se charger\,; la cavalerie de leur
droite mena assez mal celle de notre gauche, mais celle de la leur ne
tint pas. Leur infanterie fit bonne contenance après sa première
décharge, mais elle ne put résister à celle de Tallard, qui la chargea
la baïonnette au bout du fusil avec tant de vigueur, que quantité de
soldats ennemis furent tués dans les rangs et qu'ils ne purent résister.
Outre ces vingt-trois bataillons qui plièrent, ils en avaient encore
cinq autres qui se retirèrent sans avoir presque combattu. La victoire
fut complète et surprit agréablement le maréchal de Tallard, qui était
fort étourdi vers notre gauche à rétablir l'ébranlement qui y était
arrivé, et qui apprit ce grand succès de notre cavalerie de la droite et
de toute l'infanterie au moment qu'il n'espérait rien moins. Il accourut
à la victoire et y donna ses ordres partout. Il avait plus de cavalerie
qu'eux et un bataillon de moins. On leur prit tout leur canon, presque
tous leurs drapeaux et quantité d'étendards. Le soir même Laubanie manda
à Tallard, qui était sur le champ de bataille, que la chamade était
battue, mais qu'il lui conseillait de ne rien précipiter pour la
capitulation. Labaume, fils du maréchal, arriva le 20 novembre, sur les
cinq heures à Versailles, avec cette grande nouvelle que le roi manda
aussitôt à Monseigneur, qui était à Paris à l'Opéra. Ce prince fit
cesser les acteurs pour l'apprendre aux spectateurs. Pracontal,
lieutenant général et gendre de Montchevreuil, y fut tué. C'était un
homme fort appliqué, avec de la valeur et de la capacité, et qui aurait
justement fait une fortune. Il s'était fort attaché au maréchal de
Boufflers, et M\textsuperscript{me} de Maintenon le protégeait
particulièrement. Sa femme eut le gouvernement de Menin à vendre que
Pracontal avait acheté. Meuse, colonel de cavalerie de la maison de
Choiseul, Calvo, colonel du régiment Royal infanterie et brigadier,
neveu du lieutenant général et chevalier de l'ordre, garçon de beaucoup
de valeur et d'entendement et fort bien voulu de tout le monde,
Beaumanoir, qui venait d'épouser une fille du duc de Noailles, y furent
aussi tués avec force autres moins distingués. Ce dernier ne porta pas
loin la malédiction que son père lui donna en mourant au cas qu'il fît
ce mariage, comme je l'ai rapporté en son temps. Il ne laissa point
d'enfants, et en lui finit cette maison ancienne et illustre. Sa
lieutenance générale de Bretagne fut quelque temps après donnée au
maréchal de Châteaurenauld, et servit bientôt après pour une seconde
fois de dot à une autre Noailles que son fils épousa. Le régiment Royal
infanterie fut donné à Denonville, fils aîné d'un sous-gouverneur des
enfants de France, pour qui Mgr le duc de Bourgogne avait beaucoup de
bonté. Ce prince parut douloureusement affligé en cette occasion de ce
que le roi ne lui avait jamais voulu permettre d'achever la campagne,
qu'on lui fit croire finie après la prise de Brisach. Le chevalier de
Croissy, qui vint apporter les drapeaux et le détail, rapporta que les
ennemis avaient perdu six mille hommes, outre quatre mille prisonniers,
parmi lesquels trois officiers généraux et six colonels. Le jeune comte
de Frise, qui en fut du nombre, fut envoyé le soir même de la bataille
par le maréchal de Tallard coucher à Landau, dont son père était
gouverneur, pour lui apprendre la vérité de cette journée. On prétendit
que l'armée ne perdit pas plus de quatre ou cinq cents hommes, mais
beaucoup plus à proportion d'officiers.

Landau reçut une capitulation honorable de quatre mille hommes qui
étaient dedans il n'en sortit que mille sept cents sous les armes, et
fort peu d'officiers qui furent conduits à Philippsbourg, et on assura
qu'on n'avait pas eu plus de mille hommes tués ou blessés au siège. Le
prince de Hesse fit merveille de tête et de valeur. Il devait être joint
le lendemain par six mille hommes, à qui on avait donné des chariots
pour arriver plus diligemment. On sut après qu'il y avait eu deux
princes de Hesse de tués. Labaume fut fait brigadier, et Laubanie eut le
gouvernement de Landau. Peu après l'armée du Rhin entra dans ses
quartiers d'hiver, ainsi que celle de Flandre, où les ennemis avaient
pris Limbourg.

Tessé était dans Chambéry et avait occupé presque toute la Savoie. Avant
de partir il avait été destiné à commander l'armée de M. de Vaudemont,
qui, prévoyant les difficultés que la défection de M. de Savoie allait
apporter à la guerre d'Italie, ne voulait pas s'exposer aux événements
problématiques entre ses anciens protecteurs et ses nouveaux maîtres, et
avait pris son parti de se retirer à Milan et de s'y préparer à en
emporter les dépouilles si nous le perdions ou à y demeurer le maître si
ce duché restait au roi d'Espagne. L'état de sa santé, dont il a tiré
dans tous les divers temps un merveilleux parti, lui servit de prétexte,
et Tessé, son ami, pour ne pas dire son client, eut ordre d'aller
prendre le commandement de son armée quand il en serait temps.

M. de Vendôme, avant de parvenir au généralat en chef, avait fort pressé
le roi de le faire maréchal de France. Le roi, sur le point de le faire,
en fut retenu par la grandeur de ses bâtards et la similitude qu'il
avait avec eux. Il lui dit donc qu'après y avoir mieux pensé il trouvait
que le bâton ne lui convenait point, et en même temps l'assura qu'il n'y
perdrait rien. En effet, on a vu qu'il sut bien lui tenir parole\,;
ancré à la tête de l'armée d'Italie, et se voyant par son rang à un
comble inespéré, il essaya d'obtenir une patente pour commander les
maréchaux de France\,; le roi, qui n'a élevé ses bâtards que par degrés,
et qui de l'un n'a jamais imaginé de les porter à l'autre, se choqua de
la proposition à ne laisser pas d'espérance la plus légère. Au
commencement de cette campagne, Vendôme, jugeant que le mécontentement
que sa demande avait donné au roi était passé, en hasarda une autre
modifiée. Il proposa une patente qui, sans être maréchal de France,
puisque le roi avait jugé qu'il ne lui convenait pas de l'être, le remît
au même droit que s'il l'avait été, puisque Sa majesté lui avait promis
qu'il ne perdrait rien à ne l'être pas, c'est-à-dire qu'il le laissât
obéir aux maréchaux de France plus anciens lieutenants généraux que lui,
mais qu'il le fît commander à ceux d'entre eux qui étaient ses cadets,
et à qui il aurait commandé sans difficulté si le roi l'avait fait
maréchal de France en son rang.

Quelque plausible que fût cette proposition, le roi ne put se résoudre à
lui laisser commander aucun maréchal de France par voie d'autorité. Il
en parla au maréchal de Villeroy, au mieux alors avec lui, qui se récria
contre, émut les maréchaux de France et l'empêcha\,; en sorte que
Vendôme en fut refusé. Villeroy lui-même me l'a conté en s'en
applaudissant. Tessé le savait comme les autres, mais, en courtisan qui
ne voulait rien hasarder, il en reparla au roi en recevant ses ordres
pour le Dauphiné et l'Italie, et lui proposa, en homme qui voulait
plaire et ne se pas attirer les bâtards, d'éviter de se trouver avec M.
de Vendôme, et de ne prendre que la plus petite armée, qui avait été
commandée un temps par le grand prieur comme le plus ancien des
lieutenants généraux. Le roi lui répondit en ces mêmes termes\,: qu'il
ne fallait pas accoutumer ces messieurs-là à être si délicats, qu'il
avait trouvé très mauvais que M. de Vendôme eût osé songer à commander
des maréchaux de France, et qu'en deux mots il ne voulait point de
ménagements là-dessus ni pour prendre le commandement de la principale
armée ni pour se trouver avec M. de Vendôme et le commander lui-même\,;
que ces messieurs-là en avaient bien assez, et qu'il ne fallait ni lui
ne voulait les gâter davantage\,; qu'ils l'étaient bien assez\,;
qu'ainsi sans avoir aucun égard à cette considération-là, il fit tout ce
qu'il croirait devoir faire pour le bien de la chose et pour l'utilité
de ses affaires en Italie. Tessé, qui me l'a plus d'une fois raconté, en
fut surpris au dernier point, mais, en nez fin, il ne laissa pas de
biaiser pour plaire à M. de Vendôme et encore plus à M. du Maine. M. de
Vendôme, de sa part, ne lui disputa rien, et il évita sagement d'en être
obombré. On verra que M. du Maine, par M\textsuperscript{me} de
Maintenon et par tout ce qu'elle sut employer, ne laissa pas longtemps
le roi dans cette humeur. Pour M. de Vaudemont, gouverneur général du
Milanais avec patente de général des armées du roi d'Espagne, il ne
commandait ni obéissait aux maréchaux de France ni à M. de Vendôme. Ils
vivaient ensemble et agissaient de concert en partité\footnote{Partage
  du commandement.} de commandement, presque jamais ensemble que peu de
jours, et en passant, et Vaudemont toujours à Milan ou avec un corps
séparé.

Lorsque Tessé, après avoir commandé peu de temps en Dauphiné, et occupé
la Savoie, fut sur le point de passer à Milan, on vit un prodige de la
faveur de Chamillart. On a vu en plus d'un endroit de ces Mémoires
quelle avait été la conduite de La Feuillade, et quel était
l'éloignement du roi pour lui, jusqu'à avoir été empêché avec peine de
le casser. Il faut se rapprocher encore ce qui se passa entre le roi et
Chamillart, lorsqu'il eut défense de plus penser à ce mariage pour un
homme qui ne le faisait que par ambition, et pour qui le roi était
déterminé à ne jamais rien faire, enfin avec quelle mauvaise grâce il
consentit enfin par importunité que Chamillart en fît son gendre sans se
départir de sa résolution. Le ministre aidé de sa toute-puissante
protectrice, et du faible que le roi eut toujours pour ses ministres et
pour lui plus que pour aucun qu'il ait jamais eu, si on en excepte le
Mazarin, tourna si bien que, sous prétexte que La Feuillade avait le
gouvernement de Dauphiné, il lui en procura le commandement, et que de
colonel réformé qu'il était trois mois auparavant, lorsqu'il fut fait
maréchal de camp avec les autres, il le poussa au commandement en chef
de deux provinces frontières, et d'un corps d'armée complet. Pour faire
un peu moins crier, il ne mit sous lui que deux maréchaux de camp, ses
cadets\,; la surprise de la cour fut extrême, celle des troupes ne fut
pas moindre, ni l'étonnement amer des premiers officiers généraux. La
Feuillade prit Annecy avec quelques volées de canon, et nettoya quelques
pais postes que Tessé avait exprès laissés pour faire sa cour au
ministre, et il ne resta au duc de Savoie en deçà des Alpes que la
vallée de Tarentaise, où le marquis de Sales s'était retiré avec ses
troupes. On peut juger combien on fit valoir ces bagatelles. Chamillart
enivré de son gendre était dans le ravissement, et La Feuillade en
partant ne tenait pas dans sa peau.

Le comte de Toulouse revint à la cour, et peu de jours après le maréchal
de Cœuvres\,; ils avaient passé un long temps à Toulon, leurs forces
n'étant pas bastantes pour se mesurer avec les Anglais et les
Hollandais. Quand ces flottes se furent éloignées, ils firent un tour à
la mer, où le comte commandait au maréchal comme amiral, et non comme
bâtard à un maréchal de France, toutefois et avec raison soumis à son
conseil, et ayant défense du roi de rien faire que de son avis.

Villars arriva aussi, et ce fut à Marly, mais sans y coucher\,: il était
trop appuyé pour n'être pas bien reçu. Le roi lui fit même une honnêteté
sur ce qu'il n'y avait aucun logement de vide. Il parut avec sa
confiance accoutumée pour ne pas dire son audace, et il eut la
hardiesse, en rendant compte au roi chez M\textsuperscript{me} de
Maintenon à Versailles, de toucher l'étrange corde des contributions\,:
il fit valoir celles qu'il avait fait payer au profit du roi\,; puis
ajouta qu'il était trop bon maître pour vouloir qu'on se ruinât à son
service\,; qu'il savait qu'il était né sans bien\,; qu'il ne lui
dissimulait pas qu'il s'était un peu accommodé, mais que c'était aux
dépens de ses ennemis, se gardant bien d'avouer rien de la Bavière, et
qu'il regardait cela comme une grâce pécuniaire que Sa Majesté lui
faisait sans qu'elle lui coûtât rien. Avec cette pantalonnade et le
sourire gracieux de M\textsuperscript{me} de Maintenon tout passa de la
sorte, et ces démêlés si indécents avec l'électeur de Bavière, et si
funestes aux succès, furent comptés pour rien.

Tallard, à mains plus nettes, salua le roi plus modestement\,; ce fut
peu de jours après. Il arriva comme le roi s'habillait après dîner,
ayant pris médecine. Au lieu de s'en approcher, il gagna par derrière le
monde la porte du cabinet, et y fit sa révérence comme le roi y passa.
Le roi le reçut comme il méritait de l'être, le fit entrer avec lui,
l'entretint peu avant le conseil, et le remit au lendemain chez
M\textsuperscript{me} de Maintenon.

Le cardinal d'Estrées arriva presque en même temps et salua le roi
sortant de chez M\textsuperscript{me} de Maintenon pour aller à son
souper. Il l'embrassa par deux fois, lui fit un grand accueil, et
l'entretint à quelques jours de là dans son cabinet. Quelques jours
après, Louville arriva à Paris, où je causai avec lui tout à mon aise et
à beaucoup de longues reprises.

Rouillé, revenant de l'ambassade de Portugal d'où il était parti avant
la rupture, fut aussi très bien reçu. C'était un homme fort sage, fort
avisé et fort instruit, qui avait conclu le traité qu'on ne put tenir.
Châteauneuf, qui avait été ambassadeur à Constantinople, était allé le
relever, et alla par l'Espagne jusqu'aux frontières de Portugal, où il
trouva qu'il n'avait plus rien à faire.

La guerre devenant très prochaine en Espagne du côté du Portugal, le roi
d'Espagne fit venir de Flandre le comte de Serclaës pour y commander ses
troupes avec quelques autres officiers généraux sous lui, que le roi
gracieusa fort en passant. Il résolut aussi d'y envoyer un corps
d'armée, et choisit le duc de Berwick pour le commander, et Puységur
pour y servir sous lui d'une façon principale, et y être le directeur
unique de l'infanterie, cavalerie et dragons. C'était un simple
gentilhomme de Soissonnais, mais de très bonne et ancienne noblesse, du
père duquel il y a d'excellents Mémoires imprimés, et qui était pour
aller fort loin à la guerre et même dans les affaires. Celui-ci avait
percé le régiment du roi infanterie jusqu'à en devenir
lieutenant-colonel\,; le roi, qui distinguait ce régiment sur toutes ses
autres troupes, et qui s'en mêlait immédiatement comme un colonel
particulier, avait connu Puységur par là. Il avait été l'âme de tout ce
que M. de Luxembourg avait fait de beau en ses dernières campagnes en
Flandre, où il était maréchal des logis de l'armée, dont il était le
chef et le maître pour tous les détails de marches, de campements, de
fourrages, de vivres, et très ordinairement de plans. M. de Luxembourg
se reposait de tout sur lui avec une confiance entière, à laquelle
Puységur répondit toujours avec une capacité supérieure, une activité et
une vigilance surprenante, et une modestie et une simplicité qui ne se
démentit jamais dans aucun temps de sa vie ni dans aucun emploi. Elle ne
l'empêcha pourtant, par aucune considération que ce pût être, de dire la
vérité tout haut, et au roi qui l'estimait fort et qui l'entretenait
souvent tête à tête, et quelquefois chez M\textsuperscript{me} de
Maintenon, et il sut très bien résister au maréchal de Villeroy et à M.
de Vendôme, malgré toute leur faveur, et montrer qu'il avait raison. On
l'a vu ci-dessus succéder avec Montriel, aussi capitaine au régiment du
roi, aux deux gentilshommes de la manche qui furent chassés d'auprès de
Mgr le duc de Bourgogne, à la disgrâce de l'archevêque de Cambrai. Nous
verrons désormais nager Puységur en plus grande eau. Le roi lui fit
quitter sa lieutenance colonelle pour s'en servir plus utilement et plus
en grand. À la fin il est devenu maréchal de France avec
l'applaudissement public, malgré le ministre qui le fit, et qui, après
une longue résistance, n'osa se commettre au cri public et au déshonneur
qu'il aurait fait au bâton, s'il ne le lui avait pas donné, et par le
bâton il le fit après chevalier de l'ordre avec les mêmes délais et la
même répugnance. À la valeur, aux talents et à l'application dans toutes
les parties militaires, Puységur joignit toujours une grande netteté de
mains, une grande équité à rendre justice par ses témoignages, un cœur
et un esprit citoyen qui le conduisit toujours uniquement et très
souvent au mépris et au danger de sa fortune avec une fermeté dans les
occasions qui la demandèrent souvent qui ne faiblit jamais, et qui
jamais aussi ne le fit sortir de sa place. Vingt bataillons, sept
régiments de cavalerie et deux de dragons marchèrent en même temps en
Espagne, où plusieurs officiers généraux eurent ordre de se rendre en
même temps que Villadarias, commandant en Andalousie, inquiétait fort
les Portugais dans les Algarves, où il était entré avec six mille
hommes, avant qu'il fût encore arrivé rien en Portugal de ce que ses
nouveaux alliés avaient promis.

Cependant M\textsuperscript{me} des Ursins, embarrassée de l'éclat de la
retraite des deux cardinaux et de l'expulsion de tous les anciens
ministres qui avaient mis la couronne sur la tête de Philippe V, par le
testament de Charles II, fit une vraie espièglerie. Ce fut une nouvelle
junte qu'elle composa de don Manuel Arias, gouverneur du conseil de
Castille, qu'elle retint par l'autorité du roi, comme il partait pour
son archevêché de Séville\,; du marquis de Mancera, dont j'ai assez
parlé ailleurs pour qu'il ne me reste rien à y ajouter\,; et de l'abbé
d'Estrées comme ambassadeur de France\,; elle la conserva tant qu'elle
se la crut nécessaire pour apaiser le bruit. En attendant elle sut bien
empêcher qu'il ne s'y fît rien de sérieux. Elle ne la laissait s'occuper
que des amusettes d'un bas conseil, tandis que les véritables affaires
se délibéraient et se décidaient chez la reine, fort souvent chez elle
entre elles deux et Orry avec le roi\,; puis on faisait expédier, par
Rivas et par les autres secrétaires d'État de la guerre et des affaires
étrangères, ce qui était résolu et qui avait besoin d'expédition. Arias
seul l'embarrassait par son poids et sa capacité\,; de l'abbé elle s'en
jouait après s'être délivrée de son oncle. C'était un homme bien fait,
galant, d'un esprit très médiocre, enivré de soi, de ses talents, des
grands emplois, et du lustre de sa famille et de ses ambassades jusqu'à
la fatuité, et qui, avec de l'honneur et grande envie de bien faire, se
méprenait souvent et se faisait moquer de lui. Ses mœurs l'avaient exclu
de l'épiscopat\,; la considération des siens, surtout du cardinal, son
oncle, couvrirent ce dégoût par des emplois étrangers qu'il ne tint pas
à lui qu'on ne crût fort importants, et où néanmoins il y avait peu et
souvent rien à faire. Il n'était pas riche, et regardait fort à ses
affaires. Il évita de faire son entrée étant ambassadeur en Portugal, et
le cardinal d'Estrées, qui ne retenait pas volontiers ses bons mots,
même sur sa famille, disait plaisamment de lui qu'il était sorti de
Portugal sans y être entré. Pour Mancera, sa grande vieillesse mettait
la princesse des Ursins fort à l'aise avec lui. On verra bientôt comme
elle sut se défaire de ce reste d'image de conseil.

Ce fut dans ce même temps, peut-être quinze jours après l'établissement
de cette junte, que le roi d'Espagne établit quatre compagnies des
gardes du corps, précisément sur le modèle en tout de celles de France,
excepté qu'il les distingua par nations\,: deux espagnoles, les
premières, qu'il donna au connétable de Castille et au comte de Lémos
que j'ai fait connaître ailleurs\,; l'italienne au duc de Popoli,
chevalier du Saint-Esprit, dont j'aurai lieu de parler\,; la wallonne ou
flamande, qui fut la dernière, à Serclaës, que nous venons de voir
passer de Flandre par Paris, en Espagne, pour y aller commander les
troupes espagnoles. Cette nouveauté fit grand bruit à Madrid, où on ne
les aime pas. Les rois d'Espagne jusqu'alors n'avaient jamais eu de
gardes, que quelques méchants lanciers déguenillés qui ne les suivaient
guère, et en très petit nombre, et qui demandaient à tout ce qui entrait
au palais comme de vrais gueux qu'ils étaient, et qui furent cassés, et
une espèce de compagnie de hallebardiers, qui était l'ancienne garde de
tout temps, et qui fut conservée, qui ne peut être plus justement
comparée qu'à la compagnie des Cent-Suisses de la garde du roi. On
choisit exprès des seigneurs les plus élevés et les plus distingués des
trois nations pour ces quatre charges, afin de les faire passer moins
difficilement\,; et ce fut à cette occasion qu'arriva l'affaire du
banquillo, que j'ai expliquée d'avance en parlant des grands d'Espagne,
lors de l'exil en France des ducs d'Arcos et de Baños pour leur mémoire
contre la réciprocité des rangs, honneurs, etc., des ducs de France et
des grands d'Espagne, presque aussitôt que Philippe V fut monté sur le
trône.

Le duc d'Albe, nommé ambassadeur en France, au lieu de l'amirante de
Castille, était arrivé à Paris avec la duchesse sa femme, et son fils
unique encore enfant, qu'il faisait appeler le connétable de Navarre. Ce
nom est devenu si célèbre sous Charles-Quint et sous Philippe II, par le
fameux duc d'Albe, que je crois lui devoir une légère digression. Henri
IV, roi de Castille, fit, en 1469, duc d'Albe don Garcia Alvarez de
Tolède, troisième comte d'Albe, qui est une terre fort considérable et
fort étendue vers Salamanque, que le roi Jean II donna en titre de comté
en 1430 à don Gutierez Gomez de Tolède, successivement évêque de
Palencia et archevêque de Séville et de Tolède. Ce prélat donna ce comté
au fils de son frère, père du premier duc d'Albe, et ce premier duc
d'Albe fut bisaïeul de mâle en mâle du fameux duc d'Albe. Celui-ci
mourut en janvier 1582. Son fils aîné, qui fut aussi premier duc
d'Huesca, mourut sans enfants, et laissa le fils de son frère son
héritier, qui par sa mère doña Briande de Beaumont hérita aussi du comté
de Lérin, qui est une grandesse, et des titres héréditaires de grand
connétable et de grand chancelier du royaume de Navarre. Ce cinquième
duc d'Albe fut père du septième, et celui-là du huitième, dont le fils
unique est le duc d'Albe, ambassadeur en France.

Son père, qui mourut en novembre 1701, avait épousé la tante paternelle
des ducs d'Arcos et de Baños, c'est-à-dire une Ponce de Léon\,; il était
veuf, chevalier de la Toison d'or, avait eu des emplois distingués, et
été enfin conseiller d'État. C'était un homme de beaucoup d'esprit, avec
du savoir, mais fort extraordinaire. Lorsque Philippe V arriva en
Espagne, il en témoigna beaucoup de joie et lâcha force traits plaisants
et mordants sur la maison d'Autriche et sur quelques seigneurs qu'on lui
croyait attachés. Louville fut convié de l'aller voir à Madrid. Il le
trouva assez malproprement entre deux draps, couché sur le côté droit,
où il était sans avoir changé de place, ni laissé faire son lit depuis
plusieurs mois\,; il se disait hors d'état de remuer et se portait
pourtant très bien. Le fait était qu'il entretenait une maîtresse qui,
lasse de lui, avait pris la fuite. Il en fut au désespoir, la fit
chercher par toute l'Espagne, fit dire des messes et d'autres dévotions
pour la retrouver, tant la religion des pays d'inquisition est éclairée,
et finalement fit vœu de demeurer au lit et sans bouger de dessus le
côté droit, jusqu'à ce qu'elle fût retrouvée. Il avoua enfin cette folie
à Louville comme une chose forte, capable de lui rendre sa maîtresse, et
tout à fait raisonnable. Il recevait chez lui grand monde, et la
meilleure compagnie de la cour, étant lui-même d'excellente
conversation. Avec ce vœu, il ne fut de rien à la mort de Charles II ni
à l'avènement de Philippe V, qu'il ne vit jamais, et à qui il fit faire
toutes sortes de protestations, et il poussa l'extravagance jusqu'à sa
mort, sans être jamais levé ni branlé de dessus son côté droit. Cette
manie est si inconcevable, et pourtant si certaine, que je l'ai crue
digne d'être remarquée d'un homme sage d'ailleurs, sensé et plein
d'esprit dans tout le reste.

Son fils unique, don Antoine Martin de Tolède, ambassadeur en France,
qu'il n'appelait jamais que Martin, qui est assez la façon des
Espagnols, était un homme de mine assez basse, mais beaucoup d'esprit et
fort instruit, très sage, très mesuré, poli avec dignité et qui exerça
son ambassade dans les temps les plus tristes avec beaucoup de courage
et de jugement, à la satisfaction de sa cour et de la nôtre, qui eut
pour lui une véritable estime et une considération très marquée. Sa
femme, sœur des ducs d'Arcos et de Baños, extrêmement vive, encore plus
laide, divertit un peu le monde qui à la fin s'y accoutuma. L'un et
l'autre dans une grande dévotion, le mari plus solide, la femme plus à
l'espagnole, vivaient ici avec magnificence. Le duc d'Albe salua le roi
en particulier dans son cabinet en arrivant. Sa femme fut présentée au
roi dans son cabinet après son souper, en arrivant aussi, par la
duchesse du Lude qu'il avait nommée pour cela. Le roi demeura debout et
l'entretint longtemps. La duchesse du Lude la conduisit de là par la
galerie chez M\textsuperscript{me} la duchesse de Bourgogne, où tout
était plus éclairé qu'à l'ordinaire, laquelle, après le souper du roi,
au lieu de le suivre à l'ordinaire dans son cabinet, était allée
attendre chez elle. Elle la reçut debout et la baisa en entrant et en
sortant. Le roi ne la baisa qu'en entrant\,; de là elle fut chez Madame
sans la duchesse du Lude et chez M\textsuperscript{me} la duchesse
d'Orléans. On fut bien aise de lui faire cette réception extraordinaire
d'autant plus que le duc d'Harcourt avait rendu compte, dès qu'il était
en Espagne, de son inclination française marquée en plusieurs occasions.

\hypertarget{chapitre-xii.}{%
\chapter{CHAPITRE XII.}\label{chapitre-xii.}}

1703

~

{\textsc{Mariage du duc de Mortemart avec la fille du duc de
Beauvilliers.}} {\textsc{- Mariage du marquis de Roye et de la fille de
Ducasse.}} {\textsc{- Fortune et caractère de Ducasse.}} {\textsc{-
Mariage du duc de Saint-Pierre avec la sœur de Torcy, veuve de Rénel.}}
{\textsc{- Prince de Rohan capitaine des gens d'armes de la garde.}}
{\textsc{- Mort de la duchesse de Mantoue.}} {\textsc{- Mort de La
Rongère.}} {\textsc{- Mort de Briord.}} {\textsc{- Mort de Courtin\,;
ses emplois, son caractère.}} {\textsc{- Curiosité sur le vêtement des
gens de plume et de robe.}} {\textsc{- M\textsuperscript{me} de
Varangeville.}} {\textsc{- Étrange vol procuré à Courtin par Fieubet.}}
{\textsc{- Caractère et retraite de Fieubet.}} {\textsc{- Dispute pour
le décanat du conseil entre La Reynie et l'archevêque de Reims, qui le
gagne.}} {\textsc{- Affaire de la quête.}} {\textsc{- Colère du roi
contre les ducs, en particulier contre moi.}} {\textsc{- Audience que
j'eus du roi, dont je sortis content.}} {\textsc{- Raisons de m'être
étendu sur l'affaire de la quête.}} {\textsc{- Effroi de l'empereur des
mécontents.}} {\textsc{- Fanatiques soutenus par la Hollande et
Genève.}} {\textsc{- Rochegude arrêté.}}

~

M. de Beauvilliers qui avait deux fils fort jeunes, et dont toutes les
filles s'étaient faites religieuses à Montargis, excepté une seule, la
maria tout à la fin de cette année au duc de Mortemart qui n'avait ni
les mœurs ni la conduite d'un homme à devenir son gendre. Il était fils
de la sœur cadette des duchesses de Chevreuse et de Beauvilliers. Le
désir d'éviter de mettre un étranger dans son intrinsèque entra pour
beaucoup dans ce choix\,; mais une raison plus forte le détermina. La
duchesse de Mortemart, fort jeune, assez piquante, fort au gré du monde,
et qui l'aimait fort aussi, et de tout à la cour, la quitta subitement
de dépit des romancines\footnote{Ce mot se trouve plusieurs fois dans
  Saint-Simon avec le sens de chansons satiriques, ou simplement de
  reproches vifs et piquants.} de ses sœurs, et se jeta à Paris dans une
solitude et dans une dévotion plus forte qu'elle, mais où pourtant elle
persévéra. Le genre de dévotion de M\textsuperscript{me} Guyon
l'éblouit, M. de Cambrai la charma. Elle trouva dans l'exemple de ses
deux sages beaux-frères à se confirmer dans son goût, et dans sa liaison
avec tout ce petit troupeau séparé, de saints amusements pour s'occuper.
Mais ce qu'elle y rencontra de plus solide fut le mariage de son fils.
L'unisson des sentiments dans cet élixir à part d'une dévotion
persécutée où elle figurait sur le pied d'une grande âme, de ces âmes
d'élite et de choix, imposa à l'archevêque de Cambrai, dont les conseils
déterminèrent contre ce que toute la France voyait, qui demeura surprise
d'un choix si bizarre, et qui ne répondit que trop à ce que le public en
prévit. Ce fut sous de tels auspices que des personnes qui ne perdaient
jamais la présence de Dieu au milieu de la cour et des affaires, et qui
par leurs biens et leur situation brillante avaient à choisir sur toute
la France, prirent un gendre qui n'y croyait point et qui se piqua
toujours de le montrer, qui ne se contraignit, ni devant ni après,
d'aucun de ses caprices ni de son obscurité, qui joua et but plus qu'il
n'avait et qu'il ne pouvait, et qui s'étant avisé sur le tard d'un
héroïsme de probité et de vertu, n'en prit que le fanatisme sans en
avoir jamais eu la moindre veine en réalité. Ce fléau de sa famille et
de soi-même se retrouvera ailleurs.

Pontchartrain fit en même temps le mariage d'un de ses beaux-frères
capitaine de vaisseau, et lors à la mer, avec la fille unique de
Ducasse, qu'on croyait riche d'un million deux cent mille livres.
Ducasse était de Bayonne, où son frère et son père vendaient des
jambons. Il gagna du bien et beaucoup de connaissances au métier de
flibustier, et mérita d'être fait officier sur les vaisseaux du roi, où
bientôt après il devint capitaine. C'était un homme d'une grande valeur,
de beaucoup de tête et de sang-froid et de grandes entreprises, et fort
aimé dans la marine par la libéralité avec laquelle il faisait part de
tout, et la modestie qui le tenait en sa place. Il eut de furieux
démêlés avec Pointis, lorsque ce dernier prit, et pilla Carthagène. Nous
verrons ce Ducasse aller beaucoup plus loin. Outre l'appât du bien, qui
fit d'une part ce mariage, et de l'autre la protection assurée du
ministre de la mer, celui-ci trouva tout à propos à acheter pour son
beau-frère, de l'argent de Ducasse, la charge de lieutenant général des
galères, qui était unique, donnait le rang de lieutenant général, et
faisait faire tout à coup ce grand pas à un capitaine de vaisseau\,;
elle était vacante, par la mort du bailli de Noailles, et n'avait pas
trouvé d'acheteur depuis.

Un troisième mariage qui surprit fort fut celui du duc de Saint-Pierre
avec M\textsuperscript{me} de Rénel, sœur de M. de Torcy, ayant tous
deux des enfants de leur premier mariage. Saint-Pierre était Spinola, sa
première femme aussi. Il avait acheté de Charles II la grandesse de
première classe. Il était fort riche, et, pour se donner un petit État
en Italie, il avait acheté celui de Sabionette fort chèrement.
L'empereur, à qui il convenait, s'en était emparé pendant la précédente
guerre, avant que l'acquéreur s'en fût mis en possession, qui pendant ce
que dura la paix de Ryswick n'en put jamais obtenir la restitution. Je
ne sais si cet objet n'entra pas pour quelque chose dans le mariage
qu'il fit avec une sœur du ministre des affaires étrangères, qui, voyant
presque toutes les filles des ministres assises, fut flatté de faire
aussi asseoir sa sœur. L'âge était cruellement disproportionné\,; le
vieux galant passait pour être garni de cautères, et pour être
extrêmement jaloux et avare quoique avec un extérieur magnifique\,; des
cautères, je n'en sais rien, mais pour la jalousie il tint très
exactement parole à ceux qui l'avaient donné pour tel. Sa galanterie
alla jusqu'à faire l'amoureux, et l'amoureux jusqu'à l'impatience. Il ne
put attendre le courrier qu'il envoya en Espagne pour l'agrément de
cette cour\,; il supplia le roi d'en vouloir bien être garant, et,
moyennant cette légère faveur, il passa outre à épouser. La nouvelle
duchesse était fort jolie. Elle ne vit point les princesses du sang, à
qui le duc de Saint-Pierre ne voulait pas donner l'Altesse pour n'en
recevoir que l'Excellence. Cela se passa assez désagréablement, mais il
tint ferme avec hauteur. Le mariage fait, il ne demeura pas bien
longtemps en France, et emmena sa femme, qu'on ne revit de plusieurs
années et encore avec lui en passant. C'était un homme de beaucoup
d'esprit, qui avait vu, lu et retenu, et qui se retrouvera ailleurs.

En ce même temps M. de Soubise, déjà fort vieux, se démit de sa charge
des gens d'armes qui fut donnée à son fils. Ce n'était pas en soi une
grâce bien difficile, M\textsuperscript{me} de Soubise était accoutumée
à mieux.

Le duc de Mantoue perdit sa femme, d'une branche cadette de sa maison,
personne d'une vertu, d'un mérite et d'une piété singulière, qui avait
bien eu à souffrir de ses fantaisies, de son avarice, et d'un sérail
entier qu'il entretint toute sa vie. Il n'en avait point d'enfants et
songea tout aussitôt à se remarier à une Française. Cette affaire
reviendra bientôt à raconter.

La Rongère, chevalier d'honneur de Madame et chevalier de l'ordre de sa
présentation, mourut en même temps. C'était un gentilhomme du pays du
Maine, qui, avec un nom ridicule, était de fort bonne noblesse. Il
s'appelait Quatre-Barbes. C'était un fort honnête homme, très court
d'esprit, mais de taille et de visage à se louer sur le théâtre pour
faire le personnage des héros et des dieux. Briord, que nous avons vu
ci-devant ambassadeur à Turin et à la Haye, mourut aussi après avoir été
taillé, et laissa une place de conseiller d'État d'épée vacante. C'était
un très homme d'honneur et de valeur, qui avait du sens, quelque esprit,
et beaucoup d'amis qui firent si bien pour lui, que son attachement à M.
le Prince, dont il était premier écuyer, ne nuisit point à sa fortune,
chose fort extraordinaire avec le roi et peut-être unique.

M. Courtin le suivit quelques jours après. C'était un très petit homme,
qui paraissait avoir eu le visage agréable et qui avait été fort galant.
Il avait beaucoup d'esprit, de grâces et de tour, mais rien de guindé,
extrêmement l'air et les manières du grand monde, avec lequel il avait
passé sa vie dans les meilleures compagnies, sans aucune fatuité ni
jamais sortir de son état. Poli, sage, ouvert quoiqu'en effet réservé,
modeste et respectueux, surtout les mains fort nettes et fort homme
d'honneur. Il brilla de bonne heure au conseil et devint intendant de
Picardie. M. de Chaulnes, qui avait toutes ses terres, et qui était fort
de ses amis, les lui recommanda beaucoup\,; et Courtin se fit un grand
plaisir de les soulager. L'année suivante, faisant sa tournée, il vit
que, pour faire plaisir au duc de Chaulnes, il avoir surchargé d'autres
paroisses. La peine qu'il en eut lui fit examiner le tort qu'il leur
avait fait, et il trouva qu'il allait à quarante mille livres. Il n'en
fit point à deux fois, il les paya et les répartit de son argent, puis
demanda à être rappelé. On était si content de lui qu'on eut peine à lui
accorder sa demande\,; mais il représenta si bien qu'il ne pouvait
passer sa vie à faire du mal et à ne pouvoir soulager personne, ni faire
plaisir à qui que ce fût, qu'il obtint enfin de n'être plus
intendant\footnote{Quoique cette anecdote ait déjà été racontée par
  Saint-Simon (t. Ier, p.~393), nous n'avons pas cru devoir supprimer ce
  passage qui n'est pas la reproduction littérale du précédent.}. Il se
tourna aux négociations et eut plusieurs ambassades où il réussit
parfaitement. Il signa les traités de Heilbronn, de Breda, et plusieurs
autres, et fut longtemps et utilement ambassadeur en Angleterre, où, par
M\textsuperscript{me} de Portsmouth, il faisait faire au roi Charles II
tout ce qu'il voulait. Il le lui rendit bien dans la suite.

Revenue en France et Charles II mort, elle y était avec peu de
considération par la vie qu'elle y menait dans Paris. Il revint au roi
qu'on s'était licencié chez elle, et elle-même de parler fort librement
de lui et de M\textsuperscript{me} de Maintenon\,; sur quoi M. de
Louvois eut ordre d'expédier une lettre de cachet pour l'exiler fort
loin. Courtin était ami intime de M. de Louvois. Il avait une petite
maison à Meudon, et il était sur le pied d'entrer librement dans son
cabinet à toutes heures. Un soir qu'il y entra et que M. de Louvois
écrivait seul, et qu'il continuait d'écrire, Courtin vit cette lettre de
cachet sur son bureau. Quand Louvois eut fini, Courtin lui demanda avec
émotion ce que c'était que cette lettre de cachet. Louvois lui dit la
cause. Courtin s'écria que c'était sûrement quelque mauvais office\,;
mais que, quand le rapport serait vrai, le roi était payé pour n'aller
pas contre elle au delà d'un avis d'être plus circonspecte, et qu'il le
priait et le chargeait de le dire de sa part au roi, avant que de
l'envoyer\,; et que, si le roi ne voulait pas l'en croire sur sa parole,
il fît au moins, avant de passer outre, voir les dépêches de ses
négociations d'Angleterre, surtout ce qu'il y avait obtenu d'important
par M\textsuperscript{me} de Portsmouth lors de la guerre de Hollande et
pendant toute son ambassade\,; et qu'après de tels services rendus par
elle, c'était se déshonorer que les oublier. Louvois, qui s'en souvenait
bien, et à qui Courtin en rappela plusieurs traits considérables,
suspendit l'envoi de la lettre de cachet et rendit compte au roi de
l'aventure et de ce que Courtin lui avait dit\,; et sur ce témoignage
qui rappela plusieurs faits au roi, il fit jeter au feu la lettre de
cachet, et fit dire à la duchesse de Portsmouth d'être plus réservée.
Elle se défendit fort de ce qu'on lui imputait, et, vrais ou faux, elle
prit garde désormais aux propos qui se tenaient chez elle.

Courtin avait gagné, à ses ambassades, la liberté de paraître devant le
roi, et partout, sans manteau, avec une canne et son rabat. Pelletier de
Sousy avait obtenu, par son travail avec le roi sur les fortifications,
la même licence tous deux conseillers d'État et tous deux les seules
gens de robe à qui cela fût toléré, excepté les ministres qui
paraissaient de même. Il y avait même peu que les secrétaires d'État
s'habillaient comme les autres courtisans, quoique de couleurs et de
dorure plus modestes, et Chamillart ne prît l'habit gris avec de simples
boutons d'or que depuis qu'il fut secrétaire d'État. Desmarets a été le
seul contrôleur général qui, tout à la fin de la vie du roi, ait pris
l'habit gris, la cravate et le bouton d'or. Pomponne, à son retour,
était aussi vêtu de même, mais il avait été longtemps secrétaire d'État.
Le roi aimait et considérait fort Courtin, et se plaisait avec lui.
Jamais il ne paraissait au souper du roi une ou deux fois la semaine que
le roi ne l'attaquât aussitôt de conversation qui, d'ordinaire, durait
le reste du souper. Il demeura pourtant simple conseiller d'État,
quoique fort distingué, parce qu'il ne vaqua rien parmi les ministres
tant que son âge et sa santé lui auraient permis d'en profiter. En ces
temps-là, et jusqu'à la mort du roi, nul homme du parlement ne
paraissait à la cour sans robe, ni du conseil sans manteau, ni
magistrat, ni avocat nulle part dans Paris sans manteau, où même
beaucoup du parlement avaient toujours leur robe. M. d'Avaux, seul,
conserva la cravate et l'épée, avec un habit toujours noir, au retour de
ses ambassades\,; aussi s'en moquait-on fort, jusque-là que ses amis et
le chancelier lui en parlèrent. Le roi, qui en riait aussi, eut pitié de
cette faiblesse et ne voulut pas lui faire dire de reprendre son rabat
et son manteau. Le président de Mesmes, son frère, ne l'approuvait pas
plus que les autres. Ce pauvre homme, avec sa charge de l'ordre et son
cordon bleu en écharpe, se comptait faire passer pour un chevalier de
l'ordre et se croyait bien distingué des conseillers d'État de robe,
dont il était, par ce ridicule accoutrement. Nous avons vu Courtin
refuser une place de conseiller au conseil royal des finances, et la
première place parmi les ambassadeurs du roi à Ryswick, quoique le roi
lui eût permis, à cause de ses mauvais yeux, de mener avec lui
M\textsuperscript{me} de Varangeville, sa fille, qui était veuve depuis
longtemps et demeurait avec lui, de lui confier le secret des affaires,
et de se servir de sa main pour tout ce qu'il ne voudrait pas confier à
des secrétaires.

M\textsuperscript{me} de Varangeville était une grande femme, très bien
faite et lors encore fort belle et de grand air, qui avait beaucoup
d'esprit et de monde. Elle avait épousé, sans biens, une espèce de
manant de Normandie, fort riche, dont le nom était Rocq, mais qui avait
de l'esprit et du mérite et qui fut longtemps ambassadeur à Venise. Il
mourut peu après son retour, et aurait été plus loin s'il avait vécu. Il
laissa deux filles\,; le président de Maisons en épousa une, dont
j'aurai occasion de parler, et Villars l'autre, qui tôt après ce mariage
devint maréchale et enfin duchesse. Mais je ne puis quitter Courtin sans
conter son aventure unique avec Fieubet.

C'était un autre conseiller d'État très capable, d'un esprit charmant,
dans le plus grand monde de la ville et de la cour et dans les
meilleures compagnies, recherché par toutes les plus distinguées,
quelquefois gros joueur, et qui avait été chancelier de la reine. Il
menait Courtin à Saint-Germain au conseil, et on volait fort dans ce
temps-là. Ils furent arrêtés et fouillés, et Fieubet y perdit gros qu'il
avait dans ses poches. Comme les voleurs les eurent laissés, et que
Fieubet se plaignait de son infortune, Courtin s'applaudit d'avoir sauvé
sa montre et cinquante pistoles qu'il avait fait, à temps, glisser dans
sa brayette. À l'instant voilà Fieubet qui se jette par la portière à
crier après les voleurs et à les rappeler, si bien qu'ils vinrent voir
ce qu'il voulait. «\,Messieurs, leur dit-il, vous me paraissez
d'honnêtes gens dans le besoin, il n'est pas raisonnable que vous soyez
les dupes de monsieur que voilà, qui vous a escamoté cinquante pistoles
et sa montre\,;» et, se tournant à Courtin\,: «\,Monsieur, lui dit-il en
riant, vous me l'avez dit, croyez-moi, donnez-les de bonne grâce et sans
fouiller. » L'étonnement et l'indignation de Courtin furent tels qu'il
se les laissa prendre sans dire une seule parole\,; mais les voleurs
retirés, il voulut étrangler Fieubet, qui était plus fort que lui et qui
riait à gorge déployée. Il en fit le conte à tout le monde à
Saint-Germain\,; leurs amis communs eurent toutes les peines du monde à
les raccommoder. Fieubet était mort longtemps avant lui, retiré aux
Camaldules de Gros-Bois. C'était un homme de beaucoup d'ambition, qui se
sentait des talents pour la soutenir, qui soupirait après les premières
places, et qui ne put parvenir à aucune. Le dépit, la mort de sa femme
sans enfants, des affaires peu accommodées, de l'âge et de la dévotion
sur le tout, le jetèrent dans cette retraite. Pontchartrain envoya son
fils le voir, qui, avec peu de discrétion, s'avisa de lui demander ce
qu'il faisait là. «\,Ce que je fais\,? lui répondit Fieubet, je
m'ennuie\,; c'est ma pénitence, je me suis trop diverti.\,» Il s'ennuya
si bien, mais sans se relâcher sur rien, que la jaunisse le prit et
qu'il y mourut d'ennui au bout de peu d'années.

Il y avait déjà longtemps que Courtin, très infirme, presque aveugle (et
il le devint à la fin), ne sortait plus de sa maison, où il ne recevait
même presque plus personne, lorsqu'il mourut, fort vieux, d'une longue
maladie. Il était doyen du conseil. La Reynie, célèbre pour avoir
commencé à mettre la place de lieutenant de police sur le pied où on la
voit, mais néanmoins homme d'honneur et grand et intègre juge, suivait
Courtin et prétendit être doyen, lorsque l'archevêque de Reims,
conseiller d'État d'Église, entre-deux, le prétendit aussi. La Reynie se
récria\,; il demanda à l'archevêque ce qu'il en prétendait faire, lui
qui par sa dignité de pair précédait le doyen du conseil, et qui par ses
richesses ne pouvait être touché de quelques milliers d'écus que le
doyen avait de plus que les autres conseillers d'État. L'archevêque
convint qu'il n'avait que faire du décanat pour rien, mais que lui
échéant, il le voulait recueillir pour ne pas nuire aux conseillers
d'État d'Église qui n'auraient pas les mêmes raisons de rang et de biens
pour ne s'en pas soucier, et n'en voulut jamais démordre. Cela fit une
question qui fut portée devant le roi au conseil des dépêches, entre les
conseillers d'État d'Église et d'épée d'une part, et ceux de robe de
l'autre c'est-à-dire de six contre vingt-quatre. Outre qu'il ne se
trouva aucune, raison de disparité ni d'exclusion, M. de Reims allégua
des exemples, entre autres, d'un archevêque de Bourges et d'un abbé qui
avaient été conseillers d'État, puis doyens du conseil, et il gagna sa
cause tout d'une voix dans le commencement de l'année suivante.

Une autre affaire finit l'année, à laquelle je pris plus de part. Il y
avait plusieurs jours de grandes fêtes où le roi allait à la grand'messe
et à vêpres, auxquelles une dame de la cour quêtait pour les pauvres\,;
et c'était la reine, ou, quand il n'y en avait point, la Dauphine qui
nommait à chaque fois celle qui devait quêter, et dans l'intervalle des
deux Dauphines, M\textsuperscript{me} de Maintenon prenait soin d'en
faire avertir. Tant qu'il y a eu des filles de la reine ou de
M\textsuperscript{me} la Dauphine, c'était toujours l'une d'elles. Après
que les chambres des filles eurent été cassées, on nomma de jeunes
dames, comme je viens de l'expliquer. La maison de Lorraine, qui n'a
formé son rang que par des entreprises du temps de la Ligue, adroitement
soutenue depuis et augmentée par son attention et son industrie
continuelle, et, à son exemple, celles qui peu à peu se sont fait donner
le même rang par le roi, attentives à tout, évitèrent imperceptiblement
la quête pour se faire après une distinction, et prétendre ne point
quêter, et s'assimiler, en cela comme en leurs fiançailles, aux
princesses du sang. On fut longtemps sans y prendre garde et sans y
songer. À la fin, la duchesse de Noailles, la duchesse de Guiche sa
fille, la maréchale de Boufflers s'en aperçurent. Quelques autres aussi
y prirent garde, s'en parlèrent et m'en parlèrent aussi.
M\textsuperscript{me} de Saint-Simon se trouvant habillée aux vêpres du
roi, un jour de la Conception qu'il n'y avait point de grand'messe et
que M\textsuperscript{me} la duchesse de Bourgogne avait oublié de
nommer une quêteuse, lui jeta la bourse au moment de quêter. Elle quêta,
et nous ne nous doutions pas encore que les princesses songeassent à se
fabriquer un avantage de ne point quêter.

Après que j'en fus averti, je me promis bien que les duchesses
deviendraient aussi adroites qu'elles là-dessus, jusqu'à ce qu'il
arrivât quelque occasion de rendre la chose égale. La duchesse de
Noailles en parla à la duchesse du Lude qui, molle et craignant tout, se
contentait de hausser les épaules\,; et il se trouvait toujours quelque
duchesse neuve et ignorante ou basse, qui de fois à autre quêtait. Enfin
la duchesse du Lude, poussée à bout par M\textsuperscript{me} de
Noailles, en parla à M\textsuperscript{me} la duchesse de Bourgogne,
qui, trouvant la chose telle qu'elle était, voulut voir ce que les
princesses feraient, et à la première fête fit avertir
M\textsuperscript{me} de Montbazon. Elle était fille de M. de Bouillon,
belle et jeune, très souvent à la cour, et de tous côtés propre à faire
la planche. Elle était à Paris, comme elles y allaient toutes aux
approches de ces fêtes depuis nombre d'années. Elle s'excusa, et quoique
se portant fort bien, répondit qu'elle était malade, se mit une
demi-journée au lit, puis alla et vint à son ordinaire. Il n'en fut
autre chose pour lors que de rendre le projet certain. La duchesse du
Lude n'osa pousser la chose\,; M\textsuperscript{me} la duchesse de
Bourgogne non plus, quoiqu'elle se sentît piquée\,; mais cela fit
pourtant qu'aucune duchesse ne voulut ou n'osa plus quêter. Les dames de
qualité effective ne furent pas Longtemps à s'en apercevoir. Elles
sentirent que la quête demeurerait à elles seules et commencèrent aussi
à l'éviter, de manière qu'elle tomba en toutes sortes de mains et
quelquefois même on en manqua. Cela alla si loin, que le roi s'en fâcha
et qu'il fut sur le point de faire quêter M\textsuperscript{me} la
duchesse de Bourgogne. J'en fus averti par les dames du palais, qui
voulaient que nous n'allassions point à Paris pour la fête, et qui
essayèrent de me faire peur que l'orage ne tombât sur moi, qui n'étais
pas encore revenu auprès du roi d'avoir quitté le service. Je n'allais
point à Marly et j'étais encore dans la situation avec lui que j'ai
représentée en son lieu, et que ces dames me flattaient qui pourrait
cesser par là. J'y consentis, à condition que j'aurais sûreté que ma
femme ne serait point nommée pour la quête\,; et comme on ne me la put
donner, nous nous en allâmes à Paris. La maréchale de Cœuvres, comme
grande d'Espagne, refusait toutes les quêtes, et la duchesse de
Noailles, sa mère, donnait pour elle la comtesse d'Ayen, sa belle-fille.
À une autre fête, les deux filles duchesses de Chamillart, qui n'avaient
pu éviter cette fois-là de se trouver à Versailles, furent averties pour
quêter et refusèrent l'une et l'autre. Cela servit à faire crever la
bombe.

Le roi, ennuyé de ces manèges, ordonna lui-même à M. le Grand de faire
quêter sa fille le premier jour de l'an 1704, qui, par nécessité, en sut
faire sa cour aux dépens de qui il lui plut. Il ne m'avait pas pardonné
le pardon demandé par la princesse d'Harcourt à la duchesse de Rohan.
Dès le lendemain je fus averti par la comtesse de Roucy, à qui
M\textsuperscript{me} la duchesse de Bourgogne, qui était présente,
l'avait conté, que le roi était entré très sérieux chez
M\textsuperscript{me} de Maintenon, à qui il avait dit, d'un air de
colère, qu'il était très malcontent des ducs, en qui il trouvait moins
d'obéissance que dans les princes, et que, tandis que toutes les
duchesses refusaient la quête, il ne l'avait pas plutôt proposée à M. le
Grand pour sa fille, qu'il l'avait acceptée. Il ajouta qu'il y avait
deux ou trois ducs dont il se souviendrait toujours.
M\textsuperscript{me} la duchesse de Bourgogne ne les avait pas voulu
nommer à elle, mais bien à M\textsuperscript{me} de Dangeau, à
l'oreille, qui un moment après l'avait chargée de m'avertir d'être sage,
parce qu'il grondait un orage sur ma tête. Cet avis me fut donné chez le
chancelier, lui en tiers, qui ne douta point, ni moi non plus, que je ne
fusse un des trois dont le roi avait parlé. Je lui expliquai ce qui
s'était passé et lui demandai son avis, qui fut d'attendre pour ne point
aller à tâtons. Le soir M\textsuperscript{me} Chamillart me dit que le
roi en avait parlé fort aigrement à son mari. Tous deux étaient fort au
fait de cette affaire. Je les y avais mis de bonne heure, et c'était
eux-mêmes qui avaient fait refuser la quête aux deux duchesses leurs
filles.

Je vis, le lendemain, Chamillart fort matin, qui me conta que, la
veille, chez M\textsuperscript{me} de Maintenon, avant d'avoir eu le
temps d'ouvrir son sac, le roi lui demanda en colère ce qu'il disait des
ducs, en qui il trouvait moins d'obéissance qu'aux princes\,; et tout de
suite lui dit que M\textsuperscript{lle} d'Armagnac quêterait.
Chamillart lui répondit que, ces choses-là n'allant guère jusqu'à son
cabinet, il ne l'avait appris que la veille, mais que les ducs étaient
bien malheureux qu'il leur imputât à crime de ne l'avoir pas deviné, et
les princes fort heureux qu'il leur sût gré d'une chose que les ducs se
seraient empressés de faire s'il leur en eût dit autant qu'à M. le
Grand. Le roi, sans répondre qu'à soi-même, continua que c'était une
chose étrange que, depuis que j'avais quitté son service, je ne
songeasse qu'à étudier les rangs et à faire des procès à tout le
monde\,; que j'étais le premier auteur de celui-ci, et que, s'il faisait
bien, il m'enverrait si loin, que je ne l'importunerais de longtemps.
Chamillart répondit que si j'examinais ces choses de plus près, c'était
que j'étais plus capable et plus instruit que les autres, et que, cette
dignité me venant des rois, Sa Majesté me devait savoir gré de la
vouloir soutenir. Puis, se prenant à sourire, il ajouta, pour le calmer,
qu'on savait bien qu'il pouvait envoyer les gens où il lui plaisait\,;
mais que ce n'était guère la peine d'user de ce pouvoir, quand d'un mot
on pouvait également ce qu'on voulait, et que, quand on ne l'avait pas,
ce n'était que faute de le dire. Le roi point apaisé répliqua\,: que ce
qui le piquait le plus était le refus de ses filles par leurs maris, et
surtout de la cadette, apparemment à mon instigation. Sur quoi
Chamillart répondit que l'un des deux était absent, et que l'autre
n'avait que fait conformer sa femme à ce que faisaient les autres, ce
qui n'avait point ramené le roi, qui, toujours fâché, avait encore
grondé un moment, puis commencé le travail. Après l'avoir remercié
d'avoir si bien parlé sur les ducs en général, et sur moi en
particulier, il me conseilla de parler au roi et au plus tôt, un mot sur
les ducs et la quête, puis sur moi dont il était malcontent, et me dit
la substance de ce qu'il me conseillait de lui dire. Ces propos du roi
étaient le fruit d'une audience assez longue qu'il avait donnée au grand
écuyer avant de passer chez M\textsuperscript{me} de Maintenon.

Au sortir d'avec Chamillart, j'allai conter au chancelier ce que j'en
venais d'apprendre. Il fut du même avis que je parlasse, et tôt\,;
qu'attendre ne ferait que confirmer le roi dans ce qui l'irritait, et ne
rien faire après en lui parlant\,; qu'il fallait donc se commettre à
l'événement, lui demander à lui parler dans son cabinet, et si, comme je
le craignais, il s'arrêtait et se redressait pour m'écouter tout de
suite, lui dire que je voyais bien qu'il ne me voulait pas faire la
grâce pour l'heure de m'entendre, que j'espérais que ce serait une autre
fois, et me retirer tout de suite. Ce n'était pas peu à mon âge, et
doublement mal avec le roi, de l'aller attaquer de conversation. Je
n'avais pas coutume de rien faire sans l'avis du duc de Beauvilliers.
M\textsuperscript{me} de Saint-Simon n'en fut pas que je le prisse,
sûre, ce me dit-elle, qu'il me conseillerait d'écrire et point de
parler, ce qui n'aurait ni la même grâce ni la même force, outre qu'une
lettre ne répond point, et que cet avis contraire à celui des deux
autres ministres me jetterait dans l'embarras. Je la crus et allai
attendre que le roi passât de son dîner dans son cabinet, où je lui
demandai permission de le suivre. Sans me répondre, il me fit signe
d'entrer, et s'en alla dans l'embrasure de la fenêtre.

Comme j'allais parler, je vis passer Fagon et d'autres gens intérieurs.
Je ne dis mot que lorsque je fus seul avec le roi. Alors je lui dis
qu'il m'était revenu qu'il était mécontent de moi sur la quête\,; que
j'avais un si grand désir de lui plaire, que je ne pouvais différer de
le supplier de me permettre de lui rendre compte de ma conduite
là-dessus. À cet exorde il prit un air sévère, et ne répondit pas un
mot. «\,Il est vrai, sire, continuai-je, que depuis que les princesses
ont refusé de quêter, je l'ai évité pour M\textsuperscript{me} de
Saint-Simon\,; j'ai désiré que les duchesses l'évitassent aussi, et
qu'il y en a que j'ai empêchées parce que je n'ai point cru que Votre
Majesté le désirât. --- Mais, interrompit le roi d'un ton de maître
fâché, refuser la duchesse de Bourgogne, c'est lui manquer de respect,
c'est me refuser moi-même\,!» Je répondis que, de la manière que les
quêteuses se nommaient, nous ne pensions point que M\textsuperscript{me}
la duchesse de Bourgogne y fût de part, que c'était la duchesse du Lude,
souvent la première dame du palais qui s'y trouvait, qui indiquait qui
elle voulait. «\,Mais, monsieur, interrompit le roi encore du même ton
haut et fâché, vous avez tenu des discours\,? --- Non, sire, lui dis-je,
aucun. --- Quoi, vous n'avez point parlé\,?\ldots\,» Et de ce ton élevé
poursuivait, lorsqu'en cet endroit j'osai l'interrompre aussi, et,
élevant ma voix au-dessus de la sienne\,: «\,Non, sire, vous dis-je, et
si j'en avais tenu, je l'avouerais à Votre Majesté, tout de même que je
lui avoue que j'ai évité la quête à ma femme, et que j'ai empêché
d'autres duchesses de l'accepter. J'ai toujours cru et eu lieu de croire
que, puisque Votre Majesté ne s'expliquait point là-dessus, qu'elle
ignorait ce qui se passait, ou que, le sachant, elle ne s'en souciait
point. Je vous supplie très instamment de nous faire la justice d'être
persuadé que les ducs, et moi en particulier, eussions pu penser que
Votre Majesté le désirât le moins du monde, toutes se seraient
empressées de le faire, et M\textsuperscript{me} de Saint-Simon, à
toutes les fêtes, et si cela n'eût pas suffi de sa part à vous témoigner
mon désir de vous plaire, j'aurais moi aussi plutôt quêté dans un plat
comme un marguillier de village. Mais, sire, continuai-je, Votre Majesté
peut-elle imaginer que nous tenions aucune fonction au-dessous de nous
en sa présence, et une encore que les duchesses et les princesses font
tous les jours encore dans les paroisses et les couvents de Paris, et
sans aucune difficulté\,? Mais il est vrai, sire, que les princes sont
si attentifs à se former des avantages de toutes choses, qu'ils nous
obligent à y prendre garde, surtout ayant refusé la quête une fois. ---
Mais ils ne l'ont point refusée, me dit le roi d'un ton plus radouci\,;
on ne leur a point dit de quêter. --- Ils l'ont refusée, sire, repris-je
fortement, non pas les Lorraines, mais les autres (par où je lui
désignais M\textsuperscript{me} de Montbazon). La duchesse du Lude en a
pu rendre compte à Votre Majesté, et l'a dû faire, et c'est ce qui nous
a fait prendre notre parti\,; mais comme nous savons combien Votre
Majesté se trouve importunée de tout ce qui est discussion et décision,
nous avons cru qu'il suffisait d'éviter la quête, pour ne pas laisser
prendre cet avantage aux princes, persuadés, comme j'ai eu l'honneur de
vous le dire, que Votre Majesté n'en savait rien, ou ne s'en souciait
point, puisqu'elle n'en témoignait aucune chose. --- Oh bien\,!
monsieur, me répondit le roi d'un ton bas et tout à fait radouci, cela
n'arrivera plus, car j'ai dit à M. le Grand que je désirais que sa fille
quêtât le premier jour de l'an, et j'ai été bien aise qu'elle en donnât
l'exemple par l'amitié que j'ai pour son père.\,» Je répliquai toujours,
regardant le roi fixement, que je le suppliais encore une fois, et pour
moi, et pour tous les ducs, de croire que personne ne lui était plus
soumis que nous, ni plus persuadés, et moi plus qu'aucun, que nos
dignités, émanant de la sienne et nos personnes remplies de ses
bienfaits, il était, comme roi et comme bienfaiteur de nous tous,
despotiquement le maître de nos dignités, de les abaisser, de les
élever, d'en faire comme d'une chose sienne et absolument dans sa main.
Alors, prenant un ton tout à fait gracieux et un air tout à fait de
bonté et de familiarité, il me dit à plusieurs reprises que c'était là
comme il fallait penser et parler, qu'il était content de moi, et des
choses pareilles et honnêtes. J'en pris l'occasion de lui dire que je ne
pouvais lui exprimer la douleur où j'étais de voir que, tandis que je ne
songeais qu'à lui plaire, on ne cessait de me faire auprès de lui les
desservices les plus noirs\,; que je lui avouais que je ne pouvais le
pardonner à ceux qui en étaient capables, et que je n'en pouvais
soupçonner que M. le Grand, «\,lequel, ajoutai-je, depuis l'affaire de
la princesse d'Harcourt, ne me l'a pas pardonné, parce que, ayant eu
l'honneur de vous en rendre compte, Votre Majesté vit que je lui disais
vrai, et non pas M. le Grand, dont je crois que Votre Majesté se
souvient bien, et que je ne lui répète point pour ne la pas fatiguer.\,»
Le roi me répondit qu'il s'en souvenait bien, et en eût je crois écouté
la répétition patiemment, à la façon réfléchie, douce et honnête avec
laquelle il me le dit\,; mais je ne jugeai pas à propos de le tenir si
longtemps. Je finis donc par le supplier que, lorsqu'il lui reviendrait
quelque chose de moi qui ne lui plairait pas, il me fît la grâce de m'en
faire avertir, si Sa Majesté ne daignait me le dire elle-même, et qu'il
verrait que cette bonté serait incontinent suivie ou de ma
justification, ou de mon aveu et du pardon que je lui demanderais de ma
faute. Il demeura un moment après que j'eus cessé de parler, comme
attendant si j'avais plus rien à lui dire\,; il me quitta ensuite avec
une petite révérence très gracieuse, en me disant que cela était bien,
et qu'il était content de moi. Je me retirai en lui faisant une profonde
révérence, extrêmement soulagé et content d'avoir eu le loisir de tout
ce que je lui avais placé sur moi, sur les ducs, sur les princes, en
particulier sur le grand écuyer, et plus persuadé que devant, par le
souvenir du roi de l'affaire de la princesse d'Harcourt, et son silence
sur M. le Grand, que c'était à lui que je devais ce que je venais encore
une fois de confondre.

Sortant du cabinet du roi, l'air très satisfait, je trouvai M. le Duc et
quelques courtisans distingués, qui attendaient son botter dans sa
chambre, qui me regardèrent fort passer, dans la surprise de la durée de
mon audience, qui avait été de demi-heure, chose très rare aux
particuliers chargés de rien que d'en obtenir, et dont aucune n'allait à
la moitié du temps de celle que j'avais eue. Je montai chez moi tirer
M\textsuperscript{me} de Saint-Simon d'inquiétude, puis j'allai chez
Chamillart, que je trouvai sortant de table, au milieu de sa nombreuse
audience\,; où était la princesse d'Harcourt. Dès qu'il me vit, il
quitta tout, et vint à moi. Je lui dis à l'oreille que je venais de
parler au roi longtemps dans son cabinet, tête à tête, que j'étais fort
content\,; mais que, comme cela avait été fort long et qu'il était alors
accablé de gens, je reviendrais le soir lui tout conter. Il voulut le
savoir à l'heure même, parce que, devant, me dit-il, travailler ce
jour-là extraordinairement avec le roi, il voulait être bien instruit,
certain qu'il était que le roi ne manquerait pas de lui en parler, et
qu'il voulait se mettre en état de me servir. Je lui contai donc toute
mon audience. Il me félicita d'avoir si bien parlé.

M\textsuperscript{me} Chamillart et ses filles furent très surprises, et
me surent grand gré de ce que j'avais pris sur moi leur refus de la
quête. Je les trouvai irritées des propos sur elles du grand écuyer et
du comte de Marsan son frère, pourtant leurs bons amis. J'attisai ce
feu, mais j'eus beau faire, les bassesses et les souplesses des Lorrains
auprès d'elles raccommodèrent tout, en sorte qu'au bout d'une quinzaine
il n'y parut plus, et Chamillart aussi piqué qu'elles n'y résista pas
plus longtemps. Il m'apprit au retour de son travail qu'avant d'ouvrir
son sac, le roi lui avait dit qu'il m'avait vu, conté toute la
conversation, et paru tout à fait revenu sur moi, mais encore blessé
contre les ducs, sans qu'il eût pu le ramener entièrement, tant la
prévention, le faible pour M. le Grand et la préférence déclarée de sa
Maintenon pour les princes contre les ducs le tenaient obscurci contre
l'évidence et contre son propre aveu même à Chamillart, d'être content
de moi, dont la conduite ne pouvait toutefois être séparée des autres
par les choses mêmes que je lui avais dites\,; mais c'était un prince
très aisé à prévenir, qui donnait très rarement lieu à l'éclaircir, qui
revenait encore plus rarement, et jamais bien entièrement, et qui ne
voyait, n'écoutait, ne raisonnait plus dès qu'on avait l'adresse de
mettre son autorité le moins du monde en jeu, sur quoi que ce pût être,
devant laquelle justice, droits, raison et évidence, tout disparaissait.
C'est par cet endroit si dangereusement sensible que ses ministres ont
su manier avec tant d'art, qu'ils se sont rendus les maîtres despotiques
en lui faisant accroire tout ce qu'ils ont voulu, et le rendant
inaccessible aux éclaircissements et aux audiences.

Le chancelier fut étonné de ma hardiesse, et ravi du succès. Je me tirai
d'affaires après, avec le duc de Beauvilliers, comme
M\textsuperscript{me} de Saint-Simon me l'avait conseillé, et je trouvai
qu'elle avait eu raison. Je dis au duc que n'ayant pas eu le moment de
le voir avant le dîner du roi, j'avais pris mon parti de lui parler. Il
me témoigna être fort aise que cette audience se fût si bien passée,
mais qu'il m'aurait conseillé de l'éviter et d'écrire dans la situation
où j'étais, quoique par l'événement j'eusse beaucoup mieux fait.
Plusieurs ducs me parlèrent de cette affaire, qui fit du bruit. Rien
n'égala la surprise et la frayeur de M. de Chevreuse, avec qui j'étais
intimement, et à qui je contai tout\,; mais quand il entendit que
j'avais dit au roi que nous savions qu'il craignait toute discussion et
toute décision, il recula six pas\,: «\,Vous avez dit cela au roi,
s'écria-t-il, et en propres termes\,? Vous êtes bien hardi. --- Vous ne
l'êtes guère, lui répondis-je, vous autres vieux seigneurs, qui êtes si
bien et en familiarité avec lui, et bien faibles de ne lui oser dire
mot\,; car s'il m'écoute moi jeune homme, point accoutumé avec lui, mal
d'ailleurs avec lui, et de nouveau encore plus par ceci, et si la
conversation amenée avec colère finit après de tels propos par de la
bonté et des honnêtetés après qu'elle a duré tant que j'ai voulu, que
serait-ce de vous autres si vous aviez le courage de profiter de la
manière dont vous êtes avec lui, et de lui dire ce qu'il lui faudrait
dire, et que vous voyez que je lui dis non seulement impunément, mais
avec succès pour moi\,!» Chevreuse fut ravi que j'eusse parlé de la
sorte, mais il en avait encore peur\,; la maréchale de Villeroy,
extrêmement de mes amies, et qui avait infiniment d'esprit et beaucoup
de dignité et de considération personnelle, trouva que j'avais très bien
fait et dit, et que cette conversation me tournerait à bien. En effet,
je sus par M. de Laon que le roi avait dit à Monseigneur que je lui
avais parlé avec beaucoup d'esprit, de force et de respect, qu'il était
content de moi, que les choses étaient bien différentes de ce que M. le
Grand lui avait dit, et que les princesses avaient refusé la quête, ce
que Monseigneur lui confirma.

M. de Laon était frère de Clermont, dont j'ai raconté la disgrâce, que
Monseigneur aimait toujours. Il m'apprit que Monseigneur se moquait
souvent des prétentions des princes et des idées de son amie
M\textsuperscript{lle} de Lislebonne là-dessus, quelquefois jusque
devant elle, et qu'il n'était point mené par elle ni par
M\textsuperscript{me} d'Espinoy là-dessus. Il avait su ce propos du roi
à Monseigneur par M\textsuperscript{lle} Choin, avec qui par son frère
il était demeuré dans la liaison la plus intime. Il me conta plusieurs
détails là-dessus qui m'ôtèrent d'inquiétude sur Monseigneur pour les
rangs. Je les contai au duc de Montfort, mon ami intime, qui n'en était
pas moins en peine que moi, mais je ne nommai pas mon auteur, qui ne le
voulait pas être. Le rare est qu'il était en grande liaison avec ce
prélat par les Luxembourg\,; Il lui en gardait le secret, et me l'avait
bien voulu confier, tellement que le duc de Montfort, qui ne me voyait
en nulle liaison avec Monseigneur ni avec personne de sa cour
particulière, ne pouvait imaginer d'où je les avais sus, et pensait
presque qu'il fallait que le diable me l'eût dit.

Je me suis peut-être trop étendu sur une affaire qui se pouvait beaucoup
plus resserrer. Mais, outre qu'elle est mienne, il me semble que c'est
plus par des récits détaillés de ces choses de cour particulières qu'on
la fait bien connaître, et surtout le roi si enfermé et si difficile à
pénétrer, si rare à approcher, si redoutable à ses plus familiers, si
plein de son despotisme, si aisé à irriter par ce coin-là et si
difficile à en revenir, même en voyant la vérité d'une part et la
tromperie de l'autre, et toutefois capable d'entendre raison quand il
faisait tant que de vouloir bien écouter, et que celui qui lui parlait
la lui montrait même avec force, pourvu qu'il le flattât sur son
despotisme, et assaisonnât son propos du plus profond respect\,: tout
cela se touche au doigt, par les récits mieux que par toutes les autres
paroles\,: et c'est ce qui se voit bien naturellement dans celui-ci, et
dans ce que j'ai raconté en son temps de l'affaire de
M\textsuperscript{me} de Saint-Simon, et de M\textsuperscript{me}
d'Armagnac, et de la princesse d'Harcourt avec la duchesse de Rohan.

Le roi et l'empereur n'étaient pas en repos chez eux. Outre la guerre
extérieure, les mécontents de Hongrie, en nombre effrayant et appuyés de
plusieurs seigneurs et de beaucoup de noblesse, s'étaient emparés des
villes, des montagnes de Hongrie et d'une partie des mines. Quantité de
châteaux s'étaient rendus à eux où ils avaient trouvé beaucoup de
canons. Ils étaient descendus dans la plaine, et se montraient à main
armée autour de Presbourg. Leurs partis mettaient le feu à des villages
dont l'incendie se faisait voir de Vienne, et l'empereur pensa être
surpris dans un château où il dînait à une partie de chasse. L'effroi
qu'il en eut lui fit ordonner d'apporter de Presbourg à Vienne la
couronne de Hongrie, qui depuis les premières invasions des Turcs, avait
été apportée de Bude, capitale du royaume, à Presbourg. C'est une
couronne d'or qui, envoyée de Rome vers l'an 1000, au duc de Pologne qui
s'était fait baptiser et se voulait faire déclarer roi, fut enlevée par
Étienne, duc de Hongrie, qui en prit le titre de roi. Il fut reconnu
saint dans la suite, et la vénération de cette couronne a passé jusqu'à
la superstition parmi les Hongrois.

Les fanatiques du Languedoc et des Cévennes occupaient des troupes qui
en écharpaient quelques pelotons de temps en temps, mais qui ne
faisaient pas grand mal au gros. On surprit des Hollandais qui leur
portaient de l'argent et des armes avec de grandes promesses de secours.
Genève les soutenait aussi de tout ce qu'elle pouvait sourdement, et les
fournissait de prédicants. Le plus embarrassant était leurs
intelligences dans le pays même. Rochegude, gentilhomme de dix à douze
mille livres de rente, fut entre autres arrêté, accusé par un officier
Hollandais qui fut pris, et qui, pour n'être point pendu, le décela et
promit de découvrir beaucoup d'autres choses. C'était à Rochegude que
lui et ses camarades avaient ordre de s'adresser, quand ils auraient
besoin d'argent, d'armes et de vivres, et il y avait plusieurs gens
distingués dans ce pays-là, qui ne donnaient aucun soupçon, et qui se
trouvèrent des plus avant dans cette révolte.

\hypertarget{chapitre-xiii.}{%
\chapter{CHAPITRE XIII.}\label{chapitre-xiii.}}

1704

~

{\textsc{Année 1704.}} {\textsc{- Duchesse de Nemours rappelée.}}
{\textsc{- Mariage de Nangis et de M\textsuperscript{lle} de La
Hoguette.}} {\textsc{- Mariage du vidame d'Amiens et de
M\textsuperscript{lle} de Lavardin.}} {\textsc{- Visite du roi, de la
reine et des filles de France, etc.\,; époque de leur cessation.}}
{\textsc{- Deuils d'enfants et leur cause.}} {\textsc{- Messages ou
envois.}} {\textsc{- Réception d'un valet de pied envoyé par le roi au
duc de Montbazon.}} {\textsc{- Comte d'Ayen duc par démission de son
père.}} {\textsc{- Mort de Sainte-Mesme.}} {\textsc{- Mort du baron de
Bressé.}} {\textsc{- Mort de M\textsuperscript{me} de Boisdauphin.}}
{\textsc{- Mort de Termes et sa cruelle aventure.}} {\textsc{- Mort de
l'infante de Portugal.}} {\textsc{- Tessé en Italie\,; sa bassesse.}}
{\textsc{- Petit combat en Italie.}} {\textsc{- Conduite de Vendôme.}}
{\textsc{- Flatterie artificieuse de Vaudémont.}} {\textsc{- Autre
action en Italie.}} {\textsc{- Tessé en Savoie.}} {\textsc{- La
Feuillade en Dauphiné, fait lieutenant général seul.}} {\textsc{- Grand
prieur général d'armée.}} {\textsc{- Le fils unique de Vaudémont
feld-maréchal des armées de l'empereur.}} {\textsc{- Maréchal de
Villeroy et la marquise de Bedmar à Versailles.}} {\textsc{- Grande
sévérité du conseil de guerre de Vienne.}} {\textsc{- Progrès des
mécontents de Hongrie.}} {\textsc{- Villeroy en Flandre.}} {\textsc{-
Baron Pallavicin.}} {\textsc{- Mariage du fils aîné de Tallard avec la
fille unique de Verdun.}} {\textsc{- Tallard sur le Rhin\,; Coigny sur
la Moselle.}} {\textsc{- Deux cent mille livres d'augmentation de brevet
de retenue au maréchal de Boufflers sur sa charge, qui ne sert point.}}
{\textsc{- Adoration de la croix ôtée aux ducs.}} {\textsc{- Mort du duc
d'Aumont\,; sa dépouille.}} {\textsc{- Mort du cardinal Norris.}}
{\textsc{- Mort de M\textsuperscript{me} de Lyonne\,; ses enfants.}}
{\textsc{- Mort et deuil d'un fils de l'électeur de Bavière.}}
{\textsc{- Duchesse de Ventadour gouvernante survivancière des enfants
de France.}} {\textsc{- Maréchal de Châteaurenauld lieutenant général de
Bretagne.}} {\textsc{- Walstein mis en liberté.}} {\textsc{- Phélypeaux
et Vernon échangés.}} {\textsc{- Mort d'Harlay, conseiller d'État.}}
{\textsc{- Mort de Cohorn.}} {\textsc{- Villars en Languedoc et
Montrevel en Guyenne.}} {\textsc{- On me fait une opération pour une
saignée.}} {\textsc{- Chamillart m'avait raccommodé avec le roi\,;
Maréchal achève.}} {\textsc{- Avidité mal reçue du comte de Marsan.}}
{\textsc{- Mort du célèbre Bossuet, évêque de Meaux, et du cardinal de
Fürstemberg\,; leur dépouille.}}

~

Cette année commença par un acte de bonté du roi, dont il est vrai qu'il
aurait pu s'épargner la matière. Puysieux, ambassadeur en Suisse, avait
son frère le chevalier de Sillery attaché de toute sa vie au prince de
Conti plus de cœur encore que d'emploi. Il était son premier écuyer, et
intimement avec son frère. La conduite de M\textsuperscript{me} de
Nemours, de ses gens d'affaires et de ses partisans à Neuchâtel, avait
fort embarrassé les vues et les démarches de ce prince, et souvent
déconcerté tous ses projets. Il était ardent sur cette affaire, dont ses
envieux lui reprochaient que la richesse lui tenait bien plus au cœur
que n'avait fait la couronne de Pologne. Puysieux le servit autant, et
plus même que ne lui permettait son caractère, et l'impartialité du roi
entre les prétendants. Il n'y en avait aucun de plus opposé au prince de
Conti, ni de plus aimé et autorisé à Neuchâtel, que
M\textsuperscript{me} de Nemours, qui possédait ce petit État depuis si
longtemps, et qui en voulait disposer en faveur de ce bâtard de Soissons
qu'elle avait déclaré son héritier, et de ses filles. Elle fut desservie
auprès du roi, et Puysieux l'eut beau à la donner comme peu mesurée avec
un prince du sang, et trop altière sur l'exécution des ordres du roi
dans sa conduite, si bien qu'enfin elle fut exilée en sa maison de
Coulommiers. Elle en reçut l'ordre et l'exécuta sans se plaindre, avec
une fermeté qui tint encore plus de la hauteur, et, de ce lieu, agit
dans ses affaires avec la même vivacité et aussi peu de mesure contre le
prince de Conti, sans qu'il lui échappât ni plainte, ni reproche, ni
excuse, ni le moindre désir de se voir en liberté. À la fin, on eut
honte de cette violence qui durait depuis trois ans sur une princesse de
plus de quatre-vingts ans, et pour des affaires de son patrimoine. Elle
fut exilée sans l'avoir mérité, elle fut rappelée sans l'avoir demandé.
Elle vit le roi deux mois après, qui lui fit des honnêtetés, et presque
des excuses.

Nangis, le favori des danses, épousa, dans les premiers jours de cette
année, une riche héritière, fille du frère de l'archevêque de Sens, La
Hoguette.

En même temps il s'en fit un autre qui surprit un peu le monde\,: ce fut
celui du vidame d'Amiens, second fils du duc de Chevreuse, avec l'aînée
des deux filles que le marquis de Lavardin avait laissées de son second
mariage avec la sœur du duc et du cardinal de Noailles, laquelle était
morte devant lui. Ces filles, d'un nom illustre mais éteint, étaient
riches par la mort de leur frère, tué, comme on l'a vu, à la bataille de
Spire. Elles étaient sous la tutelle des Noailles qui seuls pouvaient
disposer d'elles. Le duc de Noailles avait, depuis longues années, de
ces procès piquants avec M. de Bouillon pour la mouvance de ses terres
du vicomté de Turenne. Ils avaient pris toutes sortes de formes dans
cette longue durée et pour les tribunaux et pour la conciliation. M. de
Chevreuse s'en était fort mêlé, et les choses semblaient fort adoucies,
lorsque depuis peu M. de Bouillon fit envoyer des troupes dans cette
vicomté pour y châtier une révolte de plusieurs vassaux contre lui,
qu'il publia excités et protégés par M. de Noailles. L'éclat entre eux
se renouvela. M. de Noailles en fut peiné\,; M. de Chevreuse s'entremit
encore, et on prétendit que les Noailles se hâtèrent de proposer et de
brusquer ce mariage pour gagner M. de Chevreuse, et sortir d'affaires
par son moyen. Le vidame avait père et mère et un frère aîné qui avait
des enfants, force dettes du père et du frère, et la succession du duc
de Chaulnes, qui le regardait après M. de Chevreuse, fort obérée. On ne
lit point dans l'avenir, et personne n'imaginait alors que ce cadet
vidame aurait la charge de son père, serait fait duc et pair, et
deviendrait maréchal de France.

Il faut ici placer l'époque de la cessation des visites de
M\textsuperscript{me} la duchesse d'Orléans aux dames non titrées, et
reprendre cette matière de plus haut. Jusqu'en 1678 la reine allait voir
les duchesses à leur mariage, à leurs couches, à la mort des parents
dont elles drapaient. Le roi avait cessé de venir exprès à Paris
quelques années auparavant, et les avait toujours visitées jusque-là,
même les ducs. Il haïssait le duc de Lesdiguières, de l'orgueil duquel
il était choqué. C'était un seigneur qui, par soi et par l'héritière de
Retz qu'il avait épousée, se trouvait des biens immenses, qui dépensait
plus qu'à proportion, et qui, avec le gouvernement de Dauphiné où il
était adoré et qu'il avait eu après ses pères, depuis le connétable de
Lesdiguières, faisait sa cour comme autrefois et non comme le roi
voulait qu'on la lui fît. Avec une brillante valeur, des talents pour la
guerre, et ceux encore d'y plaire, il avait capté les troupes. Avec
moins de vent et plus de réflexion, c'eût été un homme en tout temps
dans un royaume. Il n'était pas moins considéré à la cour, et à la mode
parmi les dames et dans le monde. Il mourut à trente-six ans, en mai
1681, d'une pleurésie qu'il prit pour avoir bu à la glace au sortir
d'une partie de paume, à Saint-Germain. Le roi, qui pourtant envoya de
Versailles savoir de ses nouvelles, car cela était encore alors sur ce
pied-là, ne put cacher son soulagement de cette mort. Il ne laissa qu'un
fils unique, né en octobre 1678, que nous avons vu en son temps épouser
une fille de M. de Duras, mourir sans enfants ensuite, et laisser sa
dignité au vieux Canaples, en qui enfin elle s'éteignit.
M\textsuperscript{me} de Lesdiguières était une manière de fée qui
dédaignait tous les devoirs, qui par conséquent était peu aimée et qui
se consola aisément d'un mari qui ne vivait pas uniquement pour elle,
qui forçait son humeur impérieuse et particulière par une maison
toujours ouverte, et qui la laissait maîtresse de tout dans la plus
grande opulence.

Ce fut donc par elle que le roi commença à retrancher aux duchesses, et
en même temps aux princesses étrangères, les visites de la reine.
Quelque soumise qu'elle fût en tout au roi, quelque soigneuse qu'elle
fût de lui plaire, quelque pure que fût sa vertu, sans jamais avoir
donné lieu au plus léger soupçon, quelque incapable que fût d'ailleurs
son génie doux et borné de donner la moindre inquiétude, le roi ne
laissait pas de s'importuner de son attachement pour les Carmélites de
la rue du Bouloi où elle venait souvent. Ces filles en étaient devenues
importantes. Il se trouva des femmes qui, faute de mieux, s'intriguèrent
avec elles et y voyaient la reine. Il y en eut même tout à fait de la
cour. Le roi voulut rendre ces visites plus rares pour rompre peu à peu
ce commerce. Le prétexte des visites à faire aux occasions servait à se
rabattre aux Carmélites. Tout cela, joint avec ce goût inspiré par les
ministres d'abaisser tout, fit de ce tout ensemble une occasion qui
attira cette décision du roi que la reine ne visiterait plus que les
princesses du sang.

Sur cet exemple, M\textsuperscript{me} la Dauphine qui a passé les dix
années qu'elle a vécu en France, grosse, en couche ou malade de la
longue maladie dont elle mourut en 1690, ne sortit point de Versailles
et ne visita point\,; et, de l'un à l'autre, Madame, farouche et
particulière, avec sa couche de gloire, n'en voulut pas faire plus que
M\textsuperscript{me} la Dauphine\,; de là M\textsuperscript{me} la
duchesse de Bourgogne en usa de même, puis M\textsuperscript{me} la
duchesse de Berry. Monseigneur cessa aussi comme le roi de faire des
visites\,; mais Monsieur n'y manquait point à Versailles et à Paris, et
les trois fils de Monseigneur à Versailles seulement, mais sans aller à
Paris. Ils allaient même quelquefois chez des dames non titrées, mais
fort rarement et par une distinction très marquée. Pour les
petites-filles de France, elles allaient non seulement chez les dames
titrées en toutes occasions, mais aussi chez toutes les dames de
qualité. Les trois filles de Gaston n'y ont jamais manqué. Mademoiselle,
sous prétexte de ne faire de visites qu'avec Madame, n'alla point, mais
M\textsuperscript{me} la duchesse de Chartres puis d'Orléans alla
partout. Elle continua longtemps encore après la mort de Monsieur\,;
puis, sous prétexte d'incommodité, après de paresse, et que ces visites
ne finissaient point, elle se rendit plus rare chez les femmes non
titrées, et finalement se laissa entendre à ces mariages du marquis de
Roye, de Nangis et du vidame, qu'elle n'irait plus chez pas une que chez
celles à qui, par amitié seulement et non plus par un devoir qui la
fatiguait, elle voudrait bien faire cette distinction. On s'en plaignit
et ce fut tout. On voulait plaire, aller à Marly, et par conséquent ne
pas se brouiller avec elle, quoiqu'à dire vrai elle n'influât en rien.
Mais telle est la misère du monde. Le roi mort et M. le duc d'Orléans
régent, il se défit de tous devoirs et de toutes visites sous prétexte
qu'il n'en avait pas le temps, et M\textsuperscript{me} sa femme se
laissa entendre qu'elle ne visiterait plus que les princesses du sang.
Ainsi elle fit comme la reine, et comme M. le duc d'Orléans était alors
roi pour longtemps, dans le bas âge du véritable, cela passa sans que
personne osât souffler. Tels ont été les progrès sur les visites. Tout
ce qui en est resté sont celles des princes et des princesses du sang,
que les prétextes de Marly et d'autres absences retranchent tant
qu'elles peuvent. Mais quelques usurpations qu'elles aient faites en
tout genre, elles n'en sont pas venues encore, en 1741, à déclarer
qu'elles ne visiteraient plus même les femmes non titrées.

Il faut dire tout de suite que, dans les premiers jours de cette année,
M. le prince de Conti perdit son second fils à l'âge de sept mois. On
n'avait point porté le deuil des enfants du roi et de la reine, ni de
ceux de Monsieur, morts en nombre jusqu'à l'âge de sept ans, ni fait de
compliment sur ces pertes. Le désir de relever les bâtards avait fait
porter le deuil d'un maillot de M. du Maine et lui faire des
compliments. Il n'y eut donc pas moyen de l'éviter pour celui du prince
de Conti. Au lieu d'un gentilhomme ordinaire que le roi envoyait
toujours aux princes du sang, il envoya un maître de sa garde-robe à M.
le Prince, qui le devait avoir depuis qu'à la mort de Monsieur il avait
eu les honneurs de premier prince du sang, et à M. le prince de Conti
qui, simple prince du sang, ne devait avoir qu'un gentilhomme ordinaire.
Cela fut fait pour les bâtards, à qui, dans les occasions, le roi envoya
comme aux princes du sang un maître de sa garde-robe, et bien que dans
la suite cela ne se fît pas toujours, il fut rare que les uns et les
autres n'eussent pas le message d'un maître de la garde-robe.

Aux mêmes occasions où la reine visitait, et aux personnes qu'elle
visitait, même aux ducs et aux princes étrangers qu'elle ne visitait
pas, le roi envoie jusqu'à aujourd'hui un gentilhomme ordinaire\,; on
lui présente un fauteuil, on l'invite à s'y asseoir et à se couvrir\,;
on lui donne la main, on le conduit au carrosse, et les duchesses au
milieu de leur seconde pièce. La reine et les deux Dauphines envoyèrent
un de leurs maîtres d'hôtel\,; celui de la reine était traité comme
gentilhomme ordinaire, celui des Dauphines sans descendre le degré. Je
ne sais qui a avisé cette reine-ci\footnote{Marie Leczinska, fille de
  Stanislas Leczinski, roi détrôné de Pologne, mariée à Louis XV le 15
  août 1725.} de n'envoyer qu'un page\,; ce n'est pas qu'elle soit plus
reine que l'épouse de Louis XIV, ni qu'elle soit tout à fait de si bonne
maison. Ce page aussi est reçu et traité fort médiocrement. Monseigneur
et les trois princes ses fils, un écuyer\,; car ces trois derniers ne
visitaient qu'à la cour, et ne venaient point à Paris.

J'ai ouï conter au feu roi qu'étant encore fort jeune, mais majeur, il
avait écrit à M. de Montbazon par un de ses valets de pied. M. de
Montbazon était grand veneur et gouverneur de Paris, où il y avait lors
bien des affaires dont ce duc se mêlait. Le valet de pied, parti de
Saint-Germain, ne le trouva point à Paris et l'alla chercher à Couperay
où il était. M. de Montbazon s'allait mettre à table. Il reçut la
lettre, y répondit, la donna au valet de pied qui lui fit la révérence
pour s'en retourner. «\,Non pas cela, lui dit le duc de Montbazon, vous
êtes venu de la part du roi, vous me ferez l'honneur de dîner avec
moi\,;» le prit par la main et le mena dans la salle, le faisant passer
devant lui aux portes. Ce valet de pied confondu et qui ne s'attendait à
rien moins, se fit tirer d'abord, puis tout éperdu se laissa faire et
mettre à la belle place. Il y avait force compagnie à dîner, ce que le
roi n'oublia pas, et toujours le valet de pied servi de tout le premier
par le duc de Montbazon. Il but à la santé du roi, et pria le valet de
pied de lui dire qu'il avait pris cette liberté avec toute la compagnie.
Au sortir de table, il mena le valet de pied sur le perron, et n'en
partit point qu'il ne l'eût vu monter à cheval. «\,Cela s'appelle savoir
vivre,\,» ajouta le roi. Il a fait ce conte couvent, et toujours avec
complaisance, et, je pense, pour instruire les gens de ce qui lui était
dû, et de quelle sorte les seigneurs anciens savaient en faire leur
devoir.

Le duc de Noailles, au commencement de cette année, obtint enfin le
consentement de M\textsuperscript{me} de Maintenon pour céder son duché
à son fils, le comte d'Ayen, qui prit le nom de duc de Noailles et le
père celui de maréchal. M\textsuperscript{me} de Maintenon ne voulut
jamais que sa nièce fût assise en se mariant, et lui fit acheter son
tabouret par le délai de quelques années. Elle avait de ces modesties
qui sentaient fort le relan de son premier état, mais qui pourtant ne
passaient pas l'épiderme.

Sainte-Mesme, d'une branche séparée de celle des maréchaux de L'Hôpital
et de Vitry, mourut en ce commencement d'année. Je le remarque par la
grande réputation qu'il s'était acquise parmi tous les savants de
l'Europe\,; grand géomètre, profond en algèbre et dans toutes les
parties des mathématiques\,; ami intime, et d'abord disciple du P.
Malebranche, et si connu lui-même par son livre des \emph{Infiniment
petits}. Sa mauvaise vue et son goût dominant pour ces sciences
abstraites l'avaient retiré de bonne heure de la guerre et pour ainsi
dire du monde.

En même temps mourut le baron de Bressé à Paris, celui même dont j'ai
parlé sur le siège de Namur\,; il était fort vieux et cassé, et avait du
roi autour de vingt mille livres de rente, et lieutenant général.

M\textsuperscript{me} de Boisdauphin mourut aussi à Paris à
quatre-vingts ans. Elle était sœur de Barentin, président au grand
conseil, et fort riche héritière. Elle avait épousé en premières noces
M. de Courtenvaux, premier gentilhomme de la chambre, fils du maréchal
de Souvré, gouverneur de Louis XIII, dont elle n'avait eu que
M\textsuperscript{me} de Louvois, et elle était veuve en secondes noces,
sans enfants, du frère aîné de M. de Laval, père de la maréchale de
Rochefort. M. de Louvois, toute sa vie, avait eu une grande
considération pour elle, et ses enfants après lui\,: c'était une femme
aussi qui savait se faire rendre.

Termes mourut aussi presque en même temps. M. de Montespan et lui
étaient enfants des deux frères. Il était pauvre, avait été fort bien
fait, et très bien avec les dames en sa jeunesse\,; je ne sais par quel
accident il avait un palais d'argent qui lui rendait la parole fort
étrange\,; mais ce qui surprenait c'est qu'il n'y paraissait plus dès
qu'il chantait avec la plus belle voix du monde. Il avait beaucoup
d'esprit et fort orné, avait peu servi et avait bonne réputation pour le
courage. Sans avoir bougé de la cour, à peine y put-il obtenir une très
petite subsistance. Je pense que le mépris qu'il s'y attira l'y perdit.
Il eut la bassesse de vouloir être premier valet de chambre, et personne
ne doutait qu'il ne rapportât tout au roi, tellement qu'il n'était reçu
dans aucune maison, ni abordé de personne. Il était poli et accostant,
mais à peine lui répondait-on en fuyant, tellement qu'il vivait dans une
solitude entière au milieu du plus grand monde. Le roi lui parlait
quelquefois, et lui permettait d'être à Marly dans le salon et à ses
promenades dans ses jardins tous les voyages, sans demander, mais aussi
sans avoir jamais de logement\,: il louait une chambre au village. Il
reçut une fois à Versailles une grêle de bastonnade de quatre ou cinq
Suisses qui l'attendaient sortant de chez M. le Grand, à une heure après
minuit, et l'accompagnèrent, toujours frappant, tout le long de la
galerie. Il en fut moulu et plusieurs jours au lit. Il eut beau s'en
plaindre et le roi se fâcher, les auteurs se trouvèrent sitôt qu'ils ne
se trouvèrent plus. Quelques jours auparavant, M. lé Duc et M. le prince
de Conti avaient fait un souper chez Langlée, à Paris, après lequel il
s'était passé des choses assez étranges. Le roi leur en lava la tête\,;
ils crurent bien être assurés d'en avoir l'obligation à Termes, et le
firent régaler comme je viens de dire, incontinent après. Cela fit un
grand vacarme\,; mais on n'en fit que rire, et le roi fit semblant
d'ignorer les auteurs. Il était vieux, brouillé avec sa femme, qui était
fort peu de chose, et ne laissa qu'une fille religieuse, et un frère
obscur, connu de personne et qui ne se maria point.

L'infante aînée de Portugal mourut bientôt après. Elle avait huit ans,
et, nonobstant ce peu d'âge, on avait flatté la cour de Lisbonne que
l'archiduc l'épouserait.

Tessé, qui n'avait servi que de chausse-pied en Dauphiné à La Feuillade,
l'y avait bientôt laissé en chef et s'en était allé à Milan. Il prévit
en habile et bas courtisan que M. du Maine et M\textsuperscript{me} de
Maintenon l'emporteraient tôt ou tard sur la fermeté que le roi lui
avait marquée en prenant ses derniers ordres contre le désir des
bâtards, et leur compétence à établir avec les maréchaux de France\,; il
prévit de plus que, quoi qu'il pût arriver, cette protection pour lui
était plus solide que le plaisir de prendre le commandement sur M. de
Vendôme. Il n'en voulut pas perdre l'occasion\,: il prit celle d'une
apparence d'action, s'en alla en poste seul et en carabin joindre M. de
Vendôme\,; mit dans sa poche sa commission pour commander l'armée et M.
de Vendôme même, et ne prétendit qu'à l'état de volontaire. Vendôme ne
lui fit pas la moindre civilité d'aucune déférence, et continua en sa
présence à donner l'ordre et à commander, comme si Tessé n'y eût pas
été. C'était bien connaître le roi et le crédit de son intérieur, que
d'en user ainsi après ce qu'il lui avait si positivement ordonné au
contraire, et en même temps faire peu de cas de son bâton et de soi, en
comparaison de sa fortune, que toutefois, au point où il était arrivé,
il pouvait trouver être faite.

Peu de jours après, M. de Vendôme battit une partie de l'arrière-garde
du comte de Staremberg, général des Impériaux\,: quatre cents hommes
tués, cinq cents prisonniers, trois chariots remplis de pain firent du
bruit à Versailles. M. de Vendôme assaisonna cette nouvelle de la
promesse d'attaquer les ennemis le lendemain. Il savait bien qu'il n'en
ferait rien. Ses courriers étaient sans nombre, ou pour des bagatelles
qu'il faisait valoir et qui trouvaient des prôneurs, ou pour des
assurances de choses qui ne s'exécutaient point et qui trouvaient leurs
excuses dans les mêmes personnes, et le roi s'en laissait persuader. M.
de Vaudemont écrivit de Milan au roi sur cette bagatelle une
félicitation, comme assuré que ses ennemis seraient incontinent chassés
d'Italie. C'était la même cabale et les mêmes applaudissements\,: tout
cela s'avalait et réussissait à merveille. Mais pour cette fois, M. de
Vendôme fit encore quelque chose\,: il culbuta huit cents chevaux et six
bataillons de l'arrière-garde de Staremberg dans l'Orba. Bezons et
Saint-Frémont, à la tête de notre cavalerie, et Albergotti avec quinze
cents grenadiers, firent cette expédition. Elle ne fut pas sans perte et
beaucoup de blessés. Il en coûta mille hommes aux Impériaux, tués ou
pris. Solari, qui commandait ceux-ci, tué, et le prince de Lichtenstein
pris fort blessé.

Tessé s'en était retourné à Pavie, d'où il regagna Milan, et au
commencement de février s'en retourna commander en Savoie. En même temps
La Feuillade fut fait lieutenant général seul, demeura en son
gouvernement de Dauphine, et fut destiné pour l'armée de M. de Vendôme.
Ainsi maréchal de camp tout d'un coup, en chef en Dauphiné aussitôt
après, et sans presque aucun intervalle lieutenant général, c'est le
train que Chamillart mena un homme pour qui le roi lui avait déclaré
qu'il ne ferait jamais rien. Tout de suite le grand prieur, si mal avec
le roi et qui avait eu tant de peine à servir, puis à aller avec son
frère, fut envoyé commander les troupes dans le Mantouan et le Milanais,
et incontinent après eut une petite armée avec le nom, la patente, les
appointements et le service de général d'armée en chef, séparément de M.
de Vendôme, avec qui il fut comme sont deux maréchaux de France, qui ont
chacun une armée à part dans les mêmes pays, qui se concertent, mais
dont l'ancien des deux conserve la supériorité sur l'autre. En même
temps le fils unique de Vaudemont fut fait feld-maréchal par l'empereur,
avec Staremberg, Heister et Rabutin, qui est, à l'égard du militaire, ce
que sont nos maréchaux de France\,: ainsi Vaudemont prospérait des deux
côtés, et le roi lui savait toujours le meilleur gré du monde.

Le maréchal de Villeroy, demeuré pour tout l'hiver à Bruxelles, vint à
la mi-janvier faire un tour à la cour, où le roi le reçut, après neuf
mois d'absence, avec des marques de faveur très distinguées. La marquise
de Bedmar, venant d'Espagne, s'y trouva en même temps, allant joindre
son mari en Flandre. La duchesse du Lude la présenta au roi dans son
cabinet, dont les portes demeurèrent ouvertes. La duchesse d'Albe et la
maréchale de Cœuvres, comme grandes d'Espagne, l'accompagnèrent. Le roi
la baisa et lui fit toutes sortes d'honnêtetés\,; il lui dit qu'il avait
résolu de faire son mari chevalier de l'ordre. M\textsuperscript{me} la
duchesse de Bourgogne la baisa chez elle, où ce même cortège se trouva.
On ne s'assit point au souper. La marquise de Bedmar, comme grande
d'Espagne, prit son tabouret, et après le souper congé du roi, qui, en
passant pour entrer dans son cabinet, lui fit encore des merveilles, et
lui dit qu'il avait ordonné dans toutes les places par lesquelles elle
passerait qu'on l'y reçût avec les mêmes honneurs que dans celles de la
Flandre espagnole.

Le conseil de guerre de Vienne donna, vers ces temps-ci, un grand
exemple de sévérité. Par son jugement, le comte d'Arco eut la tête
coupée, pour avoir mal défendu Brisach avec Marcilly, à qui le bourreau
cassa l'épée et lui en donna plusieurs coups sur la tête\,; le
lieutenant de roi, comme nous parlons en France, et le major de la place
furent dégradés des armes. La mauvaise humeur des progrès des mécontents
put un peu contribuer à cette sévérité, qui fit beaucoup murmurer les
officiers impériaux.

Ces mécontents inquiétaient l'empereur jusque dans Vienne, dans les
faubourgs duquel ils avaient osé aller prendre des bateaux pour passer
dans l'île de Schutt, en sorte que le prince Eugène fut obligé de faire
faire des redoutes le long du Danube\,; ils ne laissèrent pas de piller
un autre faubourg de cette capitale. Ils s'emparèrent d'Agria, des
quatre villes des montagnes où sont les mines, de quelques autres jusque
auprès de Presbourg, qui n'est qu'à dix lieues de Vienne, se firent voir
dans l'Autriche, la Silésie et la Moravie, et refusèrent les
propositions qui leur furent faites par le comte Palfi de la part de
l'empereur. Strigonie, autrement Gran, se soumit à eux avec presque
toute sa garnison. Ils coupèrent la communication de la Bohème à Vienne,
et le prince Eugène, ne se croyant plus en sûreté à Presbourg, se retira
à Vienne. Ils pillèrent une île du Danube, que l'empereur avait donnée à
ce prince, prirent ses équipages et ravagèrent toute la grande île de
Schutt. Ils se divisèrent en plusieurs corps qui prirent la forteresse
de Mongatz et Hermanstadt, capitale de la Transylvanie, s'établirent en
divers postes de Moravie et de Styrie, prirent Canise, firent des
courses jusqu'à Gratz, capitale de Styrie, et obligèrent le général
Heister de se retirer sous Vienne avec cinq mille hommes qu'il
commandait. Ils brûlèrent les environs de cette demeure impériale, d'où
on voyait les feux et d'où on ne pouvait sortir ni entrer librement,
faute de troupes pour les écarter, et où la consternation fut d'autant
plus grande, que l'envoyé de hollande à Vienne s'employa inutilement
auprès d'eux, et qu'ils rejetèrent les propositions qu'il leur fit de la
part de l'empereur.

Le maréchal de Villeroy s'en retourna à Bruxelles après quelque séjour à
la cour\,; il s'y prit d'affection pour le baron Pallavicin, dont il fit
bientôt après son homme de confiance dans son armée, où il alla servir.
Ce baron était un grand homme très bien fait, de trente-cinq ans ou
environ, point marié, et de beaucoup d'esprit, de valeur et de talents
pour la guerre et pour l'intrigue, dont on n'a jamais bien démêlé
l'histoire. Il avait été fort bien avec M. de Savoie, dont son père
était grand écuyer, et sa mère dame d'honneur d'une des deux duchesses.
Il fut arrêté avec les troupes de ce prince et donna sa parole. M. de
Savoie lui manda de revenir en Piémont, il s'en excusa sur la parole
qu'il avait donnée. M. de Savoie lui écrivit que, s'il ne revenait, il
s'attirerait son indignation. Là-dessus Pallavicin abandonna le service
de Savoie et se donna à celui de France, sans qu'on ait jamais pu savoir
la cause du procédé du maître ni du sujet. Il eut deux mille écus de
pension en arrivant. Le maréchal de Villeroy, qui aimait les étrangers
et les aventuriers, s'infatua de celui-ci qui devint son homme de
confiance dans la suite, à la cour comme à l'armée, où cette faveur du
général excita beaucoup de jalousie.

Le maréchal de Tallard s'en alla en Forez marier son fils aîné à la
fille unique de Verdun, très riche héritière et qui en avait aussi
l'humeur et la figure. Tallard et Verdun étaient enfants des deux frères
et avaient ensemble des procès à se ruiner que ce mariage termina.
Verdun était un homme de beaucoup d'esprit, mais singulier, qui n'avait
jamais guère servi ni vu de monde qu'à son point et à sa manière, et qui
n'avait jamais fait grand cas de son cousin Tallard, ni guère aussi de
la cour et de la fortune. Tallard partit bientôt après vers le Rhin et
Coigny sur la Moselle, commander un corps comme faisait ordinairement M.
d'Harcourt. Le maréchal de Boufflers ne servit point cette année, le roi
tâcha de l'en consoler par une augmentation de deux cent mille livres de
brevet de retenue sur sa charge.

J'étais allé passer la semaine sainte à la Ferté et à la Trappe, d'où je
revins à Versailles le mercredi de Pâques. J'appris en arrivant le grand
parti que M. le Grand venait de tirer de la quête de sa fille. Le matin
du vendredi saint, il vint trouver le roi et lui demanda avec un
audacieux empressement d'aller avec ceux de sa maison à l'adoration de
la croix. Les ducs y allaient de tout temps en rang d'ancienneté après
le dernier prince du sang, et depuis peu d'années après les bâtards\,;
et après les ducs, les grands officiers de la maison du roi dans le rang
de leurs charges, sans qu'aucun prince étranger y eût jamais été admis.
Le roi, surpris de la demande, refusa et répondit que cela ne se
pouvait, parce que les ducs y allaient. C'est où le grand écuyer
l'attendait. Il demanda à les précéder, non qu'il l'espérât, mais pour
réussir à ce qui arriva. Le roi fut embarrassé. M. le Grand insista,
appuyé sur la faiblesse qu'il connaissait au roi pour lui, qui en sortit
par lui dire que ni ducs ni princes n'iraient. En donnant l'ordre, il
dit au maréchal de Noailles, capitaine des gardes en quartier, d'en
avertir les ducs, qui répondit mollement, en représentant leur droit
usité de tout temps. Le parti du roi était pris, et le peu que dit M. de
Noailles, et d'un ton à peu imposer, n'était pas pour le faire changer.
Il n'y avait presque aucun duc à Versailles, même des plus à portée du
roi, qui profitaient de ces jours de dévotions pour les leurs et pour
leurs affaires. M. de La Rochefoucauld montait en carrosse de chez le
cardinal de Coislin lorsqu'on lui vint dire cette nouveauté. Il se mit à
pester, et n'osa jamais aller trouver le roi. Il partit et alla ronger
son frein aux Basses-Loges de Saint-Germain, où il allait tous les ans à
pareil jour se retirer. Ainsi cette distinction fut perdue en échange de
celle que les princes étrangers s'étaient voulu faire de la quête, et
qui avait avorté, et personne n'alla plus depuis à l'adoration de la
croix que les princes du sang et les bâtards. Je m'en allai tout de
suite à Paris sur cette nouvelle, et je ne revins de plusieurs jours à
la cour.

Le duc d'Aumont mourut d'apoplexie le matin du mercredi saint.
Villequier, son fils aîné, qui était premier gentilhomme de la chambre
en survivance, eut le gouvernement de Boulogne et du pays boulonnais
qu'avait son père, et prit le nom de duc d'Aumont.

Le cardinal Norris, moine augustin, a laissé un si grand nom parmi les
savants que je ne veux pas omettre sa mort qui arriva en ce temps-ci. Il
était d'origine irlandaise\,; il y a encore de son nom en Irlande et en
Angleterre, et aujourd'hui encore l'amiral Norris fait parler de lui
avec les escadres anglaises. Ce docte cardinal fut des congrégations de
Rome les plus importantes, et il avait succédé au cardinal Casanata, si
célèbre par son savoir et par cette bibliothèque si nombreuse et si
recherchée qu'il avait assemblée, et qu'il donna à la Minerve dans la
place de bibliothécaire de l'Église. Il n'est pas de mon sujet de
m'étendre sur ce grand cardinal, il suffira ici de n'avoir pas oublié de
faire mention de lui.

M\textsuperscript{me} de Lyonne mourut quelques jours après à Paris\,:
elle était Payen, d'une famille de Paris, veuve de M. de Lyonne,
secrétaire d'État, mort en 1671, le plus grand ministre du règne de
Louis XIV. C'était une femme de beaucoup d'esprit, de hauteur, de
magnificence et de dépense, et qui se serait fait compter avec plus de
mesure et d'économie, mais elle avait tout mangé il y avait longtemps,
et vivait dans la dernière indigence dans sa même hauteur, et l'apparent
mépris de tout, mais à la fin dans la piété depuis plusieurs années. Sa
fille avait été première femme du duc d'Estrées, fils de l'ambassadeur à
Rome. De ses trois fils, l'aîné survivancier de son père perdit avec lui
la charge de secrétaire d'État qui fut donnée à Pomponne, et il eut une
charge de maître de garde-robe dont il ne fit pas deux années de
fonctions quoiqu'il l'ait gardée longtemps. C'était un homme qui avait
très mal fait ses affaires, qui vivait très singulièrement et
obscurément, et qui passait sa vie à présider aux nouvellistes des
Tuileries. Il n'eut qu'un fils fort bien fait et distingué à la guerre,
mais qui se perdit par son mariage avec la servante d'un cabaret de
Phalsbourg, dont il n'eut point d'enfants, et qu'il voulut faire casser
dans la suite sans y avoir pu réussir. Elle l'a survécu et le survit
encore, retirée dans une communauté à Paris\,; et elle a toujours mené
une vie très sage, et qui l'a fait estimer. On verra en leur temps les
deux autres fils de M. et M\textsuperscript{me} de Lyonne, l'un riche
abbé débauché, l'autre évêque de Rosalie \emph{in partibus} et
missionnaire à Siam et à la Chine. Je ne parle pas d'un quatrième,
chevalier de Malte, qui n'a point paru\,; et voilà ce que deviennent les
familles des ministres\,! Celles des derniers de Louis XIV ont été plus
heureuses, les Tellier, les Colbert, les Chamillart, les Desmarets
surtout à bien surprendre.

L'électeur de Bavière perdit aussi un de ses fils. Le roi, pour le
gratifier, en prit le deuil pour quinze jours. Il avait l'honneur d'être
beau-frère de Monseigneur, mais sa parenté avec le roi était fort
éloignée.

On a vu comment la duchesse de Ventadour s'était mise à Madame pour
échapper à son mari et au couvent, la figure qu'elle fit auprès d'elle,
et les vues qui la lui firent quitter. Son plus que très intime ami dès
leur jeunesse, le maréchal de Villeroy, travaillait depuis longtemps à
leur succès auprès de M\textsuperscript{me} de Maintenon, avec qui il
fut toujours très bien, et qui, par raison de ressemblance, aimait bien
mieux les repenties que celles qui n'avaient pas fait de quoi se
repentir. M\textsuperscript{me} de Ventadour, dont l'âge avait dépassé
de beaucoup celui de la galanterie, s'était faite dévote depuis quelque
temps, et quoiqu'elle alliât ses anciens plus qu'amis, un gros jeu et
continuel, et bien d'autres choses avec sa dévotion, la coiffe, la
paroisse, la chapelle, l'assiduité aux offices et des jargons de
dévotion à propos, l'avaient lavée de toute tache, et les maux que ces
taches lui avaient causés ne parurent pas même un obstacle à la place de
gouvernante. Le roi dit donc un matin, à la fin de mars, à la maréchale
de La Mothe, qui par cette place lui faisait sa cour à ces heures-là
dans son cabinet, qu'il s'était trouvé si bien d'elle auprès de ses
enfants et auprès de ceux de Monseigneur, qu'il la destinait à ceux de
Mgr le duc de Bourgogne, mais qu'en même temps, pour ménager sa santé,
il lui adjoignait la duchesse de Ventadour, sa fille, pour survivancière
et pour la soulager dans les soins pénibles de cette charge. La
maréchale se trouva fort étourdie\,; elle aimait sa fille, mais non pas
jusqu'à se l'associer. On avait eu beau la tourner de toutes les façons,
jamais elle n'y avait voulu entendre. Elle disait qu'il était ridicule
de mettre auprès des enfants de France une femme qui n'avait jamais eu
d'enfants, et balbutiait pis entre ses dents, de telle sorte qu'allant
toujours à la parade elle leur fit prendre le parti de l'emporter à son
insu. Aussi parut-elle fort mécontente\,; la bonne femme craignait de
n'être plus maîtresse et de passer pour radoter, et ne se contraignit
pas sur son dépit aux compliments du monde, et beaucoup moins sur sa
fille, qu'elle reçut fort mal. Elle était à Paris, d'où elle arriva sur
cette nouvelle et entra par derrière dans ce cabinet de
M\textsuperscript{me} de Maintenon, où, tandis que le roi travaillait
dans la pièce joignante, elle présente, M\textsuperscript{me} la
duchesse de Bourgogne jouait avec des dames familières et les deux fils
de France, entrant quand elle voulait, mais seule, où était le roi.
M\textsuperscript{me} de Ventadour y arriva donc, si transportée, si
éperdue de joie, qu'oubliant ce qu'elle était, elle se jeta à genoux en
entrant et se traîna ainsi jusqu'à M\textsuperscript{me} la duchesse de
Bourgogne, qui alla l'embrasser et la relever. Elle en fit autant
lorsque, après les premiers compliments, cette princesse la mena où
était le roi, dont la surprise de cette action fut extrême\,; jamais
personne ne fut si hors de soi. Elle eut douze mille livres
d'augmentation de pension aux huit mille qu'elle avait déjà.

Le maréchal de Châteaurenauld eut bientôt après la lieutenance générale
de Bretagne, vacante depuis la mort de Lavardin, comme je l'ai dit
d'avance.

Le roi permit en même temps à Walstein, ambassadeur de l'empereur à
Lisbonne, pris sur mer en s'en retournant, de s'en aller, et fit partir
Vernon, ambassadeur de Savoie, toujours accompagné de son gentilhomme
ordinaire, pour aller sur la frontière de Provence et des États de
Savoie être échangé avec Phélypeaux.

En ce même temps mourut Harlay, conseiller d'État, qui avait été premier
ambassadeur plénipotentiaire à la paix de Ryswick, duquel j'ai assez
parlé précédemment pour n'avoir plus rien à en dire.

Les ennemis perdirent le meilleur des officiers Hollandais, qui de plus
était leur Vauban pour les places et les sièges, qui était le général
Cohorn, qui mourut à la Haye.

L'affaire des fanatiques ne finissait point et occupait des troupes. La
Hollande et M. de Savoie les soutenaient par des armes, de l'argent et
quelques hommes, et Genève par des prédicants. Villars, de retour de
Bavière, était oisif. Il avait été reçu comme s'il n'eût pas pris des
trésors, et qu'il n'eût pas empêché les progrès des armées pour les
amasser. M\textsuperscript{me} de Maintenon le protégeait ouvertement,
et conséquemment Chamillart, alors au plus haut point de la faveur. Ils
voulaient remettre Villars en selle, qui, profitant de ce qu'il pouvait
sur l'un et sur l'autre, voulait absolument être de quelque chose.
L'Allemagne ne lui convenait plus depuis qu'il s'était brouillé avec
l'électeur de Bavière\,; la Flandre et l'Italie étaient occupées par
Villeroy et Vendôme, plus en crédit que lui. Il ne se trouva que le
Languedoc à lui donner, pour le décorer au moins de finir cette petite
guerre. Montrevel n'avait que le roi pour lui, cela lui servit au moins
à ne pas demeurer par terre. On lui fit faire un troc désagréable. La
Guyenne était entièrement paisible et n'avait nul besoin de
commandant\,; Montrevel y fut envoyé avec le même pouvoir et les mêmes
appointements qu'il avait en Languedoc. Ce changement l'affligea fort,
mais il fallut céder et aller jouer au lansquenet à Bordeaux. Villars,
avec son effronterie ordinaire, voulant faire valoir le petit emploi où
il allait, dit assez plaisamment qu'on l'y envoyait comme un empirique
où les médecins ordinaires avaient perdu leur latin. Ce mot outra
Montrevel, qui fit si bien que, tandis que Villars était en chemin, il
battit deux fois les fanatiques et la dernière fois en personne et avec
un grand succès, et tout de suite s'en alla droit à Bordeaux, où il n'y
avait personne depuis que Sourdis n'y commandait plus.

Je tombai en ce temps-là dans un fâcheux accident. Je me fis saigner
parce que je sentais que le sang me portait à la tête, et il me sembla
l'avoir été fort bien. Je sentis la nuit une douleur au bras, que Le
Dran, fameux chirurgien, qui m'avait saigné, m'assura ne venir que d'une
ligature trop serrée. Pour le faire court, en deux jours le bras s'enfla
plus gros que la cuisse, avec la fièvre et de grandes douleurs\,; on me
tint autres deux jours avec des applications dessus pour dissiper le mal
par l'ouverture de la saignée, de l'avis des plus grands chirurgiens de
Paris. M. de Lauzun, qui me trouva avec raison fort mal, insista pour
avoir Maréchal, et s'en alla à Versailles le demander au roi, sans la
permission duquel il ne venait point à Paris, et il ne découchait
presque jamais du lieu où le roi était. Il eut permission de venir, de
découcher, et même de séjourner auprès de moi. En arrivant le matin, il
m'ouvrit le bras d'un bout à l'autre. Il était temps, l'abcès gagnait le
coffre, et se manifestait par de grands frissons. Il demeura deux jours
auprès de moi, vint après plusieurs jours de suite, puis de deux jours
l'un. L'adresse et la légèreté de l'opération, des pansements, et de me
mettre commodément passe l'imagination. Il prit prétexte de cet
accident, pour parler de moi au roi, qui après que je fus guéri
m'accabla de bontés. Chamillart était enfin venu à bout de me
raccommoder avec lui quelque temps auparavant. Tout ce que dit Maréchal
acheva. J'avais fait un léger effort du bras le jour de la saignée
auquel j'attribuais l'accident, et je voulus que Le Dran me saignât dans
le cours de cette opération pour ne le pas perdre. Maréchal et Fagon ne
doutèrent pas que le tendon n'eût été piqué. Par des poids qu'on me fit
porter, mon bras demeura dans sa longueur ordinaire, et je ne m'en suis
pas senti depuis. J'avais jour et nuit un des meilleurs chirurgiens de
Paris auprès de moi, qui se relevaient. Tribouleau, qui l'était des
gardes françaises avec beaucoup de réputation, me conta qu'il fallait
que M. de Marsan fût bien de mes amis, qu'il l'avait arrêté dans les
rues, qu'il lui avait demandé de mes nouvelles avec des détails et un
intérêt infini. La vérité était qu'il voulait mon gouvernement et qu'il
le demanda. Le roi lui demanda à son tour si je n'avais pas un fils, et
le rendit muet et confus. Chamillart, sans qu'on l'en eût prié, s'en
était assuré pour mon fils, en cas que je n'en revinsse pas, et n'y
avait pas perdu de temps. Je ne fis pas semblant dans la suite de savoir
le procédé de M. de Marsan, avec qui d'ailleurs, comme avec tous ces
Lorrains, je n'étais en aucun commerce.

L'Église et le siècle perdirent en ce même temps les deux prélats qui
fussent alors chacun à l'une et à l'autre avec le plus d'éclat, le
fameux Bossuet, évêque de Meaux, pour l'un, et le célèbre cardinal de
Fürstemberg, pour l'autre. Tous deux sont trop connus pour que j'aie à
rien dire de ces deux hommes si grandement et si diversement illustres,
le premier toujours à regretter, et qui le fut universellement, et dont
les grands travaux faisaient encore honte, dans cette vieillesse si
avancée, à l'âge moyen et robuste des évêques, des docteurs, et des
savants les plus instruits et les plus laborieux. L'autre, après avoir
si longtemps agité et intéressé toute l'Europe, était devenu depuis
longtemps un poids inutile à la terre. Chamillart eut la charge de
premier aumônier de M\textsuperscript{me} la duchesse de Bourgogne, pour
l'imbécile évêque de Senlis, son frère, et La Hoguette, archevêque de
Sens, la place de conseiller d'État d'Église. Bissy, évêque de Toul, se
laissa enfin persuader d'accepter Meaux. Un diocèse si près de Paris lui
parut plus propre à avancer sa fortune que ses querelles avec le duc de
Lorraine qui lui avaient suffisamment frayé le chemin à Rome\,; aussi
avait-il mieux aimé se tenir à Toul, qu'accepter Bordeaux. Mais il
espéra tout de Meaux qui, en le tenant sans cesse à portée, favoriserait
son savoir-faire qu'il ne fut pas longtemps à manifester.

\hypertarget{chapitre-xiv.}{%
\chapter{CHAPITRE XIV.}\label{chapitre-xiv.}}

1704

~

{\textsc{L'archiduc par l'Angleterre à Lisbonne\,; mal secouru.}}
{\textsc{- L'amirante de Castille tombé dans le mépris.}} {\textsc{-
Disgrâce de la princesse des Ursins, rappelée d'Espagne avec ordre de se
retirer droit en Italie\,; détails raccourcis de son gouvernement.}}
{\textsc{- Motifs qui firent passer Berwick en Espagne et Puységur.}}
{\textsc{- Négligence, impudence et crime d'Orry.}} {\textsc{- Joug
étrange de la princesse des Ursins sur l'abbé d'Estrées, et son plus que
surprenant abus.}} {\textsc{- Princesse des Ursins intercepte et
apostille de sa main une lettre de l'abbé d'Estrées au roi.}} {\textsc{-
Abbé d'Estrées obtient son rappel.}} {\textsc{- Abbé d'Estrées
commandeur de l'ordre sur l'exemple de l'abbé des Chastelliers\,; quel
était l'abbé des Chastelliers.}} {\textsc{- Cardinal d'Estrées abbé de
Saint-Germain des Prés.}} {\textsc{- Le roi d'Espagne à la tête de son
armée en Portugal.}} {\textsc{- Princesse des Ursins chassée\,; son
courage\,; ses mesures.}} {\textsc{- Son départ vers Bayonne.}}
{\textsc{- Duc de Grammont ambassadeur en Espagne\,; son caractère.}}
{\textsc{- Son misérable mariage.}} {\textsc{- Duc de Grammont déclare
son indigne mariage, et, par l'insensé raffinement d'en vouloir faire sa
cour, s'attire la colère du roi et de M\textsuperscript{me} de
Maintenon.}} {\textsc{- Princesse des Ursins insiste sur la permission
d'aller à Versailles.}} {\textsc{- Princesse des Ursins exilée à
Toulouse.}} {\textsc{- Des Pennes, confident de M\textsuperscript{me}
des Ursins, rappelé d'Espagne.}} {\textsc{- Orry rappelé d'Espagne.}}
{\textsc{- Folle prétention du connétable de Castille.}} {\textsc{-
Conduite du duc de l'Infantado.}} {\textsc{- Appointements du duc de
Grammont.}} {\textsc{- Franchise des ambassadeurs\,; abus qui s'en fait
à Venise par Charmont.}} {\textsc{- Plaintes de la république de
Venise\,; Charmont protégé.}}

~

L'archiduc, après un long séjour dans la basse Allemagne et la Hollande,
en attendant que tout frit prêt pour son trajet, avait essuyé une
terrible tempête qui le jeta deux fois en Angleterre, où la première
fois il vit la reine et ses ministres. Il était arrivé en Portugal avec
fort peu de secours\,; il trouva que tout lui manquait. Ce grand
contretemps et la fidélité des Espagnols ne répondait pas aux promesses
de l'amirante qui leur avait persuadé que tout se révolterait en
Espagne\,; et comme rien n'y branla, ni à l'arrivée de l'archiduc, ni
depuis, que deux ou trois particuliers au plus, mais bien longtemps dans
les suites, l'amirante tomba dans un discrédit total. Le Portugal,
abandonné presque à sa faiblesse, s'en prenait à lui de l'avoir comme
engagé dans ce péril, et l'archiduc d'avoir pressé son arrivée sur des
espérances dont il ne voyait aucun effet. Il se défendait sur l'espèce
d'abandon où ses alliés et l'empereur même le laissaient, qui
décourageait de lever le masque en sa faveur. Ces contrastes qui
laissèrent l'amirante sans ressources, tant du côté de la cour de
Portugal que de celle de l'archiduc, le mirent souvent en danger d'être
assommé par le peuple, et le firent tomber dans le dernier mépris.

J'ai différé l'événement suivant et quelques autres, pour raconter tout
de suite ce qui aurait été moins intelligible et moins agréable par
morceaux, à mesure que les diverses choses se sont passées, d'autant que
le principal de tous, et pour lequel j'ai différé les autres, ne dépasse
pas la fin de mai. Il faut se souvenir de ce qui a été rapporté
ci-devant de la brillante situation de la princesse des Ursins en
Espagne, et de ses solides appuis à Versailles, où elle avait trouvé
moyen de sevrer les ministres du roi du secret et du maniement des
affaires, qui se traitaient réciproquement d'elle à
M\textsuperscript{me} de Maintenon et au roi, le seul Harcourt, ennemi
de nos ministres, dans la confidence. M. de Beauvilliers, qui n'y vit
point de remède, prit enfin le parti de prier le roi de le dispenser de
se mêler plus d'aucune chose qui regardât l'Espagne. Le chancelier n'en
entendait plus parler il y avait déjà quelque temps. Chamillart, trop
occupé des finances et de la guerre, n'aurait peut-être pas été suspect
aux deux dames, sans sa liaison intime avec les ducs de Chevreuse et de
Beauvilliers, mais il n'avait pas loisir de s'occuper de plus que de sa
besogne, et on s'en tenait à son égard, sous prétexte de ménagement, à
ne lui parler d'Espagne que superficiellement pour les ordres et les
expéditions qui le regardaient nécessairement pour les troupes et
l'argent. Restait Torcy qui aurait bien voulu n'en entendre jamais
parler, et à qui il ne restait que les choses sèches et résolues sur
lesquelles on ne pouvait se passer de son expédition.

En Espagne M\textsuperscript{me} des Ursins s'était, comme on l'a vu,
défaite des cardinaux d'Estrées et Portocarrero, d'Arias qui au départ
du cardinal d'Estrées s'était retiré une seconde fois, et était allé
attendre dans son archevêché de Séville le chapeau auquel le roi
d'Espagne l'avait nommé, de Louville, de tous ceux qui avaient eu part
au testament de Charles II, ou à quelque faveur du roi indépendamment
d'elle. Rivas, qui avait écrit ce fameux testament, le seul laissé dans
le conseil, y était réduit aux simples expéditions, sans oser dire un
mot, sans crédit ni considération, en attendant qu'elle pût le renvoyer
comme les autres. La princesse et Orry gouvernaient seuls, seuls étaient
maîtres des affaires et des grâces, et tout se décidait entre eux deux,
souvent d'Aubigny en tiers, et la reine présente quand elle voulait, qui
ne voyait que par leurs yeux. Le roi dont toutes les journées étaient
réglées par la reine, et qui, s'il voulait changer quelque chose à ce
qui était convenu pour ses heures et ses amusements comme chasse, mail
ou autre chose, le lui envoyait demander par Vaset, huissier français,
dévoué à M\textsuperscript{me} des Ursins, et qui se gouvernait par ce
qu'il lui rapportait\,; le roi, dis-je, peu à peu établi dans cette
dépendance, venait les soirs chez la reine, le plus souvent chez
M\textsuperscript{me} des Ursins, où il trouvait d'ordinaire Orry et
quelquefois d'Aubigny où il apprenait ce qui avait été résolu, et leur
donnait les mémoriaux qu'il avait pris au conseil pour être décidés le
lendemain par eux, et portés par lui ensuite au conseil, où il n'y avait
point à opiner, mais seulement à savoir pour la forme ce que Rivas
recevait du roi pour expédier. L'abbé d'Estrées, qui depuis le départ de
son oncle entrait de ce conseil, n'osait s'y opposer à rien, et s'il
avait quelque représentation à faire, c'était en particulier à
M\textsuperscript{me} des Ursins et à Orry, qui l'écoutaient à peine et
allaient leur chemin sans s'émouvoir de ce qu'il leur pouvait dire. La
princesse régnait ainsi en plein, et ne songeait qu'à écarter tout ce
qui pouvait troubler ou partager le moins du monde sa puissance. Il
fallait une armée sur les frontières de Portugal contre l'archiduc, par
conséquent un général français pour commander les troupes françaises, et
peut-être aussi les espagnoles. Elle avait connu de tout temps la reine
d'Angleterre qui était Italienne, elle l'avait extrêmement cultivée dans
les longs séjours qu'elle avait faits à Paris, elle était demeurée en
commerce de lettres et d'amitié avec elle\,; elle imagina donc de faire
donner au duc de Berwick le commandement des troupes françaises en
Espagne.

Elle le connaissait doux, souple, fort courtisan, sans aucun bien, avec
une famille. Elle compta par ces raisons de faire tout ce qu'elle
voudrait d'un homme entièrement dépendant du roi et de la reine
d'Angleterre, qui lui aurait l'obligation de sortir de l'état commun des
lieutenants généraux et qui aurait un continuel besoin d'elle pour
s'élever et pour s'enrichir, et s'éviter ainsi d'avoir à compter avec un
Français qui aurait une consistance indépendante d'elle. Elle en fit
donc sa cour à Saint-Germain et le proposa à Versailles. Le roi qui, par
égard pour le roi d'Angleterre et par la similitude de ses bâtards,
avait fait servir celui-ci peu de campagnes sans caractère, puis tout
d'un coup {[}en qualité{]} de lieutenant général dans une grande
jeunesse, fut ravi d'une occasion si naturelle de le distinguer d'eux en
lui donnant une armée à commander. Il avait toujours servi en Flandre\,;
sa souplesse et son accortise l'avaient attaché et lié extrêmement avec
M. de Luxembourg et ses amis, avec M. le Duc et M. le prince de Conti,
ensuite avec le maréchal de Villeroy. Ces deux généraux d'armée
l'avaient traité comme leur enfant et à la guerre et à la cour. Il avait
des talents pour l'une et pour l'autre\,; ils l'avaient fort vanté au
roi et en avaient fait leur cour. Le roi, déjà si bien disposé, se fit
un plaisir d'accorder ce général à la prière du roi et de la reine
d'Angleterre, à la demande de M\textsuperscript{me} des Ursins, et aux
témoignages qui lui avaient été si souvent rendus de son application et
de sa capacité. Le hasard fit que Berwick, qui avait le nez bon et qui
avait cultivé Harcourt de bonne heure, comme un homme tourné à la
fortune, était devenu fort de ses amis, et que celui-ci, se trouvant
seul dans cette bouteille d'Espagne, acheva de déterminer. C'est ainsi
que ce choix fut fait\,; mais comme il n'avait jamais été en chef, le
roi lui voulut donner Puységur qu'il connaissait fort pour avoir
longtemps commandé son régiment d'infanterie, dans tous les détails
duquel il entrait, et pour avoir été employé par lui, comme on l'a vu,
en beaucoup de projets et d'exécutions importantes sur lesquels il avait
souvent travaillé avec lui, et dont Puységur lui avait rendu bon compte.
Il avait été l'âme de l'armée de Flandre\,; ainsi le duc de Berwick
l'avait aussi fort courtisé et le connaissait très particulièrement.
Avec ce secours et en chargeant Puységur du détail de toutes les
troupes, comme unique directeur, et du soin supérieur des magasins et
des vivres, c'est-à-dire de les diriger, de les examiner et d'en
disposer, le roi crut avoir pris toutes les précautions qui se pouvaient
prendre pour la guerre en Espagne.

Puységur partit le premier. Il trouva tout à merveille, depuis les
Pyrénées jusqu'à la hauteur de Madrid, pour la subsistance des troupes
françaises, et en rendit un compte fort avantageux. Il travailla en
arrivant à Madrid avec Orry, qui, papier sur table, lui montra tous ses
magasins faits, tant pour la route jusqu'à la frontière de Portugal que
sur la frontière même, pour la subsistance abondante de l'armée, et tout
son argent prêt pour que rien ne manquât dans le courant de la campagne.
Puységur, homme droit et vrai, qui avait trouvé tout au meilleur état du
monde depuis les Pyrénées, n'imagina pas qu'Orry eût pu manquer de soins
pour la frontière, dans une conjoncture si décisive que celle où
l'Espagne se trouvait d'y terminer promptement la guerre avant que
l'archiduc fût mieux secouru\,; et beaucoup moins qu'un ministre chargé
de tout eût l'effronterie de lui montrer en détail toutes ses
précautions, s'il n'en avait pris aucune. Content donc au dernier point,
il manda au roi de grandes louanges d'Orry, par conséquent de
M\textsuperscript{me} des Ursins et de leur bon et sage gouvernement, et
donna les espérances les plus flatteuses du grand usage qui s'en pouvait
tirer. Plein de ces idées, il partit pour la frontière de Portugal pour
y reconnaître tout par lui-même et y ajuster les choses suivant les
projets, afin qu'il n'y eût plus qu'à exécuter à l'arrivée des troupes
françaises et de leur général. Mais quelle fut sa surprise lorsque, de
Madrid à la frontière, il ne trouva rien de ce qui était nécessaire pour
la marche des troupes, et qu'en arrivant à la frontière même, il ne
trouva quoi que ce soit de tout ce qu'Orry lui avait montré sur le
papier comme exécuté\,! Il eut peine à ajouter foi à ce qui lui revenait
de toutes parts d'une négligence si criminelle. Il se porta dans tous
les lieux où les papiers que lui avait montrés Orry indiquaient les
magasins. Il les trouva tous vides et nul ordre même donné. On peut
juger quel fut son dépit de se trouver si loin de tout ce sur quoi il
avait eu lieu de compter avec tant de certitude, et ce qu'il en manda à
Madrid. Il en rendit compte au roi en même temps, et il avoua sa faute,
si c'en était une, d'avoir cru Orry et à ses papiers, et se donna en
même temps tout le mouvement qu'il put, non plus pour avoir de quoi
faire, comme il l'avait espéré, puisque la chose était devenue
impossible, mais au moins pour que l'armée pût subsister et ne fût pas
réduite à manquer de tout et à ne pouvoir entrer et agir quelque peu en
campagne.

Cette conduite d'Orry, et plus, s'il se peut, son impudence à oser
tromper un homme qui va incontinent après voir de ses yeux son mensonge,
sont des choses qui ne se peuvent comprendre. On comprend de tout temps
que les fripons volent, mais non pas qu'ils le fassent avec l'audace de
persuader contre les faits sitôt et si aisément prouvés. Toutefois,
c'est ce qu'Orry s'était promis de l'appui de la princesse et de la
fascination de Versailles à leur égard.

L'aveuglement fut tel que dans ce même temps, où ils doivent être si en
peine de l'effet de leur conduite, M\textsuperscript{me} des Ursins y
mit le comble. Elle avait si bien lié et garrotté le pauvre abbé
d'Estrées, qui se promettait je ne sais comment une fortune en se
cramponnant comme que ce fût dans son triste emploi en Espagne, qu'il
avait consenti à l'inouïe proposition que lui, ambassadeur de France,
n'écrirait au roi et à sa cour que de concert avec elle, et bientôt
après qu'il n'y enverrait aucune {[}lettre{]} sans la lui avoir montrée.
Une dépendance si gênante pour qui que ce fût, si folle pour un
ambassadeur, et si destructive de son devoir et de son ministère, devint
à la fin insupportable à l'abbé d'Estrées. Il commença donc à lui
souffler quelques dépêches. Son, adresse n'y fut pas telle que la
princesse, si attentive à tout, si crainte, et si bien obéie, n'en eût
le vent par le bureau de la poste. Elle prit ses mesures pour être
avertie à temps la première fois que cela arriverait\,; elle la fut, et
n'en fit pas à deux fois. Elle envoya enlever la dépêche de l'abbé
d'Estrées au roi. Elle l'ouvrit, et, comme elle l'avait bien jugé, elle
n'eut pas lieu d'en être contente\,; mais ce qui la piqua le plus, ce
fut que l'abbé détaillant sa conduite et ce conseil où tout se portait
et se décidait, composé d'elle, d'Orry et très souvent de d'Aubigny,
exagérant l'autorité de ce dernier, ajoutait que c'était son écuyer,
qu'on ne doutait point qu'elle n'eût épousé. Outrée de rage et de dépit,
elle mit en marge à côté, de sa main\,: \emph{Pour mariée, non}, montra
la lettre en cet état au roi et à la reine d'Espagne et à beaucoup de
gens de cette cour avec des clameurs étranges, et ajouta à cette folie
celle d'envoyer cette même lettre, ainsi apostillée, au roi, avec les
plaintes les plus emportées contre l'abbé d'Estrées d'avoir écrit sans
lui montrer sa lettre, comme ils en étaient convenus, et de l'injure
atroce qu'il lui faisait sur ce prétendu mariage.

L'abbé d'Estrées, de son côté, ne cria pas moins haut de la violation de
la poste, de son caractère, et du respect dû au roi, méprisé au point
d'intercepter, ouvrir, apostiller, rendre publique une lettre de
l'ambassadeur du roi à Sa Majesté. La reine d'Espagne, animée par
M\textsuperscript{me} des Ursins dont elle avait épousé les intérêts
sans bornes, éclata contre l'abbé d'Estrées de manière à mettre les
choses au point que sa demeure en Espagne devint incompatible avec son
autorité. Pour le roi son époux, il se mêla peu dans la querelle, mais
ce peu fut en, faveur de la princesse des Ursins, soit qu'avec un bon
sens qu'il eut toujours et droit en toutes choses, mais qu'il retenait
lui-même captif sous sa lenteur et sa glace, il sentît l'énormité du
fait, soit qu'il ne fût pas capable de prendre vivement l'affirmative
pour personne, par sa tranquillité naturelle. Cette lettre, apostillée
par la princesse, accompagnée de ses plaintes et de la justice
exemplaire qu'elle demandait de l'abbé d'Estrées, arriva au roi fort peu
après celles de Puységur, datées de la frontière de Portugal. Ces
dernières avaient étrangement indisposé le roi contre Orry et contre la
princesse, qui n'étaient considérés que conjointement en tout, et qui
avait écrit pour soutenir les mensonges d'Orry de toutes ses forces. Nos
ministres, qui n'avaient abandonné les affaires d'Espagne que de dépit,
ne perdirent pas une occasion si essentielle de tomber sur ce
gouvernement, et de profiter du mécontentement que le roi laissa
échapper pour se revendiquer une portion si considérable de leurs
fonctions. Harcourt, qui en sentit tout le danger, soutint tant qu'il
put M\textsuperscript{me} de Maintenon à protéger Orry dans une occasion
où il y allait de tout pour lui et pour M\textsuperscript{me} des
Ursins, empêcher le renversement de leur puissance et le retour naturel
du maniement des affaires d'Espagne aux ministres, qui ne les lui
laisseraient plus retourner, en quoi lui-même était le plus intéressé.
Cette lutte balança jusqu'à ne savoir qui l'emporterait, lorsque cette
lettre fatale arriva, et les plaintes amères de l'abbé d'Estrées au roi
et aux ministres. Le cardinal d'Estrées, déjà de retour à la cour, leur
donna tout le courage qu'il put pour profiter d'une occasion unique de
perdre M\textsuperscript{me} des Ursins, et de se délivrer une fois pour
toutes d'une usurpation d'une portion si principale de leur ministère.
L'éclat était trop grand et trop public pour que le roi ne leur en
parlât pas. Il avait déjà agité avec eux les plaintes de Puységur et les
moyens d'y remédier au moins en partie, de manière que ce surcroît
arrivé si fort en cadence forma un tout qui accabla Orry et la
princesse\,: dès lors l'un et l'autre furent perdus.
M\textsuperscript{me} de Maintenon eût trop grossièrement montré la
corde d'entreprendre la protection d'un manque de respect d'une telle
hardiesse, et dont le roi lui parut si offensé\,; toute l'adresse
d'Harcourt échoua contre cet écueil. Le parti fut donc pris de la
renvoyer à Rome et de rappeler Orry\,; mais l'embarras fut la crainte
d'une désobéissance formelle, et que le roi d'Espagne ne pût résister
aux cris que ferait la reine. Après le trait qui venait d'arriver, les
plus grandes extrémités étaient à prévoir\,; et c'est ce qui fit prendre
le tour de ne rien précipiter pour frapper le coup sans risque de le
manquer. Le roi fit à la princesse une réprimande sévère d'une hardiesse
sans exemple, qui attaquait si directement le respect dû à sa personne
et le secret qui devait être sacré de son ambassadeur à lui. En même
temps on manda à l'abbé d'Estrées cette réprimande, qu'il avait juste
occasion de se plaindre, mais rien de plus.

L'abbé d'Estrées, qui comptait que M\textsuperscript{me} des Ursins en
serait chassée, tomba dans le désespoir quand il l'en vit quitte pour si
peu de chose, et lui sans satisfaction, exposé à la haine et aux
insultes de la princesse et même de la reine, et à voir cette puissance
plus établie que jamais, puisqu'elle avait échappé à une action si
inouïe, tellement que, de dépit et de désespoir de ne pouvoir plus se
rien promettre de l'Espagne, il demanda son congé. Il fut pris au mot,
et ce fut un nouveau triomphe pour la princesse de s'être défait si
scandaleusement de lui, qui avait toute raison, et dont l'affaire était
celle du roi même, tandis qu'elle demeurait pleinement maîtresse, elle
qui avait eu loisir de sentir et de craindre les suites naturelles d'un
emportement si audacieux. Mais en même temps que ce panneau et cette
apparente victoire amusait M\textsuperscript{me} des Ursins, le cardinal
d'Estrées, autant pour la piquer que par affection pour son neveu,
soutenu des ministres par le même sentiment, et des Noailles par
l'amitié et la proximité de l'alliance, se servit avantageusement du
rappel de l'abbé d'Estrées, sans aucun tort de sa part, après un éclat
de cette nature, pour un dédommagement de la satisfaction qu'il avait
été si fort en droit d'obtenir, et qui marquât du moins celle que le roi
avait de sa conduite. Le faire évêque\,? il était encore assez jeune et
bien fait, il avait eu des galanteries, et il était du nombre de ces
abbés sur qui le roi s'était expliqué qu'il n'en élèverait aucun d'eux à
l'épiscopat. Des abbayes\,? cela ne remplissait pas leur but de quelque
chose d'éclatant. Ils se tournèrent tous sur l'ordre du Saint-Esprit,
comme sur un honneur qui marquerait continuellement sur sa personne la
satisfaction que le roi avait eue de sa conduite, une distinction très
grande dans le clergé par le petit nombre de ces places, et une place
d'autant plus flatteuse qu'elle était comme sans exemple.

En effet, le seul prêtre commandeur de l'ordre qui ne fût point évêque
était un Daillon du Lude\footnote{Guy de Daillon, comte du Lude, etc.,
  était fils de Jean de Daillon et d'Anne de Batarnay.}, fils d'une
Batarnay et du premier comte du Lude, gouverneur de Poitou, la Rochelle
et pays d'Aunis, et lieutenant général de Guyenne, qui parut fort en son
temps\,; et cet abbé, parent des Joyeuse et des Montmorency par sa mère,
était frère du second comte du Lude, gouverneur de Poitou, sénéchal
d'Anjou et chevalier du Saint-Esprit en 1581. Ses trois sœurs épousèrent
trois seigneurs, tous trois chevaliers du Saint-Esprit\footnote{Françoise
  de Daillon épousa, en 1558, Jacques Goyon, seigneur de Matignon, comte
  de Thorigny et maréchal de France\,; Anne de Daillon fut mariée à
  Philippe de Volvire, marquis de Ruffec\,; enfin une seconde Françoise
  de Daillon à Jean de Chourses, seigneur de Malicorne.}. Le maréchal de
Matignon, Philippe de Volvire, marquis de Ruffec, gouverneur de
Saintonge et d'Angoumois\,; et François, seigneur de Malicorne et
gouverneur de Poitou après son beau-frère. Le frère de René de Daillon,
commandeur de l'ordre, fut trisaïeul du comte du Lude, mort duc à brevet
et grand maître de l'artillerie. J'ai détaillé exprès cette courte
généalogie pour montrer quel fut ce René de Daillon, qui de plus s'était
jeté dans Poitiers avec ses frères, en 1569, pour le défendre contre les
huguenots. Mais il y avait une disparité avec l'abbé d'Estrées. René de
Daillon avait été nommé évêque de Luçon\,; il n'en voulut point et prit
en échange l'abbaye des Chastelliers, dont il porta le nom suivant
l'usage de ce temps-là qui a duré longtemps depuis. Ce fut sous cette
qualité qu'il eut l'ordre en la première promotion où Henri III fit des
cardinaux et des prélats\,; et assez peu de temps après, l'abbé des
Chastelliers fut fait et sacré évêque de Bayeux. Toute cette petite
fortune fut fort courte, car il mourut en 1600.

Cette différence fit au roi quelque difficulté outre l'unicité de
l'exemple\,; mais il s'en trouvait encore plus à rencontrer quelque
autre chose de compatible avec la prêtrise\,; et le roi, sur l'exemple
d'autres occasions de promesse de la première place vacante, se
détermina enfin à déclarer qu'il réservait à l'abbé d'Estrées le premier
cordon bleu dont il aurait à disposer pour un ecclésiastique. Il n'eut
pas longtemps à attendre. Le cardinal de Fürstemberg mourut presque
aussitôt après, qui fut une autre occasion de triomphe pour les Estrées.
Le roi apprit sa mort en se levant. Aussitôt il envoya Bloin au cardinal
d'Estrées, qui était à Versailles, lui dire que, se doutant que la
modestie l'empêcherait de demander Saint-Germain des Prés, il la lui
donnait. Ces deux grâces si considérables, et si près à près, faites à
l'oncle et au neveu, les comblèrent de joie\,; et le cardinal,
d'ailleurs tout à fait noble et désintéressé, ne se contenait pas, et
disait franchement que toute sa joie était du dépit qu'aurait la
princesse des Ursins. En effet, cela lui donna fort à penser.

La campagne était commencée en Portugal malgré tous les manquements
d'Orry. Le roi d'Espagne voulut la faire\,; M\textsuperscript{me} des
Ursins, qui ne voulait pas le perdre de vue, mit tout son crédit et
celui de la reine pour l'en empêcher, ou du moins pour mener la reine.
Le roi, qui suivait toujours son dessein, avait déjà mandé au roi son
petit-fils, qu'ayant été chercher ses ennemis jusqu'en Lombardie, et
ayant son compétiteur en personne dans le continent des Espagnes, il
serait honteux et indécent qu'il ne se mît pas à la tête de son armée
contre lui. Il le soutint fortement dans cette résolution, et il
s'opposa nettement à ce qu'il se fît accompagner de la reine, dont
l'embarras et la dépense seraient préjudiciables. Il rompit donc le
voyage de la reine, qui demeura à Madrid, et pressa si bien le départ du
roi son petit-fils, qu'il parut à la tête de son armée à la mi-mars, où
l'abbé d'Estrées eut ordre de l'accompagner en attendant l'arrivée de
son successeur. C'était le point où le roi avait voulu venir. La reine
avait un tel ascendant sur le roi son mari, et elle s'était si
éperdument abandonnée à la princesse des Ursins, qu'il n'espéra pas être
obéi sans des fracas qu'il voulut éviter en tenant le roi son petit-fils
éloigné de la reine. Sitôt que cela fut exécuté, il lui écrivit sur
l'éloignement pour toujours de la princesse des Ursins, d'un style à lui
en persuader la nécessité pressante et le parti pris à ne rien écouter.
En même temps il écrivit encore avec plus d'autorité à la reine, et
envoya un ordre à la princesse des Ursins de partir incontinent de
Madrid, de sortir tout de suite d'Espagne, et de se retirer en Italie.

Ce coup de foudre mit la reine au désespoir, sans accabler celle sur qui
il tombait. Elle ouvrit alors les yeux sur tout ce qui s'était passé
depuis cette lettre apostillée\,; elle sentit que tout s'était fait avec
ordre et dessein pour la chasser pendant la séparation du roi d'Espagne
et de la reine, et la vanité du triomphe dont elle s'était flattée
quelques moments. Elle comprit qu'il n'y avait nulle ressource pour
lors\,; mais elle ne désespéra pas pour un autre temps, et n'en perdit
aucun à se les préparer en Espagne, d'où elle fondait son principal
secours en attendant qu'elle pût s'ouvrir quelque porte en France. Elle
ne fit remuer la reine du côté des deux rois que pour gagner quelques
jours. Elle les employa à donner à la reine la duchesse de Montellano
pour camarera-mayor, sûre de la déplacer si elle revenait en Espagne.
Elle était sœur du feu prince d'Isenghien, la meilleure, la plus douce
femme du monde, mais la plus bornée, la plus timide, la plus désireuse
de plaire\,: je l'ai connue en Espagne camarera-mayor de la reine, fille
de M. le duc d'Orléans. Elle choisit une des femmes de la reine
entièrement à elle et qui avait de l'esprit et du manège, par qui elle
établit son commerce avec elle, et se ménagea des voies sûres d'être
instruite de tout et de donner ses ordres. Elle-même instruisit la reine
de tout ce qu'elle devait faire selon les occasions, en l'une et l'autre
cour, pour obtenir son retour auprès d'elle, et conserver cependant son
crédit. Elle lui nomma et lui dépeignit les divers caractères de ceux
sur qui, et jusqu'à quel point elle pouvait compter, et les divers
usages qu'elle en pouvait tirer pour en entourer le roi. En un mot elle
arrangea toutes ses machines, et sous prétexte de la nécessité du
préparatif d'un voyage si long et si précipité elle laissa
tranquillement redoubler les ordres et les courriers, et ne partit point
qu'elle n'eût achevé de dresser et d'établir, tout son plan. Elle alla
cependant faire ses adieux par la ville, ne regrettant, disait-elle, que
la reine, se taisant sur le traitement qu'elle recevait, et le
supportant avec un courage mâle et réfléchi, sans hauteur pour ne pas
irriter davantage, encore plus sans la moindre odeur de bassesse.

Enfin elle partit une quinzaine après en avoir reçu l'ordre, et s'en
alla à Alcala, que les nombreux et savants collèges que le célèbre
cardinal Ximénès y a si magnifiquement bâtis et fondés pour toutes
sortes de sciences ont rendue fameuse. Cette petite ville est à sept
lieues de Madrid, à peu près comme de Paris à Fontainebleau., Le plus
pressé était fait, mais elle avait encore des mesures à prendre qui
pouvaient souffrir cet éloignement, de sorte que sous toutes sortes de
prétextes elle y tint bon contre les ordres réitérés qu'elle y reçut de
partir. La reine la conduisit à deux lieues de Madrid, et n'oublia rien
qui pût persuader qu'elle et la princesse ne seraient jamais qu'une.
Elle l'avait persuadée aussi que son éloignement, pour peu qu'il durât,
serait la fin de son autorité et le commencement de ses malheurs. Ainsi
elle se pleurait elle-même en pleurant cette séparation. On crut que
d'Alcala elle avait été plus d'une fois à Madrid, ce qui était très
possible. Enfin au bout de cinq semaines d'opiniâtre séjour en ce lieu,
toutes ses trames bien ourdies et bien assurées, avec une présence
d'esprit qui ne se peut trop admirer dans ce court espace si traversé de
dépit, de rage, de douleur, et dans l'accablement d'une si profonde
chute, elle s'avança vers Bayonne aux plus petites journées et aux plus
fréquents séjours qu'elle put et qu'elle osa.

Cependant le successeur de l'abbé d'Estrées était nommé, qui ne surprit
pas peu tout le monde. Ce fut le duc de Grammont qui avait pour lui son
nom, sa dignité et une figure avantageuse, mais rien de plus\,; fils du
maréchal de Grammont si adroit à être et à se maintenir bien avec tous
les personnages, par là à se faire compter de tous, surtout à ne se pas
méprendre sur ceux qui devaient demeurer les maîtres des autres. Sans se
détacher de personne, et néanmoins sans se rendre suspect, il était
parvenu à la plus grande fortune et à la première considération par son
intimité avec les cardinaux de Richelieu et Mazarin, dont il eut la
confiance toute leur vie, conséquemment du dernier, l'amitié et la
confiance de la reine et du roi son fils\,; en même temps il sut
s'acquérir celle de Gaston et celle de M. le Prince, qui eut toujours et
dans tous les temps une sorte de déférence pour lui qui ne se démentit
point. Ce fut lui qui fut chargé d'aller faire la demande de la reine,
qu'il exécuta avec tant de magnificence et de galanterie, puis de
l'ambassade pour l'élection de l'empereur Léopold avec M. de Lyonne. Les
folies galantes de son fils aîné, le comte de Guiche, devinrent la
douleur de sa vie, qui ôtèrent le régiment des gardes de sa famille, où
il l'avait mis, et qu'il ne put jamais faire passer de l'aîné au cadet,
qu'on appelait Louvigny et qui est le duc de Grammont dont je parle.
Avec de l'esprit, le plus beau visage qu'on pût voir et le plus mâle, la
considération de son père le mit dans tous les plaisirs de la jeunesse
du roi et lui en acquit la familiarité pour toujours. Il épousa la fille
du maréchal de Castelnau, avec qui il avait poussé la galanterie un peu
loin. Son frère qui mourut depuis, et qui la laissa fort riche,
n'entendit pas raillerie, et fit faire le mariage haut à la main.
L'épouseur n'avait point acquis bon bruit sur le courage, il ne l'avait
pas meilleur au jeu ni sur les choses d'intérêt, où dans son
gouvernement de Bayonne, Béarn, etc., on avait soin de tenir sa bourse
de près. Ses mœurs n'étaient pas meilleures, et sa bassesse passait tous
ses défauts. Après les grands plaisirs du premier âge et le jeu du
second, où le duc de Grammont suivit toujours les parties du roi, le
sérieux qui succéda ne laissant plus d'accès particuliers et journaliers
au duc de Grammont, il imagina de s'en conserver quelque chose par la
flatterie et par le faible du roi pour les louanges, et se proposa à lui
pour écrire son histoire. En effet, un écrivain si marqué plut au roi,
et lui procura des particuliers pour le consulter sur des faits, et lui
montrer quelques essais de son ouvrage. Il en fit part dans la suite,
comme en grande confidence, à des gens dont il espérait que
l'approbation en reviendrait au roi, et de cette manière il se soutint
auprès de lui. Sa plume toutefois n'était pas taillée pour une si vaste
matière, et qu'il n'entreprenait que pour faire sa cour\,; aussi
fut-elle peu suivie.

Lié aux Noailles par le mariage de son fils, et beau-père du maréchal de
Boufflers, il se mit en tête plus que jamais d'être de quelque chose. Il
brigua les ambassades, même jusqu'à celle de Hollande. C'est à quoi il
était aussi peu propre qu'à composer des histoires\,; mais à force de
persévérance, il attrapa celle-ci dans une conjoncture où peu de gens
eurent envie d'aller essuyer la mauvaise humeur de la catastrophe de
M\textsuperscript{me} des Ursins. La surprise néanmoins en fut grande.
On le connaissait dans le monde, et de plus il venait d'achever de se
déshonorer en épousant une vieille gueuse qui s'appelait La Cour. Elle
avait été femme de chambre de la femme du premier médecin Daquin, puis
de M\textsuperscript{me} de Livry. Des Ormes, contrôleur général de la
maison du roi, frère de Bechameil, et dont la charge a des rapports
continuels avec celle de premier maître d'hôtel du roi qu'avait Livry,
allait chez lui toute la journée. Il trouva cette créature à son gré, il
lui en conta et l'entretint publiquement plusieurs années. Le duc de
Grammont jouait aussi fort chez Livry, il était ami de des Ormes\,; et
tant qu'il entretint cette fille, c'est-à-dire le reste de sa vie, le
duc de Grammont soupait continuellement en tiers ou en quart avec eux,
ainsi il n'ignorait pas leur façon d'être. À la mort de des Ormes, il la
prit et l'entretint, et l'épousa enfin quoique devenue vieille, laide et
borgnesse. Cet épisode à l'occasion d'un particulier n'est pas assez
intéressant (si ce n'est pour sa famille qui en fut aux hauts cris et au
dernier désespoir), pour avoir place ici sans ce qui va suivre.

Le mariage fait en secret, puis déclaré par le duc de Grammont, il se
mit dans la tête d'en faire sa cour au roi par la plus délicate de
toutes les approbations qui est l'imitation, et plus encore à
M\textsuperscript{me} de Maintenon, puisque lui-même avait déclaré son
mariage. Il employa des barbes sales de Saint-Sulpice et de ces cagots
abrutis de barbichets des Missions qui ont la cure de Versailles, pour
faire goûter ce grand acte de religion et le tourner en exemple. On peut
juger si le roi et M\textsuperscript{me} de Maintenon s'en trouvèrent
flattés. Le moment choisi pour cela, qui fut celui de sa mission en
Espagne, et le prétexte, celui d'y mener cette gentille duchesse, parut
mettre le comble à cette folie, qui réussit tout au contraire de ce
qu'il en avait espéré. La comparaison prétendue mit en fureur
M\textsuperscript{me} de Maintenon, et le roi si en colère, que le duc
de Grammont fut plusieurs jours sans oser se présenter devant lui. Il
lui envoya défendre de laisser porter ni prendre à sa femme aucune
marque ni aucun rang de duchesse en quelque lieu que ce fût, et
d'approcher jamais de la cour, surtout de ne s'aviser pas de lui laisser
mettre le pied en Espagne. L'ambassade était déclarée depuis le mariage
(ce ne fut que depuis l'ambassade que cette folie de comparaison et d'en
faire sa cour avait eu lieu, sous prétexte de faire prendre son tabouret
à cette créature, et de la mener après en Espagne)\,; quelque dépit
qu'en eussent conçu le roi et M\textsuperscript{me} de Maintenon, il n'y
eut pas moyen d'ôter l'ambassade, cela eût trop montré la corde\,; mais
l'indignation n'y perdit rien. Il n'y avait que le duc de Grammont au
monde capable d'imaginer de plaire par une si odieuse comparaison. Il
était infatué de cette créature qui le mena par le nez tant qu'il
vécut\,; il était naturel qu'elle pensât en servante de son état,
qu'elle voulût faire la duchesse, et que tout lui parût merveilleux pour
y parvenir. Elle mit donc cette belle invention dans la tête de son
mari, qui s'en coiffa aussitôt comme de tout ce qui venait d'elle, et
qui même après le succès ne put se déprendre de la croire aveuglément
sur tout.

Il eut défense expresse de voir la princesse des Ursins, qu'il devait
rencontrer sur sa route. Quelque peu écoutée qu'elle pût espérer d'être
à Versailles, dans ces moments si proches de la foudre qui en était
partie et qui l'écrasait, son courage ne l'y abandonna pas plus qu'à
Madrid. Tout passe avec le temps dans les cours, même les plus terribles
orages, quand on est bien appuyé et qu'on sait ne pas s'abandonner au
dépit et aux revers. M\textsuperscript{me} des Ursins, s'avançant
toujours à lents tours de roue, ne cessait d'insister sur la permission
de venir se justifier à la cour. Ce n'était pas qu'elle osât l'espérer,
mais à force d'instances et de cris d'éviter l'Italie, et d'obtenir un
exil en France, d'où avec le temps elle saurait peut-être se tirer.
Harcourt, par l'Italie, perdait jusqu'à l'espérance de tous les secrets
détails par lesquels il se maintenait, et M\textsuperscript{me} de
Maintenon toute celle de part directe au gouvernement de l'Espagne. Ils
sentirent l'un et l'autre le poids de cette perte\,; après les premiers
temps de l'éclat ils reprirent leurs esprits. Le roi était obéi, il
jouissait de sa vengeance. L'ordre à l'abbé d'Estrées et l'abbaye de
Saint-Germain à son oncle la comblait. C'était un surcroît d'accablement
pour une dictatrice de cette qualité aussi roidement tombée et chassée
avec si peu de ménagement. La pitié put avoir lieu après une exécution
si éclatante\,; et la réflexion qu'il ne fallait pas pousser la reine
d'Espagne à bout sur des choses qui n'influaient plus sur les affaires,
et qui ne compromettaient point l'autorité. Ce fut le biais que prit
M\textsuperscript{me} de Maintenon pour arrêter la princesse des Ursins
en France. Cela parait l'Italie, cela suffisait pour lors\,; mais il
fallait ménager le roi si ferme sur l'Italie, il n'était pas temps de
lui laisser naître aucun soupçon. C'est ce qui détermina à fixer à
Toulouse le séjour qui fut accordé enfin comme une grâce à
M\textsuperscript{me} des Ursins, et même avec beaucoup de peine.

C'était le chemin à peu près pour gagner de Bayonne, par où elle entrait
en France, le Dauphiné ou la Provence, pour de là passer les Alpes, ou
par mer en Italie. C'était une grande ville où elle aurait toutes ses
commodités et la facilité nécessaire pour ses commerces en Espagne d'où
elle ne l'éloignait point, et à Versailles par le grand abord d'une
capitale de Languedoc, siège d'un parlement, et un grand passage où on
cache mieux ses mouvements que dans de petites villes et dans des lieux
écartés. Un châtiment mis en évidence sur ce théâtre de province, qui
eût été un grand surcroît de dépit et de peine dans toute autre
conjoncture, parut une grâce à l'exilée et une certitude de retour. Elle
comprit par ce premier pas qu'il n'y avait qu'à attendre, et cependant
bien ménager sans se décourager\,; et dès lors elle se promit tout de
ses appuis et plus encore d'elle-même. Avec un aussi grand intérêt que
celui de M\textsuperscript{me} de Maintenon\,; un agent aussi à portée,
aussi habile, aussi audacieux que Harcourt porté par son intérêt le plus
cher d'ambition et de haine des ministres, et un ami capable de tout
imaginer et de tout entreprendre avec feu et suite, et l'expérience
d'une vie toute tissue des plus grandes intrigues tel qu'était Cosnac,
archevêque d'Aix, la reine d'Angleterre, pour porter de certains coups
qui auraient trop démasqué M\textsuperscript{me} de Maintenon et
d'autres amis en sous-ordre que son frère savait organiser et conduire,
tout aveugle qu'il était, il parut impossible à M\textsuperscript{me}
des Ursins d'être laissée longtemps en spectacle à Toulouse, maîtresse
et en commodité de faire agir le roi et la reine d'Espagne en cadence de
ces grands ressorts.

On fit revenir en même temps le chevalier des Pennes, qui passait pour
la créature de M\textsuperscript{me} des Ursins la plus attachée à elle.
Elle l'avait fait enseigne des gardes du corps\,; il était à Palencia
auprès du roi d'Espagne, et il était enfermé trois heures tête à tête
avec lui tous les jours, lorsqu'il reçut cet ordre en même temps que la
princesse des Ursins reçut le sien. Le roi d'Espagne lui envoya quinze
cents pistoles quoiqu'il eût sûrement plus besoin qu'elle, et que, sans
le crédit de l'abbé d'Estrées qui trouva cent mille écus, il n'eût pu
sortir de Madrid. Orry eut ordre en même temps de venir rendre compte de
l'impudence de ses mensonges et d'une administration qui sauvait
l'archiduc, et empêchait la conquête du Portugal que les progrès des
armées de France et d'Espagne, nonobstant des manquements de tout si
universels, montrèrent avoir été facile et sûre, si ou eût trouvé la
moitié seulement de ce que cet audacieux fripon avait dit et assuré à
Puységur être partout dans les magasins établis sur cette frontière.

Plusieurs grands suivirent le roi d'Espagne. Le connétable de Castille
qui en voulait être s'en abstint, sur la folle prétention de faire à
l'armée les mêmes fonctions et avec la même autorité que le connétable
de France commande les nôtres. Cette charge de connétable de Castille
est devenue un nom et rien davantage par une hérédité qui, sans cette
sage réduction, le rendrait beaucoup plus grand que le roi d'Espagne. On
parlera ailleurs plus à fond de ces titres vains et héréditaires en
Espagne. Le duc de l'Infantado, du nom de Silva, partit de Madrid pour
aller à une de ses terres quelques jours avant le roi, sans prendre
congé de lui, et y rentra le soir même que le roi en partit. Cette
conduite scandalisa fort. Je la remarque parce qu'elle a été soutenue
toute sa vie, et qu'il y aura encore occasion d'en parler.

Laissons aller et demeurer la princesse des Ursins à Toulouse, qui à
Bayonne avait encore reçu ordre de s'acheminer droit en Italie, et le
duc de Grammont en Espagne. Il eut soixante mille livres pour son
équipage\,; douze mille livres par an pour le dédommager du droit de
franchise que les ambassadeurs avaient pour les provisions de leurs
maisons, et que l'abus qui s'en faisait a fait retrancher\,; et cinq
mille livres par mois. À Venise ils étaient en usage. Charmont, qui de
procureur général du grand conseil s'était fait secrétaire du cabinet
pour le plaisir de ne rien faire, d'aller à Versailles et de porter une
brette, en avait obtenu l'ambassade, et n'avait pas résolu de s'y
appauvrir. Il eut force prises sur ces franchises, tant qu'à la fin les
Vénitiens attrapèrent de ses passeports qu'il avait donnés à des
marchands qui faisaient sortir les sels de l'État de la république, pour
les porter dans ceux de l'empereur au bout du golfe sans payer aucuns
droits. Ils les envoyèrent à Paris à leur ambassadeur qui les porta à M.
de Torcy, et fit de grandes plaintes au roi de la part de la république,
dans une audience demandée uniquement pour cela. Un homme de qualité
aurait mal passé son temps, mais Charment était Hennequin. Les ministres
le protégèrent, et l'affaire se passa fort doucement. La fin fut
pourtant qu'il fut rappelé, mais au bout de son temps achevé, et avec
des ménagements admirables. Il fut même fort bien reçu à son retour, et
il eut la plume de Mgr le duc de Bourgogne par le choix du roi.

\hypertarget{chapitre-xv.}{%
\chapter{CHAPITRE XV.}\label{chapitre-xv.}}

1704

~

{\textsc{Comte de Toulouse et maréchal de Cœuvres s'embarquent à
Brest.}} {\textsc{- Duc de Mantoue incognito à Paris\,; voit le roi à
Versailles.}} {\textsc{- Trente mille livres de pension au cardinal
Ottobon.}} {\textsc{- Cinq cent mille livres de brevet de retenue au duc
de Beauvilliers.}} {\textsc{- La Queue et sa femme, et leur chétive
fortune.}} {\textsc{- Mort de l'abbé Boileau, le prédicateur.}}
{\textsc{- Mort de Mélac.}} {\textsc{- Mort de Rivaroles.}} {\textsc{-
Mort de la duchesse de Verneuil.}} {\textsc{- Mort de Grancey.}}
{\textsc{- Quatre cent mille livres de brevet de retenue à La
Vrillière.}} {\textsc{- Troisvilles élu et refusé du roi pour
l'Académie\,; sa vie et son caractère.}} {\textsc{- Villars voit
Cavalier, un des chefs des fanatiques\,; ses demandes\,; ce que devint
cet aventurier.}} {\textsc{- Barbezières rendu à Casal.}} {\textsc{-
Manèges de MM. de Vendôme.}} {\textsc{- Mort du fils unique de
Vaudémont.}} {\textsc{- Mot du premier maréchal de Villeroy sur les
ministres.}} {\textsc{- Complaisance de Tessé qui laisse La Feuillade en
chef en Savoie et en Dauphiné, qui devient général d'armée, prend Suse
et les vallées.}} {\textsc{- Phélypeaux salue le roi\,; sa conduite, son
caractère\,; celui de son frère l'évêque de Lodève\,; est fait
conseiller d'État d'épée.}} {\textsc{- Le duc de Grammont voit en chemin
la princesse des Ursins.}} {\textsc{- Succès du duc de Berwick.}}
{\textsc{- Comte d'Aguilar premier colonel du régiment des gardes
espagnoles.}} {\textsc{- Mouvements des armées de Flandre et du Rhin.}}
{\textsc{- Combat de Donawerth.}} {\textsc{- Comte d'Arco commande nos
lieutenants généraux et obéit aux maréchaux de France.}} {\textsc{-
Bruges, puis Namur bombardés.}} {\textsc{- Verceil pris par le duc de
Vendôme.}} {\textsc{- Fanatiques secourus.}} {\textsc{- Abbé de La
Bourlie et La Bourlie son frère\,; leur extraction et leur fin
misérable.}} {\textsc{- Augicourt, personnage curieux\,; sa mort.}}
{\textsc{- Fortune de Vérac et de Marillac\,; mort du premier.}}
{\textsc{- Harley secrétaire d'État d'Angleterre.}} {\textsc{- Le Blanc
intendant d'Auvergne.}} {\textsc{- Leczinski élu roi de Pologne\,;
depuis beau-père du roi.}} {\textsc{- Abbé de Caylus évêque d'Auxerre.}}
{\textsc{- Castel dos Rios part pour le Pérou, où il meurt.}} {\textsc{-
Comte d'Albret en Espagne, attaché à l'électeur de Bavière.}} {\textsc{-
Abbé d'Estrées de retour.}} {\textsc{- Rebours et Guyet nouveaux
intendants des finances.}} {\textsc{- Mort et caractère de l'abbesse de
Fontevrault\,; sa nièce lui succède.}}

~

Le comte de Toulouse partit dans ces temps-là, précédé de quelques jours
par le maréchal de Cœuvres, pour Brest, et ils montèrent enfin tous deux
sur le même vaisseau.

M. de Mantoue, mal à son aise dans son État devenu le théâtre de la
guerre, qui l'avait livré au roi de bonne grâce, et avait en cela rendu
le plus important service pour la guerre d'Italie, voulut venir faire un
tour en France, où il ne pouvait douter qu'il ne fût très bien reçu. Il
se détourna pour aller faire un tour à Charleville qui lui appartenait,
et il arriva à Paris la surveille de la Pentecôte avec une grande suite.
Il descendit à Luxembourg, meublé pour lui magnifiquement des meubles de
la couronne, ses gens du commun logés rue de Tournon à l'hôtel des
Ambassadeurs extraordinaires, et fut servi de sept tables par jour, soir
et matin, aux dépens et par les officiers du roi, pendant tout son
séjour, et d'autres tables encore pour le menu domestique. Il fut
incognito sous le nom du marquis de San-Salvador\,; mais de cet
incognito dont M. de Lorraine introduisit l'étrange usage sous les
auspices de Monsieur, et qu'on ne voulut pas retrancher, après cet
exemple qui depuis a mené bien loin, à un prince qui, en nous livrant sa
capitale, avait donné au roi la clef de l'Italie. Le lendemain de la
Pentecôte, il alla à Versailles dans des carrosses drapés avec ses
chiffres seulement, qu'on fit entrer dans la grande cour où n'entrent
que ceux qui ont les honneurs du Louvre. Il descendit à l'appartement de
M. le comte de Toulouse, où il trouva toutes sortes de rafraîchissements
servis. De là il monta par le petit degré dans les cabinets du roi, où
il fut reçu sans que le roi s'avançât du tout vers lui. Il parla d'abord
et assez longtemps\,; le roi lui répondit, le combla de civilités, et
après, lui montra Monseigneur, les deux princes ses fils, M. le duc
d'Orléans, M. le Duc et M. le prince de Conti, puis M. du Maine en les
lui nommant\,: il n'y avait outre ces princes que les entrées. Ensuite
M. de Mantoue demanda permission au roi de lui présenter les principaux
de sa suite. De là le roi, suivi de tout ce qui était dans le cabinet,
sortit directement dans la galerie, et le mena chez
M\textsuperscript{me} la duchesse de Bourgogne qui était incommodée et
se trouvait naturellement au lit où il y avait force dames parées, à la
ruelle de laquelle le roi lui présenta M. de Mantoue. La conversation y
dura près d'un quart d'heure, après quoi le roi mena M. de Mantoue tout
du long de la galerie qu'il lui fit voir avec les deux salons, et rentra
avec lui dans son cabinet, où, après une courte conversation, mais de la
part du roi toujours fort gracieuse, le duc prit congé et revint à
Paris. Le roi fut toujours découvert et debout. Huit jours après il
retourna à Versailles, vit les jardins et le roi par le petit degré dans
ses cabinets, n'y ayant que Torcy en tiers. Quelques jours après,
Monseigneur lui donna un grand dîner à Meudon, où étaient les deux
princes ses fils, M. le duc d'Orléans, M\textsuperscript{me} la
princesse de Conti, quelques dames et quelques courtisans. MM. d'Elfian
et Strozzi, les deux principaux de sa suite, mangèrent à la table de
Monseigneur, où, contre l'ordinaire de ces sortes de repas, il fut gai
et M. de Mantoue de bonne compagnie. Il galantisa et loua fort la beauté
de la duchesse d'Aumont. Monseigneur lui montra sa maison et le promena
fort dans ses jardins en calèche. Une autre fois il alla voir les
écuries et le chenil de Versailles, la Ménagerie et Trianon. Il retourna
encore à Versailles, y coucha dans l'appartement de M. le comte de
Toulouse, vit tous les chevaux du roi, s'alla promener à cheval dans les
hauts de Marly et soupa chez Dangeau avec beaucoup de dames. Dangeau
aimait fort à faire les honneurs de la cour, et il est vrai qu'il les
faisait fort bien. M. de Mantoue vit plusieurs fois le roi, et toujours
par le petit degré dans son cabinet, en tête à tête, ou Torcy en tiers.

Parlant d'étrangers, le cardinal Ottobon, qui avec des biens immenses
s'était fort obéré, s'attacha à la France et en eut une pension de dix
mille écus.

Le roi donna aussi cinq cent mille livres de brevet de retenue au duc de
Beauvilliers sur sa charge.

Il fit, vers le même temps, La Queue, capitaine de cavalerie, mestre de
camp par commission, grâce qu'il se fit demander par M. de Vendôme et
qui n'a guère mené cet officier plus loin. Ce La Queue, seigneur du lieu
dont il portait le nom, à six lieues de Versailles et autant de Dreux,
était un gentilhomme fort simple et assez médiocrement accommodé, qui
avait épousé une fille que le roi avait eue d'une jardinière. Bontems,
l'homme de confiance du roi pour ses secrets domestiques, avait fait le
mariage et stipulé sans déclarer aucun père ni mère, que La Queue savait
à l'oreille et s'en promettait une fortune. Sa femme fut confinée à La
Queue, et ressemblait fort au roi. Elle était grande, et pour son
malheur elle savait qui elle était, et elle enviait fort ses trois sœurs
reconnues et si grandement mariées. Son mari et elle vécurent fort bien
ensemble et ont eu plusieurs enfants, demeurés dans l'obscurité. Ce
gendre ne paraissait presque jamais à la cour, et comme le plus simple
officier et le moins recueilli dans la foule, à qui Bontems ne laissait
pas de donner de temps en temps de l'argent. La femme vécut vingt ans
tristement dans son village, sans presque voir personne, de peur que ce
qu'elle était se divulguât, et mourut sans en être sortie.

L'abbé Boileau mourut en ce temps-ci assez promptement d'une opération
au bras fort semblable à la mienne, pour avoir fait un effort en prenant
un in-folio de trop haut. C'était un gros homme, grossier, assez
désagréable, fort homme de bien et d'honneur, qui ne se mêlait de rien,
qui prêchait partout assez bien, et qui parut à la cour plusieurs avents
et carêmes, et qui, avec toute la protection de Bontems dont il était
ami intime, ne put parvenir à l'épiscopat.

Mélac, retiré avec deux valets en un coin de Paris, ne voulant voir qui
que ce fût depuis sa belle défense de Landau et le bâton de Villars,
mourut subitement. Le roi lui donnait dix mille écus par an et quelque
chose de plus. Il avait près de quatre-vingts ans. Je l'ai assez fait
connaître pour n'avoir rien à y ajouter.

Rivaroles, autre fort bon lieutenant général, mourut en même temps.
C'était un Piémontais qui s'était attaché au service de France et qui y
était estimé. Un coup de canon lui avait emporté une jambe il y avait
fort longtemps\,; un autre lui emporta sa jambe de bois à Neerwinden et
le culbuta. On le releva sans mal\,; il se mit à rire. «\,Voilà de
grands sots, dit-il, et un coup de canon perdu\,! Ils ne savaient pas
que j'en ai deux autres dans ma valise.\,» Il était grand-croix de
Saint-Lazare, puis de Saint-Louis à l'institution. Il laissa des enfants
peu riches, qui ont servi et qui n'ont pas fait fortune. Ce Rivaroles,
qui était un grand homme, fort bien fait, adroit et vigoureux, était,
avec sa jambe de bois, un des meilleurs joueurs de paume, et y jouait
souvent.

La duchesse de Verneuil les suivit à quatre-vingt-deux ans, ayant encore
grande mine et des restes d'avoir été fort belle. Elle était fille du
chancelier Séguier, dans le carrosse duquel elle voulut être quand il
courut un si grand péril aux Barricades de Paris, et que le maréchal de
La Meilleraye l'alla délivrer avec des troupes. Elle était mère du duc
de Sully, fait chevalier de l'ordre en 1688, et de la duchesse du Lude.
De son second mari, elle n'eut point d'enfants et devint princesse du
sang longtemps après sa mère, à titre de sa veuve. Le roi en prit le
deuil pour quinze jours, mais il ne lui fit faire aucun honneur
particulier à ses obsèques. M\textsuperscript{me} de Laval, sa sœur
aînée, mère du duc, cardinal et chevalier de Coislin en premières noces,
et de la maréchale de Rochefort en secondes, jalouse de son rang et qui
d'ailleurs n'aimait rien et tombait volontiers sur chacun, dit, en
apprenant sa mort, qu'elle avait toujours bien cru que sa sœur mourrait
jeune par tous les remèdes qu'elle faisait.

Le vieux Grancey mourut en même temps et au même âge, marié pour la
quatrième fois depuis six semaines. Il était lieutenant général avant la
paix des Pyrénées. En ces temps-là on allait vite, puis choisi ou
laissé\,; et c'est ainsi qu'on fait des généraux utiles, et non pas des
gens usés dont le corps ne peut plus aller. Celui-ci était demeuré
depuis obscur et dans la débauche, toujours chez lui en Normandie, et
sans avoir rien de recommandable que d'être le fils et le père de deux
maréchaux de France.

Le roi donna quatre cent mille livres de brevet de retenue à La
Vrillière sur sa charge de secrétaire d'État.

Il refusa en même temps Troisvilles, que l'usage fait prononcer
Tréville, pour être de l'Académie française, où il avait été élu\,; il
répondit qu'il ne l'approuvait pas et qu'on en élût un autre.
Troisvilles était un gentilhomme de Béarn, de beaucoup d'esprit et de
lecture, fort agréable et fort galant. Il débuta très heureusement dans
le monde, où il fut fort recherché et fort recueilli par des dames du
plus haut parage, et de beaucoup d'esprit et même de gloire, avec qui il
fut longtemps plus que très bien. Il ne se trouva pas si bien de la
guerre que de la cour, les fatigues ne convenaient pas à sa paresse, ni
le bruit des armes à la délicatesse de ses goûts. Sa valeur fut accusée.
Quoi qu'il en fût, il se dégoûta promptement d'un métier qu'il ne
trouvait pas fait pour lui. Il ne put être supérieur à l'effet que
produisit cette conduite\,; il se jeta dans la dévotion, abdiqua la
cour, se sépara du monde. Le genre de piété du fameux Port-Royal était
celui des gens instruits, d'esprit et de bon goût. Il tourna donc de ce
côté-là, se retira tout à fait, et persévéra dans la solitude et la
grande dévotion plusieurs années. Il était facile et léger. La diversion
le tenta\,; il s'en alla en son pays, il s'y dissipa\,; revenu à Paris,
il s'y livra aux devoirs pour soulager sa faiblesse, il fréquenta les
toilettes, le pied lui glissa, de dévot il devint philosophe\,; il se
remit peu à peu à donner des repas recherchés, à exceller en tout par un
goût difficile à atteindre, en un mot il se fit soupçonner d'être devenu
grossièrement épicurien. Ses anciens amis de Port-Royal, alarmés de
cette vie et des jolis vers auxquels il s'était remis, dont la
galanterie et la délicatesse étaient charmantes, le rappelèrent enfin à
lui-même et à ce qu'il avait été\,; mais il leur échappa encore, et sa
vie dégénéra en un haut et bas de haute dévotion, et de mollesse et de
liberté qui se succédèrent par quartiers, et en une sorte de problème,
qui, sans l'esprit qui le soutenait et le faisait désirer, l'eût tout à
fait déshonoré et rendu parfaitement ridicule. Ses dernières années
furent plus suivies dans la régularité et la pénitence, et répondirent
mieux aux commencements de sa dévotion. Ce qu'il en conserva dans tous
les temps fut en entier éloignement de la cour, dont il ne rapprocha
jamais après l'avoir quittée, une fine satire de ce qui s'y passait, que
le roi lui pardonna peut-être moins que l'attachement à Port-Royal.
C'est ce qui lui attira ce refus du roi pour l'Académie, si déplacée
d'ailleurs avec cette haute profession de dévotion. Le roi ne lui manqua
pas ce coup de verge faute de meilleure occasion. Il s'en trouvera dans
la suite de voir quel crime c'était, non de lèse-majesté, mais de
lèse-personne de Louis XIV, que faire profession de ne le jamais voir,
qu'il était acharné à venger. Troisvilles était riche et ne fut jamais
marié.

Les fanatiques, battus et pris en diverses rencontres, demandèrent, vers
la mi-mai, à parler sur parole à Lalande, qui servait d'officier général
sous le maréchal de Villars. Cavalier, leur chef, qui était un
{[}aventurier{]}, mais qui avait de l'esprit et de la valeur, demanda
amnistie pour lui, pour Roland, un autre de leurs chefs, pour un de
leurs officiers qui avait pris le nom de Catinat, et pour quatre cents
hommes qu'ils avaient là avec eux, un passeport et une route pour eux
tous jusque hors du royaume, permission à tous les autres qui voudraient
sortir du royaume d'en sortir à leurs dépens, liberté de vendre leurs
biens à tous ceux qui désireraient de s'en défaire, enfin le pardon à
tous les prisonniers de leur parti. Cavalier vit ensuite le maréchal de
Villars avec une égalité de précautions et de gardes qui fut trouvée
fort ridicule. Il quitta les fanatiques moyennant douze cents livres de
pension et une commission de lieutenant-colonel\,; mais Roland ne
s'accommoda point et demeura le chef du parti, qui continua à donner de
la peine. Ce fut un concours de monde scandaleux pour voir Cavalier
partout où il passait. Il vint à Paris et voulut voir le roi, à qui
pourtant il ne fut point présenté. Il rôda ainsi quelque temps, ne
laissa pas de demeurer suspect, et finalement passa en Angleterre, où il
obtint quelque récompense. Il servit avec les Anglais\,; et il est mort
seulement cette année fort vieux dans l'île de Wight, où il était
gouverneur pour les Anglais depuis plusieurs années, avec une grande
autorité et de la réputation dans cet emploi.

Enfin, à la mi-mai, Barbezières, sorti des prisons de Gratz, fut remis
dans Casal à M. de Vendôme. Il avait été gardé à vue avec la dernière
dureté et si mal traité qu'il en tomba fort malade. Averti de son état,
il demanda un capucin\,; quand il fut seul avec lui, il le prit à la
barbe, qu'il tira bien fort pour voir si elle n'était point fausse et si
ce n'était point un capucin supposé. Ce moine se trouva un bon homme
qui, gagné par la compassion, alla lui-même avertir M. de Vendôme. Outre
le devoir de général, il aimait particulièrement Barbezières, tellement
qu'il manda aux ennemis qu'il était informé de leur barbarie sur un
lieutenant général des armées du roi, et qu'il allait traiter de même
tous les prisonniers qu'il tenait, et sur-le-champ l'exécuta. Cela fit
traiter honnêtement Barbezières et en prisonnier de guerre, jusqu'à ce
qu'il fut enfin renvoyé.

M. de Vendôme et son frère repaissaient le roi toutes les semaines par
des courriers que chacun d'eux envoyait de son armée, et souvent plus
fréquemment de projets et d'espérances d'entreprises qui s'allaient
infailliblement exécuter deux jours après, et qui toutes s'en allaient
en fumée. On comprenait aussi peu une conduite si propre à décréditer,
que la persévérance du roi à s'en laisser amuser et à être toujours
content d'eux\,; et cette suite si continuelle et si singulière de
toutes leurs campagnes prouve peut-être plus l'excès du pouvoir qu'eut
toujours auprès de lui leur naissance et la protection pour cela même de
M. du Maine, conséquemment de M\textsuperscript{me} de Maintenon, que
tout ce qu'on lui a vu faire avant et depuis pour les bâtards comme
tels. De temps en temps quelque petite échauffourée soutenait leur
langage, dans un pays si coupé où deux grandes armées jouaient aux
échecs l'une contre l'autre. À la mi-mai M. de Vendôme tenta l'exploit
de chasser de Trin quelques troupes impériales\,; il y arriva trop tard,
à son ordinaire, et trouva les oiseaux envolés. Il fit tomber sur une
arrière-garde qui se trouva si bien protégée par l'infanterie postée en
divers lieux avantageux sur leur retraite, qu'elle se fit très bien
malgré lui. Il leur tua quatre cents hommes et prit force prisonniers,
entre autres Vaubrune, un de leurs officiers généraux, grand partisan et
fort hasardeux. Qui compterait exactement ce que M. de Vendôme mandait
au roi chaque campagne qu'il tuait ou prenait aux ennemis ainsi en
détail, y trouverait presque le montant de leur armée. C'est ainsi qu'en
supputant les pertes dont les gros joueurs se plaignent le long de
l'année, il s'est trouvé des gens qui, à leur dire, avaient perdu plus
d'un million, et qui en effet n'avaient jamais perdu cinquante mille
francs. La licence et la débauche, l'air familier avec les soldats et le
menu officier faisait aimer M. de Vendôme de la plupart de son armée.

L'autre partie, rebutée de sa paresse, de sa hauteur, surtout de
l'audace de ce qu'il avançait en tout genre, et retenue par la crainte
de son crédit et de son autorité, laissait ses louanges poussées à
l'excès sans contradiction aucune, qui en faisaient un héros à grand
marché\,; et le roi, qui se plaisait à tout ce qui en pouvait donner
cette opinion, devenait sans cesse le premier instrument de la tromperie
grossière dans laquelle il était plongé à cet égard.

Le fils de Vaudemont, nouveau feld-maréchal de l'empereur, et qui
commandait son armée à Ostiglia, y mourut en quatre jours de temps. Ce
fut pour lui, pour sa sœur et pour ses deux nièces une très sensible
affliction. La politique leur fit cacher autant qu'ils le purent une
douleur inutile puisqu'il n'y a point de remède. M\textsuperscript{lle}
de Lislebonne et M\textsuperscript{me} d'Espinoy ne purent s'empêcher
d'en laisser voir la profondeur à quelques personnes, ou par confiance,
ou peut-être plus encore de surprise. Cette remarque suffit pour fournir
aux réflexions.

Le vieux maréchal de Villeroy, grand routier de cour, disait plaisamment
qu'il fallait tenir le pot de chambre aux ministres tant qu'ils étaient
en puissance, et de leur renverser sur la tête sitôt qu'on s'apercevait
que le pied commençait à leur glisser. C'est la première partie de ce
bel apophtegme que nous allons voir pratiquer au maréchal de Tessé, en
attendant que nous lui voyions accomplir pleinement l'autre partie. Avec
la même bassesse qu'il s'était conduit en Italie avec le duc de Vendôme,
malgré les ordres si précis du roi de prendre sans ménagement le
commandement sur lui, avec la même accortise il fit la navette avec La
Feuillade en Dauphiné et en Savoie, pour le laisser en chef quelque part
et y accoutumer le roi. D'accord avec Chamillart, il fit le malade quand
il en fut temps, le fut assez longtemps pour se rendre inutile et
obtenir enfin un congé qui laissât La Feuillade pleinement en chef d'une
manière toute naturelle, et en état de recevoir comme nécessairement la
patente, le caractère et les appointements de général d'armée sans que
le roi s'en pût dédire. C'est aussi ce qui s'exécuta de la sorte. Après
ce qu'on avait fait pour lui et la situation et la conjoncture où il se
trouvait, le roi, obsédé de son ministre, ne put reculer et ne voulut
pas même le laisser apercevoir qu'il en eût envie. La Feuillade succéda
donc en tout à Tessé dans les parties du Dauphiné, de la Savoie et des
vallées. Il fallait en profiter pour, de ce chausse-pied, aller à mieux
et en attendant faire parler de soi. Il alla donc former le siège de
Suse, d'où il envoya force courriers. Le fort de la Brunette pensa lui
faire abandonner cette place. Il ne manqua pas de jouer sur le mot avec
un air de galanterie militaire que son beau-père sut faire valoir. Ce
fort pris, Bernardi, gouverneur de Suse, se défendit si mal qu'il
capitula le 16 juin, sans qu'il y eût aucune brèche, ni même qu'il pût y
en avoir sitôt. Le chevalier de Tessé en apporta la nouvelle. Cette
honnêteté était bien due à la complaisance de son père. L'exploit fut
fort célébré à la cour, après lequel ce nouveau général d'armée se
tourna à de nouveaux, mais ce ne fut que contre les barbets\footnote{Les
  barbets étaient les Vaudois du Piémont. Ils tiraient, dit-on, leur nom
  de ce qu'ils avaient pour chefs des ministres, qu'ils appelaient
  \emph{barbes} ou anciens.} des vallées. Il ne fallut pas demeurer
oisif, mais peloter en attendant partie, et se conserver cependant en
exercice de général d'armée pour le devenir plus solidement.

En même temps, en ce mois de juin, Phélypeaux arriva de Turin et salua
le roi, qui aussitôt l'entretint longtemps dans son cabinet. C'était un
grand homme bien fait, de beaucoup d'esprit et de lecture, naturellement
éloquent, satirique, la parole fort à la main, avec des traits et
beaucoup d'agrément, et quand il le voulait de force. Il mit ces talents
en usage, et sans contrainte, pour se plaindre de tout ce qu'il avait
souffert les six derniers mois qu'il avait demeuré en Piémont, ou à
Turin, ou à Coni, où il fut gardé étroitement et où on lui refusait
jusqu'au nécessaire de la vie. Ses derniers propos avec M. de Savoie
furent assommants pour un prince qui se sentait autant que celui-là, et
ses réponses encore plus piquantes, par leur sel et leur audace, aux
messages qu'il lui envoya souvent depuis. Il dit même aux officiers qui
le gardaient à Coni qu'il espérait que le roi serait maître de Turin
avant la fin de l'année, que lui en serait fait gouverneur, qu'il y
ferait raser d'abord la maison où il avait été arrêté, et qu'il y ferait
élever une pyramide avec une inscription en plusieurs langues, par
laquelle il instruirait la postérité des rigueurs avec lesquelles M. de
Savoie avait traité un ambassadeur de France, contre le droit des gens,
contre l'équité et la raison. Il avait fait une relation de ce qui
s'était passé à son égard depuis les premiers événements de la rupture,
très curieuse et bien écrite, où il n'épargnait pas M. de Savoie ni sa
cour. Il en montra quelques copies, qui furent fort recherchées et qui
méritent de l'être toujours. Le malheur de l'État, attaché à la fortune
de La Feuillade, ne permit pas à Phélypeaux de jouir de sa vengeance, ni
la longueur de sa vie de voir les horreurs dans lesquelles M. de Savoie
finit la sienne. Ce Phélypeaux était un vrai épicurien qui croyait tout
dû à son mérite, et il était vrai qu'il avait des talents de guerre et
d'affaires, et tout possible par l'appui de ceux de son nom qui étaient
dans le ministère\,; mais particulier et fort singulier, d'un commerce
charmant quand il voulait plaire ou qu'il se plaisait avec les gens\,;
d'ailleurs épineux, difficile, avantageux et railleur. Il était pauvre
et en était fâché pour ses aises, ses goûts très recherchés et sa
paresse.

Il était frère d'un évêque de Lodève, plus savant, plus finement
spirituel et plus épicurien que lui, plus aisé aussi dans sa caisse,
qui, par la tolérance de Bâville et l'appui de ceux de son nom dans le
ministère, maniait fort le Languedoc depuis la chute du cardinal Bonzi.
Il survécut son frère, entretenait des maîtresses publiquement chez lui,
qu'il y garda jusqu'à sa mort, et tout aussi librement ne se faisait
faute de montrer, et quelquefois de se laisser entendre, qu'il ne
croyait pas en Dieu. Tout cela lui fut souffert toute sa vie sans le
moindre avis de la cour, ni la plus légère diminution de crédit et
d'autorité. Il n'avait fait que cela toute sa vie, mais il s'appelait
Phélypeaux. Il s'en fallait bien que le cardinal Bonzi, avec tous ses
talents, ses services, ses ambassades, eût jamais donné le quart de ce
scandale\,; et il en fut perdu\,! Ce Lodève ne sortait presque point de
sa province, mourut riche et vieux, car il sut aussi s'enrichir, et
laissa un tas de bâtards. Phélypeaux eut en arrivant la place de
conseiller d'état d'épée vacante par la mort de Briord.

Le duc de Grammont avait eu enfin la permission de voir la princesse des
Ursins sur sa route. Ce fut le premier adoucissement qu'elle obtint
depuis sa disgrâce. Le désir de préparer à mieux fit accorder cette
liberté. Le prétexte en fut de ne pas aigrir la reine pour une bagatelle
et ne pas mettre le duc de Grammont hors d'état de pouvoir traiter
utilement avec elle\,; mais il ne sut pas en profiter. Battu de
l'oiseau, à son départ, sur la déclaration de son mariage, il craignit
tout et ne fut point assez avisé pour se bien mettre avec cette femme si
importante dans un tête-à-tête dont le roi ne pouvait savoir le détail,
et s'aplanir par là toutes les épines que la sécheresse de sa part en
cette entrevue éleva contre lui de toutes parts à la cour d'Espagne.

Il y arriva les premiers jours de juin. Il trouva le roi avec l'abbé
d'Estrées sur la frontière de Portugal, où, malgré la criminelle disette
de tout ce qui est nécessaire à l'entretien des troupes, des places et
de la guerre, Puységur avait fait des prodiges pour y suppléer, dont le
duc de Berwick avait su profiter par un détail de petits avantages qui
découragèrent les ennemis et lui facilitèrent les entreprises\,; il prit
à discrétion Castelbranco, où il se trouva quantité de farines qui
furent d'un grand secours, beaucoup d'armes et les tentes de la suite du
roi de Portugal. De là il marcha au général Fagel, qui fut battu et fort
poursuivi\,; il pensa être pris\,; il y eut six cents prisonniers avec
tous leurs officiers\,; et sans les montagnes, pour vingt hommes qu'il
en coûta au duc, rien ne serait échappé du corps de Fagel, qui s'y
dispersa en désordre. Portalègre et d'autres places suivirent ces succès
et augmentèrent bien le crime d'Orry, comme je l'ai dit ailleurs, par la
conquête du Portugal, alors sans secours, qu'avec les précautions sur
lesquelles on comptait à l'ouverture de la campagne, il aurait été
facile de faire, au lieu que les secours ayant eu le temps d'arriver
avant le printemps suivant, ce côté-là devint le plus périlleux, et
celui par lequel l'Espagne fut plus d'une fois au moment d'être perdue.
Berwick avait d'abord pris Salvatierra avec dix compagnies à discrétion,
et fait divers autres petits exploits. Ce fut pendant cette campagne que
le roi d'Espagne se forma un régiment des gardes espagnoles dont le
comte d'Aguilar fut fait colonel. Ce grand d'Espagne reviendra plusieurs
fois sur la scène. On le fera connaître dans la suite.

Les armées de Flandre et d'Allemagne étaient dans un grand mouvement
depuis l'ouverture de la campagne l'empereur serré de près par les
mécontents de Hongrie, ce royaume tout révolté, le commerce intercepté
dans la plupart des provinces héréditaires qui en sont voisines, Vienne
même dans la confusion par les dégâts et les courses que souffraient non
seulement sa banlieue, mais ses faubourgs qui étaient insultés, et
l'empereur qui avait vu brûler sa ménagerie et avait éprouvé en personne
le danger des promenades au dehors\,; une situation si pénible porta
toute son attention sur la Bavière. Il craignit tout des succès d'un
prince qui, à la tête d'une armée française et de ses propres troupes,
pourrait donner la loi à l'Allemagne et l'enfermer entre les mécontents
et lui à n'avoir plus d'issue. Le danger ne parut pas moins grand à ses
alliés\,; de sorte que la résolution fut prise de porter toutes leurs
forces dans le cœur de l'empire. C'est ce qui rendit les premiers temps
de la campagne de Flandre si incertains par le soin que les ennemis
eurent de cacher leur projet pour dérober des marches au maréchal de
Villeroy, et gagner le Rhin longtemps avant lui, s'il était possible. Le
maréchal de Tallard, qui avait passé le Rhin de bonne heure, s'avançait
cependant vers les gorges des montagnes\,; il n'y trouva aucune
difficulté, et il passa la journée du 18 mai avec l'électeur de Bavière.

Le duc de Marlborough, avancé vers Coblentz, laissait en incertitude
d'une entreprise sur la Moselle, ou de vouloir seulement attirer le gros
des troupes de ce côté-là\,; mais bientôt, pressé d'exécuter son projet,
il marcha à tire-d'aile au Rhin et le passa à Coblentz le 26 et le 27
mai. Le maréchal de Villeroy venu jusqu'à Arlon craignit encore un
Hoquet, que l'Anglais, embarquant son infanterie, la portât en Flandre
bien plus tôt qu'il n'y pourrait être retourné, et ne fît quelque
entreprise vers la mer. Dans ce soupçon, il laissa une partie de son
infanterie assez près de la Meuse pour pouvoir joindre le marquis de
Bedmar à temps, et lui avec le reste de sa cavalerie se mit à suivre
l'armée ennemie, tandis que M. de Bavière et le prince Louis de Bade se
côtoyaient de fort près. Tallard, sur les nouvelles de la cour et du
maréchal de Villeroy, avait quitté l'électeur et fait repasser le Rhin à
son armée. Il s'était avancé à Landau, et le maréchal de Villeroy avait
passé la Moselle entre Trèves et Thionville. Le marquis de Bedmar était
demeuré en Flandre à commander les troupes françaises et espagnoles qui
y étaient restées, et M. d'Overkerke celles des ennemis. Marlborough
cependant passa le Mein entre Francfort et Mayence, et passa par le
Bergstras sur Ladenbourg pour y passer le Necker. Les maréchaux de
Villeroy et de Tallard se virent, et se concertèrent, les troupes du
premier sur Landau, celles du second sous Neustadt, d'où Tallard remena
son armée passer le Rhin sur le pont de Strasbourg le 1er juillet. Alors
celle de Marlborough était arrivé à Ulm, et le prince Eugène, parti de
Vienne, s'était rendu à Philippsbourg, d'où il était allé camper à
Rothweil pour couvrir le Würtemberg, et ce dessein manqué mena son armée
à Ulm, où il conféra avec le prince Louis de Bade et le duc de
Marlborough qui avaient les leurs à portée.

Le maréchal de Villeroy suivit Tallard et passa le Rhin\,; il entra dans
le commencement des vallées de manière à pouvoir communiquer avec
Tallard, et de le joindre même au besoin par des détachements avancés.
Tous deux avaient perdu dans le Palatinat une précieuse quinzaine en
revues et en fêtes et en attente des ordres de la cour. Villeroy,
accoutumé à maîtriser Tallard son cousin, son courtisan et son protégé,
toute sa vie, n'en rabattit rien pour le voir à la tête d'une armée
indépendante de lui. Tallard, devenu son égal au moins en ce genre,
trouva cette hauteur mal placée et voulut secouer un joug trop dur, et
que l'autre n'avait aucun droit de lui imposer. Cela fit des scènes
assez ridicules, mais qui n'éclatèrent pas jusque dans le gros de
armées. Tallard plus sage comprit pourtant qu'à la cour leur égalité
cesserait, et le besoin de ne se pas brouiller avec son ancien
protecteur les remit un peu plus en mesure. Cette perte de temps fut le
commencement des malheurs que le roi éprouva en Allemagne. Tallard
devait passer et le maréchal de Villeroy garder les gorges\,; cela se
fit, mais trop tard. Donawerth est un passage très important sur le
Danube. La ville ne vaut rien\,: on fit des retranchements à la hâte sur
l'arrivée de tant de troupes des alliés, et le comte d'Arco, maréchal
des troupes de Bavière, se mit dedans\,; il fut attaqué avant que ses
retranchements fussent achevés. Il soutint très bravement et avec
capacité ses retranchements depuis six heures du soir jusqu'à neuf que,
se voyant hors d'état d'y tenir davantage, il se retira en bon ordre à
Donawerth qu'il abandonna le lendemain, passa le Danube, puis le Lech,
et se retira à Rhein, d'où il compta pouvoir empêcher aux ennemis le
passage de la rivière. Arco avait du talent pour la guerre et une grande
valeur\,; il était Piémontais d'origine, et avait toujours été attaché
au service de Bavière\,; il y était parvenu avec réputation au premier
et unique grade militaire de ce pays-là, qui est maréchal, et M. de
Bavière avait obtenu qu'obéissant sans difficulté aux maréchaux de
France, il commanderait nos lieutenants généraux et ne roulerait point
avec eux, en sorte que, par cet expédient que la facilité du roi accepta
par les liaisons étroites où il était avec l'électeur, le comte d'Arco,
qui se faisait appeler franchement le maréchal d'Arco, commandait nos
troupes jointes à celles de l'électeur en l'absence de ce prince et des
maréchaux de France, qui était une sorte de réciproque avec eux, et pour
les honneurs militaires il les avait pareils à eux dans ses troupes, et
dans les nôtres fort approchant des leurs. On prétendit que les
Impériaux eurent en ce combat presque tous leurs généraux et leurs
officiers tués ou blessés, six mille morts et huit mille blessés\,; ce
qu'il y a de plus avéré, c'est qu'on n'y perdit guère que mille François
et cinq à six cents Bavarois. M. d'Arco présuma trop et se trompa. Les
Impériaux passèrent le Danube tout de suite après avoir occupé Donawerth
qu'il n'avait pu tenir, traversèrent le Lech sans lui donner loisir de
se reconnaître, l'obligèrent de leur quitter Rhein, où il s'était
retiré, d'où ils dirigèrent leur marche droit sur Munich. L'électeur,
effrayé de cette rapidité, et qui avait déjà Marlborough en tête, cria
au secours. Tallard, qui avait ordre de s'établir dans le Würtemberg, et
qui pour cela assiégeait Villingen, que nous disons Fillingue, abandonna
ce projet et se mit en marche droit vers l'électeur. Il faut ici faire
une pause pour ne perdre pas haleine dans les tristes succès d'Allemagne
en les racontant tout de suite, et retourner un peu en arrière avant de
revenir au Danube.

Cependant Overkerke voulut profiter de la faiblesse dans laquelle le
marquis de Bedmar avait été laissé aux Pays-Bas. Le Hollandais bombarda,
dix heures durant, Bruges où il ne fit presque point de dommage, et se
retira très promptement tout au commencement de juillet\,; et, à la fin
du même mois, il jeta pendant deux jours trois mille bombes dans Namur,
qui brûlèrent deux magasins de fourrages et coûtèrent à. la ville
environ cent cinquante mille livres de dommage.

M. de Vendôme assiégea enfin Verceil. Il le promettait au roi depuis
longtemps\,; il y ouvrit la tranchée le 16 juin. La place capitula le 19
juillet, mais Vendôme les voulut prisonniers de guerre. Il leur permit
seulement les honneurs militaires et de sortir par la brèche au bas de
laquelle ils posèrent les armes. Trois mille trois cents hommes
sortirent sous les armes. On trouva dedans tout le nécessaire pour le
plus grand siège. Ce fut le prince d'Elbœuf qui apporta cette nouvelle.

M. de Savoie ne cessait de secourir les fanatiques\,; le chevalier de
Roannais prit une tartane\footnote{Les tartanes étaient de grosses
  barques de pêcheurs, qui allaient à rames et à voiles. Elles étaient
  en usage sur la Méditerranée.} pleine d'armes et de réfugiés, et en
coula une autre à fond, chargée de même. Toutes deux étaient parties de
Nice\,; une troisième, pareillement équipée, échoua et fut prise sur les
côtes de Catalogne, que le vent avait séparée de ces deux. Il y avait de
plus un vaisseau rempli d'armes, de munitions et de ces gens-là qu'il ne
put prendre. L'abbé de La Bourlie y était embarqué, après être sorti du
royaume sans aucun prétexte ni cause de mécontentement. Il s'était
arrêté longtemps à Genève, puis avait été trouver M. de Savoie, qui le
jugea propre à aller soutenir les fanatiques en Languedoc. Comptant y
arriver incessamment, il s'y était annoncé en y faisant répandre
quantité de libelles très insolents, et très séditieux, où il prenait la
qualité de chef des mécontents et de Farinée des hauts alliés en France.
On surprit aussi de ses lettres à La Bourlie, son frère, qu'il conviait
à le venir trouver et se mettre à la tête de ces braves gens, et les
réponses de ce frère, qui témoignaient l'horreur qu'il avait de cette
folie. Celui-ci venait d'en faire plus d'une\,: c'était un homme d'une
grande valeur, mais un brigand, et d'ailleurs intraitable. Il avait le
régiment de Normandie, qu'il quitta étant brigadier pour de fâcheuses
affaires qu'il s'y fit, et se retira dans sa province. Quelque temps
après il fut volé dans sa maison\,; il soupçonna un maître valet, à qui,
de son autorité privée, il fit donner en sa présence une très rude
question. Cette affaire éclata, en renouvela d'autres fort vilaines qui
s'étaient assoupies. Il fut arrêté et mené à Paris dans la Conciergerie.
L'abbé avait beaucoup de bénéfices, violent et grand débauché, comme La
Bourlie. Nous les verrons finir tous deux très misérablement, l'un en
France, l'autre en Angleterre. Ces deux frères furent de cruels pendants
d'oreilles pour Guiscard, leur aîné, dans sa fortune et sa richesse.
Leur père, qui s'appelait La Bourlie, qui est leur nom, était un
gentilhomme de valeur qui avait été à mon père et qui en eut le don de
quelques métairies au marais de Blaye, lorsque mon père prit soin de le
faire dessécher. La Bourlie fit fortune\,; il succéda à Dumont dans la
place de sous-gouverneur du roi, et eut après le gouvernement de Sedan.
Il conserva toute sa vie de l'attachement et de la reconnaissance pour
mon père. C'était aussi un fort galant homme. Guiscard s'en est toujours
souvenu avec moi, avec son cordon bleu et ses ambassades, ses
gouvernements et ses commandements.

Augicourt mourut ayant six mille livres de pension du roi et deux mille
sur l'ordre de Saint-Louis, sans ce qui ne se savait pas et qu'on avait
lieu de croire aller haut par son peu de bien et les commodités qu'il se
donnait et avec une cassette toujours bien fournie. C'était un
gentilhomme de Picardie, né sans biens, avec beaucoup d'esprit,
d'adresse, de valeur et de courage d'esprit. M. de Louvois, qui
cherchait à s'attacher des sujets de tête et de main dont il pût se
servir utilement en beaucoup de choses, démêla celui-ci dans les
troupes, qui, sans bien, n'espérant pas d'y faire aucune fortune,
consentit volontiers à quitter son emploi pour entrer chez M. de
Louvois. Il n'y fut pas longtemps sans être employé\,; il s'acquitta
bien de ce dont il était chargé, et mérita de l'être d'affaires secrètes
et d'autres à la guerre en différentes occasions. Il y fit bien les
siennes et parvint à une grande confiance de M. de Louvois, qui le fit
connaître au roi avec qui ces affaires secrètes lui procurèrent divers
entretiens pour lui rendre un compte direct ou recevoir directement ses
ordres. La bourse grossissait, mais ce métier subalterne qui ne menait
pas à une fortune marquée dégoûta à la fin un homme gâté par la
confiance d'un aussi principal ministre qu'était Louvois et qui se
mêlait de tout, et par quelque part aussi en celle du roi, et un homme
devenu audacieux et né farouche. Après un assez long exercice de ce
train de vie, il fut accusé de faire sa cour au roi aux dépens du maître
qui le lui avait produit. Quoi qu'il en soit, M. de Louvois le chassa de
chez lui avec éclat et s'en plaignit, mais sans rien articuler de
particulier, comme du plus ingrat, du plus faux, du plus indigne de tous
les hommes.

Augicourt fut aussi réservé en justification que M. de Louvois en
accusation. Il se contenta de dire qu'il l'avait bien servi, mais qu'il
n'y avait plus moyen de durer avec lui. Le roi ne se mêla point du tout
de cette rupture, mais il continua toujours de le voir en particulier et
de s'en servir en plusieurs choses secrètes. Il ne lui prescrivit rien à
l'égard de Louvois, le laissa paraître publiquement à la cour et
partout, lui augmenta de temps en temps ses bienfaits publiquement, mais
par mesure. En secret, il lui donnait gros souvent, lui faisait toutes
les petites grâces qu'il lui pouvait faire, et assez volontiers à ceux
pour qui il les demandait. Outre les audiences secrètes, Augicourt
parlait au roi très souvent et longtemps, allant à la messe ou chez
M\textsuperscript{me} de Maintenon. Quelquefois le roi l'appelait et lui
parlait ainsi en allant, et il était toujours bien reçu et bien écouté,
et paraissait fort libre avec le roi en l'approchant, et le roi avec
lui. Il voyait aussi, et quand il voulait, M\textsuperscript{me} de
Maintenon en particulier, et il était d'autant mieux avec elle, qu'elle
était plus mal avec Louvois. Après sa mort, et Barbezieux en sa place,
Augicourt vécut et fut toujours traité comme il l'avait été
jusqu'alors\,; il ne craignait point de rencontrer ces ministres ni
leurs parents, et ce fut un grand crève-cœur pour Louvois et pour
Barbezieux ensuite et pour tous les Tellier, de voir cet homme se
conserver sur le pied où il était. Du reste, haï, craint, méprisé comme
le méritait sa conduite avec M. de Louvois, soupçonné d'être rapporteur,
et personne ne voulant se brouiller pour Augicourt avec les Tellier qui
l'abhorraient, il n'entrait dans aucune maison de la cour que chez Livry
et chez M. le Grand, qui étaient des maisons ouvertes, où on jouait dès
le matin, toute la journée et fort souvent toute la nuit. Augicourt
était gros joueur et net, mais de mauvaise humeur, et au lansquenet,
public il jouait chez Monsieur avec lui, et à la cour avec Monseigneur.
En aucun temps, il ne fréquenta aucuns ministres ni aucuns généraux
d'armée\,: il était assez, vieux et point marié.

Verac venait de mourir depuis peu. Il s'appelait Saint-Georges, et il
était homme de qualité\,: la lieutenance générale de Poitou, où il avait
des terres, fit sa fortune. Il avait été huguenot. Lui et Marillac,
intendant de Poitou, lors de la révocation de l'édit de Nantes et des
barricades qui furent exercées contre les huguenots, tous deux crurent y
trouver leur fortune, tous deux se signalèrent en cruautés, en
conversions, tous deux donnèrent le ton aux autres provinces, tous deux
en obtinrent ce qu'ils s'en étaient proposé. Verac en fut chevalier de
l'ordre en 1688, et Marillac conseiller d'État, par une grande
préférence sur ses anciens\,: il en a joui jusqu'à être doyen du
conseil, mais il a vu mourir ses deux fils sans enfants, qui lui
donnaient de justes et d'agréables espérances, l'un dans la robe,
l'autre à la guerre, sa fille et son gendre La Fayette, lieutenant
général, dont la fille unique fut grand'mère du duc de La Trémoille
d'aujourd'hui, morte encore avant son grand-père. Verac a été plus
heureux. Son fils est mort cette année 1741, estimé, aimé et considéré,
lieutenant général et chevalier de l'ordre en 1724, dont les enfants ne
sont pas tournés à la fortune, l'un par un asthme qui l'empêche de
servir, l'autre par être cadet et encore capitaine de cavalerie.

Deux mois depuis la mi juin jusqu'au 15 août de cette année, virent
diverses élévations de quatre hommes qui chacun fort différents ont eu
de grandes et de curieuses suites\,; on pourrait ajouter les plus
incroyables, et de ces choses dans lesquelles paraît toute la grandeur
de Dieu qui se joue des hommes, et qui prépare et tire de rien et de
néant les plus grands et les plus singuliers événements, ou qui dans un
ordre inférieur, selon le monde, découvre ce que c'est que la faiblesse
des instruments par lesquels il daigne soutenir sa vérité et l'Église.
Harley, auparavant orateur de la chambre basse, devint secrétaire
d'État\,; Le Blanc, intendant d'Auvergne\,; Leczinski, roi de Pologne\,;
et l'abbé de Caylus, évêque d'Auxerre\,; qui tous quatre, chacun en son
très différent genre, peuvent fournir les plus abondantes et, les plus
curieuses matières aux réflexions. On en verra assez sur Harley, dans
les Pièces\footnote{Voy., sur les pièces, t. Ier, p.~437, note.}, à
l'occasion de la paix d'Utrecht, et de ce qui la précéda à Londres, pour
que je n'aie rien {[}à dire{]} ici de lui. M. Le Blanc se trouvera en
son temps ici en entier. Du roi de Pologne, devenu beau-père du roi, il
n'y a qu'à admirer, et se mettre, non pas un doigt, mais tous les doigts
sur la bouche, et la main tout entière\,; et de M. d'Auxerre, les
bibliothèques sont pleines de lui, et il se trouvera lieu d'en parler.

Castel dos Rios, cet heureux ambassadeur d'Espagne, qui se trouva ici
lors de la mort de Charles II, eut ordre de se rendre à Cadix pour s'y
embarquer et aller au Pérou, dont il avait été nommé vice-roi, et où il
mourut après avoir rempli ce grand emploi et fort dignement pendant
plusieurs années.

Monasterol revint à Paris de la part de l'électeur de Bavière, et
présenta le comte d'Albert venu avec lui, qui, chassé du service de
France pour son duel, comme il a été dit en son temps, s'était attaché à
celui de Bavière, où il était maréchal de camp. Il allait de la part de
l'électeur en Espagne, où il devait aussi servir. L'abbé d'Estrées
arriva aussi d'Espagne dans l'épanouissement, et fut très bien reçu.

Chamillart fit en même temps deux nouveaux intendants des finances\,:
Rebours, son cousin germain et de sa femme, et Guyet, maître des
requêtes, dont la fille unique avait malheureusement pour elle épousé le
frère de Chamillart. Rien de si ignorant, ni en récompense de si
présomptueux et de si glorieux que ces deux nouveaux animaux. Le premier
s'était sûrement moulé sur le marquis de Mascarille\,; il l'outrait
encore. Tout était en lui parfaitement ridicule. L'autre, grave et
collet monté, faisait grâce de prêter l'oreille, à condition pourtant
qu'il ne comprenait rien de ce qu'on lui disait. Jamais un si sot homme
que celui-ci, jamais un si impertinent que l'autre\,; jamais rien de
plus indécrottable que tous les deux, et voilà les choix et les environs
des ministres, et ce que sont leurs familles quand ils ont la faiblesse
d'y vouloir trouver et avancer. Ils n'y trouvent aucun secours, ils
excitent le cri public, et ils préparent de loin leur propre perte.

La mort de l'abbesse de Fontevrault dans un âge encore assez peu avancé,
arrivée dans ce temps-ci, mérite d'être remarquée\,: elle était fille du
premier duc de Mortemart, et sœur du duc de Vivonne, de
M\textsuperscript{me} de Thianges et de lime de Montespan\,; elle avait
encore plus de beauté que cette dernière, et ce qui n'est pas moins
dire, plus d'esprit, qu'eux tous avec ce même tour, que nul autre n'a
attrapé qu'eux, ou avec eux par une fréquentation continuelle, et qui se
sent si promptement, et avec tant de plaisir. Avec cela très savante,
même bonne théologienne, avec un esprit supérieur pour le gouvernement,
une aisance et une facilité qui lui rendait comme un jeu le maniement de
tout son ordre et de plusieurs grandes affaires qu'elle avait
embrassées, et où il est vrai que son crédit contribua fort au succès\,;
très régulière et très exacte, mais avec une douceur, des grâces et des
manières qui la firent adorer à Fontevrault et de tout son ordre. Ses
moindres lettres étaient des pièces à garder, et toutes ses
conversations ordinaires, même celles d'affaires ou de discipline,
étaient charmantes, et ses discours en chapitre les jours de fête,
admirables. Ses sœurs l'aimaient passionnément, et malgré leur impérieux
naturel gâté par la faveur au comble, elles avaient pour elle une vraie
déférence. Voici le contraste. Ses affaires l'amenèrent plusieurs fois
et longtemps à Paris. C'était au fort des amours du roi et de
M\textsuperscript{me} de Montespan. Elle fut à la cour et y fit de
fréquents séjours, et souvent longs. À la vérité elle n'y voyait
personne, mais elle ne bougeait de chez M\textsuperscript{me} de
Montespan, entre elle et le roi M\textsuperscript{me} de Thianges et le
plus intime particulier. Le roi la goûta tellement qu'il avait peine à
se passer d'elle. Il aurait voulu qu'elle fût de toutes les fêtes de sa
cour, alors si galante et si magnifique. M\textsuperscript{me} de
Fontevrault se défendit toujours opiniâtrement des publiques, mais elle
n'en put éviter de particulières. Cela faisait un personnage extrêmement
singulier. Il faut dire que son père la força à prendre le voile et à
faire ses veaux, qu'elle fit de nécessité vertu, et qu'elle fut toujours
très bonne religieuse. Ce qui est très rare, c'est qu'elle conserva
toujours une extrême décence personnelle dans ces lieux et ces parties
où son habit en avait si peu. Le roi eut pour elle une estime, un goût,
une amitié que l'éloignement de M\textsuperscript{me} de Montespan ni
l'extrême faveur de M\textsuperscript{me} de Maintenon ne purent
émousser. Il la regretta fort et se fit un triste soulagement de le
témoigner. Il donna tout aussitôt cette unique abbaye à sa nièce, fille
de son frère, religieuse de la maison et personne d'un grand mérite.

\hypertarget{chapitre-xvi.}{%
\chapter{CHAPITRE XVI.}\label{chapitre-xvi.}}

1704

~

{\textsc{Naissance du premier duc de Bretagne.}} {\textsc{- Progrès des
mécontents.}} {\textsc{- Mesures des alliés pour la défense de
l'Allemagne.}} {\textsc{- Mouvements dans nos armées.}} {\textsc{-
Première faute principale.}} {\textsc{- Faute du maréchal de Villeroy.}}
{\textsc{- Marche et dispositions des armées.}} {\textsc{- Bataille
d'Hochstedt.}} {\textsc{- Bon et sage avis de l'électeur méprisé.}}
{\textsc{- Électeur de Bavière passe à Strasbourg, et par Metz à
Bruxelles.}} {\textsc{- Obscurité et rareté des nouvelles d'Allemagne.}}
{\textsc{- Silly, prisonnier, vient rendre compte au roi de la bataille
d'Hochstedt.}} {\textsc{- Digression sur Silly et sa catastrophe.}}
{\textsc{- Fautes de la bataille d'Hochstedt.}} {\textsc{- Cri public\,;
consternation, embarras\,; contraste des fêtes continuées pour la
naissance du duc de Bretagne.}}

~

Je devais marquer un peu plus tôt la naissance du fils aîné de Mgr le
duc de Bourgogne, arrivée à Versailles à cinq heures après midi, le
mercredi 25 juin. Ce fut une grande joie pour le roi, à laquelle la cour
et la ville prirent part jusqu'à la folie par l'excès des démonstrations
et des fêtes. Le roi en donna une à Marly et y fit les plus galants et
les plus magnifiques présents à M\textsuperscript{me} la duchesse de
Bourgogne, alors relevée. Malgré la guerre et tant de vifs sujets de
mécontentement de M. de Savoie, le roi lui écrivit pour lui donner part
de cette nouvelle, mais il adressa le courrier à M. de Vendôme pour
qu'il envoyât la lettre au duc de Savoie. On eut tout lieu de se
repentir de tant de joie, puisqu'elle ne dura pas un an, et de tant
d'argent dépensé si mal à propos en fêtes dans les conjonctures où on
était.

La grande alliance avait grande raison de tout craindre pour l'empereur,
et de porter toutes ses forces à sa défense. Les mécontents, devenus
maîtres d'Agria et de toute l'île de Schutt une deuxième fois depuis
l'avoir abandonnée, n'avaient pu en être chassés\,; le comte Forgatz, à
la tête de trente mille hommes entré en Moravie, y avait défait quatre
mille Danois et six mille hommes des pays héréditaires, leur avait tué
deux mille hommes, pris toute leur artillerie et leurs bagages, et
acculé le général Reizthaw, Danois, qui les commandait, dans un château.
Le même Forgatz défit ensuite le général Heister avec tout ce qu'il
avait pu rassembler de troupes pour s'opposer à eux et couvrir Vienne,
où la consternation et la frayeur furent extrêmes. Que n'avait-on point
à espérer dans une conjoncture si singulièrement heureuse, pour peu que
les armées des maréchaux de Marsin et de Tallard jointes à celle de
l'électeur de Bavière eussent eu le moindre des succès que promettaient
tant de forces unies au cœur de l'Allemagne, avec l'armée du maréchal de
Villeroy en croupe\,! On va voir ce que peut la conduite et la fortune,
ou pour mieux dire la Providence, qui se joue de l'orgueil et de la
prudence des hommes, et qui dans un instant relève et atterre les plus
grands rois.

Tallard arriva à Ulm le 28 août\footnote{Saint-Simon a écrit 28 août
  pour 28 juillet.}, et y séjourna deux jours pour laisser reposer son
armée\,; l'amena le 2 août sous Augsbourg, et joignit le 4 l'électeur et
le maréchal de Marsin. Dès lors l'électeur était poussé par Blainville,
à qui les mains démangeaient d'autant plus qu'avec les grandes parties
de guerre qu'il avait fait voir durant celle-ci et la considération
singulière qu'il s'était acquise, il n'espérait rien moins que le bâton
d'une action heureuse, porté par son ancienneté de lieutenant général et
par la faveur de sa famille. Legal, qu'une jolie action venait de faire
lieutenant général, comme je crois l'avoir marqué en son lieu, et qui
revenait de la cour où l'électeur l'avait envoyé comme un homme
intelligent et de confiance, secondait Blainville auprès de lui en
audacieux qui espère tout et ne regarde point d'où il est parti, et
l'électeur, plein de valeur et à la tête de trois armées complètes et
florissantes, pétillait de lui-même d'ardeur de s'en servir et de se
rendre maître de l'Allemagne par le gain d'une bataille qui aurait mis
l'empereur à sa merci, entre des mécontents victorieux déjà et les
armées de l'électeur triomphantes. Ces idées si flatteuses le perdirent.
Il ne discerna pas l'incertitude du succès d'avec la sûreté de celui de
ne rien entreprendre. Il se trouvait dans l'abondance et dans une
abondance durable, par les pays gras et neufs dont il était maître et
qu'il avait dans ses derrières et à l'un de ses côtés. Le vis-à-vis de
lui était ruiné par les armées ennemies qui, par le nombre de leurs
troupes, de leurs marches circulaires et croisées, de leur séjour, était
mangé. Leur derrière ne l'était pas moins. Il y avait peu de distance au
delà jusqu'au ravage qu'avaient fait les courses des mécontents. En un
mot, ces pays épuisés ne pouvaient fournir huit jours de subsistance à
ce grand nombre de troupes des alliés, et sans rien faire que les
observer, il fallait que, faute de subsistance, ils lui quittassent la
partie, et se retirassent assez loin pour chercher à vivre, pour que
l'électeur trouvât tout ouvert devant lui. N'avoir pas pris ce parti fut
la première faute et la faute radicale.

Marsin ne songeait, depuis qu'il était en Bavière, qu'à se rendre
agréable à l'électeur, et Tallard, gâté par sa victoire de Spire, et
cherchant aussi à plaire en courtisan, ne mit aucun obstacle à
l'empressement de l'électeur de donner une bataille. Il ne fut donc plus
question que de ce but, qui se trouva d'autant plus facile à atteindre,
qu'une bataille était tout le désir et toute la ressource des alliés
dans la position où ils se trouvaient. Le prince Louis de Bade
assiégeait Ingolstadt, et ne le pouvait prendre si la faim chassait le
duc de Marlborough, qui était l'armée opposée à l'électeur. Le prince
Eugène amusait le maréchal de Villeroy, destiné à la garde des
montagnes\,; il croyait avoir tout fait que d'avoir établi la
communication entre l'électeur et lui par de gros postes semés entre eux
deux. Il en avait sur le haut des montagnes, qui voyaient à revers le
camp du prince Eugène. Le maréchal le comptait uniquement occupé à
garder ses retranchements de Bihel, et l'empêcher de les attaquer. Il
fut averti que ce prince avait un autre dessein\,; il n'en voulut rien
croire. Le prince Eugène, informé de moment en moment des mouvement de
l'électeur, et qui n'était dans ses retranchements {[}que{]} pour
occuper le maréchal de Villeroy, et l'empêcher d'aller grossir les trois
armées de la sienne, se mesura assez juste pour l'amuser jusqu'au bout,
et partir précisément pour aller joindre Marlborough, de manière qu'il y
arrivât sûrement à temps, mais sans donner au maréchal celui d'en
profiter, ni sur son arrière-garde, ni par de nombreux détachements pour
fortifier l'électeur\,; c'est ce qu'il exécuta avec une capacité qui
dépassait de loin celle du maréchal de Villeroy, qui n'y sut pas
remédier après ne l'avoir pas voulu prévoir, et qui, après quelques
mouvements, demeura avec toute son armée dans ces gorges.

Cependant l'électeur marchait aux ennemis avec une merveilleuse
confiance\,: il arriva le matin du 12 août dans la plaine d'Hochstedt,
lieu de bon augure par la bataille qui y avait été gagnée. L'ordre de
celle de l'électeur fut singulier. On ne mêla point les armées\,: celle
de l'électeur occupa le centre commandée par d'Arco, Tallard avec la
sienne formait l'aile droite, et Marsin avec la sienne l'aile gauche,
sans aucun intervalle plus grand qu'entre le centre et les ailes d'une
même armée. L'électeur commandait le tout, mais Tallard présidait, et
comme il ne voyait pas à dix pas devant lui, il tomba en de grandes
fautes qui ne trouvèrent pas, comme à Spire, qui les réparât
sur-le-champ. Peu d'heures après l'arrivée de l'électeur dans la plaine
d'Hochstedt, il eut nouvelle que les ennemis venaient au-devant de lui,
c'est-à-dire, Marlborough et le prince Eugène, qui joignit son armée
avec la sienne, dans la marche de la veille. Rien ne fut mesuré plus
juste. Il avait laissé dix-sept bataillons et quelque cavalerie au comte
de Nassau-Weilbourg dans les retranchements de Bihel, pour continuer d'y
amuser le maréchal de Villeroy tant qu'il pourrait, et se retirer dès
que le maréchal désabusé tournerait sur lui\,; le prince Louis de Bade
était demeuré à son siège d'Ingolstadt. Nos généraux eurent toute la
journée à choisir leur champ de bataille et à faire toutes leurs
dispositions. Il était difficile de réussir plus mal à l'un et à
l'autre. Un ruisseau assez bon et point trop marécageux coulait
parallèlement au front de nos trois armées\,; une fontaine formait une
large et longue fondrière qui séparait presque les deux lignes du
maréchal de Tallard\,: situation étrange quand on est maître de choisir
son terrain dans une vaste plaine, et qui devint aussi très funeste.
Tout à fait à sa droite, mais moins avancé qu'elle, était le gros
village de Bleinheim, dans lequel, par un aveuglement sans exemple, il
mit vingt-six bataillons de son armée avec Clérembault, lieutenant
général, et Blansac, maréchal de camp, soutenus de cinq régiments de
dragons dans les haies du même village, et d'une brigade de cavalerie
derrière\,; c'était donc une armée entière pour garder ce village et
appuyer sa droite, et se dégarnir d'autant. La première bataille
d'Hochstedt, gagnée en ce même terrain, était un plan bon à suivre, et
une leçon présente dont beaucoup d'officiers généraux qui se trouvaient
là avaient été témoins\,; il paraît qu'on n'y songea pas. Entre deux
partis à prendre, ou de border le ruisseau parallèle au front des armées
pour en disputer le passage aux ennemis, et celui de les attaquer dans
le désordre de leur passage, tous deux bons, et le dernier meilleur, on
en prit un troisième\,: ce fut de leur laisser un grand espace entre nos
troupes et le ruisseau, et de leur laisser passer à leur aise pour les
culbuter après dedans, dit-on. Avec de telles dispositions, il n'était
point possible de douter que nos chefs fussent frappés d'aveuglement. Le
Danube coulait assez près de Bleinheim, qui eût été un appui de la
droite, en s'en approchant, meilleur que ce village, et qui n'avait pas
besoin d'être gardé.

Les ennemis arrivèrent le 13 août, se portèrent d'abord sur le ruisseau,
et y parurent presque avec le jour. Leur surprise dut être grande d'en
aviser nos armées si loin, qui se rangeaient en bataille. Ils
profitèrent de l'étendue du terrain qu'on leur laissait, passèrent le
ruisseau presque partout, se formèrent sur plusieurs lignes au deçà,
puis s'étendirent à leur aise sans recevoir la plus légère opposition.
Voilà de ces vérités exactes, mais sans aucune vraisemblance, et que la
postérité ne croira pas. Il était près de huit heures du matin quand
toute leur disposition fut faite, que nos armées leur virent faire sans
s'émouvoir. Le prince Eugène avec son armée avait la droite, et le duc
de Marlborough la gauche avec la sienne, qui fut ainsi opposée à celle
du maréchal de Tallard. Enfin elles s'ébranlèrent l'une contre l'autre,
sans que le prince Eugène pût obtenir le moindre avantage sur Marsin,
qui au contraire en eut sur lui, et qui était en état d'en profiter sans
le malheur de notre droite. Sa première charge ne fut pas heureuse. La
gendarmerie ploya, et porta un grand désordre dans la cavalerie qui la
joignait, dont plusieurs régiments firent merveilles. Mais deux
inconvénients perdirent cette malheureuse armée\,: la seconde ligne,
séparée de la première par la fondrière de cette fontaine, ne la put
soutenir à propos, et par le long espace qu'il fallait marcher pour
gagner la tête de cette fondrière et en faire le tour, le ralliement ne
se put faire parce que les escadrons des deux lignes ne purent passer
dans les intervalles les uns des autres, ceux de la seconde pour aller
ou pour soutenir la charge, ceux de la première pour se rallier derrière
la seconde\,; quant à l'infanterie, vingt-six bataillons dans Bleinheim
y laissèrent un grand vide, non en espace, car on avait rapproché les
bataillons restés en ligne, mais en front et en force. Les Anglais qui
s'aperçurent bientôt de l'avantage que leur procurait ce manque
d'infanterie, et du désordre extrême du ralliement de la cavalerie de
notre droite, en surent profiter sur-le-champ, avec la facilité de gens
qui se maniaient aisément dans la vaste étendue d'un bas terrain. Ils
redoublèrent les charges, et pour le dire en un mot, ils défirent toute
cette armée, dès cette première charge, si mal soutenue par les nôtres
que la fermeté de plusieurs régiments qui çà, qui là, ni la valeur et le
dépit des officiers généraux et particuliers ne purent jamais rétablir.
L'armée de l'électeur, entièrement découverte, et prise en flanc par les
mêmes Anglais, s'ébranla à son tour. Quelque valeur que témoignassent
les Bavarois, quelque prodige que fît l'électeur, rien ne put remédier à
cet ébranlement, mais la résistance au moins y fut grande. Ainsi l'armée
de Tallard battue et enfoncée dans le plus grand désordre du monde,
celle de l'électeur soutenant avec vigueur, mais ne pouvant résister par
devant et par le flanc tout à la fois, l'une en fuite, l'autre en
retraite, celle de Marsin chargeant et gagnant sur le prince Eugène, fut
un spectacle qui se présenta tout à la fois, pendant lequel le prince
Eugène crut plus d'une fois la bataille fort hasardée pour eux. En même
temps ceux de Bleinheim vigoureusement attaqués, non seulement surent se
défendre, mais poursuivre par deux fois les ennemis fort loin dans la
plaine, après les avoir repoussés, lorsque Tallard, voyant son armée
défaite, en fuite, poussa à Bleinheim pour en retirer les troupes avec
le plus d'ordre qu'il pourrait, et tâcher d'en faire quelque usage. Il
en était d'autant plus en peine, qu'il leur avait très expressément
défendu de le quitter, et d'en laisser sortir un seul homme quoi qu'il
pût arriver. Comme il y poussait à toute bride avec Silly et un
gentilhomme à lui, tous trois seuls, il fut reconnu, environné, et tous
trois pris.

Pendant tous ces désordres, Blansac était dans Bleinheim, qui ne savait
ce qu'était devenu Clérembault, disparu depuis plus de deux heures.
C'est que, de peur d'être tué, il était allé se noyer dans le Danube. Il
espérait le passer à la nage sur son cheval, avec son valet sur un
autre, apparemment pour se faire ermite après\,; le valet passa et lui y
demeura. Blansac donc, sur qui le commandement roulait en l'absence de
Clérembault qui ne paraissait plus sans que personne sût ce qu'il était
devenu, se trouva fort en peine de l'extrême désordre qu'il voyait et
entendait, et de ne recevoir aucun ordre du maréchal de Tallard.
L'éparpillement que cause une confusion générale fit que Valsemé,
maréchal de camp, et dans la gendarmerie, passa tout près du village, en
lieu où Blansac le reconnut\,; il cria après lui, y courut et le pria de
vouloir bien aller chercher Tallard, et lui demander ce qu'il lui
ordonnait de faire et de devenir. Valsemé y fut très franchement, mais
en l'allant chercher il fut pris\,; ainsi Blansac demeura sans ouïr
parler d'aucun ordre ni d'aucun supérieur. Je ne dirai ici que ce que
Blansac allégua pour une justification qui fut également mal reçue du
roi et du public, mais qui n'eut point de contradicteurs, parce que
personne ne fut témoin de ce qui se passa à Bleinheim que ceux qui y
avaient été mis, que les principaux s'accordèrent à un même plaidoyer,
et que la voix de ces vieux piliers de bataillons qui perça ne fit
pourtant pas une relation suivie, sur laquelle on pût entièrement
compter, niais qui fut assez forte pour accabler à la cour, et dans le
public, les officiers principaux à qui ils furent obligés d'obéir.
Ceux-là donc, au milieu de ces peines et livrés à eux-mêmes,
s'aperçurent que la poudre commençait à manquer, que leurs charrettes
composées s'en étaient allées doucement sans demander congé à personne,
que quelques soldats en avaient pris l'alarme et commençaient à la
communiquer à d'autres, lorsqu'ils virent revenir Denonville, qui avait
été pris à cette grande attaque du village dont j'ai parlé, et qui était
accompagné d'un officier qui, le mouchoir en l'air, demandait à parler
sur parole.

Denonville était un jeune homme, alors fort beau et bien fait, fils aîné
du sous-gouverneur de Mgr le duc de Bourgogne, et colonel du régiment
Royal-infanterie, que la faveur de ce prince un peu trop déclarée avait
rendu présomptueux et quelquefois audacieux. Au lieu de parler, au moins
en particulier à Blansac et aux autres officiers principaux, puisqu'il
avait fait la folie de se charger d'une mission si étrange, Denonville,
dis-je, qui avait de l'esprit, du jargon, et grande opinion de
lui-même\,; se mit à haranguer les troupes qui bordaient le village pour
leur persuader de se rendre prisonniers de guerre, pour se conserver
pour le service du roi. Blansac, qui vit l'ébranlement que ce discours
causait dans les troupes, le fit taire avec la dureté que son propos
méritait, le fit retirer et se mit à haranguer au contraire\,; mais
l'impression était faite, il ne tira d'acclamations que du seul régiment
de Navarre, tout le reste demeura dans un triste silence. J'avertis
toujours que c'est d'après Blansac que je parle.

Quelque peu de temps après que Denonville et son adjoint furent
retournés aux ennemis, revint de leur part un milord, qui demanda à
parler au commandant sur parole. Il fut conduit à Blansac, auquel il dit
que le duc de Marlborough lui mandait qu'il était là avec quarante
bataillons et soixante pièces de canon, maître d'y faire venir de plus
tout ce qu'il voudrait de troupes\,; qu'il commençait à l'environner de
toutes parts\,; que le village n'avait plus rien derrière soi pour le
soutenir\,; que l'armée de Tallard était en fuite, et ce qui restait
ensemble de celle de l'électeur était en marche pour se retirer\,; que
Tallard même et force officiers généraux étaient pris\,; que Blansac
n'avait aucun secours à espérer\,; qu'il ferait donc mieux d'accepter
une capitulation, en se rendant tous prisonniers de guerre, que de faire
périr tant de braves gens et de si bonnes troupes de part et d'autre,
puisqu'à la fin il faudrait bien que le plus petit nombre fût accablé
par le plus grand. Blansac voulut le renvoyer tout court\,; mais sur ce
que l'Anglais le pressa de s'avancer avec lui sur parole jusqu'à deux
cents pas de son village pour voir de ses yeux la vérité de la défaite
de l'armée électorale de sa retraite et des préparatifs pour l'attaquer,
Blansac y consentit. Il prit avec lui Hautefeuille, mestre de camp
général des dragons, et ils s'avancèrent avec ce milord. Leur
consternation fut grande lorsque par leurs yeux ils ne purent douter de
la vérité de tout ce que cet Anglais venait de leur dire. Ramenés par
lui dans Bleinheim, Blansac assembla les officiers principaux à qui il
rendit compte de la proposition qui leur était faite, et de ce que, par
ses propres yeux et ceux d'Hautefeuille, il venait de voir. Tous
comprirent combien affreuse serait pour eux la première inspection de
leur reddition prisonniers de guerre\,; mais, tout bien considéré, celle
de leur situation les frappa davantage, et ils conclurent tous à
accepter la proposition qui leur était faite, en prenant les précautions
qu'ils purent pour conserver au roi ces vingt-six bataillons et les
douze escadrons de dragons, par échange ou par rançon, pour leur
traitement et leurs traites. Cette horrible capitulation fut donc tout
aussitôt jetée sur le papier et signée de Blansac, des officiers
généraux et de tous les chefs de corps, hors de celui, je crois, de
Navarre, qui fut le seul qui refusa, et tout aussitôt exécutée.

Cependant Marsin, qui avait toujours non seulement soutenu mais repoussé
le prince Eugène avec avantage, averti de la déroute de l'armée de
Tallard et d'une grande partie de celle de l'électeur, découverte et
entraînée par l'autre, ne songea plus qu'à profiter à l'intégrité de la
sienne pour faire une retraite et recueillir tout ce qu'il pourrait de
ses débris, et il l'exécuta sans être poursuivi. Marlborough lui-même
était surpris d'un si prodigieux bonheur, le prince Eugène ne le pouvait
comprendre, le prince Louis de Bade, à qui ils le mandèrent, ne se le
pouvait persuader, et fut outré de n'y avoir point eu de part. Il leva
suivant leur avis, le siège d'Ingolstadt qui, après un événement aussi
complet ne se pouvait soutenir et tomberait de soi-même. L'électeur fut
presque le seul à qui la tête ne tourna point, et qui proposa peut-être
le seul bon parti à prendre\,: c'était de se maintenir dans son pays à
la faveur des postes et des subsistances commodes et abondantes. On
sentit trop tard la faute de ne l'avoir pas cru. Son pays, livré à
soi-même et soutenu de peu de ses troupes, se soutint tout l'hiver
contre toutes les forces impériales. Mais notre sort n'était pas de
faire des pertes à demi, l'électeur ne put être écouté\,; on ne songea
qu'à se retirer sur l'armée du maréchal de Villeroy et à la joindre. Les
ennemis n'y apportèrent pas le moindre obstacle, ravis de voir prendre à
nos armées un parti d'abandon auquel, après leur victoire, ils auraient
eu peine à les forcer. Cette jonction se fit donc, si différente des
précédentes, le 25 août, à Doneschingen, où l'armée du maréchal de
Villeroy s'était avancée. Chamarande y amena tout ce qu'il avait été
ramasser à Augsbourg, Ulm, etc., et Marsin ne ramena pas plus de deux
mille cinq cents soldats et autant de cavaliers, dont dix-huit cents
démontés, de l'armée de Tallard, qui perdit trente-sept bataillons,
savoir\,: les vingt-six qui se rendirent prisonniers de guerre à
Bleinheim, et onze tués et mis en pièces\,; la gendarmerie en
particulier, et en général presque toute la cavalerie de Tallard fut
accusée d'avoir très mal fait. Ils tirèrent au lieu de charger l'épée à
la main, ce que fit la cavalerie ennemie, qui avait auparavant coutume
de tirer\,; ainsi l'une et l'autre changea son usage et prit celui de
son ennemi, qui fut une chose très fatale. Enfin nos armées arrivèrent,
le dernier août, sous le fort de Kehl, au bout du pont de Strasbourg, et
le prince Eugène dans ses lignes de Stolhofen, faisant contenance de
vouloir passer le Rhin.

L'électeur passa de sa personne de Strasbourg à Metz, d'où il gagna
Bruxelles, tout droit comme il put. Il aurait fort voulu aller voir le
roi, mais cette triste entrevue ne fut pas du goût de Sa Majesté,
quoique ce prince, dans l'intervalle de la bataille à son passage du
Rhin, eût refusé des propositions fort avantageuses, s'il avait voulu
abandonner son alliance. Il vit l'électrice et ses enfants en passant à
Ulm, leur donna ses instructions avec beaucoup de courage et de
sang-froid, et les renvoya à Munich pour s'y soutenir, avec ce qu'il
laissait de ses troupes, le plus longtemps qu'il serait possible.
Blainville, Zurlauben, lieutenants généraux, furent tués et beaucoup
d'autres, les prisonniers furent infinis. Labaume, fils aîné de Tallard,
survécut peu de jours à sa blessure. Le duc de Marlborough, qui avait
tout fait avec son armée, garda le maréchal de Tallard et les officiers
les plus distingués qu'il envoya à Hanau, jusqu'à ce qu'il fût temps
pour lui de passer en Angleterre, pour en orner son triomphe. De tous
les autres, il en donna la moitié au prince Eugène. Ce fut pour eux une
grande différence. Celui-ci les traita durement\,; le duc de Marlborough
avec tous les égards, les complaisances, les politesses les plus
prévenantes en tout, et une modestie peut-être supérieure à sa victoire.
Il eut soin que ce traitement fût toujours le même jusqu'à leur passage
avec lui, et le commun des prisonniers qu'il se réserva reçut par ses
ordres tous les ménagements et toutes les douceurs possibles.

Le roi reçut cette cruelle nouvelle le 21 août par un courrier du
maréchal de Villeroy, à qui les troupes laissées par le prince Eugène
sous le comte de Nassau-Weilbourg dans leurs lignes de Stolhofen,
envoyèrent un trompette, avec des lettres de plusieurs de nos officiers
prisonniers à qui ou avait permis de donner de leurs nouvelles à leurs
familles. Par ce courrier, le roi apprit que la bataille donnée le 13
avait duré depuis huit heures du matin jusque vers le soir\,; que
l'armée entière de Tallard était tuée ou prise\,; qu'on ne savait ce que
ce maréchal était devenu\,; aucune lettre ne le disait, ni n'expliquait
si l'électeur et le maréchal de Marsin avaient été à l'action. Il y en
avait de Blansac, de Hautefeuille de Montpéroux, du chevalier de Croissy
et de Denonville, mais sans aucun détail, et de gens éperdus. Dans cette
terrible inquiétude, le roi ouvrit ces lettres, il trouva quelque chose
de plus dans celle de Montpéroux, mais pourtant sans détail\,: il
écrivait à sa femme, qu'il appelait sa chère petite Palatine. Quand le
roi, longtemps après, fut éclairci, il demanda au maréchal de Boufflers
ce que c'était que ce petit nom de tendresse dont il n'avait jamais ouï
parler. Le maréchal lui apprit que le nom propre de Montpéroux était
Palatin de Dio. Il aurait pu ajouter que \emph{Palatin} était un titre
familier dans ces provinces de Bourgogne et voisines, resté en nom
propre après avoir été des concessions des empereurs\,; ainsi c'était
palatin, ou sous un titre plus éminent, seigneur de Dio.

Le roi demeura six jours dans cette situation violente de savoir tout
perdu en Bavière, et d'ignorer le comment. Le peu de gens dont il arriva
des lettres se contentaient de mander de leurs propres nouvelles, tout
au plus de quelques amis. Personne n'était pressé de raconter le
désastre. On craignait pour ses lettres, et on n'osait s'y expliquer sur
les choses ni sur les personnes. Marsin, tout occupé de sa retraite, se
contenta de donner de ses nouvelles au maréchal de Villeroy, uniquement
relatives à cet objet. L'électeur, outré de ses pertes et de la
contradiction qu'il avait trouvée à son avis de demeurer dans son pays,
n'écrivit au roi que deux mots de respect et de fermeté dans son
alliance, en passant le Rhin\,; tellement qu'on n'apprenait rien que par
lambeaux, et rares et médiocres, qui ne faisaient qu'augmenter
l'inquiétude sur la chose générale et sur le sort des particuliers. La
cruelle capitulation de Bleinheim fut pourtant démêlée la première, par
deux mots qui s'en trouvèrent dans les lettres de Denonville, de Blansac
et d'Hautefeuille. D'autres officiers particuliers s'échappèrent sans
détail contre la gendarmerie et contre quelques officiers généraux,
parmi lesquels le comte de Roucy n'était pas bien traité, et qui
relevaient amèrement sa contusion si longuement pansée, si fort dans les
derrières, pendant tout l'effort de la bataille de la Marsaille où il ne
parut plus. Lui et Blansac son frère étaient fils de la sœur bien-aimée
de M. le maréchal de Lorges. Ils avaient passé leur vie chez lui comme
ses enfants. M. de La Rochefoucauld, aîné de leur maison, les traitait,
aux secours près, de même. Leurs femmes, avec qui je vivais fort,
m'envoyèrent chercher partout, et me conjurèrent de voir Chamillart
sur-le-champ pour obtenir de lui tout ce qu'il pourrait auprès du roi en
leur faveur. Je le fis si efficacement qu'il leur sauva des choses
fâcheuses.

Le roi, jusque par lui-même, cherchait des nouvelles, il en demandait,
il se faisait apporter ce qui arrivait de la poste, et il n'y arrivait
rien, ou presque rien qui l'instruisît\,; on mettait bout à bout ce que
chacun savait pour en faire un tout qui ne contentait guère. Le roi ni
personne ne comprenait point une armée entière placée dedans et autour
d'un village, et cette armée rendue prisonnière de guerre par une
capitulation signée. La tête en tournait. Enfin les détails grossissant
peu à peu, qui d'une lettre, qui d'une autre, arriva Silly à l'Étang, le
matin du 29 août. Chamillart l'amena à Meudon où le roi était, qui
s'enferma longtemps avec eux avant son dîner. Tallard, avec qui il fut
pris, obtint du duc de Marlborough la permission de l'envoyer au roi lui
rendre compte de son malheur, avec parole qu'il reviendrait incontinent
après où il lui ordonnerait de se rendre. Comme il n'apprit rien que je
n'aie raconté ici, il servira quelques moments à faire une assez
curieuse diversion à une matière aussi désagréable dont les suites se
reprendront après.

Silly, du nom de Vipart, était un gentilhomme de Normandie des plus
minces qu'il y eût, entre Lisieux et Séez, et en biens et en naissance.
C'était un grand garçon, parfaitement bien fait, avec un visage agréable
et mâle, infiniment d'esprit, et l'esprit extrêmement orné\,; une grande
valeur et de grandes parties pour la guerre\,; naturellement éloquent
avec force et agrément\,; d'ailleurs d'une conversation très aimable\,;
une ambition effrénée, avec un dépouillement entier de tout ce qui la
pouvait contraindre, ce qui faisait un homme extrêmement dangereux, mais
fort adroit à le cacher, appliqué au dernier point à s'instruire, et
ajustant tous ses commerces, et jusqu'à ses plaisirs, à ses vues de
fortune. Il joignait les grâces à un air de simplicité qui ne put se
soutenir bien longtemps, et qui, à mesure qu'il crût en espérance et en
moyens, se tourna en audace. Il se lia tant qu'il put avec ce qu'il y
avait de plus estimé dans les armées, et avec la plus brillante
compagnie de la cour. Son esprit, son savoir qui n'avait rien de pédant,
sa valeur, ses manières plurent à M. le duc d'Orléans. Il s'insinua dans
ses parties, mais avec mesure, de peur du roi, et assez pour plaire au
prince, qui lui donna son régiment d'infanterie. Un hasard le fit
brigadier longtemps avant son rang, et conséquemment lieutenant général
de fort bonne heure.

Silly, colonel de dragons, dès lors fort distingué, et qui depuis a
pensé, et peut-être aurait dû être maréchal de France, fut fait
brigadier dans cette promotion immense, où je ne le fus point, et qui me
fit quitter le service, comme je l'ai dit en son temps. Chamillart
arrivait dans la place de secrétaire d'État de la guerre. C'était la
première promotion de son temps\,; il ne connaissait pas un officier.
Sortant de chez M\textsuperscript{me} de Maintenon, où la promotion
s'était faite à son travail ordinaire, il rencontre Silly et lui dit
d'aller remercier le roi qui venait de le faire brigadier. Silly, qui
n'en était pas à portée, eut la présence d'esprit de cacher sa surprise.
Il se douta de la méprise entre lui et Silly des dragons, mais il compta
en tirer parti, et alla remercier le roi, sortant de chez
M\textsuperscript{me} de Maintenon pour aller souper. Le roi, bien
étonné de ce remerciement, lui dit qu'il n'avait pas songé à le faire.
L'autre, sans se démonter, allégua ce que Chamillart lui venait de dire,
et de peur d'une négative qui allât à l'exclusion, se dérobe dans la
foule, va trouver Chamillart, et s'écrie qu'après avoir remercié sur sa
parole, il n'a plus qu'à s'aller pendre s'il reçoit l'affront de n'être
pas brigadier. Chamillart, honteux de sa méprise, crut qu'il y allait du
sien de la soutenir. Il l'avoua au roi dès le lendemain, et tout de
suite fit si bien que Silly demeura brigadier. Il s'attacha le plus
qu'il put à M. le prince de Conti et à ceux qu'il voyait le plus.
C'était alors le bon air comme il l'a été toujours, et Silly n'y était
pas indifférent. Il tourna le maréchal de Villeroy\,; ses grandes
manières et ses hauteurs le rebutèrent. Il trouva mieux son compte avec
l'esprit, le liant et la coquetterie de Tallard, qui se voulait faire
aimer jusque des marmitons. Faits prisonniers ensemble, Tallard, fort en
peine de soi à la cour, crut n'y pouvoir envoyer un meilleur chancelier
que Silly. Il le servit si bien qu'on en verra bientôt des fruits. Mais
au retour, je ne sais ce qui arriva entre eux. Ils se brouillèrent
irréconciliablement, apparemment sur des choses qui ne faisaient honneur
à l'un ni à l'autre, puisque chacun d'eux a tellement gardé le secret
là-dessus, que leurs plus intimes amis n'y ont pu rien deviner, et que
la cause de cette rupture, tous deux l'ont emportée en l'autre monde,
même le survivant des deux qui fut Tallard, et qui n'avait rien à
craindre d'un mort qui ne laissait ni famille ni amis.

Le roi mort, Silly fit un moment quelque figure dans la régence\,; mais,
peu content de n'être d'aucun conseil, il se tourna aux richesses. Il
était né fort pauvre, et n'avait pu que subsister. Sa fortune allait
devant tout\,; mais, foncièrement avare, l'amour du bien suivait
immédiatement en lui. Il fit sa cour à Law qu'il séduisit par son
esprit. La mère du vieux Lassay était Vipart\,; il était très bien avec
son fils, qui depuis bien des années disposait du cœur, de l'esprit, de
la conduite et de la maison de M\textsuperscript{me} la Duchesse.
M\textsuperscript{me} la Duchesse, en cela seulement, une avec M. le
Duc, était tout système. Law, après M. le duc d'Orléans, avait mis ses
espérances en la maison de Condé, dont l'avidité héréditaire se gorgea
de millions par le dévouement de ce Law. Silly s'y fraya accès par
Lussé, qui était la voie exquise auprès de M\textsuperscript{me} la
Duchesse. Il y devint bientôt un favori important sous la protection du
véritable, et se gorgea en sous-ordre. M. le Duc, devenu premier
ministre, ne put refuser à sa mère quelques colliers de l'ordre dans la
nombreuse promotion de 1724, où il fourra tant de canailles. Silly en
eut un, que M\textsuperscript{me} la Duchesse arracha avec peine. Il
avait attrapé de M. le duc d'Orléans une place de conseiller d'État
d'épée. Alors riche et décoré, il revêtit le seigneur. Cette fortune
inespérable ne fit que l'exciter à la combler. Rien ne lui parut
au-dessus de son mérite. Morville, secrétaire d'État des affaires
étrangères, en fut ébloui. Silly le domina. Il devint son conseil pour
sa conduite et pour les affaires. Une position si favorable à son
ambition lui donna d'idée de l'ambassade d'Espagne, d'y être fait grand,
de revenir après dans le conseil comme un homme déjà imbu des affaires,
de se faire duc et pair\,; et de là tout ce qu'il pourrait. Ce fut un
château en Espagne et le pot au lait de la bonne femme. M. le Duc fut
remercié, et Morville congédié.

Un grand homme ne s'abandonne pas soi-même. Silly comprit avec tout le
monde que M. de Fréjus, incontinent après cardinal Fleury, était tout
seul le maître des grâces et des affaires, et Chauvelin sous lui.
C'était pour lui deux visages tout nouveaux, à qui il était très
inconnu. L'opinion qu'il avait de soi le persuada qu'avec un peu d'art
et de patience il viendrait à bout de faire d'eux comme de Morville\,;
mais ils avaient trop peu de loisirs et lui trop peu d'accès. Dans la
peine du peu de succès de ses essais, il se mit dans la tête de venir à
bout du cardinal, par une assiduité qui lui plût, comme il n'en doutait
pas, et qui, l'accoutumant à lui, lui frayât le chemin de son cabinet,
ou, une fois entré, il comptait bien le gouverner. Il se mit donc à ne
bouger de Versailles, et quoiqu'il n'eût de logement qu'à la ville, d'y
donner tous les jours un dîner dont la délicatesse attirât. Il y menait
des gens de guerre qu'il trouvait sous sa main, le peu de gens d'âge
qui, autrefois à la cour, venaient pour quelque affaire à Versailles, et
des conseillers d'État. Là on dissertait, et Silly tenait le dé du
raisonnement et de la politique, en homme qui se ménage, qui croit déjà
faire une figure, et qui la veut augmenter. En même temps il s'établit
tous les jours à la porte du cardinal pour le voir passer. Cela dura
plus d'un an, sans rien rendre que quelques dîners chez le cardinal,
encore bien rarement\,; soit que le cardinal fût averti du dessein de
Silly, soit que sa défiance naturelle prît ombrage d'une assiduité si
remarquable. Un jour qu'il rentrait un moment avant son dîner, il
s'arrêta à la porte de son cabinet, et demanda à Silly d'un air fort
gracieux s'il désirait quelque chose et s'il avait à lui parler. Silly,
se confondant en compliments et en respects, lui répond que non, et
qu'il n'est là que pour lui faire sa cour en passant. Le cardinal lui
répliqua civilement, mais haussant la voix pour être entendu de tout ce
qui était autour d'eux, qu'il n'était pas accoutumé à voir des gens
comme lui à sa porte, et ajouta fort sèchement qu'il le priait de n'y
plus revenir quand il n'aurait point affaire à lui.

Ce coup de foudre, auquel Silly s'était si peu attendu, le pénétra
d'autant plus qu'il s'y trouva plus de témoins. Il avait compté
circonvenir le cardinal par ses plus intimes amis à qui il faisait une
cour basse et assidue, après avoir trouvé divers moyens de s'introduire
chez eux, et mène de leur plaire. Il sentit avec rage toutes ses
espérances perdues, et s'en alla chez lui, où il trouva force compagnie.
Le comte du Luc, qui me conta cette aventure, était à la porte du
cardinal, où il entendit tout le dialogue, d'où il alla dîner chez
Silly, qui auparavant l'en avait convié, et où ils se trouvèrent
plusieurs. Silly y parut outré et assez longtemps morne. À la fin il
éclata à table contre le cardinal à faire baisser les yeux à tout le
monde. Il continua le reste du repas à se soulager de la sorte. Personne
ne répondit un mot. Il sentait bien qu'il embarrassait, et qu'il ne
faisait par ces propos publics que se faire à lui-même un mal
irrémédiable\,; mais le désespoir était plus fort que lui. Il se passa
près d'un an depuis, tantôt à Paris, tantôt à Versailles, n'osant plus
approcher du cardinal, qu'il aurait voulu dévorer, et cherchant dans son
esprit des expédients et des issues qu'il ne pût lui fournir. À la fin,
il s'en alla chez lui pour y passer l'hiver. Il avait accru et ajusté sa
gentilhommière qu'il avait travestie en château.

Il n'y fut pas longtemps sans renvoyer le peu de gens qui venaient le
voir\,; je dis le peu, car ses nouveaux airs de seigneur, auxquels ses
voisins n'étaient pas accoutumés chez lui, en avaient fort éclairci la
compagnie. Il dit qu'il était malade, et se mit au lit. Il y demeura
cinq ou six jours. Le peu de valets qu'il y avait se regardaient ne le
voyant point malade. Son chirurgien, que j'ai vu après à M. de Lévi, ne
lui trouvait point de fièvre. Le dernier jour il se leva un moment, se
recoucha, et fit sortir tous ses gens de sa chambre. Sur les six heures
du soir, inquiets de cette longue solitude, et sans rien prendre, ils
entendirent quelque bruit dans les fossés, plus pleins de boue que
d'eau\,; là-dessus ils entrèrent dans sa chambre, et se mirent à la
cheminée à écouter un peu. Un d'eux sentit un peu de vent d'une
fenêtre\,; il la voulut aller fermer. En même temps un autre s'approche
du lit, et lève doucement le rideau\,; mais quel fut l'étonnement de
tous les deux, lorsque l'un ne trouva personne dans le lit, et l'autre
deux pantoufles au bas de la fenêtre dans la chambre\,! Les voilà à
s'écrier et à courir tous aux fossés. Ils l'y trouvèrent tombé de façon
à avoir pu gagner le bord s'il eût voulu. Ils le retirèrent palpitant
encore, et fort peu après il mourut entre leurs bras. Il n'était point
marié, et avait une sœur fille, qu'il laissait à la lettre manquer de
tout et mourir de faim, qui trouva dans sa riche succession une ample
matière à se consoler d'une si funeste catastrophe. Avec tout son esprit
il fit une sottise qui fâcha extrêmement le roi. Après l'avoir entretenu
longtemps dans son cabinet en arrivant à Meudon, il l'aperçut sur le
soir à sa promenade sans épée. Gela piqua le roi à l'excès, et il le
marqua par le ton avec lequel il lui demanda ce qu'il en avait fait.
Silly répondit qu'étant prisonnier, il croyait n'en devoir point porter.
«\,Qu'est-ce que cela veut dire\,? reprit le roi fort ému, allez en
prendre une tout à l'heure.\,» Cela, joint aux tristes nouvelles dont il
avait apporté le détail, ne le fit pas briller pendant ce court voyage,
et rie contribua pas peu à lui donner de l'impatience d'aller retrouver
Tallard à Hanau, comme il fit peu de jours après avoir été à un voyage
de Marly pour la première fois de sa vie.

On n'était pas accoutumé aux malheurs. Celui-ci était très
raisonnablement inattendu\,; quatre armées au delà du Rhin, dont les
trois dans le cœur de l'Allemagne avec la puissance des mécontents,
faisaient tout attendre d'elles. Qu'on n'eût point combattu, on était
maître de tout par la retraite forcée des ennemis, et imminente, et fort
éloignée pour trouver de la subsistance. Que le maréchal de Villeroy qui
n'avait rien à faire qu'à observer le prince Eugène, le suivre, le
barrer, ne s'en fût point laissé amuser, puis moquer en s'échappant,
jamais Marlborough, sans sa jonction, n'eût osé prêter le collet à nos
trois armées. Qu'elles eussent bordé le ruisseau de leur front, jamais
ils ne se seraient commis à le passer devant elles, ou y auraient été
rompus et défaits. Qu'elles n'eussent laissé que peu d'intervalle entre
elles et le ruisseau pour les attaquer demi-passés, s'ils l'osaient
entreprendre, ils étaient sûrement battus et culbutés dedans. Qu'elles
eussent au moins pris un terrain où le vaste laissait le choix libre,
qui ne mît pas une large et longue fondrière entre les deux lignes de
Tallard, encore auraient-elles eu au moins partie égale. Qu'on n'eût pas
pris vingt-six bataillons et douze escadrons de dragons de cette armée
pour mettre dedans et autour d'un village, pour appuyer la droite qu'on
était maître de mettre tout près de là au Danube, on n'aurait pas
affaibli cette armée, qui tenait lieu d'aile droite, à être enfoncée, et
le centre, qui était celle de l'électeur, à être pris en flanc. Qu'au
moins une armée entière, établie dans ce village de Bleinheim, eût eu le
courage de s'y défendre, elle eût donné le temps à l'armée de Marsin qui
faisait la gauche, qui était entière, qui avait toujours battu, de
profiter du temps et de l'occupation qu'aurait donnée ce village, de se
rallier aux deux tiers de l'armée de l'électeur qui soutenait encore, et
à la faveur d'une défense de vingt-six bons bataillons et de douze
escadrons de dragons, d'y porter la bataille et tout l'effort des armes
qui peut-être eût été heureux. Mais il était écrit que la honte, les
fautes, le dommage seraient extrêmes du côté du roi, et que toutes
seraient comblées par le tournoiement de tête de la dernière faute, en
abandonnant la Bavière si aisée à tenir, avec ses places, sa volonté,
son abondance, par une armée entière qui n'avait rien souffert, et par
le débris des deux autres, en prenant des postes avantageux. En vain
l'électeur ouvrit-il cet avis, la peur ne crut trouver de salut qu'à
l'abri de l'armée du maréchal de Villeroy\,; et, quand la jonction fut
faite, au lieu de profiter de ce que les passages étaient encore libres,
et de ramener cette armée toute fraîche avec eux en Bavière, où tous
ensemble se seraient trouvés aussi forts que devant la bataille, et plus
frais que les ennemis qui avaient combattu, car il était resté peu de
troupes avec le prince Louis de Bade devant Ingolstadt, on ne songea
qu'à hâter la fuite, à presser l'abandon de tant de places et de taret
de vastes et d'abondants pays. On ne se crut en sauveté qu'au Rhin, et
au bout du pont de Strasbourg, pour être maître à tous moments de le
passer. Ces prodiges d'erreurs, d'aveuglements, de ténèbres, entassés et
enchaînés ensemble, si grossiers, si peu croyables, et dont un seul de
moins eût tout changé de face, retracent bien, quoique dans un genre
moins miraculeux, ces victoires et ces défaites immenses que Dieu
accordait, ou dont il affligeait son peuple, suivant qu'il lui était
fidèle ou que son culte en était abandonné.

On peut juger quelle fut la consternation générale, où chaque famille
illustre, sans parler des autres, avait des morts, des blessés et des
prisonniers, quel fut l'embarras du ministre de la guerre et de la
finance d'avoir à réparer une armée entière détruite, tuée ou
prisonnière\,; et quelle la douleur du roi qui tenait le sort de
l'empereur entre ses mains, et qui, avec cette ignominie et cette perte,
se vit réduit, aux bords du Rhin, à défendre le sien propre. Les suites
ne marquèrent pas moins l'appesantissement de la main de Dieu. On perdit
le jugement, on trembla au milieu de l'Alsace. La cruelle méprise du
maréchal de Villeroy fut noyée dans sa faveur. Nous allons voir Tallard
magnifiquement récompensé. Marsin demeura dans l'indifférence\,; on
trouva qu'il ne méritait rien, puisqu'il n'avait point failli, car le
roi ne le blâma point de ne s'être pas roidi en Bavière. Toute la colère
tomba sur quelques régiments qui furent cassés, sur des particuliers
dont tout le châtiment fut de n'être plus employés dans les armées,
parmi lesquels quelques innocents furent mêlés avec les coupables.
Denonville seul fut honteusement cassé et son régiment donné à un autre,
tellement que, sa prison finie, il n'osa plus paraître nulle part. Je ne
veux pas dire que la proposition qu'il eut la folie de venir faire aux
barrières de Bleinheim ne l'eût bien mérité\,; mais ce ne fut pas à son
éloquence que ce village mit les armes bas et se rendit prisonnier de
guerre. Ce fut à celle d'un Anglais seul envoyé après lui. Denonville
fut le seul puni, et pas un de ceux qui remirent leur armée, car c'en
était une au pouvoir des Anglais sans tirer un seul coup depuis que la
capitulation avec la condition de prisonniers de guerre leur eut été
proposée\,; et le seul chef de troupes qui refusa de la signer n'en fut
pas reconnu ni distingué le moins du monde. En échange, le public ne se
contraignit, ni sur les maréchaux, ni sur les généraux, ni sur les
particuliers qu'il crut en faute, ni sur les troupes dont les lettres
parlèrent mal. Ce fut un vacarme qui embarrassa leurs familles. Les plus
proches furent plusieurs jours sans oser se montrer, et il y en eut qui
regrettèrent de n'avoir pas gardé une plus longue clôture.

Au milieu de cette douleur publique, les réjouissances et les fêtes pour
la naissance du duc de Bretagne ne furent point discontinuées. La ville
en donna une d'un feu sur la rivière, que Monseigneur, les princes ses
fils, et M\textsuperscript{me} la duchesse de Bourgogne vinrent voir des
fenêtres du Louvre avec force dames et courtisans, et force magnificence
de chère et de rafraîchissements, contraste qui irrita plus qu'il ne
montra de grandeur d'âme. Peu de jours après, le roi donna une
illumination et une fête à Marly, où la cour de Saint-Germain fut
invitée, et où tout fut en l'honneur de M\textsuperscript{me} la
duchesse de Bourgogne. Il remercia le prévôt des marchands du feu donné
sur la rivière, et lui dit que Monseigneur et M\textsuperscript{me} la
duchesse de Bourgogne l'avaient trouvé fort beau.

\hypertarget{chapitre-xvii.}{%
\chapter{CHAPITRE XVII.}\label{chapitre-xvii.}}

1704

~

{\textsc{Marche des alliés.}} {\textsc{- Marlborough feld-maréchal
général des armées de l'empereur et de l'empire.}} {\textsc{- Nos armées
en Alsace.}} {\textsc{- Mort du duc de Montfort\,; son caractère.}}
{\textsc{- Sa charge donnée à son frère.}} {\textsc{- Mort, famille et
dépouille du comte de Verue.}} {\textsc{- Entreprise manquée sur
Cadix.}} {\textsc{- Bataille navale gagnée près de Malaga par le comte
de Toulouse.}} {\textsc{- Faute fatale malgré le comte de Toulouse.}}
{\textsc{- Châteauneuf, ambassadeur en Portugal, arrivé d'Espagne\,; son
frère, leur fortune, leur caractère.}} {\textsc{- Orry arrivé à Paris en
disgrâce et en péril.}} {\textsc{- Aubigné bien traité à Madrid.}}
{\textsc{- Berwick rappelé d'Espagne aux instances de la reine\,; Tessé
nommé pour lui succéder.}} {\textsc{- Intrigues du mariage du duc de
Mantoue, qui refuse M\textsuperscript{lle} d'Enghien, est refusé de la
duchesse de Lesdiguières, et qui, contre le désir du roi et sa propre
volonté, épouse fort étrangement M\textsuperscript{lle} d'Elbœuf, qu'il
traite après fort mal.}}

~

Les trois chefs ennemis, maîtres de la Bavière et de tout jusqu'au Rhin,
ramenèrent leurs armées auprès de Philippsbourg, dans les derrières, et
y tinrent un pont tout prêt à y jeter sur le Rhin en trois heures.
Tandis que les troupes marchèrent et qu'ils les laissèrent se rafraîchir
dans ce camp, le prince Louis de Bade reçut dans ce voisinage au beau
château de Rastadt, qu'il avait bâti en petit sur le modèle de
Versailles, le prince Eugène et le duc de Marlborough qui vinrent s'y
reposer à l'ombre de leurs lauriers. Ce fut là que ce duc reçut de
l'empereur les patentes de feld-maréchal général des armées de
l'empereur et de l'empire, grade fort rare, pareil à celui qu'avait le
prince Eugène, et supérieur aux feld-maréchaux, qui, pour l'armée, les
troupes et les places, sont comme nos maréchaux de France\,; et la reine
d'Angleterre lui permit de l'accepter en attendant les récompenses qu'on
lui préparait en Angleterre.

Pendant ce glorieux repos nos maréchaux avaient repassé le Rhin et
s'étaient avancés sur Haguenau. Tout leur faisait craindre le siège de
Landau. Le maréchal de Villeroy ne se crut pas en état de s'y opposer\,;
il se contenta de le munir de tout le nécessaire pour un long siège, et
d'y faire entrer, outre la garnison, huit bataillons, un régiment de
cavalerie et un de dragons sous Laubanie, gouverneur, chargé de le
défendre. Rien n'était pareil à la rage des officiers de cette armée.

J'avais reçu depuis peu une lettre du duc de Montfort, qui était fort de
mes amis, qui me mandait qu'à son retour il voulait casser son épée et
se faire président à mortier. Il avait toujours été de l'armée du
maréchal de Villeroy. Sa lettre me parut si désespérée qu'avec un
courage aussi bouillant que le sien, je craignis qu'il ne fît quelque
folie martiale, et lui mandai qu'au moins je le conjurais de ne se pas
faire tuer à plaisir. Il sembla que je l'avais prévu. Il fallut envoyer
un convoi d'argent à Landau\,; on fit le détachement pour le conduire.
Il en demanda le commandement au maréchal de Villeroy, qui lui dit que
cela était trop peu de chose pour en charger un maréchal de camp. Peu
après il se fit refuser encore\,; à une troisième {[}fois{]} il
l'emporta de pure importunité. Il jeta son argent dans Landau sans aucun
obstacle. Au retour, et marchant à la queue de son détachement, il vit
des hussards qui voltigeaient\,; le voilà à les vouloir courre et faire
le coup de pistolet comme un carabin. On le retint quelque temps, mais
enfin il s'échappa sans être suivi que de deux officiers. Ces coquins
caracolèrent, s'enfuirent, s'éparpillèrent, se rapprochèrent\,; et
l'ardeur poussant le duc de Montfort sur eux, il s'en trouva tout à coup
enveloppé, et aussitôt culbuté d'un coup de carabine qui lui fracassa
les reins, et qui ne lui laissa le temps que d'être emporté comme on
put, de se confesser avec de grands sentiments de piété et de regret de
sa vie passée, et d'arriver au quartier général, où il mourut presque
aussitôt après.

Il n'avait pas encore trente-cinq ans, et en avait cinq plus que moi.
Beaucoup d'esprit, un savoir agréable, des grâces naturelles qui
réparaient une figure un peu courte et entassée, et un visage que les
blessures avaient balafré\,; une valeur qui se pouvait dire excessive,
une grande application et beaucoup de talents pour la guerre, avec
l'équité, la liberté, le langage fait pour plaire aux troupes et à
l'officier, et avec cela à s'en faire respecter\,; une grande ambition,
mais, par un mérite rare, toujours retenue dans les bornes de la
probité. Un air ouvert et gai, des mœurs douces et liantes, une vérité,
une sûreté à toute épreuve, jointe à une vraie simplicité, formaient en
lui le caractère le plus aimable et un commerce délicieux\,; avec cela
sensible à l'amitié et très fidèle, mais fort choisi dans ses amis, et
le meilleur fils, le meilleur mari, le meilleur frère et le meilleur
maître du monde, adoré dans sa compagnie des chevau-légers, ami intime
de Tallard et de Marsin, fort de M. le prince de Conti, qui l'avait fort
connu chez feu M. de Luxembourg, qui l'aimait comme son fils\,; ami
particulier de M. le duc d'Orléans, et si parfaitement bien avec M. le
duc de Bourgogne, qu'il en devenait déjà considérable à la cour.
Monseigneur aussi le traitait avec amitié, et le roi se plaisait à lui
parler et à le distinguer en tout, tellement qu'il était compté à la
cour fort au-dessus de son âge, et n'en était pas moins bien avec ses
contemporains, dont ses manières émoussaient l'envie. Une éducation
beaucoup trop resserrée, et trop longtemps, l'avait jeté d'abord dans un
grand libertinage, l'avait écarté de cette assiduité qui était d'un si
grand mérite auprès du roi, et avait étrangement gâté ses affaires. Il
revenait depuis quelque temps d'un égareraient si commun, et ce retour
lui avait tourné à grand mérite auprès du roi. Ma liaison intime avec le
duc de Chevreuse, son père, et M. de Beauvilliers, avait formé la mienne
avec lui. Une certaine ressemblance de goûts, d'inclinations,
d'aversions, de vues et de manières de penser et d'être, l'avait
resserrée jusqu'à la plus grande intimité, en sorte que pour le sérieux
nous n'avions rien de caché l'un pour l'autre. L'habitation continuelle
de la cour nous faisait fort vivre ensemble. Sa femme et
M\textsuperscript{me} de Lévi, sa sœur, étaient amies intimes de
M\textsuperscript{me} de Saint-Simon, que M\textsuperscript{me}s de
Chevreuse et de Beauvilliers traitaient comme leur fille. En absence
nous nous écrivions continuellement. Sa perte fut aussi pour moi de la
dernière amertume, et tous les jours de ma vie je l'ai sentie depuis
tant d'années. On peut juger quelle fut la douleur de sa famille. Il ne
laissa que des enfants tout enfants. Sa charge fut donnée à son frère,
le vidame d'Amiens, qui est parvenu depuis à tout.

La mort du comte de Verue, tué à cette funeste bataille, dégrilla sa
femme, qu'il tenait dans un couvent à Paris, depuis qu'elle y était
revenue d'entre les bras de M. de Savoie, comme je l'ai raconté en son
lieu, et lui donna toute liberté. Elle reviendra en son temps sur la
scène. Verue ne laissa qu'un fils d'elle, qui le survécut peu, et des
filles religieuses. Sa charge de commissaire général de la cavalerie,
qu'il venait d'acheter du maréchal de Villars, fut donnée à La Vallière,
prisonnier d'Hochstedt, et ce choix fit fort crier.

Le roi ne fut pas longtemps dans la douleur du désastre d'Hochstedt sans
recevoir quelque consolation, médiocre pour l'État, mais sensible à son
cœur. Le comte de Toulouse, qui ne ressemblait en quoi que ce pût être
au duc du Maine son frère, avait souffert impatiemment d'avoir consumé
sa première campagne d'amiral à se promener sur la Méditerranée, sans
oser prêter le collet aux flottes ennemies trop fortes pour la sienne.
Il en avait donc obtenu une cette année, avec laquelle il pût se mesurer
avec celle qui, ayant hiverné à Lisbonne, tenait la mer sous l'amiral
Rooke, en attendant les secours de Hollande et d'Angleterre. Il faut
dire, avant que d'aller plus avant, un mot d'Espagne pour l'intelligence
de ce qui va suivre.

Le prince de Darmstadt, qui avait été à la cour de Charles II, comme on
l'a vu en son lieu, et qui y avait été si bien avec la reine sa dernière
femme, s'était embarqué sur la flotte avec l'archiduc lorsque ce prince
alla en Portugal, et avec une partie projeta de surprendre Cadix, qu'il
savait fort dégarni de toutes choses. Un marchand français, armé pour
les îles de l'Amérique, moitié guerre moitié marchandises, mais qui pour
son commerce y portait sur deux gros bâtiments beaucoup de munitions de
guerre, d'armes et assez d'argent, se trouva dans ces mers, et vit à la
manœuvre de l'escadre le dessein sur Cadix. Il força de voiles, y entra
en présence de l'escadre, débarqua toute sa cargaison, mit ainsi la
place en état de se défendre, qui, faute d'armes et de munitions et
d'argent, ne pouvait autrement résister, et demeura dedans. Darmstadt
n'ayant donc pu réussir dans son dessein, après l'avoir inutilement
tenté pendant plusieurs jours, mit pied à terre et pilla les environs de
terre ferme. Les communes s'assemblèrent sous le capitaine général du
pays, les évêques voisins se surpassèrent par le prompt secours de monde
et d'argent\,; en un mot, après un mois de courses où les Anglais
perdirent bien du monde, il fallut se rembarquer, et encore à
grand'peine et faire voile vers le Portugal. On a vu les négligences
d'Orry, et ce nonobstant comme Puységur en répara tout ce qui fut
possible, et les succès du duc de Berwick sur la frontière de Portugal.
Les chaleurs séparèrent les armées, qui mirent en quartier d'été.
Berwick, Villadarias ni Serclaës, dénués de tout par cette même
négligence d'Orry, n'avaient pu pourvoir à tout, ni porter leurs troupes
partout où elles auraient été nécessaires. Gibraltar, cette fameuse
place qui commande à l'important détroit de ce nom, avait été pourvue
comme les autres, c'est-à-dire qu'il n'y avait quoi que ce soit dedans
pour la défendre, et pour toute garnison une quarantaine de gueux. Le
prince de Darmstadt, qui était bien averti, profita d'une faute si
capitale. Y aller et s'en emparer ne fut que la même chose, et la
grandeur de cette perte ne fut sentie qu'après qu'elle fut faite. D'un
autre côté, le même prince de Darmstadt, qui avait été sous Charles II
vice-roi de Catalogne, avait conservé dans cette province beaucoup
d'intelligences, et dans Barcelone quantité de créatures. On y méditait
une révolte, on la soupçonna, notre flotte y toucha. Le comte de
Toulouse y mit pied à terre, il y fut quelque temps, et déconcerta
entièrement le projet par les bonnes mesures qui furent prises. Mais il
voulait rencontrer la flotte de Rooke et la combattre. Il en avait la
permission\,; il se rembarqua et l'alla chercher.

Il la rencontra auprès de Malaga, et, le 24 septembre, il la combattit
depuis dix heures du matin jusqu'à huit heures du soir. Les flottes,
pour le nombre des vaisseaux, étaient à peu près égales. On n'avait vu
de longtemps à la mer de combat plus furieux et plus opiniâtre. Ils
eurent toujours le vent sur notre flotte. La nuit favorisa leur
retraite. Vilette, lieutenant général qui avait l'avant-garde, défit
celle des ennemis. Tout l'avantage fut du côté du comte de Toulouse,
dont le vaisseau se battit longtemps contre celui de Rooke et le démâta,
qui put se vanter d'avoir remporté la victoire, et qui, profitant du
changement du vent, poursuivit Rooke tout le 25, qui se retirait vers
les côtes de Barbarie. Ils perdirent six mille hommes, le vice-amiral
Hollandais sauté, quelques-uns coulés bas et plusieurs démâtés. Notre
flotte ne perdit ni bâtiment ni mat, mais la victoire coûta cher en gens
distingués par leurs grades et plus encore par leur mérite, outre quinze
cents soldats ou matelots tués ou blessés. Le bailli de Lorraine, fils
de M. le Grand, et chef d'escadre, Bellisle et Évrard, chefs d'escadre,
et un fils du maréchal de Châteaurenauld furent tués. Relingue,
lieutenant général, Gabaret, chef d'escadre, sorti de France pour duel,
mais que le roi d'Espagne avait envoyé sur la flotte, un capitaine de
vaisseau, neveu et du nom du maréchal de Châteaurenauld eurent chacun
une cuisse emportée et moururent quelques jours après, ainsi
qu'Herbault, capitaine de vaisseau, frère d'Herbault intendant des
armées navales. Ce dernier fut tué aux pieds de M. le comte de Toulouse,
qui empêcha qu'on le jetât à la mer avec beaucoup de présence d'esprit,
jusqu'après le combat, pour ne pas perdre ce qu'il pouvait avoir de
papiers de conséquence sur lui, et avoir le temps de le visiter.
Plusieurs de ses pages furent tués et blessés autour de lui. On ne
saurait une valeur plus tranquille qu'il fit paraître pendant toute
l'action, ni plus de vivacité à tout voir et de jugement à commander à
propos. Il avait su gagner les cœurs par ses manières douces et
affables, par sa justice, par sa libéralité. Il en emporta ici toute
l'estime. Ducasse, chef d'escadre, que nous verrons aller plus loin,
reçut une grande blessure et plusieurs autres de moindres.

Le 25 au soir, à force de vent et de manœuvre, on rejoignit Rooke de
fort près. Le comte de Toulouse voulait l'attaquer de nouveau le
lendemain\,; le maréchal de Cœuvres, sans lequel il avait défense de
rien faire, voulut assembler le conseil. Relingue, qui se mourait et qui
aimait le comte, dont il avait bien voulu être premier écuyer, lui
manda, en deux mots de sa main, qu'il battrait les ennemis et qu'il le
conjurait de les attaquer. Le comte fit valoir cette lettre écrite par
un homme d'une capacité si reconnue, et le prix d'une seconde victoire,
qui était Gibraltar. Il captiva les suffrages, il y mit de la douceur,
les raisons les plus fortes, il y ajouta ce qu'il osa d'autorité. Tous
s'y portaient lorsque d'O, le mentor de la flotte, et contre l'avis
duquel le roi avait très précisément défendu au comte de faire aucune
chose, s'y opposa avec un air dédaigneux et une froide, muette, et
suffisante opiniâtreté, qui le dispensa, à la mer, d'esprit et de
raison, comme faisait à la cour la confiance que M\textsuperscript{me}
de Maintenon et le roi avaient prise en lui. L'oracle prononcé, le
maréchal de Cœuvres le confirma malgré lui et ses lumières, et chacun se
retira à son bord consterné, le comte dans sa chambre outré de la plus
vive douleur. Ils ne tardèrent pas à apprendre avec certitude que c'en
était fait de la flotte ennemie s'ils l'eussent attaquée\,; et tout de
suite de Gibraltar, qu'ils auraient trouvé dans le même état qu'il avait
été abandonné. Le comte de Toulouse acquit un grand honneur en tout
genre en cette campagne, et son plat gouverneur y en perdit peu, parce
qu'il n'en avait guère à perdre. Le comte, mouillé devant Malaga, reçut
dans son bord la visite de Villadarias, qui obtint de lui tout ce qu'il
lui demanda pour le siège de Gibraltar. On mit à terre trois mille
hommes, cinquante pièces de gros canon, et généralement tout le
nécessaire pour ce siège, et Pointis fut détaché avec dix vaisseaux et
quelques frégates devant Gibraltar, pour servir de maréchal de camp
aussi au siège, comme étant chef d'escadre. Tous ces ordres exécutés, le
comte et sa flotte appareillèrent pour Toulon.

Châteauneuf, qui avait été ambassadeur en Portugal, et qui, depuis la
rupture, s'était par ordre du roi arrêté à Madrid, venait d'arriver à
Paris. C'était un Savoyard qui, en l'autre guerre, avait quitté son
maître, et avait été fait premier président du sénat de Chambéry par le
roi, et depuis la paix, fait conseiller au parlement, et envoyé
ambassadeur à Constantinople, où il avait très bien fait les affaires du
roi. Lui et l'abbé son frère, qu'on a vu en son temps envoyé pour
rectifier les fautes de l'abbé de Polignac en Pologne, étaient gens de
lettres, d'infiniment d'esprit et de beaucoup d'agrément. Châteauneuf
savait se manier et s'était mis fort avant dans la confiance de la
princesse des Ursins, à qui il ne fut pas inutile.

Sur ses pas arriva Orry. Le roi ne voulut pas le voir et fut au moment
de lui faire faire son procès et de le faire pendre. Il le méritait
bien, mais la chose aurait trop porté contre M\textsuperscript{me} des
Ursins, et M\textsuperscript{me} de Maintenon fut doucement à la parade.
Aubigny, resté à Madrid l'agent intime de sa maîtresse, eut en ce
temps-ci deux mille ducats de pension, malgré l'épuisement des finances,
et une maison dans Madrid, aux dépens du roi. La reine ne cessait
d'intercéder de toutes ses forces que la princesse des Ursins fût
écoutée à Versailles et lui fût après rendue. Outrée des refus, elle se
prit au duc de Berwick comme à l'auteur de la disgrâce d'Orry, par les
plaintes qu'il en avait faites, quoique dès auparavant Puységur eût
vérifié et découvert au roi sa turpitude et son crime. Elle demanda si
instamment le rappel de Berwick, que, pour ne la pas désespérer sur
tout, on le lui accorda, et le liant, l'accort Tessé, malade ou sain
suivant sa basse politique, fut nommé pour lui succéder. Harcourt et
M\textsuperscript{me} de Maintenon savaient bien ce qu'ils faisaient en
procurant ce choix, bien moins utile aux armes que propre à leurs
desseins pour le gouvernement et le cabinet.

Le duc de Mantoue était toujours à Paris. La raison principale qui l'y
avait attiré était, comme je l'ai remarqué, d'y épouser une Française,
et qu'elle lui vînt de la main du roi, toutefois à son gré. Cette vue
n'était pas cachée. M. de Vaudemont était trop son voisin et trop bien
informé pour l'ignorer, trop avisé et trop touché de l'intérêt de la
maison de Lorraine pour ne pas sentir l'importance de lui faire épouser
une princesse de cette maison, qui après sa mort prétendait le
Montferrat. Si ce mariage lui donnait des enfants, encore valait-il
mieux pour eux qu'ils fussent d'une Lorraine, qui cependant serait très
dignement mariée, et longtemps veuve par la disproportion d'âge de sa
belle-sœur {[}avec le mari{]} qu'il lui destinait, pourrait pendant le
mariage prendre de l'ascendant sur ce vieux mari, et veuve, sur ses
enfants et sur le pays par la tutelle, et faire compter avec soi le roi
même par rapport aux affaires d'Italie. M\textsuperscript{me} d'Elbœuf,
troisième femme et veuve alors du duc d'Elbœuf, était fille aînée de la
maréchale de Navailles, dont la mère, M\textsuperscript{me} de
Neuillant, avait recueilli M\textsuperscript{me} de Maintenon à son
retour des îles de l'Amérique, l'avait gardée, nourrie et entretenue
chez elle par charité, et pour s'en défaire l'avait mariée à Scarron.

M\textsuperscript{me} de Navailles, dont le mari {[}fut{]} domestique et
le plus fidèle confident de Mazarin jusque dans les temps les plus
calamiteux de sa vie, avait été dame d'honneur de la reine à son
mariage\,; elle en avait été chassée par le roi et avait coûté à son
mari la charge de capitaine des chevau-légers de la garde et le
gouvernement du Havre de Grâce, pour avoir fait trouver au roi un mur au
lieu d'une porte, par laquelle il entrait secrètement la nuit dans la
chambre des filles de la reine. Les deux reines avaient été outrées de
leur malheur, et la reine mère obtint en mourant leur rappel de leur
exil en leur gouvernement de la Rochelle. Quoique le roi n'eût jamais
bien pardonné ce trait à M\textsuperscript{me} de Navailles, qu'elle
vînt très rarement et très courtement à la cour, le roi, surtout depuis
sa dévotion, n'avait pu lui refuser son estime et des distinctions qui
la marquaient.

Sous ses auspices, M\textsuperscript{me} d'Elbœuf sa fille s'introduisit
à la cour. Avec un air brusque et de peu d'esprit et de réflexion, elle
se trouva très propre au manège et à l'intrigue. Elle trouva moyen de
faire que M\textsuperscript{me} de Maintenon se piquât d'honneur et de
souvenir de M\textsuperscript{me} de Neuillant, et le roi de
considération pour feu M. et M\textsuperscript{me} de Navailles. La
princesse d'Harcourt rompit des glaces auprès de M\textsuperscript{me}
de Maintenon\,; M. le Grand s'intéressa auprès du roi\,;
M\textsuperscript{lle} de Lislebonne et M\textsuperscript{me} d'Espinoy
l'appuyèrent partout (car rien n'est pareil au soutien que toute cette
maison se prête)\,; M\textsuperscript{me} d'Elbœuf joua, fut à Marly, à
Meudon, s'ancra, vit M\textsuperscript{me} de Maintenon quelquefois en
privance, mena sa fille, belle et bien faite, à la cour, qui fut bientôt
de tout avec M\textsuperscript{me} la duchesse de Bourgogne. Elle y
entra si avant et tellement encore dans le gros jeu, où elle avait
embarqué M\textsuperscript{me} la duchesse de Bourgogne avec elle en
beaucoup de dettes que, soit ordre, comme on le crut, soit sagesse de la
mère, elle était avec sa fille dans ses terres de Saintonge depuis plus
de huit mois, et n'en revinrent que pour trouver M. de Mantoue à Paris.
C'était M\textsuperscript{lle} d'Elbœuf que M. de Vaudemont voulait lui
donner, et dont il lui avait parlé dès l'Italie, et pour elle que toute
la maison de Lorraine faisait les derniers efforts.

M. le Prince avait une fille dont il ne savait comment se défaire,
enrichie des immenses biens de Maillé-Brézé, des connétables de
Montmorency, sa mère et sa grand'mère héritières\,; il avait oublié la
fille de La Trémoille et l'héritière de Roye dont il était sorti, et
tous les autres mariages de seigneurs et de leurs filles faits par les
diverses branches de Bourbon. Quelque grandement honorables qu'en
fussent les alliances directes, elles étaient devenues si onéreuses pour
les biens, et si fâcheuses dans les suites par les procédés, qu'il y
avait pour elles maintenant aussi peu d'empressement dans la première
noblesse que de dédains nouveaux dans les princes du sang, ce qui
rendait leurs enfants difficiles à marier, surtout les filles. Outre que
M. de Mantoue parut un débauché pour sa fille à M. le Prince, il avait
des prétentions sur le Montferrat pour une grosse créance sur la
succession de la reine Marie de Gonzague\footnote{Marie de Gonzague, et
  non M\textsuperscript{me} de Gonzague, comme on lit dans les
  précédentes éditions, avait été reine de Pologne.}, tante maternelle
de M\textsuperscript{me} la Princesse, dont toute son industrie n'avait
jamais pu rien tirer depuis tant d'années, ballotté sans cesse entre la
Pologne et la maison de Gonzague. Il espérait donc se procurer le
payement de cette dette de façon ou d'autre par sa fille devenant
duchesse de Mantoue, si elle avait des enfants, ou, si elle n'en avait
point, d'ajouter sa dot et ses droits à sa créance, et, par l'appui de
la France, mettre le Montferrat dans sa maison. Il expliqua au roi ses
vues et son dessein, qui lui permit de les suivre et qui lui promit de
l'y servir de toute sa protection.

M. le Prince, qui craignait là-dessus le crédit de M. le Grand, et son
habitude avec le roi de tout emporter d'assaut, fit sentir au roi, et
plus encore aux ministres, les prétentions des ducs de Lorraine sur le
Montferrat, fortifiées de l'engagement formel de l'empereur, pendant
cette guerre, d'y soutenir le duc de Lorraine de tout son pouvoir, si le
duc de Mantoue venait à mourir sans enfants (que la nécessité lui fit
changer depuis en faveur du duc de Savoie, mais en insistant sur un
dédommagement au duc de Lorraine, comme on le verra dans les pièces
concernant la paix d'Utrecht\footnote{Voy., sur ces Pièces, t. II,
  p.~437, note.} )\,; et le danger pour l'État de laisser mettre un pied
en Italie au duc de Lorraine qui y rendrait l'empereur son protecteur
d'autant plus puissant, et qui engagerait le roi à des ménagements même
sur la Lorraine auxquels on n'était pas accoutumé, surtout en temps de
guerre, et qui pouvaient devenir embarrassants. Ces raisons se firent
sentir, le roi promit à M. le Prince tous les bons offices qui ne
sentiraient ni la contrainte ni l'autorité\,; mais la laideur de
M\textsuperscript{lle} d'Enghien mit un obstacle invincible à cette
affaire.

M. de Mantoue aimait les femmes, il voulait des enfants\,; il s'expliqua
sur les désirs de M. le Prince d'une façon respectueuse qui ne le pût
blesser, mais si nette, qu'il n'osa plus espérer. La maison de Lorraine,
informée par Vaudemont des démarches qu'il avait faites, et que la
timidité de ce petit souverain, à l'égard du gouverneur du Milanais,
avait fait recevoir avec quelque agrément, ne trouva pas à Paris ses
dispositions si favorables. Dès avant de partir de chez lui, son choix
était fait et arrêté. Soupant avec le duc de Lesdiguières peu de temps
avant sa mort, il avait vu à son doigt un petit portrait en bague, qu'il
le pria de lui montrer\,; ayant la bague entre ses mains, il fut charmé
du portrait, et dit à M. de Lesdiguières qu'il le trouvait bien heureux
d'avoir une si belle maîtresse. Le duc de Lesdiguières se mit à rire, et
lui apprit que ce portrait était celui de sa femme. Dès qu'il fut mort,
le duc de Mantoue ne cessa de songer à cette jeune veuve. Sa naissance
et ses alliances étaient fort convenables, il s'en informa encore
secrètement, et il partit dans la résolution de faire ce mariage. En
vain lui fit-on voir M\textsuperscript{lle} d'Elbœuf comme par hasard
dans des églises et en des promenades\,: sa beauté, qui en aurait touché
beaucoup d'autres, ne lui fit aucune impression. Il cherchait partout la
duchesse de Lesdiguières, et il ne la rencontrait nulle part, parce
qu'elle était dans sa première année de veuve\,; mais lui qui voulait
finir, s'en ouvrit à Torcy comme au ministre des affaires étrangères\,;
il en rendit compte au roi, qui approuva fort ce dessein, et qui chargea
le maréchal de Duras d'en parler à sa fille. Elle en fut aussi affligée
que surprise. Elle témoigna à son père sa répugnance à s'abandonner aux
caprices et à la jalousie d'un vieil Italien débauché, l'horreur qu'elle
concevait de se trouver seule entre ses mains en Italie, et la crainte
raisonnable de sa santé avec un homme très convaincu de ne l'avoir pas
bonne.

Je fus promptement averti de cette affaire. Elle et
M\textsuperscript{me} de Saint-Simon vivaient ensemble, moins en
cousines germaines qu'en sœurs\,; j'étais aussi fort en liaison avec
elle. Je lui représentai ce qu'elle devait à sa maison prête à tomber
après un si grand éclat par la mort de mon beau-père, la conduite de mon
beau-frère, l'âge si avancé de M. de Duras, et l'état de son seul frère,
dont les deux nièces emportaient tous les biens. Je lui fis valoir le
désir du roi, les raisons d'État qui l'y déterminaient, le plaisir
d'ôter ce parti à M\textsuperscript{lle} d'Elbœuf, en un mot tout ce
dont je pus m'aviser. Tout fut inutile. Je ne vis jamais une telle
fermeté. Pontchartrain, qui la vint raisonner, y échoua comme moi, mais
il fit pis, car il l'irrita par les menaces qu'il y mêla que le roi le
lui saurait bien faire faire. M. le Prince se joignit à nos désirs,
n'ayant plus aucune espérance pour lui-même, et qui surtout craignait le
mariage d'une Lorraine. Il fut trouver M. de Duras, le pressa d'imposer
à M\textsuperscript{me} de Lesdiguières, lui dit, et le répéta au roi,
qu'il en voulait faire la noce à Chantilly comme de sa propre fille, par
sa proche parenté avec la maréchale de Duras, arrière-petite-fille comme
lui du dernier connétable de Montmorency. Je ne me rebutai point, je
m'adressai à tout ce que je crus qui pouvait quelque chose sur la
duchesse de Lesdiguières, jusqu'aux filles de Sainte-Marie du faubourg
Saint-Jacques, où elle avait été élevée, et qu'elle aimait beaucoup. Je
n'eus pas plus de succès. Cependant M. de Mantoue, irrité par les
difficultés de voir la duchesse de Lesdiguières, se résolut de l'aller
attendre un dimanche aux Minimes. Il la trouva enfermée dans une
chapelle, il s'approcha de la porte pour l'en voir sortir. Il en eut peu
de contentement, ses coiffes épaisses de crêpes étaient baissées, à
peine put-il l'entrevoir. Résolu d'en venir à bout, il en parla à Torcy,
et lui témoigna que la complaisance de se laisser voir dans une église
ne devait pas être si difficile à obtenir. Torcy en parla au roi, qui
lui ordonna devoir M\textsuperscript{me} des Lesdiguières, de lui parler
de sa part du mariage comme d'une affaire qui lui convenait et qu'il
désirait, mais pourtant sans y mêler d'autorité, de lui expliquer la
complaisance que le duc de Mantoue désirait d'elle, et de lui faire
entendre qu'il souhaitait qu'elle la lui accordât. Torcy fut donc à
l'hôtel de Duras lui exposer sa mission\,; sur le mariage, la réponse
fut ferme, respectueuse, courte\,; sur la complaisance, elle dit que les
choses ne devant pas aller plus loin, elle la trouvait fort inutile\,;
mais Torcy insistant sur ce dernier point de la part du roi, il fallut
bien qu'elle y consentît. M. de Mantoue la fut donc attendre au même
lieu où il l'avait déjà une fois si mal vue\,; il trouva
M\textsuperscript{me} de Lesdiguières déjà dans la chapelle, il, s'en
approcha comme l'autre fois. Elle avait pris M\textsuperscript{lle}
d'Espinoy avec elle\,; prête à sortir, elle leva ses coiffes, passa
lentement devant M. de Mantoue, lui fit une révérence en glissant, pour
lui rendre la sienne, et comme ne sachant pas qui il était, et gagna son
carrosse.

M. de Mantoue en fut charmé, il redoubla d'instances auprès du roi et de
M. de Duras\,; l'affaire se traita en plein conseil, comme une affaire
d'État\,: en effet c'en était une. Il fut résolu d'amuser M. de Mantoue,
et cependant de tout faire pour vaincre cette résistance, excepté la
force de l'autorité que le roi voulut bien ne pas employer. Tout fut
promis à M\textsuperscript{me} de Lesdiguières de la part du roi\,: que
ce serait Sa Majesté qui stipulerait dans le contrat de mariage\,; qui
donnerait une dot et la lui assurerait, ainsi que son retour en France
si elle devenait veuve\,; sa protection dans le cours du mariage\,; en
un mot, elle fut tentée de toutes les façons les plus honnêtes, les plus
honorables pour la résoudre. Sa mère, amie de M\textsuperscript{me} de
Creil, si connue pour sa beauté et sa vertu, emprunta sa maison pour une
après-dînée, pour que nous pussions parler plus de suite et plus à notre
aise à M\textsuperscript{me} de Lesdiguières qu'à l'hôtel de Duras. Nous
n'y gagnâmes qu'un torrent de larmes. Peu de jours après, je fus bien
étonné que Chamillart me racontât tout ce qui s'était dit de plus
particulier là-dessus entre la duchesse de Lesdiguières et moi, et
encore entre elle et Pontchartrain. Je sus bientôt après que, craignant
enfin que ses refus ne lui attirassent quelque chose de fâcheux de la
part du roi, ou ne fussent enfin forcés par son autorité absolue, elle
s'était ouverte à ce ministre à notre insu à tous, pour faire par son
moyen que le roi trouvât bon qu'il ne fût plus parlé de ce mariage,
auquel elle ne se pouvait résoudre\,; que M. de Mantoue en fût si bien
averti qu'il tournât ses pensées ailleurs, et qu'elle fût enfin délivrée
d'une poursuite qui lui était devenue une persécution très fâcheuse.
Chamillart la servit si bien que dès lors tout fut fini à cet égard, et
que le roi, flatté peut-être de la préférence que cette jeune duchesse
donnait à demeurer sa sujette sur l'état de souveraine, fit son éloge le
soir dans son cabinet à sa famille et aux princesses, par lesquelles
cela se répandit dans le monde. M. de Duras se souciait trop peu de tout
pour contraindre sa fille, et la maréchale de Duras, qui l'aurait voulu,
n'en eut pas la force. Le duc de Mantoue, informé enfin par Torcy du
regret du roi de n'avoir pu vaincre la résolution de la duchesse de
Lesdiguières de ne se point remarier, car ce fut ainsi qu'on lui donna
la chose, cessa d'espérer, et résolut de se pourvoir ailleurs.

Il faut achever cette affaire tout de suite. Les Lorrains, qui avaient
suivi de toute leur plus curieuse attention la poursuite du mariage avec
la duchesse de Lesdiguières, reprirent leurs espérances, le voyant
rompu, et leurs errements. M. le Prince, qui les suivait de près, parla,
cria, excita le roi, qui se porta jusque-là de faire dire à
M\textsuperscript{me} d'Elbœuf de sa part que ses poursuites lui
déplaisaient. Rien ne les arrêta. Ils comprirent que le roi n'en
viendrait pas jusqu'à des défenses expresses, et sûrs par l'expérience
de n'en être que mieux après, à force de flatteries et de souplesses,
ils poussèrent leur pointe avec roideur. Un certain Casado, qui se
faisait depuis peu appeler marquis de Monteléon, créature de M. de
Vaudemont, et Milanais, avait obtenu par lui l'emploi d'envoyé d'Espagne
à Gênes, puis auprès de M. de Mantoue, dont il gagna les bonnes grâces,
et qu'il accompagna à Paris. C'était un compagnon de beaucoup d'esprit,
d'adresse, d'insinuation et d'intrigue, hardi avec cela et entreprenant,
qu'on verra dans la suite devenir ambassadeur d'Espagne en Hollande et
en Angleterre, et y bien faire ses affaires, et pas mal celles de sa
cour. Il eut pour adjoint, pour marier M. de Mantoue au gré de
Vaudemont, un autre Italien subalterne, théatin renié, connu autrefois à
Paris, dans les tripots, sous le nom de Primi, et qui avait depuis pris
le nom de Saint-Mayol, homme à tout faire, avec de l'esprit et de
l'argent, dont il fut répandu quantité dans la maison. Avec ses mesures
et le congé donné par lime de Lesdiguières, ils vainquirent la
répugnance de M. de Mantoue, qui, au fond, ne pouvait être que caprice
par la beauté, la taille et la naissance de M\textsuperscript{lle}
d'Elbœuf\,; mais la sienne ne laissa pas de les embarrasser.

Avec un rang et du bien, initiée à tout à la cour, et avec une
réputation entière, elle ne se voulait point marier, ou se marier à son
gré, et disait toutes les mêmes raisons qu'avait alléguées
M\textsuperscript{me} de Lesdiguières pour ne point épouser M. de
Mantoue. Elle avait subjugué sa mère, qui trouvait même son joug pesant,
mais qui n'avait garde de s'en vanter. Elle avait donc grande envie de
s'en défaire. Elle la tint à Paris, pour l'éloigner de la cour, de ses
plaisirs, de ses semonces. Elle fit un présent considérable à une
bâtarde de son mari qui avait tout l'esprit du monde et toute la
confiance de sa fille, et lui lit envisager une fortune en Italie. Toute
la maison de Lorraine se mit après M\textsuperscript{lle} d'Elbœuf,
M\textsuperscript{lle} de Lislebonne surtout et M\textsuperscript{me}
d'Espinoy, qui vainquirent enfin sa résistance. Quand ils en furent
venus à ce point, la souplesse auprès du roi vint au secours de l'audace
d'un mariage conclu contre sa volonté qu'il leur avait déclarée. Ils
firent valoir la répugnance invincible du duc de Mantoue pour
M\textsuperscript{lle} d'Enghien, celle de la duchesse de Lesdiguières
pour lui, qui n'avait pu être surmontée, et la spécieuse raison de ne
pas forcer un souverain, son allié, et actuellement dans Paris, sur le
choix d'une épouse, lors surtout qu'il la voulait prendre parmi ses
sujettes (car les Lorrains savent très impudemment disputer, ou très
accortement avouer, selon leur convenance occasionnelle, la qualité de
sujets du roi). Sa Majesté fut donc gagnée, avec cet ascendant de M. le
Grand sur lui, à laisser faire sans rien défendre et aussi sans s'en
mêler. M. le Prince obtint que le mariage ne se ferait pas en France, et
il fut convenu que, le contrat signé entre les parties, elles s'en
iraient chacune de leur côté le célébrer à Mantoue.

M. de Mantoue qui, en six ou sept mois qu'il fut à Paris, ne vit le roi
que cinq ou six fois \emph{incognito} dans son cabinet, reçut du roi, la
dernière fois qu'il le vit à Versailles, une belle épée de diamants que
le roi avait exprès mise à son côté, et qu'il en tira pour la lui
donner, et lui mettre, lui dit-il, les armes à la main comme au
généralissime de ses armées en Italie. Il en avait eu le titre en effet
depuis la rupture avec M. de Savoie, mais pour en avoir le nom et les
honneurs, sans autorité dont il était incapable, et sans exercice dont
il aurait trop appréhendé le péril. Il voulut encore aller prendre congé
du roi à Marly, et lui demanda permission de le saluer encore, en
passant à Fontainebleau, s'en allant à cheval avec sa suite en Italie.

Il arriva à Fontainebleau le 19 septembre, et coucha à la ville chez son
envoyé. Le 20, il dîna chez M. le Grand, vit le roi dans son cabinet, et
soupa chez Torcy. Le 21, il vit encore le roi un moment, dîna chez
Chamillart, et s'en alla, toujours à cheval, coucher à Nemours et tout
de suite en Italie. En même temps lime et Aille d'Elbœuf avec
M\textsuperscript{me} de Pompadour, sœur de M\textsuperscript{me}
d'Elbœuf, passèrent à Fontainebleau sans voir personne, suivant leur
proie jusqu'où leur chemin fourchait, pour aller, lui par terre, elles
par mer, de peur que le marieur ne changeât d'avis et leur fît un
affront\,: c'était pour des personnes de ce rang un étrange personnage
que suivre elles-mêmes leur homme de si près. En chemin la frapper leur
redoubla. Arrivées à Nevers, dans une hôtellerie, elles jugèrent qu'il
ne fallait pas se commettre plus avant, sans de plus efficaces sûretés.
Elles y séjournèrent un jour\,; ce même jour, elles y reçurent la visite
de M. de Mantoue.

M\textsuperscript{me} de Pompadour qui tant qu'elle avait pu, avec son
art et ses minauderies, s'était insinuée auprès de lui dans le dessein
d'en tirer tout ce qu'elle pourrait, lui proposa de ne différer pas à se
rendre heureux par la célébration de son mariage\,; il s'en défendit
tant qu'il put. Pendant cette indécente dispute elles envoyèrent
demander permission à l'évêque. Il se mourait\,; le grand vicaire, à qui
on s'adressa, la refusa. Il dit qu'il n'était pas informé de la volonté
du roi\,; qu'un mariage ainsi célébré ne le serait pas avec la dignité
requise entre de telles personnes\,; que, de plus, il se trouverait
dépouillé des formalités indispensablement nécessaires pour le mettre à
couvert de toute contestation d'invalidité. Une si judicieuse réponse
fâcha fort les dames sans leur faire changer de dessein. Elles
pressèrent M. de Mantoue, lui représentèrent que ce mariage n'était pas
de ceux où il y avait des oppositions à craindre, le rassurèrent sur ce
que, se faisant ainsi dans l'hôtellerie d'une ville de province, le
respect au roi se trouvait suffisamment gardé, le piquèrent sur son état
de souverain qui l'affranchissait des lois et des règles ordinaires,
enfin le poussèrent tant, qu'à force de l'importuner elles l'y firent
consentir. Ils avaient dîné. Aussitôt le consentement arraché, elles
firent monter l'aumônier de son équipage, qui les maria dans le moment.
Dès que cela fut fait, tout ce qui était dans la chambre sortit pour
laisser les mariés en liberté de consommer le mariage, quoi que pût dire
et faire M. de Mantoue pour les retenir, lequel voulait absolument
éviter ce tête-à-tête. M\textsuperscript{me} de Pompadour se tint en
dehors, sur le degré, à écouter près de la porte. Elle n'entendit qu'une
conversation fort modeste et fort embarrassée, sans que les maris
s'approchassent l'un de l'autre. Elle demeura quelque temps de la sorte,
mais jugeant enfin qu'il ne s'en pouvait espérer rien de mieux, et qu'à
tout événement ce tête-à-tête serait susceptible de toutes les
interprétations qu'on lui voudrait donner, elle céda enfin aux cris que
de temps en temps le duc de Mantoue faisait pour rappeler la compagnie,
et qui demandait ce que voulait dire de s'en aller tous et de les
laisser ainsi seuls tous deux. M\textsuperscript{me} de Pompadour appela
sa sœur. Elles rentrèrent\,; aussitôt le duc prit congé d'elles, et
quoiqu'il ne fût pas de bonne heure, monta à cheval et ne les revit
qu'en Italie, encore qu'ils fissent même route jusqu'à Lyon. La nouvelle
de cette étrange célébration de mariage ne tarda guère à se répandre
avec tout le ridicule dont elle était tissue.

Le roi trouva très mauvais qu'on eût osé passer ses défenses. Les
Lorrains, accoutumés de tout oser, puis de tout plâtrer, et à n'en être
pas plus mal avec le roi, eurent la même issue de cette entreprise\,;
ils s'excusèrent sur la crainte d'un affront, et il pouvait être que M.
de Mantoue, amené à leur point à force de ruses, d'artifices, de
circonventions, n'eût pas mieux aimé que de gagner l'Italie, puis se
moquer d'eux. Ils aimèrent donc mieux encourir la honte qu'ils
essuyèrent en courant, et forçant M. de Mantoue, que celle de son dédit,
accoutumés comme ils sont à tant d'étranges façons de faire des
mariages. De Lyon M\textsuperscript{me} de Pompadour revint pleine
d'espérance de l'ordre pour son mari à la recommandation du duc de
Mantoue, qui n'eut aucun succès.

M\textsuperscript{me} d'Elbœuf et sa fille allèrent s'embarquer à Toulon
sur deux galères du roi, par une mescolance\footnote{Combinaison.} rare
d'avoir défendu à M\textsuperscript{me} d'Elbœuf de penser à ce mariage,
ou l'équivalent de cela, de n'avoir voulu dans la suite, ni le
permettre, ni le défendre, ni s'en mêler, d'avoir défendu après qu'il se
fît en France, et de prêter après deux de ses galères pour l'aller faire
ou achever. Ces galères eurent rudement la chasse par des corsaires
d'Afrique. Ce fut grand dommage qu'elles ne fussent prises pour achever
le roman. Débarquées enfin à sauveté, M. de Vaudemont les joignit. Il
persuada à M. de Mantoue de réhabiliter son mariage par une célébration
nouvelle qui rétablît tout le défectueux de celle de Nevers. Ce prince
l'avait lui-même trouvée si contraire aux défenses précises que le roi
leur avait faites de se marier en France, qu'il l'avait fait assurer par
son envoyé qu'il n'en était rien, et que ce n'étaient que des bruits
faux que ceux qui couraient de son mariage fait à Nevers\,; cette raison
le détermina donc à suivre le conseil de Vaudemont. L'évêque de Tortone
les maria dans Tortone publiquement, en présence de la duchesse d'Elbœuf
et du prince et de la princesse de Vaudemont.

Ce beau mariage, tant poursuivi\,: par les Lorrains, tant fui par M. de
Mantoue, fait avec tant d'indécence, et refait après pour la sûreté de
l'état de M\textsuperscript{lle} d'Elbœuf, n'eut pas des suites
heureuses. Soit dépit de s'être laissé acculer à épouser malgré lui,
soit caprice ou jalousie, il renferma tout aussitôt sa femme avec tant
de sévérité, qu'elle n'eut permission de voir qui que ce fût, excepté sa
mère, encore pas plus d'une heure par jour, et jamais seule, pendant les
quatre ou cinq mois qu'elle demeura avec eux. Ses femmes n'entraient
chez elle que pour l'habiller et la déshabiller précisément. Il fit
murer ses fenêtres fort haut et la fit garder à vue par de vieilles
Italiennes. Ce fut donc une cruelle prison. Ce traitement, auquel je ne
m'attendais pas, et le peu de considération, pour ne pas dire le mépris,
qu'on témoigna ici à ce prince toujours depuis son départ, me,
consolèrent beaucoup de l'invincible opiniâtreté de la duchesse de
Lesdiguières. J'eus pourtant peine à croire que, prise de son choix,
elle eût essuyé les mêmes duretés, ni lui les traitements qu'il reçut,
s'il n'eût pas fait un mariage auquel le roi se montra si contraire. Six
mois après, M\textsuperscript{me} d'Elbœuf, outrée de dépit, mais trop
glorieuse pour le montrer, revint, remplie, à ce qu'elle affectait, des
grandeurs de son gendre et de sa fille, ravie pourtant au fond d'être
défaite d'une charge devenue si pesante. Elle déguisa les malheurs de sa
fille jusqu'à s'offenser qu'on dît et qu'on crût ce qui en était, et ce
qui en revenait par toutes les lettres de nos armées. Mais à la fin,
Lorraine d'alliance non de naissance, le temps et la force de la vérité
les lui fit avouer. Fin rare, et qui montra bien tout l'art et
l'ascendant des Lorrains, elle ne fut pas moins bien traitée après ce
voyage que si elle n'eût rien fait que de la volonté du roi.

Je me suis peut-être trop étendu sur cette affaire. Il m'a paru qu'elle
le méritait par sa singularité, et plus encore pour montrer par des
faits de cette sorte quelle fut la cour du roi. Reprenons maintenant le
courant où nous l'avons laissé.

\hypertarget{chapitre-xviii.}{%
\chapter{CHAPITRE XVIII.}\label{chapitre-xviii.}}

1704

~

{\textsc{Tracy\,; sa catastrophe\,; sa mort.}} {\textsc{- Reineville
retrouvé.}} {\textsc{- Mort de Rigoville.}} {\textsc{- Mort et
conversion de la comtesse d'Auvergne.}} {\textsc{- Mort et caractère du
prince d'Espinoy.}} {\textsc{- Assassinat, extraction, caractère de
Vervins\,; singularité de sa fin.}} {\textsc{- Voyage de Fontainebleau
par Sceaux.}} {\textsc{- Maréchal de Villeroy à la cour, puis à
Bruxelles.}} {\textsc{- Électeur de Bavière à Bruxelles.}} {\textsc{-
Électeur de Cologne à Lille.}} {\textsc{- Petits exploits de La
Feuillade.}} {\textsc{- Anecdote curieuse.}} {\textsc{- État brillant de
M\textsuperscript{me} la duchesse de Bourgogne.}} {\textsc{- Nangis.}}
{\textsc{- M\textsuperscript{me} de La Vrillière.}} {\textsc{-
Maulevrier et sa femme.}} {\textsc{- Maulevrier va avec Tessé en
Espagne, passe par Toulouse, y voit la princesse des Ursins.}}
{\textsc{- Tessé grand d'Espagne en arrivant à Madrid.}} {\textsc{-
Comte de Toulouse chevalier de la Toison d'Or.}} {\textsc{- Mort du
prince de Montauban\,; caractère de sa femme.}} {\textsc{- Mort du fils
du comte de Grignan\,; mot impertinent de sa mère.}} {\textsc{- Mort de
Coigny.}} {\textsc{- Mort de M. de Duras\,; sa fortune et son
caractère.}} {\textsc{- Comédies, bienséances.}} {\textsc{- Ruse
d'orgueil de M. de Soubise inutile.}} {\textsc{- Régiment des gardes
arraché par ruse au maréchal de Boufflers pour le duc de Guiche, et le
maréchal fait capitaine des gardes du corps.}} {\textsc{- Duchesse de
Guiche.}} {\textsc{- Tallard gouverneur de la Franche-Comté\,; mot salé
de M. le duc d'Orléans.}} {\textsc{- Quarante mille livres de pension au
fils enfant du prince de Conti.}}

~

La triste destinée que le pauvre Tracy acheva en ce temps-ci put servir
de grande leçon aux ambitieux, même qui méritent les faveurs de la
fortune. C'était un gentilhomme de Bretagne, d'esprit et bien fait,
parent proche de la duchesse de Coislin, mais pauvre, qui fut exempt,
puis enseigne des gardes du corps. Il se distingua à la cour et à la
guerre par ses divers talents, et les fit servir les uns aux autres. Il
devint un des meilleurs partisans de l'armée\,; ce fut lui qui, étant
dehors, sauva l'armée de M. de Luxembourg lors du combat de Steinkerque,
comme je l'ai raconté en son lieu. Sa volonté, sa valeur, l'exécution
parfaite de tout ce dont il était très ordinairement chargé par les
généraux, lui acquirent leur estime puis leur amitié. Il entra dans
toute la confiance de M. de Luxembourg. Son service auprès de
Monseigneur lui en avait valu des bontés très particulières. Une des
filles d'honneur de M\textsuperscript{me} la princesse de Conti le
voyait de bon œil, et de meilleur encore la princesse même. Il fut
recueilli et considéré\,; il avait lieu d'attendre tout de la fortune,
et à la guerre et à la cour. Malheureusement elle ne le servit pas aussi
rapidement qu'il l'avait attendu. Sa tête s'altéra\,; on s'en aperçut\,;
on s'en tut jusqu'à ce que des disparates plus fortes firent juger
dangereux de le laisser approcher d'aussi près que le demandait son
service d'enseigne des gardes du corps en quartier. Il était brigadier,
on lui donna un régiment. Ce changement d'état acheva de lui tourner la
tête, tant qu'à la fin on lui fit entendre de ne plus venir à
Versailles. Cela combla son malheur. Son mal redoubla et se tourna
bientôt en fureur, qui obligea de le mettre à Charenton, chez les pères
de la Charité, où le roi fit prendre grand soin de lui, et où il mourut
en ce temps-ci, trois ou quatre ans après y avoir été mis. Il n'était
point marié. Ce fut grand dommage, je le connaissais extrêmement, et je
n'ai guère trouvé un plus galant homme. En ce même temps Reineville,
lieutenant des gardes du corps, qu'on a vu (t. II, p.~262) disparaître
en 1699, coulé à fond par le jeu, fut reconnu et retrouvé caché et
servant pour sa paye dans les troupes de Bavière. En même temps aussi
mourut Rigoville, lieutenant général, fort vieux et homme d'honneur, de
valeur et de mérite, qui avait longtemps commandé les mousquetaires
noirs, sous Jouvelle et Vins. Le vieux La Rablière mourut aussi à Lille,
oui il commandait depuis très longtemps. Il était lieutenant général,
grand-croix de Saint-Louis dès l'institution, frère de la maréchale de
Créqui, Il but du lait à ses repas toute sa vie, et mangeait bien et de
tout jusqu'à quatre-vingt-sept ou huit ans, et la tête entière. Il avait
été très bon officier, mais un assez méchant homme\,; il ne but jamais
de vin\,; honorable, riche, de l'esprit et sans enfants. Le maréchal de
Boufflers le protégeait fort. Il se piquait de reconnaissance pour le
maréchal de Créqui, et rendit toute sa vie de grands devoirs à la
maréchale de Créqui.

La comtesse d'Auvergne acheva aussi une courte vie par une maladie fort
étrange et assez rare, qui fut une hydropisie de vents. Elle ne laissa
point d'enfants. On a vu en son lieu qui elle était et comment se fit ce
mariage. Le comte d'Auvergne, qui avait obtenu la permission de l'amener
à Paris et à la cour quoique huguenote, désirait fort qu'elle se fît
catholique. Un fameux avocat qui s'appelait Chardon, et qui l'a été de
mon père et le mien, avait été huguenot et sa femme aussi\,; ils étaient
de ceux qui avaient fait semblant d'abjurer, mais qui ne faisaient aucun
acte de catholiques, qu'on connaissait parfaitement pour tels, qui même
ne s'en cachaient pas, mais que la grande réputation de Chardon
soutenait et le nombre des protecteurs considérables qu'elle lui avait
acquis. Ceux-là mêmes avaient fait tout ce qu'ils avaient pu pour leur
persuader au moins d'écouter\,; ils n'en purent venir à bout\,: le
moment de Dieu n'était pas venu. Il arriva enfin\,; ils étaient tous
deux vertueux, exacts à tout, et d'une piété dans leur religion qui
aurait fait honneur à la véritable. Étant un matin dans leur carrosse
tous deux arrêtés auprès de l'Hôtel-Dieu, attendant une réponse que leur
laquais fut un très long temps à rapporter, M\textsuperscript{me}
Chardon porta ses yeux vis-à-vis d'elle au hasard sur le grand portail
de Notre-Dame, et peu à peu tomba dans une profonde rêverie, qui se doit
mieux appeler réflexion.

Son mari, qui à la fin s'en aperçut, lui demanda à quoi elle rêvait si
fort, et la poussa même du coude pour l'engager à lui répondre. Elle lui
montra ce qu'elle considérait, et lui dit qu'il y avait bien des siècles
avant Luther et Calvin que toutes ces figures de saints avaient été
faites à ce portail, que cela prouvait qu'on invoquait donc alors les
saints\,; que l'opposition de leurs réformateurs à cette opinion si
ancienne était une nouveauté\,; que cette nouveauté lui rendait suspects
les autres dogmes qu'ils leur enseignaient contraires à l'antiquité
catholique\,; que ces réflexions qu'elle n'avait jamais faites lui
donnaient beaucoup d'inquiétude et lui faisaient prendre la résolution
de chercher à s'éclaircir. Chardon trouva qu'elle avait raison, et dès
ce même jour ils se mirent à chercher la vérité, puis à consulter, enfin
à se faire instruire. Cela dura plus d'un an, pendant lequel les parties
et les amis de Chardon se plaignaient qu'il ne travaillait plus, et
qu'on ne pouvait plus le voir ni sa femme. Enfin secrètement instruits
et pleinement persuadés, ils se déclarèrent tous deux, ils firent une
abjuration nouvelle, et tous deux ont passé depuis une longue vie dans
la piété et les bonnes œuvres, surtout dans un zèle ardent de procurer à
leurs anciens frères de religion la même grâce qu'ils avaient reçue.
M\textsuperscript{me} Chardon s'instruisit fort dans la controverse,
elle convertit beaucoup de huguenots. Le comte d'Auvergne l'attira chez
sa femme. L'une et l'autre avaient de l'esprit et de la douceur. La
comtesse la vit volontiers, M\textsuperscript{me} Chardon en profita,
elle en fit une très bonne catholique. Tous les Bouillon, outrés de ce
mariage, l'avaient reçue fort froidement\,; sa vertu, sa douceur, ses
manières à la fin les charma. Elle devint le lien du père et des
enfants, et elle s'acquit le cœur et l'estime d'eux tous et de tout ce
qui la connut particulièrement, dont elle fut extrêmement regrettée.

Le prince d'Espinoy ne le fut pas tant à beaucoup près. Il mourut de la
petite vérole à Strasbourg, par l'opiniâtreté d'avoir voulu changer de
linge trop tôt et faire ouvrir ses fenêtres. C'était un homme d'assez
peu agréable figure, qui avait beaucoup d'esprit et l'esprit fort orné,
avec beaucoup de valeur. J'avais été élevé comme avec lui, c'est-à-dire
à nous voir continuellement plusieurs que nous étions enfants, puis
jeunes gens. Sa mère l'avait gâté et c'était dommage, car il avait des
talents pour tout et beaucoup d'honneur. Mais je n'ai connu personne
plus follement glorieux ni plus continuellement avantageux. Il abusa
donc de tout ce qu'il avait de bon et d'utile, ne ménagea personne,
voulut surpasser chacun en tout, et fut le fléau de sa femme, parce
qu'elle était d'une maison souveraine qui avait un rang qu'il n'avait
pas, et un crédit et une considération à la cour et dans le monde dont
il ne voulait pas qu'on crût qu'il voulût dépendre. Avec ce rang des
siens et cette faveur si déclarée de Monseigneur, elle se conduisit avec
lui comme un ange, sans qu'elle ait jamais pu rendre sa condition plus
heureuse avec lui\,; aussi se trouva-t-elle bien délivrée, quoiqu'en
gardant toutes les bienséances. Presque personne de la cour ni des
armées ne le plaignit. Il laissa un fils et une fille, desquels la
catastrophe mérita, trente ans après, la compassion de tout le monde, et
combla les malheurs que leur mère avait commencé d'éprouver.

Il arriva en ce mois de septembre un étrange assassinat. Le comte de
Grandpré, chevalier de l'ordre en 1661, frère aîné du maréchal de
Joyeuse, chevalier de l'ordre en 1688, mort sans enfants, avait laissé
des enfants de deux lits. Sa seconde femme était fille et sœur des deux
marquis de Vervins, l'un après l'autre premiers maîtres d'hôtel du roi.
Le dernier des deux mourut jeune en 1663. Il était gendre du maréchal
Fabert, par conséquent beau-frère du marquis de Beuvron et de Caylus,
père de celui qui a passé en Espagne, du mari de M\textsuperscript{me}
de Caylus, nièce à la mode de Bretagne de M\textsuperscript{me} de
Maintenon, et de l'abbé de Caylus que nous venons de voir évêque
d'Auxerre. Vervins avait épousé l'aînée qu'il laissa grosse de Vervins
dont il s'agit ici, et qui se remaria depuis en Flandre au comte de
Mérode. Vervins eut force procès avec ses cousins germains, enfants de
la sœur de son père et du comte de Grandpré, dont il fut étrangement
tourmenté presque toute sa vie. Enfin il était sur le point d'achever de
les gagner tous, lorsqu'un de ses cousins germains, qui avait des
prieurés et se faisait appeler l'abbé de Grandpré, le fit attaquer comme
il passait dans son carrosse sur le quai de la Tournelle, devant la
communauté de M\textsuperscript{me} de Miramion. Il fut blessé de
plusieurs coups d'épée et son cocher aussi, qui le voulut défendre. Sur
la plainte en justice, l'abbé s'enfuit en pays étranger d'où il n'est
jamais revenu, et bientôt après, sur les preuves, {[}fut{]} condamné à
être roué vif. Il y avait longtemps que Vervins était menacé d'un
mauvais coup de sa part.

Vervins se prétendait Comminges, des anciens comtes de ce nom. Son
bisaïeul, père du premier des deux premiers maîtres d'hôtel du roi,
était ce Saubole, gouverneur de la citadelle de Metz, qui est si connu
dans la vie du duc d'Épernon, et dans les Mémoires de ces temps-là, qui
avait épousé l'héritière de Vervins qui était Coucy. Le grand-père de ce
Saubole était second fils d'Aimery, dit de Comminges, seigneur de
Puyguilhem, dont le père, nommé aussi Aimery, était cru sorti des
vicomtes de Conserans, mais dont l'union n'était pas bien prouvée. Pour
ces Conserans, leur auteur Roger était marqué comme étant quatrième fils
de Bernard II, comte de Comminges et de Diaz de Muret, qui fonda les
abbayes de Bonnefonds et de Feuillans, et qui fut tué près la ville de
Gaudens en 1150\,: voilà pour l'extraction de Vervins. Quant à lui,
c'était un grand homme fort bien fait, d'un visage assez agréable, de
l'esprit, quelque lecture, et fort le vol des femmes\,; particulier,
extrêmement paresseux, fort dans la liaison et les parties de M. le Duc,
et fort dans le grand monde. Il quitta le service de bonne heure, fit
plusieurs séjours chez lui en Picardie, toujours reçu avec empressement
quand il en revenait. À la fin, sans dire mot à personne, il se confina
dans une terre en Picardie, sans aucune cause de dégoût ni de déplaisir,
sans besoins du côté de ses affaires, il était riche, arrangé, et ne fut
jamais marié\,; sans vue de piété, il n'en eut pas la moindre veine\,;
sans occasion de santé, qu'il eut toujours parfaite\,; et sans goût
d'ouvriers, dont il n'employa aucun\,; encore moins entraîné par le
plaisir de la chasse, où il n'alla jamais. Il demeura chez lui plusieurs
années sans aucun commerce avec personne, et ce qui est
incompréhensible, sans bouger de son lit, que le temps de le faire
faire. Il y dînait et y soupait tout seul, y faisait le peu d'affaires
qu'il avait, et y recevait le peu de gens qu'il ne pouvait éconduire,
et, depuis qu'il avait les yeux ouverts jusqu'à ce qu'il les fermât, y
travaillait en tapisserie, et lisait quelquefois un peu, et a persévéré
jusqu'à la mort dans cette étrange sorte de vie, si uniquement
singulière que j'ai voulu la rapporter.

Le roi alla à Fontainebleau, où il arriva le 12 septembre, ayant
séjourné un jour à Sceaux\,; la cour de Saint-Germain y vint le 23, et y
demeura jusqu'au 6 octobre. En y arrivant le roi apprit que les armées
alliées avaient toutes passé le Rhin sur le pont de Philippsbourg, et
bientôt après que Landau était assiégé par le prince Louis de Bade, qui
attendait le roi des Romains qui y arriva le 25 septembre, et que le
prince Eugène et le duc de Marlborough commandaient l'armée
d'observation qu'ils portèrent sur la Lauter. Marsin demeura avec la
sienne sous Haguenau. Le maréchal de Villeroy et son fils s'en allèrent
de leurs personnes en Flandre, passant à Fontainebleau, où ils
demeurèrent quelques jours. Ils allèrent après trouver l'électeur de
Bavière à Bruxelles, et chemin faisant virent l'électeur de Cologne à
Lille, où il avait établi sa demeure, en même temps que son frère était
allé à Bruxelles après avoir {[}passé{]} ensemble quelques jours.

Pendant tous ces malheurs, Villars était venu à bout d'achever à peu
près de dissiper les fanatiques\,; cinq ou six de leurs chefs, les
autres tués ou accommodés et sortis du pays, obtinrent de se retirer à
Genève\,; on comptait qu'il ne restait qu'une centaine de ces gens-là
dans les hautes Cévennes, et qu'il n'était plus besoin de laisser de
troupes en Languedoc. Peu de jours après, le roi reçut la nouvelle de la
prise d'Ivrée, après un siège assez court, et qui ne coûta guère que
deux cents hommes et quatre cents blessés. M. de Vendôme eut avec la
place onze bataillons prisonniers de guerre.

La Feuillade n'épargnait pas les courriers pour annoncer ses conquêtes
dans les vallées des Alpes\,: tantôt un petit fort pris, défendu par des
milices, tantôt quelque peu de troupes réglées forcées derrière un
retranchement qui gardait quelque passage. Tout cela était célébré,
comme si c'eût été quelque chose. Chamillart, ravi, en recevait les
compliments, et savait faire valoir ces merveilles au roi et à
M\textsuperscript{me} de Maintenon.

Il se présente ici une anecdote très saga à taire, très curieuse à
écrire à qui a vu les choses d'aussi près que j'ai fait\,; ce qui me
détermine au second parti, c'est que le fait en gros n'a pas été ignoré,
et que les trônes de tous les siècles et de toutes les nations
fourmillent d'aventures pareilles. Faut-il donc le dire\,? nous avions
une princesse charmante, qui, par ses grâces, ses soins et des façons
uniques en elle, s'était emparée du cœur et des volontés du roi, de
M\textsuperscript{me} de Maintenon et de Mgr le duc de Bourgogne. Le
mécontentement extrême, trop justement conçu contre le duc de Savoie,
son père, n'avait pas apporté la plus petite altération à leur tendresse
pour elle. Le roi, qui ne lui cachait rien, qui travaillait avec les
ministres en sa présence toutes les fois qu'elle y voulait entrer et
demeurer, eut toujours l'attention pour elle de ne lui ouvrir jamais la
bouche de tout rien de ce qui pouvait regarder le duc son père, ou avoir
trait à lui. En particulier, elle sautait au cou du roi à toute heure,
se mettait sur ses genoux, le tourmentait de toutes sortes de badinages,
visitait ses papiers, ouvrait et lisait ses lettres en sa présence,
quelquefois malgré lui, et en usait de même avec M\textsuperscript{me}
de Maintenon. Dans cette extrême liberté, jamais rien ne lui échappa\,:
contre personne\,; gracieuse à tous et parant même les coups, toutes les
fois qu'elle le pouvait, attentive aux domestiques intérieurs du roi,
n'en dédaignant pas les moindres\,; bonne aux siens et vivant avec ses
dames comme une amie, et en toute liberté, vieilles et jeunes\,; elle
était l'âme de la cour, elle en était adorée\,; tous, grands et petits,
s'empressaient de lui plaire\,; tout manquait à chacun en son absence,
tout était rempli en sa présence\,; son extrême faveur la faisait
infiniment compter, et ses manières lui attachaient tous les cœurs. Dans
cette situation brillante le sien ne fut pas insensible.

Nangis, que nous voyons aujourd'hui un fort plat maréchal de France,
était alors la fleur des pois\,; un visage gracieux sans rien de rare,
bien fait sans rien de merveilleux, élevé dans l'intrigue et dans la
galanterie par la maréchale de Rochefort, sa grand'mère, et
M\textsuperscript{me} de Blansac, sa mère, qui y étaient des maîtresses
passées. Produit tout jeune par elles dans le grand monde, dont elles
étaient une espèce de centre, il n'avait d'esprit que celui de plaire
aux dames, de parler leur langage et de s'assurer les plus désirables
par une discrétion qui n'était pas de son âge et qui n'était plus de son
siècle. Personne que lui n'était alors plus à la mode\,; il avait eu un
régiment tout enfant\,; il avait montré de la volonté, de l'application,
et une valeur brillante à la guerre, que les dames avaient fort relevée
et qui suffisait à son âge\,; il était fort de la cour de Mgr le duc de
Bourgogne, et à peu près de son âge, et il en était fort bien traité. Ce
prince, passionnément amoureux de son épouse, n'était pas fait comme
Nangis\,; mais la princesse répondait si parfaitement à ses
empressements qu'il est mort sans soupçonner jamais qu'elle eût des
regards pour un autre que pour lui. Il en tomba pourtant sur Nangis, et
bientôt ils redoublèrent. Nangis n'en fut pas ingrat, mais il craignit
la foudre, et son cœur était pris.

M\textsuperscript{me} de La Vrillière qui, sans beauté, était jolie
comme les amours et en avait toutes les grâces, en avait fait la
conquête. Elle était fille de M\textsuperscript{me} de Mailly, dame
d'atours de M\textsuperscript{me} la duchesse de Bourgogne\,; elle était
de tout dans sa cour\,; la jalousie l'éclaira bientôt. Bien loin de
céder à la princesse, elle se piqua d'honneur de conserver sa conquête,
de la lui disputer, de l'emporter. Cette lutte mit Nangis dans
d'étranges embarras\,: il craignait les furies de sa maîtresse qui se
montrait à lui plus capable d'éclater qu'elle ne l'était en effet. Outre
son amour pour elle, il craignait tout d'un emportement et voyait déjà
sa fortune perdue. D'autre part, sa réserve ne le perdait pas moins
auprès d'une princesse qui pouvait tant, qui pourrait tout un jour et
qui n'était pas pour céder, non pas même pour souffrir une rivale. Cette
perplexité, à qui était au fait, donnait des scènes continuelles. Je ne
bougeais alors de chez M\textsuperscript{me} de Blansac à Paris, et de
chez la maréchale de Rochefort à Versailles\,; j'étais ami intime de
plusieurs dames du palais qui voyaient tout et ne me cachaient rien\,;
j'étais avec la duchesse de Villeroy sur un pied solide de confiance, et
avec la maréchale, tel, qu'ayant toujours été mal ensemble, je les
raccommodai si bien que jusqu'à leur mort elles ont vécu ensemble dans
la plus tendre intimité\,; la duchesse de Villeroy savait tout par
M\textsuperscript{me} d'O, et par la maréchale de Cœuvres qui était
raffolée d'elle, et qui étaient les confidentes et quelque chose de
plus\,; la duchesse de Lorges, ma belle-sœur, ne l'était guère moins et
tous les soirs me contait tout ce qu'elle avait vu et appris dans la
journée\,; j'étais donc instruit exactement et pleinement d'une journée
à l'autre. Outre que rien ne me divertissait davantage, les suites
pouvaient être grandes, et il était important pour l'ambition d'être
bien informé. Enfin toute la cour assidue et éclairée s'aperçut de ce
qui avait été caché d'abord avec tant de soin. Mais, soit crainte, soit
amour de cette princesse qu'on adorait, cette même cour se tut, vit
tout, se parla entre elle et garda le secret qui ne lui était pas même
confié. Ce manège, qui ne fut pas sans aigreur de la part de
M\textsuperscript{me} de La Vrillière pour la princesse, et quelquefois
insolemment placé, ni sans une souffrance et un éloignement doucement
marqué de la princesse pour elle, fit longtemps un spectacle fort
singulier.

Soit que Nangis, trop fidèle à son premier amour eût besoin de quelques
grains de jalousie, soit que la chose se fît naturellement, il arriva
qu'il trouva un concurrent. Maulevrier, fils d'un frère de Colbert, mort
de douleur de n'être pas maréchal de France à la promotion où le
maréchal de Villeroy le fut, avait épousé une fille du maréchal de
Tessé. Maulevrier n'avait point un visage agréable, sa figure était
d'ailleurs très commune. Il n'était point sur le pied de la galanterie.
Il avait de l'esprit, et un esprit fertile en intrigues sourdes, une
ambition démesurée, et rien qui la pût retenir, laquelle allait jusqu'à
la folie. Sa femme était jolie, avec fort peu d'esprit, tracassière, et,
sous un extérieur de vierge, méchante au dernier point. Peu à peu elle
fut admise, comme fille de Tessé, à monter dans les carrosses, à manger,
à aller à Marly, à être de tout chez M\textsuperscript{me} la duchesse
de Bourgogne, qui se piquait de reconnaissance pour Tessé qui avait
négocié la paix de Savoie et son mariage, dont le roi lui savait fort
bon gré. Maulevrier écuma des premiers ce qui se passait à l'égard de
Nangis\,; il se fit donner des privances chez M\textsuperscript{me} la
duchesse de Bourgogne par son beau-père\,; il s'y rendit assidu\,;
enfin, excité par l'exemple, il osa soupirer. Lassé de n'être point
entendu, il hasarda d'écrire\,; on prétendit que M\textsuperscript{me}
Cantin, amie intime de Tessé, trompée par le gendre, crut recevoir de sa
main des billets du beau-père, et que, les regardant comme sans
conséquence, elle les rendait. Maulevrier, sous le nom de son beau-père,
recevait, crut-on, la réponse aux billets par la même main qui les avait
remis. Je n'ajouterai pas ce qu'on crut au delà. Quoi qu'il en soit, on
s'aperçut de celui-ci comme de l'autre, et on s'en aperçut avec le même
silence. Sous prétexte d'amitié pour M\textsuperscript{me} de
Maulevrier, la princesse alla plus d'une fois pleurer avec elle, et chez
elle, dans des voyages de Marly, le prochain départ de son mari et les
premiers jours de son absence, et quelquefois M\textsuperscript{me} de
Maintenon avec elle. La cour riait\,: si les larmes étaient pour lui ou
pour Nangis, cela était douteux\,; mais Nangis toutefois, réveillé par
cette concurrence, jeta M\textsuperscript{me} de La Vrillière dans
d'étranges douleurs et dans une humeur dont elle ne fut point maîtresse.

Ce tocsin se fit entendre à Maulevrier. De quoi ne s'avise pas un homme
que l'amour ou l'ambition possède à l'excès\,! Il fit le malade de la
poitrine, se mit au lait, fit semblant d'avoir perdu la voix, et sut
être assez maître de soi pour qu'il ne lui échappât pas un mot à voix
intelligible pendant plus d'un an, et par là ne fit point la campagne,
et demeura à la cour. Il fut assez fou pour conter ce projet et bien
d'autres au duc de Lorges, son ami, par qui dans le temps même je le
sus. Le fait était que, se mettant ainsi dans la nécessité de ne parler
jamais à personne qu'à l'oreille, il se donnait la liberté de parler de
même à M\textsuperscript{me} la duchesse de Bourgogne devant toute la
cour, sans indécence et sans soupçon que ce fait en secret. De cette
sorte, il lui disait tout ce qu'il voulait tous les jours, et il prenait
son temps de manière qu'il n'était point entendu, et que parmi des
choses communes dont les réponses se faisaient tout haut, il en mêlait
d'autres dont les réponses courtes se ménageaient de façon qu'elles ne
pouvaient être entendues que de lui. Il avait tellement accoutumé le
monde à ce manège, qu'on n'y prenait plus garde, sinon de le plaindre
d'un si fâcheux état\,; mais il arrivait pourtant, que ce qui approchait
le plus M\textsuperscript{me} la duchesse de Bourgogne en savait assez
pour ne s'empresser pas autour d'elle quand Maulevrier s'en approchait
pour lui parler. Ce même manège dura plus d'un an, souvent en reproches,
mais les reproches réussissent rarement en amour\,; la mauvaise humeur
de M\textsuperscript{me} de La Vrillière le tourmentait\,; il croyait
Nangis heureux, et il voulait qu'il ne le fût pas. Enfin, la jalousie et
la rage le transportèrent au point de hasarder une extrémité de folie.

Il alla à la tribune sur la fin de la messe de M\textsuperscript{me} la
duchesse de Bourgogne. En sortant il lui donna la main et prit un jour
qu'il savait que Dangeau, chevalier d'honneur, était absent. Les
écuyers, soumis au premier écuyer son beau-père, s'étaient accoutumés à
lui céder cet honneur à cause de sa voix éteinte, pour le laisser parler
en chemin, et se retiraient par respect pour ne pas entendre. Les dames
suivaient toujours de loin, tellement qu'en pleins appartements et au
milieu de tout le monde, il avait, depuis la chapelle jusqu'à
l'appartement de M\textsuperscript{me} la duchesse de Bourgogne, la
commodité du tête-à-tête, qu'il s'était donné plusieurs fois. Ce jour-là
il chanta pouille sur Nangis à la princesse, l'appela par toutes sortes
de noms, la menaça de tout faire savoir au roi, à M\textsuperscript{me}
de Maintenon, au prince son mari, lui serra les doigts à les lui
écraser, en furieux, et la conduisit de la sorte jusque chez elle. En
arrivant, tremblante et prête à s'évanouir, elle entra tout de suite
dans sa garde-robe, et y appela M\textsuperscript{me} de Nogaret,
qu'elle appelait sa petite bonne, et à qui elle allait volontiers au
conseil, quand elle ne savait plus où elle en était. Là elle lui raconta
ce qui venait de lui arriver, et lui dit qu'elle ne savait comment elle
n'était pas rentrée sous les parquets, comment elle n'en était pas
morte, comment elle avait pu arriver jusque chez elle. Jamais elle ne
fut si éperdue. Le même jour M\textsuperscript{me} de Nogaret le conta à
M\textsuperscript{me} de Saint-Simon et à moi, dans le dernier secret et
la dernière confiance. Elle conseilla à la princesse de filer doux avec
un fou si dangereux et si fort hors de tout sens et de toute mesure, et
toutefois d'éviter sur toute chose de se commettre avec lui. Le pis fut
qu'au partir de là, il menaça, dit force choses sur Nangis, comme un
homme qui en était vivement offensé, qui était résolu d'en tirer raison
et de l'attaquer partout. Quoiqu'il n'en dît pas la cause, elle était
claire, On peut juger de la frayeur qu'en conçut la princesse, de la
peur et des propos de M\textsuperscript{me} de La Vrillière et de ce que
devint Nangis. Il était brave de reste pour n'en craindre personne, et
prêter le collet à quiconque, mais le prêter sur pareil sujet, il en
pâmait d'effroi. Il voyait sa fortune et des suites affreuses entre les
mains d'un fou furieux. Il prit le parti de l'éviter avec le plus grand
soin qu'il put, de paraître peu, et de se taire.

M\textsuperscript{me} la duchesse de Bourgogne vivait dans des mesures
et des transes mortelles, et cela dura plus de six semaines de la sorte,
sans que pourtant elle en ait eu autre chose que l'extrême peur. Je n'ai
point su ce qui arriva, ni qui avertit Tessé, mais il le fut et fit un
trait d'habile homme. Il persuada son gendre de le suivre en Espagne, où
il lui fit voir les cieux ouverts pour lui. Il parla à Fagon, qui du
fond de sa chambre et du cabinet du roi voyait tout et savait tout.
C'était un homme d'infiniment d'esprit, et avec cela un bon et honnête
homme. Il entendit à demi-mot, et fut d'avis qu'après tous les remèdes
que Maulevrier avait tentés pour son extinction de voix et sa poitrine,
il n'y avait plus pour lui que l'air des pays chauds\,; que l'hiver où
on allait entrer le tuerait infailliblement en France et lui serait
salutaire dans un pays où cette saison est une des plus belles et des
plus tempérées de l'année\,; ce fut donc sur le pied de remède et comme
l'on va aux eaux, que Maulevrier alla en Espagne. Cela fut donné ainsi à
toute la cour et au roi, à qui Fagon persuada ce qu'il voulut par des
raisonnements de médecine, où il ne craignit point de contradicteur
entre le roi et lui, et à lime de Maintenon tout de même, qui l'un et
l'autre le prirent pour bon et ne se doutèrent de rien. Sitôt que la
parole en fut lâchée, Tessé n'eut rien de plus pressé que de tirer son
gendre de la cour et du royaume, et pour mettre fin à ses folies et aux
frayeurs mortelles qu'elles causaient, et pour couper court à la
surprise et aux réflexions sur un si long voyage d'un homme en l'état
auquel Maulevrier passait pour être.

Tessé prit donc congé les premiers jours d'octobre, et partit avec son
gendre de Fontainebleau pour l'Espagne. Mais il était trop avisé pour y
aller tout droit. Il y voulait une fortune, il la savait pour ce pays-là
entre les mains de la princesse des Ursins, il en savait trop de notre
cour pour ignorer que M\textsuperscript{me} de Maintenon demeurait
sourdement sa protectrice\,; il ne crut donc pas lui déplaire de lui
représenter qu'allant en Espagne pour servir, il ne le pouvait faire
utilement qu'avec les bonnes grâces du roi et de la reine d'Espagne\,;
qu'il se gardait bien de pénétrer dans tout ce qui s'était passé sur la
princesse des Ursins, mais qu'il ne pouvait ignorer avec tout le monde
jusqu'à quel point elle tenait au cœur de Leurs Majestés Catholiques\,;
qu'une visite de sa part à M\textsuperscript{me} des Ursins ne pouvait
influer sur rien, mais que cette attention, qui plairait infiniment au
roi et à la reine d'Espagne, ferait peut-être tout le succès de son
voyage en lui conciliant Leurs Majestés Catholiques, et lui aplanirait
tout pour le service des deux rois. Avec ce raisonnement il supplia
M\textsuperscript{me} de Maintenon de lui obtenir la liberté de passer
par Toulouse, uniquement dans la vue de se mettre en état de pouvoir
bien répondre à ce qu'on attendait de lui au pays où le roi l'envoyait.
M\textsuperscript{me} de Maintenon goûta fort une proposition qui lui
donnait le moyen de charger Tessé de lettres et de choses qui, sans le
mettre dans le secret, lui étaient utiles à mander commodément et à la
princesse des Ursins d'apprendre.

Le roi, qui alors était un peu calmé sur M\textsuperscript{me} des
Ursins\,; entra dans les raisons du maréchal de Tessé, que
M\textsuperscript{me} de Maintenon sut doucement appuyer, et lui permit
de passer à Toulouse. Tessé y demeura trois jours\,; il n'y perdit pas
son temps. Ce premier rayon de retour de considération lui donna une
grande joie et lui rendit Tessé infiniment agréable. Il se livra à elle
pour tout ce qu'elle pourrait souhaiter pour les deux cours. Il partit
de Toulouse chargé de ses lettres et de ses ordres pour Madrid, où en
arrivant, c'est-à-dire le lendemain qu'il eut fait la première révérence
au roi et à la reine, il fut fait grand d'Espagne de la première classe.
Il dépêcha un courrier au roi pour lui demander la permission d'accepter
cette grande grâce, qui la lui accorda aussitôt. Tel fut le lien qui les
unit, M\textsuperscript{me} des Ursins et lui, intimement pour tout le
reste de leur vie. En même temps le roi d'Espagne envoya au comte de
Toulouse une Toison d'or de diamants admirable, et le collier de cet
ordre qu'il reçut, à son retour à Versailles, des mains de M. le duc de
Berry, dans la chambre de ce prince, et son portrait avec des diamants
au maréchal de Cœuvres.

Un frère de M. de Guéméné mourut en ce temps-ci. Il se faisait appeler
le prince de Montauban. C'était un homme obscur et débauché que personne
ne voyait jamais, et qui pour vivre avait épousé la veuve de Rannes, tué
lieutenant général et mestre de camp général des dragons, laquelle était
Bautru, sœur du chevalier de Nogent, et de Nogent, tué au passage du
Rhin, beau-frère de M. de Lauzun. On a vu (t. Ier, p.~157) comment
Monsieur escroqua au roi un tabouret pour elle. C'était une bossue, tout
de travers, fort laide, pleine de blanc, de rouge et de filets bleus
pour marquer les veines, de mouches, de parures et d'affiquets, quoique
déjà vieille, qu'elle a conservés jusqu'à plus de quatre-vingts ans
qu'elle est morte. Rien de si effronté, de si débordé, de si avare, de
si étrangement méchant que cette espèce de monstre, avec beaucoup
d'esprit et du plus mauvais, et toutefois de l'agrément quand elle
voulait plaire. Elle était toujours à Saint-Cloud et au Palais-Royal
quand Monsieur y était, à qui on reprochait de l'y souffrir, quoique sa
cour ne fût pas délicate sur la vertu. Elle n'approchait point de la
cour, et personne de quelque sorte de maintien ne lui voulait parler
quand rarement on la rencontrait. Elle passait sa vie au gros jeu et en
débauches, qui lui coûtaient beaucoup d'argent. À la fin Monsieur fit
tant que, sous prétexte de jeu, il obtint un voyage de Marly. Les Rohan,
c'est-à-dire alors M\textsuperscript{me} de Soubise, l'y voyant
parvenue, la soutint de son crédit\,; elle joua, fit cent bassesses à
tout ce qui la pouvait aider, s'ancra à force d'esprit, d'art et de
hardiesse. Le jeu l'appuya beaucoup. Son jargon à Marly amusa
M\textsuperscript{me} la duchesse de Bourgogne\,; la princesse
d'Harcourt la protégea chez M\textsuperscript{me} de Maintenon, qu'elle
vit quelquefois. Le roi la faisait causer quelquefois aussi à table\,;
en un mot, elle fut de tous les Marlys et, bien que l'horreur de tout le
monde, il n'y en eut plus que pour elle, en continuant la licence de sa
vie, ne la cachant pas, et sans se donner la peine du mérite des
repenties. Elle survécut le roi, tira gros de M. le duc d'Orléans,
quoiqu'il la méprisât parfaitement, et mourut tout comme elle avait
vécu. Elle avait un fils de son premier mari qui servait et qu'elle
traitait fort mal, et une fille du second qu'elle avait faite
religieuse.

Je perdis un ami avec qui j'avais été élevé, qui était un très galant
homme, et qui promettait fort\,: c'était le fils unique du comte de
Grignan et de cette M\textsuperscript{me} de Grignan si adorée dans les
lettres de M\textsuperscript{me} de Sévigné, sa mère, dont cette
éternelle répétition est tout le défaut. Le comte de Grignan, chevalier
de l'ordre en 1688, s'était ruiné à commander en Provence, dont il était
seul lieutenant général. Ils marièrent donc leur fils à la fille d'un
fermier général fort riche. M\textsuperscript{me} de Grignan, en la
présentant au monde, en faisait ses excuses\,; et avec ses minauderies
en radoucissant ses petits yeux, disait qu'il fallait bien de temps en
temps du fumier sur les meilleures terres. Elle se savait un gré infini
de ce bon mot, qu'avec raison chacun trouva impertinent, quand on a fait
un mariage, et le dire entre bas et haut devant sa belle-fille.
Saint-Amant, son père, qui se prêtait à tout pour leurs dettes, l'apprit
enfin, et s'en trouva si offensé qu'il ferma le robinet. Sa pauvre fille
n'en fut pas mieux traitée\,; mais cela ne dura pas longtemps. Son mari,
qui s'était fort distingué à la bataille d'Hochstedt, mourut, au
commencement d'octobre, à Thionville\,; on dit que ce fut de la petite
vérole. Il avait un régiment, était brigadier et sur le point d'avancer.
Sa veuve, qui n'eut point d'enfants, était une sainte, mais la plus
triste et la plus silencieuse que je vis jamais. Elle s'enferma dans sa
maison, où elle passa le reste de sa vie, peut-être une vingtaine
d'années, sans en sortir que pour aller à l'église et sans voir qui que
ce fût.

Coigny, dont j'ai assez parlé pour n'avoir plus rien à en dire, avait
passé le Rhin avec son corps destiné sur la Moselle, lorsque le maréchal
de Villeroy le passa après le malheur d'Hochstedt, et nos armées prêtes
à rentrer en Alsace. Il fut renvoyé avec son corps sur la Moselle. Il
n'avait pu se consoler de n'avoir pas compris l'énigme de Chamillart, et
d'avoir, sans le savoir, refusé le bâton en refusant d'aller en Bavière.
Marsin l'avait eu en sa place. Depuis l'hiver que Chamillart lui avait
achevé de dévoiler un mystère que le bâton de Marsin, déclaré à son
arrivée en Bavière, lui avait suffisamment révélé, il ne fit plus que
tomber. Le chemin où il était, et l'espérance d'y revenir ne le put
soutenir contre l'amertume de sa douleur. Il avait déjà de l'âge. Il
mourut sur la Moselle au commencement d'octobre, à la tête de ce petit
corps qu'il y commandait. Son fils fut plus heureux, et son petit-fils
aussi, à qui on voit maintenant une si brillante fortune.

Précisément en même temps mourut aussi M. le maréchal de Duras, doyen
des maréchaux de France, et frère aîné de huit ans de mon beau-père\,:
c'était un grand homme maigre, d'un visage majestueux et d'une taille
parfaite, le maître de tous en sa jeunesse et longtemps depuis dans tous
les exercices, galant et fort bien avec les dames\,; de l'esprit
beaucoup et un esprit libre et à traits perçants dont il ne se refusa
jamais aucun\,; vif, mais poli, et avec considération, choix et dignité,
magnifique en table et en équipages\,; beaucoup de hauteur sans aucune
bassesse, même sans complaisance\,; toujours en garde contre les favoris
et les ministres, toujours tirant sur eux, et toujours les faisant
compter avec lui. Avec ces qualités, je n'ai jamais compris comment il a
pu faire une si grande fortune. Jusqu'aux princes du sang et aux filles
du roi, il ne contraignait aucun de ses dits\,; et le roi même, en
parlant à lui, en éprouva plus d'une fois et devant tout le monde, puis
riait et regardait la compagnie, qui baissait les yeux. Le roi, parlant
un jour des majors, du détail desquels il s'était entêté alors, M. de
Duras qui n'aimait point celui des gardes du corps, et qui entendit que
le roi ne désapprouvait pas qu'ils se fissent haïr\,: «\,Par \ldots,
dit-il au roi derrière lequel il avait le bâton, et traînant Brissac par
le bras pour le montrer au roi, si le mérite d'un major est d'être haï,
voici bien le meilleur de France, car c'est celui qui l'est le plus.\,»
Le roi se mit à rire et Brissac confondu. Une autre fois le roi parlait
du P. de La Chaise. «\,Il sera damné, dit M. de Duras, à tous les mille
diables, mais je le comprends d'un moine dans la contrainte, la
soumission, la pauvreté, qui se tire de tout cela pour être dans
l'abondance, régner dans son ordre, se mêler de tout et avoir le clergé,
la cour et tout le monde à ses pieds\,; mais ce qui m'étonne, c'est
qu'il puisse, lui, trouver un confesseur, car celui-là se damne bien
sûrement avec lui, et pour cela n'en a pas un morceau de plus, ni un
grain de liberté, ni de considération dans son couvent. Il faut être fou
pour se damner à si bon marché.\,» Il n'aimait point les jésuites, il
lui était resté un levain contre eux du commerce qu'il avait eu avec des
prêtres attachés au Port-Royal lors de sa conversion, et qu'il avait
conservé toute la vie avec eux.

Il avait suivi M. le Prince auquel il s'était attaché plutôt par
complaisance pour ses oncles de Bouillon et de Turenne. Il était le
meilleur officier de cavalerie qu'eût eu le roi, et le plus brillant
pour mener une aile et un gros corps séparé. À la tête d'une armée, il
n'eut ni les mêmes occasions ni la même application\,: il mena pourtant
très bien le siège de Philippsbourg, et le reste de cette courte
campagne où le roi lui avait confié les premières armes de Monseigneur.
Mal d'origine avec Louvois à cause de M. de Turenne, et dégoûté des
incendies du Palatinat, et des ordres divers qu'il reçut sur le secours
de Mayence, se trouvant dans la plus haute fortune, il envoya tout
promener, et n'a pas servi depuis. Il avait fort brillé en chef à la
guerre de Hollande et aux deux conquêtes de la Franche-Comté, dont il
eut le gouvernement à la dernière. Le roi lui avait donné fort jeune un
brevet de duc pour faciliter son mariage avec M\textsuperscript{lle} de
Ventadour, qui fut longtemps heureux\,; un démon domestique les
brouilla. Ils trouvèrent à Besançon M\textsuperscript{lle} de
Beaufremont, tante paternelle de ceux-ci, laide, gueuse, joueuse, mais
qui avait beaucoup d'esprit, et qui sut leur plaire assez pour la
prendre avec eux et la mener à Paris, où ils l'ont gardée bien des
années. L'enfer n'était pas plus méchant ni plus noir que cette
créature\,: elle s'était introduite dans la maison par
M\textsuperscript{me} de Duras, elle s'empara du cœur du maréchal, fit
entre eux des horreurs qui causèrent des éclats, et qui confinèrent la
maréchale à la campagne, dont elle n'est jamais revenue que pour de
courts voyages de fort loin à loin, et où elle aimait mieux sa solitude
que la vie où elle était réduite à l'hôtel de Duras.
M\textsuperscript{lle} de Beaufremont y en fit tant dans la suite que le
maréchal la congédia, mais pour se livrer à une autre gouvernante qui ne
valait pas mieux, et qui, avec de l'esprit, de l'audace, une effronterie
sans pareille, des propos de garnison où pourtant elle n'avait jamais
été, et le jeu de même, le gouverna de façon qu'il ne pouvait s'en
passer, qu'elle le suivait exactement partout à Versailles et à Paris,
domina son domestique, ses enfants, ses affaires, en tira tant et plus,
et jusqu'à son déjeuner le matin, l'envoyait chercher chez lui.

C'était une commère au-dessus des scandales, et qui riait de celui-là
comme n'y pouvant avoir matière. Cela dura jusqu'à la mort du maréchal,
que le curé de Saint-Paul se crut obligé en conscience de la chasser de
l'hôtel de Duras avec éclat par sa résistance, quoi que pût faire la
maréchale arrivée sur cette extrémité, pour sauver cet affront. Depuis
que le maréchal était devenu doyen des maréchaux de France, on
n'appelait plus sa dame que la connétable\,; elle en riait et le
trouvait fort bon. Cette dangereuse et impudente créature était fille de
Besmaux, gouverneur de la Bastille, et femme de Saumery, sous-gouverneur
des enfants de France, dont elle eut beaucoup d'enfants, et qui, avec
toute son arrogance, était petit comme une fourmi devant elle, et lui
laissait faire et dire tout ce que bon lui semblait. Il reviendra en son
particulier sur la scène. Sa femme était une grande créature, sèche, qui
n'eut jamais de beauté ni d'agréments, et qui vit encore à plus de
quatre-vingt-dix ans.

M. de Duras, n'allant plus à la guerre, avait presque toujours le bâton
pour les autres capitaines des gardes qui servaient. Il n'aima jamais
rien que son frère, et assez M\textsuperscript{me} de Saint-Simon, avec
quoi j'avais trouvé grâce devant lui, en sorte que j'en ai toujours reçu
toutes sortes de prévenances et de marques d'amitié. De ses enfants il
n'en faisait aucun compte\,; rien ne l'affecta jamais ni ne prit un
moment sur sa liberté d'esprit et sur sa gaieté naturelle. Il le dit un
jour au roi, et il ajouta qu'il le défiait avec toute sa puissance de
lui donner jamais de chagrin qui durât plus d'un quart d'heure. Sa
propreté était extrême et poussée même fort loin. À quatre-vingts ans il
dressait encore des chevaux que personne n'avait montés. C'était aussi
le plus bel homme de cheval et le meilleur qui fût en France. Lorsque
les enfants de France commencèrent à apprendre sérieusement à y monter,
le roi pria M. de Duras de vouloir bien les voir monter et présider à
leur manège. Il y fut quelque temps, et à la grande écurie et à des
promenades avec eux, puis dit au roi qu'il n'irait plus, que c'était
peine perdue, que ses petits-fils n'auraient jamais ni grâce ni adresse
à cheval, qu'il pouvait s'en détacher, quoi que les écuyers lui pussent
dire dans la suite, et qu'ils ne seraient jamais à cheval que des paires
de pincettes. Il tint parole et eux aussi. On a vu en son lieu ce qu'il
décocha au maréchal de Villeroy lorsqu'il passa de Flandre en Italie. On
ne finirait pas à rapporter ses traits. Aussi les gens importants le
ménageaient et le craignaient plus qu'ils ne l'aimaient. Le roi se
plaisait avec lui, et il s'était fait à en tout entendre, et si M. de
Duras eût voulu, il en aurait tiré beaucoup de grâces. Il fut attaqué de
l'hydropisie dont il mourut, ayant le bâton. Il disputa quelque temps,
enfin il fallut céder, et lui-même comprit très bien qu'il n'en
reviendrait pas. Il prit congé du roi dans son cabinet, qui le combla
d'amitiés, et qui s'attendrit jusqu'aux larmes. Il lui demanda ce qu'il
pouvait faire pour lui. Il ne demanda rien et n'eut rien aussi, et il
est certain qu'il ne tint qu'à lui d'avoir sa charge ou son gouvernement
pour son fils. Il ne s'en soucia pas.

Quelque temps après, le roi alla à Fontainebleau\,; il s'y fâcha de ce
que les dames négligeaient de s'habiller pour la comédie et se passaient
d'y aller ou s'y mettaient à l'écart pour n'être pas obligées à
s'habiller. Quatre mots qu'il en dit, et le compte qu'il se fit rendre
de l'exécution de ses ordres, y rendit toutes les femmes de la cour très
assidues en grand habit. Là-dessus il nous vint des nouvelles de
l'extrémité de M. de Duras. On ne vivait pas alors comme on fait
aujourd'hui. L'assiduité dont le roi ne dispensait personne de ce qui
était ordinairement à la cour n'avait pas permis à
M\textsuperscript{me}s de Saint-Simon et de Lauzun de s'absenter de
Fontainebleau\,; mais sur ces nouvelles, elles furent dire à
M\textsuperscript{me} la duchesse de Bourgogne qu'elles s'en iraient le
lendemain, et que pour la comédie elles la suppliaient de les en
dispenser ce soir-là. La princesse trouva qu'elles avaient raison, mais
que le roi ne l'entendrait pas. Tellement qu'elles capitulèrent de
s'habiller, de venir à la comédie en même temps qu'elle ou un moment
après, qu'elles en sortiraient aussitôt sous prétexte de n'y avoir plus
trouvé place, et que la princesse le dirait au roi. Je marque cette très
légère bagatelle, pour montrer combien le roi ne comptait que lui et
voulait être obéi, et que ce qui n'aurait pas été pardonné aux nièces de
M. de Duras en l'état où il était, partout ailleurs qu'ai la cour, y
était un devoir qui eut besoin d'adresse et de protection, pour ne se
pas faire une affaire sérieuse en préférant la bienséance.

M. de Duras mourut en bon chrétien et avec une grande fermeté. La
parenté, les amis, beaucoup d'autres et la connétablie accompagnèrent
son corps à Saint-Paul. M. de Soubise alerte surtout, et dont la
belle-fille était fille unique du duc de Ventadour, frère de la
maréchale de Duras, lequel n'y était pas, envoya proposer à la famille
de mener le deuil. Celui qui le mène est en manteau et précède toute la
parenté. Je leur fis remarquer que ce n'était que pour cela que M. de
Soubise s'y offrait, et dire après qu'il avait précédé la famille, et ne
point parler qu'il eût mené le deuil. On se moqua de moi, mais je tins
ferme, et leur déclarai que si l'offre était acceptée, je me retirerais
et ne paraîtrais à rien. Cela les arrêta. M. de Soubise fut remercié, et
ce qui montra la corde, il ne vint point à l'enterrement ni son fils, et
il fut fort piqué.

La longueur de la maladie de M. de Duras avait donné le temps aux
machines. Le duc de Guiche, revenu fort mal de l'armée du maréchal de
Villeroy, se portait mieux et il était à Fontainebleau, depuis longtemps
mal avec le roi par sa conduite, et ayant reçu plusieurs dégoûts. Malgré
cela les Noailles se mirent dans la tête de lui faire tomber le régiment
des gardes qu'avait son beau-frère le maréchal de Boufflers qui était
aussi à Fontainebleau, et de le faire capitaine des gardes du corps.
Quelque belle que fût cette dernière charge, celle de colonel était sans
comparaison. Il n'y avait donc pas moyen de faire entrer Boufflers dans
cette affaire. Il vivait intimement avec le duc et la duchesse de Guiche
sa belle-sœur, et avec tous les Noailles\,; ils étaient lors au comble
de la faveur, et le maréchal n'avait garde de se défier d'eux. Le
mariage du duc de Noailles qui avait environné M\textsuperscript{me} de
Maintenon des siens, en avait plus approché sa sœur aînée la duchesse de
Guiche que pas une.

Son âge fort supérieur à celui de ses sœurs y contribuait. Quoiqu'elle
eût quitté le rouge, sa figure était encore charmante. Elle avait
infiniment d'esprit, du souple, du complaisant, de l'amusant, du
plaisant, du bouffon même\,; mais tout cela sans se prodiguer, du
sérieux, du solide\,; raffolée de M. de Cambrai, de
M\textsuperscript{me} Guyon, de leur doctrine et de tout le petit
troupeau, et dévote comme un ange. Séparée d'eux par autorité, et fidèle
à l'obéissance, tout cela était devenu des degrés de mérite auprès de
M\textsuperscript{me} de Maintenon, supérieurs à celui qu'elle tirait de
l'alliance de son frère. Sa retraite la faisait rechercher\,; elle
n'accordait pas toujours d'aller aux voyages de Marly, et
M\textsuperscript{me} de Maintenon croyait recevoir une faveur toutes
les fois qu'elle venait chez elle. Il pouvait y avoir du vrai, mais ce
vrai n'était pas sans art. Sa dévotion, montée sur le ton de ce petit
troupeau à part, qui avait ses lois et ses règles particulières, était,
comme la leur, compatible avec la plus haute et la plus vive ambition et
avec tous les moyens de la satisfaire. Quoique son mari n'eût rien
d'aimable, même pour elle, elle en fut folle d'amour toute sa vie. Pour
lui plaire, et pour se plaire à elle-même, elle ne songeait qu'à sa
fortune. Sa famille, si maîtresse en cet art, n'en avait pas moins de
passion\,; ils s'entraidèrent. Rien n'est pareil au trébuchet qu'ils
imaginèrent pour tendre au maréchal de Boufflers et dans lequel ils le
prirent\,; aussi tout était-il bien préparé à temps\,; et il n'y fut pas
perdu une minute.

M. de Duras mourut à Paris le dimanche matin, 12 octobre, et
l'après-dînée le roi le sut au sortir du salut. Le lendemain matin,
comme le roi, au sortir de son lever, eut donné l'ordre, il appela le
maréchal de Boufflers, le surprit par un compliment d'estime, de
confiance, et jusqu'à la tendresse\,; lui dit qu'il ne pouvait pas lui
en donner une plus sensible marque qu'en l'approchant au plus près de sa
personne, et la lui remettant entre les mains\,; que c'était ce qui
l'engageait à le préférer à qui que ce fait pour lui donner la charge de
M. de Duras, persuadé qu'il l'acceptait avec autant de joie et de
sentiment qu'il la lui donnait avec complaisance. Il n'en fallait pas
tant pour étourdir un homme qui ne s'attendait à rien moins, qui n'avait
aucun lieu de s'y attendre, qui avait peu d'esprit, d'imagination, de
repartie, pour qui le roi était un dieu, et qui, depuis qu'il
l'approchait et qu'il était parvenu au grand, n'avait pu s'accoutumer à
ne pas trembler en sa présence. Le roi bien préparé se contente de sa
révérence, et sans lui laisser le moment de dire une parole, dispose
tout de suite de la charge de colonel du régiment des gardes, et lui dit
qu'il compte lui faire une double grâce de la donner au duc de Guiche\,;
autre surprise, autre révérence pendant laquelle le roi tourne le dos,
se retire, et laisse le maréchal stupéfait, qui se crut frappé de la
foudre.

Il sortit donc du cabinet sans avoir pu proférer un seul mot, et chacun
lui vit les larmes aux yeux. Il s'en alla chez lui, où sa femme ne
pouvait comprendre ce qui venait d'arriver, et qui s'en prit abondamment
à ses yeux. Les bons Noailles et la douce, humble et sainte duchesse de
Guiche, leur bonne et chère sœur, avec qui ils vivaient comme telle, non
contents de lui avoir arraché sa charge, eurent le front de le prier de
demander au roi pour le duc de Guiche le même brevet de cinq cent mille
livres qu'il avait sur le régiment des gardes qui allait payer le pareil
de M. de Duras. Boufflers, hors de lui de douleur et de dépit, mais trop
sage pour donner des scènes, avala ce dernier calice, et obtint ce
brevet de retenue au premier mot qu'il en dit au roi, toujours sur le
ton de lui faire des grâces pour son beau-frère. Jamais Boufflers, ni sa
femme ne se sont consolés du régiment des gardes, jamais ils n'en ont
pardonné le rapt au due, et moins encore à la duchesse de Guiche\,; mais
en gens qui ne veulent point d'éclats et d'éclats inutiles, ils
gardèrent les mêmes dehors avec eux et avec tous les Noailles. Ils
essayèrent de consoler le maréchal comme un enfant avec un hochet. Le
roi lui dit de conserver partout le logement de colonel des gardes, et
de continuer d'en mettre les drapeaux à ses armes.

Le gouvernement de la Franche-Comté fut donné à Tallard, à l'étonnement
et au scandale de tout le monde. M. le duc d'Orléans dit là-dessus
plaisamment qu'il fallait bien donner quelque chose à un homme qui avait
tout perdu. Comme il le dit sur-le-champ et tout haut, ce bon mot vola
de bouche en bouche, et il déplut fort au roi.

Peu de jours après, le roi donna quarante mille livres de pension au
petit comte de La Marche, tout enfant, fils du prince de Conti. Cela
parut prodigieux et l'était en effet pour lors. Pour aujourd'hui, à ce
qu'en ont tiré ces princes depuis la mort du roi, ce serait une goutte
d'eau.

\hypertarget{chapitre-xix.}{%
\chapter{CHAPITRE XIX.}\label{chapitre-xix.}}

1704

~

{\textsc{Siège de Verue par le duc de Vendôme.}} {\textsc{- Retour de
Fontainebleau par Sceaux.}} {\textsc{- Rouillé sans caractère près
l'électeur de Bavière\,; son caractère et ses emplois.}} {\textsc{-
Progrès des mécontents.}} {\textsc{- Ragotzi élu prince de
Transylvanie.}} {\textsc{- Des Alleurs.}} {\textsc{- Subsides.}}
{\textsc{- La Bavière en proie à l'empereur.}} {\textsc{- Trèves et
Traarbach perdus.}} {\textsc{- Marlborough en diverses cours
d'Allemagne.}} {\textsc{- Landau rendu au roi des Romains\,; Laubanie,
aveuglé dedans, récompensé.}} {\textsc{- Séparation des armées.}}
{\textsc{- Coigny, colonel général des dragons.}} {\textsc{- Abbé de
Pomponne ambassadeur à Venise.}} {\textsc{- Puysieux\,; sa famille, son
caractère.}} {\textsc{- Son adresse le fait chevalier de l'ordre.}}
{\textsc{- Comte de Toulouse, de retour, résolu de perdre Pontchartrain,
est arrêté par sa femme.}} {\textsc{- Caractère de Pontchartrain.}}
{\textsc{- Suites funestes à l'État.}} {\textsc{- Mort de Caylus\,;
caractère de sa femme.}} {\textsc{- Cercles.}} {\textsc{- Berwick de
retour d'Espagne.}} {\textsc{- Mariage du marquis de Charost et de
M\textsuperscript{lle} Brûlart, depuis duchesse de Luynes et dame
d'honneur de la reine.}} {\textsc{- Mort de M\textsuperscript{me} de
Gamaches.}} {\textsc{- Mort duc de Gesvres.}} {\textsc{- Mort du
président Payen.}} {\textsc{- Bouligneux et Wartigny tués devant
Verue.}} {\textsc{- Singularité arrivée à des masques de cire.}}
{\textsc{- Mort de la duchesse d'Aiguillon\,; son caractère.}}
{\textsc{- Marquis de Richelieu\,; explication de sa prétention de
succéder à la dignité d'Aiguillon, rejetée par le roi.}} {\textsc{-
Denonville obtient permission de venir se justifier.}} {\textsc{-
Marlborough passe en Angleterre avec Tallard et les principaux
prisonniers.}} {\textsc{- Villars rappelé de Languedoc, où Berwick va
commander.}}

~

M. de Vendôme s'opiniâtra à vouloir assiéger Verue\,; il dépêcha, à son
ordinaire, un courrier pour mander qu'en y arrivant, le 14 octobre, il
avait emporté trois hauteurs que les ennemis avaient négligé de
retrancher, d'où il les avait chassés à la vue de M. de Savoie et de
toute sa cour, qui avaient été obligés de se retirer à toutes jambes.
Avec ces fanfaronnades il repaissait le roi à l'appui de
M\textsuperscript{me} de Maintenon par M. du Maine. Jamais siège si
follement entrepris, peu qui aient tant coûté de temps, d'hommes et
d'argent. Il influa encore sur la campagne suivante, qu'on ne put ouvrir
à temps par le délabrement de l'armée. Le terrain était extrêmement
mauvais, même dans la belle saison, et on allait se trouver dans la
mauvaise\,; et tandis que la place était attaquée d'un côté, elle était
soutenue de l'autre par un camp retranché de l'autre côté de l'eau, qui
rafraîchissait la place tout à son aise de troupes et de tout, et
inquiétait continuellement notre armée. L'opiniâtreté et l'autorité que
M. de Vendôme s'était acquise par son crédit l'emportèrent sur toute
raison de guerre et sur le sentiment de toute son armée, qui à peine
osa-t-elle témoigner ce qu'elle en pensait, tant le peu d'officiers
généraux, de ceux qui étaient le mieux avec le duc de Vendôme, furent
mal reçus dans leurs courtes et modestes représentations. Outre ces
difficultés, la subsistance de la cavalerie y était d'une difficulté
extrême, tellement qu'il fallut, dès les premiers commencements,
renvoyer presque tous les équipages de l'armée du côté d'Alexandrie, où
M. de Vaudemont leur fit donner des quartiers et du fourrage, mais pour
de l'argent, à un prix modique. On comprend ce que ce peut être pour
tous les officiers généraux et particuliers qui font un grand siège sans
investiture, vis-à-vis un camp ennemi séparé d'eux par la rivière, dans
un très mauvais terrain, sans équipages, et qui sont avec cela obligés
de les nourrir hors de leur portée à leurs dépens. Ce fut avec cette
bonne nouvelle que le roi partit de Fontainebleau, le 23 octobre, pour
retourner à Versailles par Sceaux, où il séjourna un jour. Incontinent
après, il envoya Rouillé sans caractère résider à. Bruxelles auprès de
l'électeur de Bavière, avec vingt-quatre mille livres d'appointements.
Il était président en la cour des aides, frère de Rouillé, qui avait été
directeur des finances et qui était conseiller d'État, et il était
revenu, il y avait deux ans, de Lisbonne, où il avait été ambassadeur
avec satisfaction. C'était un homme d'esprit, appliqué, capable, un peu
timide, et que les ducs de Chevreuse et de Beauvilliers protégeaient
fort. Il figurera dans la suite, et on le verra employé aux affaires les
plus importantes et les plus secrètes, où il se conduisit toujours très
bien\,: il est donc bon dès ici de le connaître.

Les mécontents de Hongrie ne se laissèrent point abattre par le grand et
inespéré succès de la bataille d'Hochstedt. Loin d'écouter les
propositions que l'empereur leur fit faire, ils prirent Neutra, et
Ragotzi fut élu prince de Transylvanie. Il en envoya donner part au
Grand Seigneur, et lui offrir pour sa protection le même tribut que
payaient à la Porte son bisaïeul et son grand-père en la même qualité.
Ils se rendirent depuis maîtres d'Eperiez et de Cassovie, et de cent
quarante pièces de canon qu'ils y trouvèrent\,; il y avait déjà du temps
que des Alleurs était secrètement, de la part du roi, auprès de Ragotzi,
à qui il donnait trois mille pistoles par mois. Il envoya en ce temps-ci
un officier de confiance à l'électeur de Bavière à Bruxelles, qui le
renvoya au roi. Ragotzi voulait quelque augmentation et moins de secret
dans la protection du roi pour se donner plus de crédit et à son armée
plus de confiance. La vérité était que personne ne doutait en Europe
qu'il ne fait soutenu par la France, quelque obscurément qu'elle le fît.
Ils prirent bientôt après Neuhausel, et obligèrent ensuite le général
Heister de se hâter de se retirer devant eux.

L'empereur cependant fit à l'électrice de Bavière des propositions si
étranges qu'elle ne les voulut pas écouter. Les Impériaux trouvant plus
de difficultés qu'ils ne pensaient à leur conquête, la cour de Vienne
changea de ton sans changer de volonté, et conclut un accommodement par
lequel il fut convenu que l'électrice retirerait toutes ses troupes des
places du Danube, et qu'elle demeurerait dans la paisible possession de
la Bavière, qui ne payerait aucune contribution\,; mais elle ne fut
obéie qu'à Passau\,; les gouverneurs d'Ingolstadt, Brunau et Kufstein
s'excusèrent sur leur serment à l'électeur, sans un ordre duquel, signé
de sa main, ils ne sortiraient pas de leurs places\,; et la cavalerie
bavaroise, qu'on voulait séparer, en répondit autant. Le prince Eugène
remarcha en Bavière, prit les places, et mit le pays et la famille
électorale en étrange état.

Marlborough, d'autre part, suivit de près son frère, qu'il avait envoyé
sur la Moselle avec un gros détachement\,; ils s'emparèrent de Trèves,
et tôt après firent le siège de Traarbach, et le prirent, pendant que le
duc de Marlborough s'alla promener en Allemagne, et voir les électeurs
de Brandebourg et d'Hanovre, le landgrave de Hesse et quelques autres
princes. Chacun après quitta les armées en Flandre, qui se séparèrent
incontinent pour les quartiers d'hiver. Il n'y eut que celle d'Alsace
qui, sous Marsin, attendait impatiemment la prise de Landau, pour
s'aller reposer de même. Cette place capitula enfin le 23 décembre.
Laubanie y avait fait merveille, même après y avoir perdu les deux yeux.
Le roi des Romains le traita avec toute la distinction que sa valeur
méritait, lui surtout et sa garnison, dont il ne sortit que la moitié de
ce qu'elle était au commencement du siège. Le roi donna à Laubanie
trente-six mille livres de pension, outre de petites qu'il avait déjà,
et sa grand'croix de Saint-Louis de six mille livres. C'était un
excellent officier et un très galant homme d'ailleurs, aveuglé dans
Landau, et qui avait très bien servi toute sa vie.

Coigny, fils de celui que nous venons de voir mourir sur la Moselle,
eut, par la protection de Chamillart, l'agrément d'acheter du duc de
Guiche la charge de colonel général des dragons, qui fut le commencement
et le fondement de la grande fortune où on le voit aujourd'hui.

Depuis le retour de Charmont de Venise, le roi, mécontent de cette
république sur plusieurs griefs, n'y avait envoyé personne, et refusé
même d'admettre son ambassadeur à son audience. Par force souplesses et
propos de respect peu solides, {[}les Vénitiens{]} se raccommodèrent
avec le roi. L'abbé de Pomponne vieillissait dans la charge d'aumônier
de quartier. Le roi s'était expliqué avantageusement sur lui, mais que
son nom d'Arnauld lui répugnait trop dans l'épiscopat pour l'y faire
jamais monter. Il fallut donc se tourner ailleurs. Il était beau-frère
de Torcy. Pomponne, son père, lui avait fait mettre le nez dans ses
papiers avec l'agrément du roi, et il continuait de même avec Torcy\,;
il avait déjà été à Rome et en diverses cours d'Italie. Tout cela
ensemble le fit choisir pour l'ambassade de Venise, et il remit sa place
d'aumônier.

Puysieux, revenu depuis peu par congé de son ambassade de Suisse, où il
faisait fort bien, avait obtenu, ainsi que l'année précédente, la
singulière faveur de rendre compte directement au roi des affaires de ce
pays-là, et dans son cabinet tête à tête. Il était petit-fils de
Puysieux, secrétaire d'État, fils du chancelier de Sillery, enveloppé
dans sa disgrâce qui lui fit perdre sa charge, et de sa seconde femme
qui était Étampes, sieur de M. de Valencey, chevalier de l'ordre en
1619, gouverneur de Montpellier, puis de Calais, et grand maréchal des
logis de la maison du roi, de l'archevêque-duc de Reims, du cardinal de
Valencey, de la seconde maréchale de La Châtre, tante paternelle de la
maréchale d'Hocquincourt, et du grand prieur de France et ambassadeur à
Rome. Elle avait un autre frère qui s'était avisé de se faire de robe,
et qui, après avoir été ambassadeur aux Grisons et en Hollande, était
devenu conseiller d'État, et beau-père du comte de Béthune, chevalier
d'honneur de la reine et chevalier du Saint-Esprit, en son temps un
personnage. M\textsuperscript{me} de Puysieux, veuve dès 1640, ne mourut
qu'en 1677, à quatre-vingts ans, avec toute sa tête et sa santé. C'était
une femme souverainement glorieuse, que la disgrâce n'avait pu abattre,
et qui n'appelait jamais son frère le conseiller d'État que «\,mon frère
le bâtard.\,» On ne pouvait avoir plus d'esprit qu'elle en avait, et
quoique impérieux, plus tourné à l'intrigue. Elle haïssait mortellement
le cardinal de Richelieu pour la disgrâce de son beau-père et de son
mari, et elle était dans l'intime confiance de la reine. Revenue de
Sillery dès 1640, cette amitié se resserra de plus en plus par les
besoins et par les intrigues, en sorte que, lorsque la reine fut
régente, chacun compta avec M\textsuperscript{me} de Puysieux, et y a
compté tant qu'elle a vécu. Le roi et Monsieur, dans leur enfance, ne
bougeaient de chez elle\,; dans leur jeunesse ils continuèrent à y
aller, et tant qu'elle a été au monde, le roi l'a toujours
singulièrement distinguée et considérée. Elle était magnifique et ruina
elle et ses enfants.

On portait en ces temps-là force points de Gênes qui étaient extrêmement
chers\,: c'était la grande parure et la parure de tout âge\,; elle en
mangea pour cent mille écus en une année à ronger entre ses dents celle
qu'elle avait autour de sa tête et de ses bras. Elle eut des fils
comblés d'abbayes, une fille abbesse, une autre mariée au fils du
maréchal d'Étampes, et son fils aîné, M. de Sillery, qui épousa une
fille de M. de La Rochefoucauld, si connu par son esprit, et par la
figure qu'il fit dans la minorité de Louis XIV. Sillery ruiné servit
peu\,; il était fort aimable, et fort du grand monde. M. de La
Rochefoucauld, son beau-frère, les retira chez lui à Liancourt où ils
sont morts. Ils laissèrent plusieurs enfants, dont Puysieux, duquel je
parle ici, fut l'aîné.

C'était un petit homme, fort gros et entassé, plein d'esprit, de traits
et d'agréments, tout à fait joyeux, doux, poli et respectueux, et le
meilleur homme du monde. Il savait beaucoup, avec goût et avec une
grande modestie\,; il était d'excellente compagnie, et un répertoire de
mille faits curieux\,; tout le monde l'aimait. Il servit tant qu'il
put\,; mais M. de Louvois le prit en aversion, et l'arrêta tout court.
Il était maréchal de camp, et déjà gouverneur d'Huningue, fort bien avec
le roi, qui se souvenait toujours de sa grand'mère avec amitié, et
d'avoir passé sa première jeunesse à jouer chez elle avec ses enfants.
Après la mort de Louvois, il fut employé en haute Alsace, et fait enfin
lieutenant général. Il trouva l'ambassade de Suisse tout auprès de lui
et à sa bienséance. M. de La Rochefoucauld la lui obtint, et il y servit
à merveille. Ses anciennes privances et M. de La Rochefoucauld lui
obtinrent ces audiences du roi tête à tête à ses retours, pour lui
rendre un compte direct de son ambassade, ce qui ne fut jamais accordé à
nul autre. Torcy était le seul ministre que M. de La Rochefoucauld vît
sur un pied d'amitié et de familiarité. Il fallait tout ce préambule
pour comprendre ce qui va suivre.

Puysieux, arrivant de Suisse par congé, après le retour de Fontainebleau
cette année, fut fort bien traité du roi dans l'audience qu'il en eut.
Comme il avait beaucoup d'esprit et de connaissance du roi, il s'avisa
tout à coup de tirer hardiment sur le temps, et comme le roi lui
témoignait de l'amitié et de la satisfaction de sa gestion en Suisse, il
lui demanda s'il était bien vrai qu'il fût content de lui, si ce n'était
point discours, et s'il y pouvait compter. Sur ce que le roi l'en
assura, il prit un air gaillard et assuré et lui répondit que pour lui
il n'était pas de même, et qu'il n'était pas content de Sa Majesté.
«\,Et pourquoi donc, Puysieux\,? lui dit le roi. --- Pourquoi, sire\,?
parce qu'étant le plus honnête homme de votre royaume, vous ne laissez
pas pourtant de me manquer de parole depuis plus de cinquante ans. ---
Comment, Puysieux, reprit le roi, et comment cela\,? --- Comment cela,
sire\,? dit Puysieux, vous avez bonne mémoire et vous ne l'aurez pas
oublié. Votre Majesté ne se souvient-elle pas qu'ayant l'honneur de
jouer avec vous à colin-maillard, chez ma grand'mère, vous me mîtes
votre cordon bleu sur le dos pour vous mieux cacher au colin-maillard,
et que lorsque après le jeu je vous le rendis, vous me promîtes de m'en
donner un quand vous seriez le maître\,; il y a pourtant longtemps que
vous l'êtes, et bien assurément, et toutefois ce cordon bleu est encore
à venir.\,» Le roi s'en souvint parfaitement, se mit à rire, et lui dit
qu'il avait raison\,; qu'il lui voulait tenir parole et qu'il tiendrait
un chapitre exprès avant le premier jour de l'an pour le recevoir ce
jour-là. En effet, le jour même il en indiqua un pour le chapitre et dit
que c'était pour Puysieux. Ce fait n'est pas important, mais il est
plaisant. Il est tout à fait singulier avec un prince aussi sérieux et
aussi imposant que Louis XIV\,; et ce sont de ces petites anecdotes de
cour qui ont leur curiosité.

En voici une plus importante et de laquelle l'État se sent encore.
Pontchartrain, secrétaire d'État de la marine, en était le fléau, comme
de tous ceux qui étaient sous sa cruelle dépendance. C'était un homme
qui avait de l'esprit, du travail, de l'adresse\,; mais gauche à tout,
désagréable et pédant à l'excès, volontiers le précepteur grossier de
tout le monde\,; suprêmement noir, et aimant le mal précisément pour le
mal\,; jaloux jusque de son père, qui s'en plaignait amèrement à ses
plus intimes amis\,; tyran cruel jusque de sa femme qui, avec beaucoup
d'esprit, était l'agrément, la douceur, la complaisance, la vertu même
et l'idole de la cour\,; barbare jusqu'avec sa mère\,; un monstre, en un
mot, qui ne tenait au roi que par l'horreur de ses délations de son
détail de Paris, et une malignité telle qu'elle avait presque rendu
d'Argenson bon. Un amiral était sa bête, et un amiral bâtard du roi son
bourreau. Il n'y avait rien qu'il n'eût fait contre sa charge et pour
l'empêcher de la faire\,; point d'obstacles qu'il n'eût semés sur son
chemin\,; rien qu'il n'eût employé pour l'empêcher de commander la
flotte, et, après, pour rendre cette flotte inutile, comme il y avait
réussi l'année précédente de celle-ci. Il lui disputa tous ses honneurs,
toutes ses distinctions, ses pouvoirs encore davantage, et lui en fit
retrancher des uns et des autres qui, par leur nature et par leur
exemple, ne pouvaient être et n'avaient pas été contestés.

Cela fut hardi contre un fils de la personne bien plus que si c'eût été
contre un fils de France\,; mais il sut prendre le roi par son faible,
balancer le père naturel par le maître, s'identifier avec le roi, et lui
persuader qu'il ne s'agissait de l'autorité qu'entre le roi et l'amiral.
Ainsi le fils de l'amour disparut aux yeux d'un maître, toujours maître
de préférence infinie à tout autre sentiment. Sous ce voile, le
secrétaire d'État le fut entièrement, et nourrit le comte de Toulouse de
contretemps pour le faire échouer, et de dégoûts à le mettre au
désespoir, sans qu'il pût que très légèrement se défendre. Ce fut un
spectacle public à la mer, et dans les ports où la flotte toucha, qui
indigna toute la marine, où Pontchartrain était abhorré, et le comte
adoré par son accès facile, sa douceur, sa libéralité, son application,
sa singulière équité. Le maréchal de Cœuvres, M. d'O et tous les autres
chefs de degré ou de confiance ne furent pas mieux traités, tellement
qu'ils excitèrent tous le comte à ce qu'ils s'étaient déjà proposé, qui
était de perdre Pontchartrain en arrivant, pour montrer au net les
contretemps et leurs suites, et le secrétaire d'État comme l'auteur de
malices méditées, et de là, par effort de crédit auprès du roi. Il
fallait l'audace de Pontchartrain pour s'être mis en ce danger, prévu et
déploré souvent et inutilement par son sage père, par sa mère et par sa
femme. L'ivresse dura jusqu'au retour du comte de Toulouse que la
famille fut avertie de toutes parts de l'orage, et Pontchartrain
lui-même par l'accueil qu'il reçut de l'amiral et des principaux de la
flotte. Aussi abject dans le danger qu'audacieux dans la bonace, il
tenta tout à la fois pour prévenir sa chute, et n'en remporta que des
dédains.

Enfin, le jour venu où le comte devait travailler seul à fond avec le
roi pour lui rendre un compte détaillé de son voyage, et de tout faire
pour perdre Pontchartrain, sa femme prit sur sa modestie et sur sa
timidité naturelle de l'aller trouver chez M\textsuperscript{me} la
duchesse d'Orléans, et le forcer à entrer seul avec elle dans un,
cabinet. Là, fondue en larmes, reconnaissant tous les torts de son mari,
exposant quelle serait sa condition à elle s'il était perdu selon ses
mérites, elle désarma l'amiral et en tira parole de tout oublier, pourvu
qu'à l'avenir le secrétaire d'État ne lui donnât pas lieu de rappeler
l'ancien avec le nouveau. Il avoua qu'il n'avait jamais pu résister à la
douceur et à la douleur de M\textsuperscript{me} de Pontchartrain, et
que, quelque résolution qu'il eût faite, les armes lui étaient tombées
des mains, en considérant quel serait le malheur de cette pauvre femme
entre les mains d'un cyclope furieux de sa chute, qui n'aurait plus rien
à faire dans son délaissement que de la tourmenter. Ce fut ainsi que
Pontchartrain fut sauvé, mais il en coûta cher à l'État. La peur qu'il
eut de succomber sous la gloire ou sous la vengeance d'un amiral fils du
roi le détermina à perdre lui-même la marine, pour la mettre hors d'état
de revoir l'amiral à la mer. Il se le promit et se tint exactement
parole\,; cela ne fut que trop bien vérifié depuis par les faits, et que
les débris de la marine ne l'appauvrirent pas. Le comte de Toulouse ne
revit plus ni ports ni vaisseaux, et il ne sortit depuis que de très
faibles escadres, et le plus rarement qu'il se put. Pontchartrain eut
l'impudence de s'en applaudir devant moi.

Au commencement de novembre, mourut, sur la frontière de Flandre, un
homme qui fit plaisir à tous les siens\,: ce fut Caylus, frère de celui
d'Espagne et de l'évêque d'Auxerre, cousin germain d'Harcourt, qui avait
épousé la fille de Villette, lieutenant général des armées navales,
cousin germain de M\textsuperscript{me} de Maintenon qui avait toujours
pris soin d'elle comme de sa propre nièce. Jamais un visage si
spirituel, si touchant, si parlant, jamais une fraîcheur pareille,
jamais tant de grâces ni plus d'esprit, jamais tant de gaieté et
d'amusement, jamais de créature plus séduisante. M\textsuperscript{me}
de Maintenon l'aimait à ne se pouvoir passer d'elle, au point de fermer
les yeux sur une conduite que M\textsuperscript{me} de Montchevreuil
avait autrefois trop éclairée, et qui, n'étant pas devenue meilleure
dans le fond, avait encore des saillies trop publiques. Son, mari,
blasé, hébété depuis plusieurs années de vin et d'eau-de-vie, était tenu
à servir, hiver et été, sur la frontière pour qu'il n'approchât ni de sa
femme ni de la cour. Lui aussi ne demandait pas mieux, pourvu qu'il fût
toujours ivre. Sa mort fut donc une délivrance dont sa femme et ses plus
proches ne se contraignirent pas de la trouver telle.
M\textsuperscript{me} de Maintenon se tint toujours dans la chambre de
cette belle à son mariage à recevoir les visites\,; et la princesse
d'Harcourt, servante à tout faire, chargée des honneurs à tout ce qui y
venait. M\textsuperscript{me} de Caylus s'échappait tant qu'elle pouvait
chez M\textsuperscript{me} la Duchesse, où elle trouvait à se divertir.
Elle aimait le jeu sans avoir de quoi le soutenir, encore mieux la
table, où elle était charmante\,; elle excellait dans l'art de
contrefaire, et surpassait les plus fameuses actrices à jouer des
comédies\,; elle s'y surpassa à celles d'\emph{Esther} et
d'\emph{Athalie} devant le roi. Il ne la goûta pourtant jamais et fut
toujours réservé, même sévère avec elle\,; cela surprenait et affligeait
M\textsuperscript{me} de Maintenon. Je me suis étendu sur
M\textsuperscript{me} de Caylus, qui, après de longs revers, fit enfin
une sorte de personnage. Ce revers était arrivé\,; plusieurs imprudences
en furent cause. Il y avait trois ou quatre ans qu'elle était chassée de
la cour et réduite à demeurer à Paris.

Le feu roi, qui n'aimait la dignité que pour lui et qui aimait la
majesté de sa cour, regrettait toujours celle des cercles de la reine sa
mère, parmi lesquels il avait été nourri et dont la splendeur finit avec
elle. Il essaya de les soutenir chez la reine sa femme, dont la bêtise
et l'étrange langage les éteignirent bientôt. Le roi, qui ne s'en
pouvait départir, les releva du temps de M\textsuperscript{me} la
Dauphine, après la mort de la reine. Elle avait l'esprit, la grâce, la
dignité et la conversation très propres à cette sorte de cour. Mais les
incommodités de ses fréquentes grossesses, celles des longues suites de
ses couches, la longue maladie qui dura depuis la dernière jusqu'à sa
mort, les interrompirent bientôt. L'excessive jeunesse, pour ne pas dire
l'enfance, de M\textsuperscript{me} la duchesse de Bourgogne, ne permit
pas d'y penser depuis son arrivée jusqu'en ce temps-ci que le roi,
toujours touché des cercles, la crut assez formée pour les tenir. Il
voulut donc que tous les mardis, qui est le jour que tous les ministres
étrangers sont à Versailles, M\textsuperscript{me} la duchesse de
Bourgogne dînât seule, servie par ses gentilshommes servants\,; qu'il y
eût, à son dîner, force dames assises et debout\,; et qu'ensuite elle
tint un cercle où M\textsuperscript{me} la duchesse d'Orléans, les
princesses du sang et toutes les dames assises et debout se trouvassent
avec tous les seigneurs de la cour. Cet ordre commença à s'exécuter de
la sorte à la mi-novembre de cette année, et se continua quelque
temps\,; mais la représentation sérieuse, et l'art d'entretenir et de
faire entretenir un si grand monde, n'était pas le fait d'une princesse
vive, timide en public, et encore bien jeune. Peu à peu elle en brûla et
à la fin ils cessèrent sans qu'ils aient été rétablis depuis.

Le duc de Berwick avait appris son rappel étant à la tête de son armée
en présence des ennemis\,; il avait continué à donner ses ordres sans la
moindre émotion. Ils trouvèrent moyen de se retirer en lieu où ils ne
purent être attaqués\,; alors Berwick rendit publique la nouvelle qui le
regardait, comme s'il n'eût pas été question de lui. Outre qu'il était
froid et naturellement silencieux, fort maître de soi et grand
courtisan, peut-être que, content d'avoir dépassé les lieutenants
généraux par le commandement en chef d'une armée, il regretta peu un
pays où il avait trouvé tant de mécomptes et une cour si passionnée, où
il n'y avait de salut ni de résolution que par la reine, et par l'esprit
absent de la princesse des Ursins. Tessé et lui se rencontrèrent
arrivant à Madrid chacun de son côté. Ils conférèrent, et Berwick prit
aussitôt congé et salua le roi à Versailles, le 3 décembre.

Le marquis de Charost et les ducs, ses père et grand-père, vinrent dîner
dans ma chambre à Marly, où il y avait longtemps que je retournais,
venant faire signer au roi le contrat de mariage du marquis de Charost
et de la fille, devenue héritière, de la duchesse de Choiseul, sœur de
l'ancien évêque de Troyes Bouthillier retiré, de la maréchale de
Clérembault, etc., et de son premier mari Brûlart, mort premier
président du parlement de Dijon. C'est elle que nous voyons remariée au
duc de Luynes et dame d'honneur de la reine, lorsque la maréchale de
Boufflers, qui l'avait été malgré elle, remit cette place et se retira à
Paris.

La bonne femme Gamaches, veuve du chevalier de l'ordre, mère de Cayeux,
qui alors prit le nom de Gamaches, mourut à plus de quatre-vingts ans.
Elle était fille et sœur des deux Brienne-Loménie, secrétaires d'État,
et tante paternelle de sa belle-fille. C'était une femme aimable, de
beaucoup d'esprit toute sa vie, fort du grand monde, et qui conserva sa
tête, sa santé et des amis jusqu'à la fin. Elle avait été amie intime de
M\textsuperscript{me} de Longueville, depuis son dernier retour, et dans
la plus étroite confiance de la princesse de Conti Martinozzi. J'ai ouï
conter à mon père que toutes les semaines, à jour pris, elles venaient
toutes les deux dîner chez sa première femme, la meilleure amie qu'eut
la princesse de Conti, que mon père allait ce jour-là dîner chez ses
amis, et qu'elles dînaient toutes trois la clochette sur la table et
passaient ensemble le reste du jour. Toutes deux alors étaient fort
belles, J'en ai trouvé, à la Ferté, deux petits portraits en pied de ce
temps-là en pendants d'oreilles les plus agréables du monde que j'ai
conservés avec soin.

Enfin le vieux duc de Gesvres mourut aussi et délivra sa famille d'un
cruel fléau. Il n'avait songé qu'à ruiner ses enfants et y avait
parfaitement réussi. J'ai assez parlé de cette espèce de monstre pour
n'avoir rien à y ajouter. Le duc de Tresmes avait, depuis longtemps, la
survivance de sa charge et de la capitainerie de Monceaux\,; il eut le
lendemain de cette mort le gouvernement de Paris.

Le président Payen, homme d'esprit, de bonne compagnie, et qui était
assez parmi le grand monde et les gens de la cour, était en ce temps-ci
chez Armenonville à Rambouillet, qu'il vendit depuis au comte de
Toulouse, sortit un moment avant souper hors la cour, apparemment pour
quelque nécessité\,; et comme il avait de gros yeux sortants qui
voyaient fort peu, il tomba dans le fossé, où on le trouva mort, la tête
cassée sur la glace\,; il fut fort regretté. Le roi l'avait chargé de
gouverner les abbayes du grand prieur, et lui donnait deux mille livres
de pension. Il était vieux et point marié.

Bouligneux, lieutenant général, et Wartigny, maréchal de camp, furent
tués devant Verue\,; deux hommes d'une grande valeur, mais tout à fait
singuliers. On avait fait l'hiver précédent plusieurs masques de cire de
personnes de la cour, au naturel, qui les portaient sous d'autres
masques, en sorte qu'en se démasquant on y était trompé en prenant le
second masque pour le visage, et c'en était un véritable tout différent
dessous\,; on s'amusa fort à cette badinerie. Cet hiver-ci on voulut
encore s'en divertir. La surprise fut grande lorsqu'on trouva tous ces
masques naturels, frais et tels qu'on les avait serrés après le
carnaval, excepté ceux de Bouligneux et de Wartigny, qui, en conservant
leur parfaite ressemblance, avaient la pâleur et le tiré de personnes
qui viennent de mourir. Ils parurent de la sorte à un bal, et firent
tant d'horreur qu'on essaya de les raccommoder avec du rouge, mais le
rouge s'effaçait dans l'instant, et le tiré ne se put rajuster. Cela m'a
paru si extraordinaire que je l'ai cru digne d'être rapporté\,; mais je
m'en serais bien gardé aussi, si toute la cour n'avait pas été comme moi
témoin, et surprise extrêmement et plusieurs fois, de cette étrange
singularité. À la fin on jeta ces deux masques.

Le 18 octobre mourut à Paris la duchesse d'Aiguillon, sœur du duc de
Richelieu, qui ne fut jamais mariée. C'était une des plus
extraordinaires personnes du monde, avec beaucoup d'esprit. Elle fut un
mélange de vanité et d'humilité, de grand monde et de retraite, qui dura
presque toute sa vie\,; elle se mit si mal dans ses affaires, qu'elle
raccommoda depuis, qu'elle cessa d'avoir un carrosse et des chevaux.
Elle aurait pu, quand elle voulait sortir, se faire mener par quelqu'un
ou se faire porter en chaise. Point du tout, elle allait dans ces
chaises à roue qu'on loue, qu'un homme traîne et qu'un petit garçon
pousse par derrière, qu'elle prenait au coin de la rue. En cet équipage,
elle s'en alla voir Monsieur, qui était au Palais-Royal, et dit à son
traîneur d'entrer. Les gardes de la porte le repoussèrent\,; il eut beau
dire ce qu'il voulut, il ne put les persuader. M\textsuperscript{me}
d'Aiguillon laissait disputer en silence. Comme elle se vit éconduite,
elle dit tranquillement à son pousseur de la mener dans la rue
Saint-Honoré\,; elle y arrêta chez le premier marchand de drap, et se
fit ajuster à sa porte une housse rouge sur sa vinaigrette, et tout de
suite retourna au Palais-Royal. Les gardes de la porte, bien étonnés de
voir cet ornement sur une pareille voiture, demandèrent ce que cela
voulait dire. Alors M\textsuperscript{me} d'Aiguillon se nomma, et avec
autorité ordonna à son pousseur d'entrer. Les gardes ne firent plus de
difficultés, et elle alla mettre pied à terre au grand degré. Tout le
Palais-Royal s'y assembla\,; et Monsieur, à qui on le conta, se mit à la
fenêtre, et toute sa cour, pour voir cette belle voiture houssée.
M\textsuperscript{me} d'Aiguillon la trouva si à son gré qu'elle y
laissa sa housse, et s'en servit plusieurs années, ainsi houssée,
jusqu'à ce qu'elle pût remettre son carrosse sur pied. Elle prit et
quitta plusieurs fois le voile blanc aux filles du Saint-Sacrement de la
rue Cassette, à qui elle fit de grands biens, et dont elle faisait fort
la supérieure, sans avoir pu se résoudre à y faire profession\,; et elle
le portait depuis plusieurs années, lorsqu'elle mourut dans ce monastère
à près de soixante-dix ans. Elle avait encore beaucoup de bien et ne se
remaria jamais.

Le marquis de Richelieu, fils de son frère, et cadet du duc de
Richelieu, était un homme obscur, ruiné, débauché, qui avait été
longtemps hors du royaume pour avoir enlevé des filles Sainte-Marie de
Chaillot, une fille du duc Mazarin, qui s'est depuis rendue fameuse par
les désordres et les courses de sa vie errante, belle comme le jour.
C'était un homme enterré dans la crapule et la plus vile compagnie,
quoique avec beaucoup d'esprit, et qu'on ne voyait ni ne rencontrait
jamais nulle part. On l'annonça à Marly à Pontchartrain, comme nous
allions nous mettre à table chez lui pour souper. Toute la compagnie en
fut extrêmement surprise\,; on jugea qu'il lui était survenu quelque
affaire bien pressante, pour laquelle il était permis à tout le monde de
venir à Marly, par les derrières, chez le ministre à qui on avait à
parler, en s'en allant après tout de suite et ne se montrant point.
Tandis que Pontchartrain était allé lui parler, j'imaginai que
M\textsuperscript{me} d'Aiguillon était morte, qu'il venait pour faire
parler au roi sur le duché, conséquemment qu'il n'y avait ou point de
droit, ou un droit litigieux\,; parce qu'un fils de duc, ou un héritier
nécessaire, dont le droit est certain est duc d'abord, ne demande aucune
permission pour en prendre le nom et le rang, et vient seulement, comme
tout autre homme de qualité, faire sa révérence au roi, etc., en manteau
long, s'il ne demande la permission de se dispenser de cette cérémonie,
comme fait maintenant presque tout le monde, depuis la prostitution des
manteaux longs à toutes sortes de gens. En effet, Pontchartrain, de
retour, nous dit que la duchesse d'Aiguillon était morte, qu'elle avait
fait le marquis de Richelieu son héritier, et qu'il venait le prier
d'obtenir du roi la permission d'être duc et pair.

Le roi, à qui il en rendit compte le lendemain, lui ordonna de mander au
marquis de Richelieu d'instruire le chancelier de sa prétention, avec
lequel Sa Majesté l'examinerait à son retour à Versailles, qui fut peu
de jours après. Le fait est que le cardinal de Richelieu avait obtenu,
en 1638, une érection nouvelle d'Aiguillon en duché-pairie mâle et
femelle, pour sa chère nièce de Combalet et ses enfants, etc., si elle
se remariait, car elle était veuve sans enfants d'un Beauvoir du Roure,
avec la clause inouïe, devant et depuis cette érection\,; en cas qu'elle
n'eût point d'enfants, de choisir qui bon lui semblerait pour lui faire
don du duché d'Aiguillon, en vertu duquel dont la personne choisie
serait duc ou duchesse d'Aiguillon et pair de France, dont la dignité et
la terre passerait à la postérité. M\textsuperscript{me} de Combalet,
dès lors duchesse d'Aiguillon, et en portant le nom, mourut en 1675 sans
s'être remariée, et fit un testament, par lequel elle exerça le pouvoir
que lui donnait cette clause en faveur de sa nièce, fille de son frère,
non mariée, qui en conséquence fut sans difficulté duchesse d'Aiguillon,
pair de France, et en porta le nom. M\textsuperscript{me} de Combalet,
que je continue d'appeler ainsi pour la distinguer de sa nièce, fit une
longue substitution par son testament du duché d'Aiguillon et de tous
ses biens, par laquelle elle ne fait aucune mention de sa dignité qu'en
faveur de sa nièce, n'en dit pas un mot sur aucun autre appelé après
elle, si elle meurt sans enfants, à la terre et duché d'Aiguillon, d'où
je conclus, dans le mémoire que je fis pour le chancelier : 1° Que les
lois qui sont exceptions ou extensions du droit commun se prennent à la
rigueur, et précisément à la lettre\,; que la clause extraordinaire et
inouïe de choix en faveur de M\textsuperscript{me} de Combalet n'en
porte qu'un et non davantage, encore moins l'étend-elle à la personne
par elle choisie pour avoir droit comme elle de faire un nouveau choix à
faute d'enfants\,; 2° ce choix a été fait et consommé par
M\textsuperscript{me} de Combalet en faveur de M\textsuperscript{me}
d'Aiguillon sa nièce, et il a eu tout son effet\,; 3° que
M\textsuperscript{me} d'Aiguillon, à faute d'enfants, n'a aucun droit de
choix, ni de laisser à personne sa dignité, éteinte en elle faute de
postérité\,; 4° que M\textsuperscript{me} de Combalet, pour qui la
clause de choix a été faite, a tellement senti qu'elle n'était que pour
elle, et que son choix à elle ne se pouvait répéter par la personne
choisie par elle, ni par elle-même M\textsuperscript{me} de Combalet
après le premier, que dans toute l'étendue de sa substitution elle n'a
énoncé sa dignité avec le duché d'Aiguillon qu'en faveur de sa nièce\,;
et toutes les fois qu'elle a appelé après elle d'autres substitués au
duché d'Aiguillon elle n'a jamais fait la moindre mention de la dignité,
mais uniquement de la possession de la terre\,; 5° que le choix est
consommé dans la personne de M\textsuperscript{me} d'Aiguillon\,;
qu'elle n'a aucun titre pour en faire un autre\,; que la clause insolite
a sorti son effet et n'a plus d'existence\,; que M\textsuperscript{me}
d'Aiguillon, morte fille, par conséquent sans postérité, peut disposer
de la terre et duché d'Aiguillon comme de ses autres biens, mais non de
sa dignité qui est éteinte par le droit commun qui reprend toute sa
force sitôt qu'il n'y a plus de loi expresse qui en excepte\,; 6° que le
marquis de Richelieu peut être seigneur et possesseur du duché
d'Aiguillon, soit comme appelé à cette substitution par
M\textsuperscript{me} de Combalet sa grand'tante, soit comme héritier
testamentaire de M\textsuperscript{me} d'Aiguillon sa tante, mais qu'il
ne peut jamais recueillir d'elle la dignité de duc et pair d'Aiguillon.

Les ducs de La Trémoille, La Rochefoucauld et autres en parlèrent au
chancelier, comme s'opposant aux prétentions du marquis de Richelieu. Je
fis mon mémoire en peu d'heures, je le lus au chancelier et le lui
laissai. Il avait les pièces du marquis de Richelieu, et l'avait
amplement entretenu. Il rapporta au roi cette affaire qui tint une
partie de la matinée du lendemain, sans tiers entre le roi et lui, et il
en reçut l'ordre de rendre au marquis de Richelieu ses papiers, de lui
défendre de sa part de prendre le nom et les marques de duc, d'en
prendre aucun rang ni honneurs, ni d'en faire aucunes poursuites dans
quelque tribunal que ce pût être. La chose en demeura là jusqu'en 1711
qu'elle n'eut pas un meilleur succès. Il sera temps alors de dire ce
qu'elle est devenue depuis.

Denonville, qui avait été sous-gouverneur de Mgr le duc de Bourgogne, et
qui avait marié son malheureux fils à la fille de Lavienne, premier
valet de chambre du roi, qu'il n'a pas rendue heureuse, fit tant auprès
du roi qu'il permit qu'il vînt tâcher de se justifier de sa belle
harangue de Bleinheim. Le duc de Marlborough lui donna aussitôt un congé
de quelques mois. Il était revenu de ses voyages d'Allemagne en
Hollande, où il avait fait venir le maréchal de Tallard et tous les
prisonniers considérables. Il les fit embarquer avec lui pour orner le
triomphe de son retour en Angleterre.

Villars, qui avait à peu près vu finir l'affaire des fanatiques, tenait
par commission les états de Languedoc. Il eut ordre de revenir à Paris,
et le duc de Berwick d'aller commander dans cette province après la fin
des États et le retour du maréchal de Villars. Ce fut par où finit cette
année. On ne voulut pas laisser Berwick sans un emploi principal en
chef, après la conduite qu'il avait eue en Espagne, et la façon dont il
en était revenu.

\hypertarget{chapitre-xx.}{%
\chapter{CHAPITRE XX.}\label{chapitre-xx.}}

1705

~

{\textsc{Année 1705.}} {\textsc{- Maréchaux de France subitement nommés
chevaliers de l'ordre.}} {\textsc{- Abus et suites de cette promotion.}}
{\textsc{- Bon mot de M. de Lauzun.}} {\textsc{- Catinat refuse l'ordre
faute de pouvoir prouver.}} {\textsc{- Villars et sa naissance\,; fait
du duc vérifié.}} {\textsc{- Remarques sur la cérémonie de l'ordre où
les maréchaux de France furent reçus.}} {\textsc{- Harcourt et Bedmar
reçus extraordinairement chevaliers de l'ordre.}} {\textsc{- Caractère
de Bedmar\,; ses obligations au roi.}} {\textsc{- Action devant Verue.}}
{\textsc{- Combat naval et secours jeté dans Gibraltar.}} {\textsc{-
Marlborough grandement reçu en Angleterre.}} {\textsc{- Tallard et les
principaux prisonniers à Nottingham.}} {\textsc{- Action légère en
Italie.}} {\textsc{- Lautrec tué\,; son caractère.}} {\textsc{- Conduite
de Maulevrier à Madrid, et sa faveur.}} {\textsc{- Adresse étrange de la
reine d'Espagne.}} {\textsc{- Adresse d'Harcourt et de
M\textsuperscript{me} de Maintenon en faveur de M\textsuperscript{me}
des Ursins.}} {\textsc{- Permission accordée à la princesse des Ursins
de venir à la cour.}} {\textsc{- Réunion d'Harcourt au chancelier et à
son fils, et d'eux par lui à la princesse des Ursins.}} {\textsc{-
Politique de la princesse des Ursins.}} {\textsc{- Attente à la cour de
la princesse des Ursins.}} {\textsc{- Princesse des Ursins à Paris.}}
{\textsc{- Princesse des Ursins à Versailles.}}

~

Le premier jour de cette année, l'abbé d'Estrées et Puysieux furent
reçus dans l'ordre du Saint-Esprit, et l'abbé en rochet et camail violet
comme les évêques. Harcourt avait le bâton pendant la cérémonie, parce
que, au changement de quartier parmi les capitaines des gardes, celui
qui sort garde le bâton jusqu'au sortir de la messe du roi, et à la
porte de la chapelle le donne à celui qui le relève. Tandis que Puysieux
prêtait son serment, le roi se tourna par hasard, vit Harcourt vêtu de
son justaucorps à brevet et fut choqué que ce qui l'approchait là de si
près ne fût pas chevalier de l'ordre. Cette fantaisie, qui ne lui avait
jamais pris et qui ne lui revint plus dans la suite, le frappa tellement
pour lors et il le dit ensuite, que dans le moment il voulut faire
Harcourt\,; puis, songeant qu'il y en avait d'autres à faire s'il
faisait celui-là, il rêva qui faire et qui laisser pendant le reste de
la cérémonie. Enfin il s'arrêta aux maréchaux de France, parce que, les
faisant tous, aucun d'eux n'aurait à se plaindre, et que, se bornant à
ce petit nombre, cette borne n'excluait personne personnellement. Il y
aurait eu grandement à répondre à un raisonnement si faux.

Jamais les maréchaux de France n'avaient eu droit à l'ordre comme tels,
et plusieurs ne l'ont jamais eu. Une dignité ou plutôt un office de la
couronne purement militaire, tel qu'est celui-là, et qui est la
récompense du mérite militaire, est donné sans égard à la naissance, et
c'est pour la naissance que l'ordre a été institué. Alors même le cas en
existait. De neuf maréchaux de France qui n'avaient pas l'ordre, il y en
avait plus d'un qui n'était pas né pour cet honneur-là\,; et plus d'un
aussi qui, ayant quelque noblesse, n'était pas né pour porter l'ordre.
En un mot, le roi le conçut et l'exécuta. En sortant de la chapelle, il
fit dire de main en main aux chevaliers d'entrer dans son cabinet, au
lieu de demeurer en haie dans sa chambre, et qu'il voulait tenir
chapitre. Il le tint donc tout de suite en rentrant, et nomma en bloc
les maréchaux de France, d'où M. de Lauzun dit que le roi, comme les
grands capitaines, avait pris son parti le cul sur la selle. C'est
depuis cette promotion, d'après laquelle on s'est infatué de croire que
le bâton donne l'ordre de droit, que M. le Duc étant premier ministre,
et qui haïssait les rangs et les dignités parce qu'il leur devait ce
qu'il ne voulait devoir ni rendre à personne, tout confondre et que tout
fût égal et peuple devant les princes du sang, fit les maréchaux de
France en 1724, excepté ceux qu'il fit maréchaux de France le même jour,
et ne fit point les ducs que ceux qu'il lui plut de faire, tandis
qu'aucun d'eux, en âge et non en disgrâce marquée, n'avait jamais été
omis comme tels en pas une grande promotion, même par Louis XIV, qui tes
dépouilla et les avilit tant qu'il put toute sa vie, et qui
publiquement, au chapitre de la promotion de 1688, fit les excuses qu'on
a vues sur les trois seules qu'il ne fit pas, et en voulut bien dire les
raisons. Le cardinal Fleury, depuis son règne, a fait tous les maréchaux
de France, quoiqu'il n'ait fait que de petites promotions de l'ordre\,;
en sorte que le droit établi et suivi depuis l'institution de l'ordre en
faveur de la première dignité du royaume (et qui, au contraire de
l'office de maréchal de France, suppose tellement la grande naissance
que les érections ont menti là-dessus quand la faveur déplacée y a élevé
des gens du commun) a été pour ainsi dire aboli et transmis à un office
de la couronne, qui ne suppose et qui souvent tombe sur des gens de peu
ou d'aucune naissance, depuis que la fantaisie momentanée du feu roi a
été prise pour une loi, parce qu'on l'a voulu de la sorte, tandis que
lui-même a fait des maréchaux de France depuis, à qui il n'a jamais
songé de donner l'ordre, et qui ne l'ont eu que longtemps après sa mort.
Cela peut s'appeler un rare échange. Mais achevons tout de suite cette
promotion du Saint-Esprit.

Ces maréchaux étaient le duc d'Harcourt, Cœuvres grand d'Espagne,
Villars qui venait d'être fait duc, Catinat, Vauban, qui s'appelait
Leprêtre, était du Nivernais\,; s'il était gentilhomme, c'était bien
tout au plus. Il montra son frère aîné pour le premier qui ait servi de
leur race, et qui avait été seulement de l'arrière-ban de Nivernais, au
retour duquel il mourut en 1635. Rien donc de si court, de si nouveau,
de si plat, de si mince. Voilà ce que les grandes et uniques parties
militaires et de citoyen ne pouvaient couvrir dans un sujet d'ailleurs
si digne du bâton, et de toutes les grâces que le seul mérite doit et
peut acquérir. Rosen était de condition, on l'a vu par ce que j'en ai
rapporté sur le témoignage de M. le prince de Conti, qui s'en informa
fort en son voyage de Pologne\,; mais je ne sais si c'était bien là de
quoi faire un chevalier de l'ordre. Chamilly s'appelait Bouton\,; il
était de bonne noblesse de Bourgogne, dès avant 1400, chambellans des
ducs de Bourgogne et baillis de Dôle. Ces emplois ne se donnaient alors
qu'à des gens distingués. Ce nom assez ridicule de Bouton le fit passer
mal à propos pour peu de chose. Châteaurenauld s'appelait Rousselet, il
était de Dauphiné. Il fallait que ce ne fût rien du tout, puisque
eux-mêmes ne montrèrent rien avant le bisaïeul du maréchal, intitulé
seigneur de quelques petits fiefs ou rotures, mort en 1564, et qui dut
son être et celui de ses enfants à la sœur du maréchal et du cardinal de
Gondi qu'il épousa en 1533, en décembre, c'est-à-dire du temps
qu'Antoine de Gondi, son beau-père, était banquier à Lyon, et quelques
mois avant que Catherine de Médicis y passât après son mariage, et
qu'elle y prît Catherine de Pierrevive, sa belle-mère, à son service,
qui devint sa favorite, sa confidente, la gouvernante de ses enfants, et
qui fit la fortune des Gondi en France. Avec cela, le fils de Rousselet
ne fut que le protégé des Gondi, gouverneur de leurs châteaux de
Machecoul et de Belle-Ile, et rien de plus. Il acheta d'eux une terre en
Bretagne, et Châteaurenauld en Touraine. Le père n'ayant rien été, qui
était le beau-frère, le fils ne pouvait guère être mieux, et cela montre
le cas que le maréchal de Retz, si puissant toute sa vie, et le cardinal
son frère, faisaient de cette alliance et de leur propre neveu. Leur
petit-neveu, père du maréchal, ne fut rien du tout, dont le frère aîné
pour tout grade fut lieutenant de la maîtrise de camp du régiment des
gardes. Cela est bien neuf, bien chétif, bien éloigné de l'ordre du
Saint-Esprit. Pour le bâton, Châteaurenauld l'avait dignement mérité.
Montrevel, tout au contraire, sans aucune sorte de mérite avec une
grande naissance\,; était de plain-pied avec l'ordre, et d'une inégalité
au bâton qui faisait honte à le lui voir entre les mains parcourt, s'il
était parcourt, comme il le prétendait, valait au moins Montrevel pour
la naissance. Il était duc, et on a vu plus d'une fois ici quel
personnage ce fut.

Catinat était arrière-petit-fils du lieutenant général de Mortagne au
Perche, mort en 1584\,; c'étaient apparemment des manants de là autour,
puisque c'est le premier qu'on connaisse. Son fils et son petit-fils
furent conseillers au parlement\,; le petit-fils devint doyen de cette
compagnie, et eut Saint-Gratien de sa femme, fille d'un autre conseiller
au parlement. De ce mariage, quantité d'enfants, dont le maréchal de
Catinat fut le cinquième fils. L'aîné fut conseiller au parlement, puis
conseiller d'honneur en faveur de son frère, et laissa un fils aussi
conseiller au parlement. Catinat apprit de bonne heure à Paris la
promotion des maréchaux de France\,; il alla à Versailles, et fit
demander au roi à lui parler dans son cabinet, qui l'y fit entrer au
sortir de son dîner. Là il remercia le roi de l'honneur qu'il venait de
lui faire, et en même temps, lui dit qu'il ne pouvait le tromper, et lui
expliqua qu'il ne pouvait faire de preuves\,; il était extrêmement
mécontent et avec grande raison. Il était philosophe. Il s'accoutumait
de propos délibéré à la retraite. Cela se passa de sa part très
respectueusement, mais fort froidement, jusque-là qu'il y en eut qui
crurent qu'il n'avait pas été trop fâché de faire ce refus. Le roi le
loua fort, mais sans le presser, comme il avait fait en pareil cas à
l'archevêque de Sens, Fortin de La Hoguette, et toute la cour, qui sut
le même jour ce refus, y applaudit extrêmement. Au sortir du cabinet du
roi, il s'en alla à Paris, et s'y déroba modestement à toutes les
louanges. Ce fut donc le troisième, et tous trois du règne du roi, qui
refusa l'ordre, faute de pouvoir faire ses preuves\,: le maréchal Fabert
en 1661, et ces deux-ci. Combien d'autres en auraient dû faire de même,
sans parler des légers\,!

Venons maintenant au maréchal de Villars, le plus complètement et
constamment heureux de tous les millions d'hommes nés sous le long règne
de Louis XIV. On a vu ci-devant quel fut son père, sa fortune, son
mérite, celui que M\textsuperscript{me} Scarron lui trouva, et que,
devenue M\textsuperscript{me} de Maintenon, elle n'oublia jamais. Il
passait pour être fils du greffier de Condrieu. Son père eut pourtant un
régiment, peut-être de milice, et passa, en 1635, pour sa prétendue
noblesse. On sait assez comment se font ces recherches de noblesse\,:
ceux qui en sont chargés ne sont pas de ce corps, et plus que très
ordinairement le haïssent et ne songent qu'à l'avilir. Ils dépêchent
besogne, leurs secrétaires la défrichent, et font force nobles pour de
l'argent\,; aussi est le proverbe\,: qu'ils en font plus qu'ils n'en
défont.

La femme de ce grand-père du maréchal était Louvet, qui est le nom des
Cauvisson, et ces Cauvisson ne sont pas grand'chose. Le père de celui-là
eut, disent-ils, un guidon dans la compagnie de chevau-légers du sieur
de Peyrand, c'est-à-dire d'une compagnie levée dans le pays par qui en
voulut prendre la peine. On le donne encore pour avoir commandé à
Montluel et à Condrieu, par commission de M. d'Alincourt, gouverneur de
la province. Ce dernier eût été bien étonné, quelque fortune qu'il eût
faite, s'il eût vu celle de son fils. À quel excès l'eût-il donc été,
s'il eût pu prévoir celle de la postérité d'un manant renforcé, qu'il
trouva sous sa main à mettre dans un colombier\,! Ce même homme eut une
place dans les cent gentilshommes de la maison du roi, c'est-à-dire les
becs-de-corbin\footnote{Les \emph{becs-de-corbin}, ou gentilshommes à
  bec-de-corbin, formaient deux compagnies de la maison militaire du
  roi. Ils tiraient leur nom de leur hallebarde en forme de
  bec-de-corbin. La première compagnie avait été instituée par Louis XI
  en 1478. Charles VIII établit la seconde en 1497. Supprimées sous
  Louis XIII, ces deux compagnies furent rétablies par Louis XIV. et
  définitivement licenciées sous Louis XVI, en 1776. Les gentilshommes à
  bec-de-corbin précédaient le roi dans les grandes cérémonies en
  marchant deux à deux.}, depuis longtemps dès lors anéantis par les
compagnies des gardes du corps, et ces places s'achetaient déjà du
capitaine pour s'exempter de la taille. J'ai peine à croire que la
noblesse du Lyonnais l'ait employé en 1614 à dresser ses mémoires et à
les présenter aux états, peut-être comme un compagnon entendu et
intrigant, car on n'ose proférer le mot de député de la noblesse, qu'on
n'eût pas oublié, s'il eût eu cet honneur qui aurait constaté la sienne.
On le dit aussi chevalier de Saint-Michel\,; mais dès lors, qu'est-ce
qui ne l'était pas avec la plus légère protection, qui que l'on pût
être\,? Le père de celui-ci est donné pour avoir été mis commandant dans
Condrieu par le duc de Nemours\,; outre la petitesse de l'emploi, il ne
prouve point de noblesse. Ce qu'ils ont de mieux est un oncle paternel
de Villars, père du maréchal, archevêque de Vienne, duquel un oncle
paternel le fut aussi. De ces temps-là de troubles encore plus que de
ceux-ci, on choisissait des évêques par d'autres raisons que par la
naissance, et cette illustration, quand elle est unique, n'en est pas
une. Ils prétendent en avoir eu deux antérieures, et ainsi quatre de
suite. Mais on prétend aussi que ces deux précédents étaient de ces
anciens Villars, seigneurs de Dombes, égaux en naissance aux
dauphins\footnote{Il s'agit ici des anciens seigneurs de Dauphiné qu'on
  appelait \emph{dauphins} de Viennois.} avec qui ils avaient des
alliances directes, des filles de Savoie, et de très grandes terres\,;
que ce Villars du maréchal était aumônier du second de ces archevêques
qui le prit en amitié, l'éleva, le fit évêque \emph{in partibus}, puis
son coadjuteur. En effet, il est difficile d'ajuster ces deux premiers
Villars, archevêques de Vienne, oncle et neveu, qui ont tous deux fait
un personnage principal dans toutes les affaires de leur temps, être
fils d'un homme de rien et tout à fait inconnu, frère du juge ordinaire
de Lyon devenu lieutenant particulier civil et criminel de ce siège, et
celui-là père du deuxième de ces deux premiers archevêques et du
lieutenant général au présidial\footnote{On appelait \emph{lieutenants},
  dans l'ancienne organisation judiciaire de la France, les magistrats
  qui remplaçaient le premier officier d'un tribunal en cas d'absence.
  Ainsi le lieutenant général de la sénéchaussée de Lyon, dont parle ici
  Saint-Simon, remplaçait le sénéchal, qui était toujours un homme
  d'épée, dans la présidence du tribunal, qu'on pourrait comparer au
  tribunal de première instance de nos jours et qu'on appelait alors
  présidial. Ces tribunaux, subordonnés aux parlements, avaient une
  juridiction tout à la fois civile et criminelle. Dans certains cas,
  prévus par les ordonnances, ils jugeaient sans appel. On peut
  consulter, pour les détails, tousse, \emph{De la juridiction des
  présidiaux}.} et sénéchaussée de Lyon, qui succéda après à son
beau-père en la place de premier président au parlement de Dombes.

Voilà un préambule étrange de ce qui va suivre. Le roi et Chamillart
étaient fort étourdis d'Hochstedt et de ses grandes suites. C'était le
premier revers qu'il avait essuyé, et ce revers le ramenait de l'attaque
de la Bohême et de l'Autriche à la défense de l'Alsace, qui se regardait
comme très difficile après la perte de Landau, sans compter les États de
l'électeur de Bavière et ses enfants, en proie à la vengeance de
l'empereur. Tallard était prisonnier, Marsin semblait trop neuf et trop
futile pour se reposer sur lui d'un emploi si important. Villeroy, quel
qu'il fût, était destiné pour la Flandre avec l'électeur. Boufflers
était hors de gamme\,; et tous les autres maréchaux aussi. De princes du
sang, le roi n'en voulait pour rien à la tête de ses armées restait
Villars, car parcourt se gardait bien de se vouloir éloigner de la cour,
ni M\textsuperscript{me} de Maintenon de s'en défaire dans la crise où
ils se trouvaient pour lors\,; Villars, comme on l'a vu, avait comme
parcourt, et par les mêmes raisons paternelles, toute la protection de
M\textsuperscript{me} de Maintenon, conséquemment celle de Chamillart,
plus favori alors, s'il se peut encore, que ministre tout-puissant de la
guerre et des finances. Villars qui, dès la Bavière, avait osé prétendre
à la dignité de duc, n'avait rien rabattu de son audace pour ses
pillages et sa chute en Languedoc\,; il y triomphait de la besogne qu'il
y avait trouvée faite\,; il en donnait la consommation comme due
uniquement à lui, et Bâville, le plus haineux des hommes, et qui n'avait
jamais pu souffrir Montrevel, secondait du poids de son témoignage les
vanteries de Villars. Ce maréchal n'avait cessé d'écrire au roi, à
Chamillart, à M\textsuperscript{me} de Maintenon sur les fautes
d'Hochstedt et sur celles de ses suites, de leur mander tout ce qu'il
aurait fait, de déplorer de s'être trouvé éloigné de ses armées, en un
mot de fanfaronner avec une effronterie qui ne lui avait jamais manqué,
et qui le servit d'autant mieux en cette occasion qu'il parlait à des
gens ébranlés et dans le dernier embarras sur le choix d'un général
capable de soutenir un poids devenu si difficile du côté du Rhin et de
la Moselle, et si âpres à se flatter et à se promettre.

M\textsuperscript{me} de Maintenon tira sur le temps\,; elle sentit
l'embarras et le besoin, elle vit les pillages de Villars, et ses
insolences avec l'électeur effacées\,; elle comprit quelles pouvaient
être les grâces d'un homme devenu comme nouveau\,; elle en profita, et
Villars, qui sentit ses lettres goûtées, fit sentir aussi combien il se
trouvait affligé sur la manière dont ses espérances d'être duc avaient
été reçues. Quand le roi se fut bien laissé mettre dans la tête qu'il
n'y avait que Villars dont il se pût servir dans la conjoncture
présente, il fut aisé de lui persuader qu'il ne s'en fallait pas servir
mécontent et offensé, et de là, le ministre, et la dame qui le faisait
agir, parvinrent à faire qu'il serait duc en arrivant. Il reçut donc un
courrier qu'il lui porta ordre de finir le plus promptement qu'il lui
serait possible les états de Languedoc qu'il avait la commission de
tenir, et de se rendre en même temps à la cour le plus diligemment qu'il
lui serait possible. Il arriva à Versailles le 15 janvier, et fit la
révérence au roi comme il arrivait de se promener à Marly. Le roi, en
descendant de carrosse, lui dit de monter en haut et qu'il lui
parlerait. Étant rhabillé et entré chez M\textsuperscript{me} de
Maintenon, il l'y fit appeler, et dès qu'il le vit\,: «\,Je n'ai pas
maintenant, lui dit-il, le temps de vous parler, mais je vous fais
duc\,;» ce monosyllabe valait mieux que toutes les audiences dont aussi
pour le maréchal il était le but. Il sortit transporté de la plus
pénétrante joie, et en apprenant la grâce qu'il venait de recevoir,
causa la plus étrange surprise pour ne pas dire au delà, et la plus
universelle consternation dans toute la cour, qui, contre sa coutume, ne
s'en contraignit pas. Jusqu'à M. le Grand jeta chez lui feu et flammes
devant tout le monde, et tous les Lorrains s'en expliquèrent avec le
même ressentiment et aussi peu de ménagement. Les ducs, ceux qui
aspiraient à l'être, ceux qui n'y pouvaient penser, furent également
affligés. Tous furent indignés d'avoir, les uns un égal de cette espèce,
les autres d'en être précédés et distingués, les princes du sang d'avoir
à lui rendre, et les autres princes d'avoir à céder ou à disputer à une
fortune aussi peu fondée en naissance. Le murmure fut donc plus grand
pour cette fois que la politique\,; les compliments froids et courts, et
le nouveau duc les cherchant, se les attirant, et allant assez
infructueusement au-devant de chacun, montrant, au travers de beaucoup
d'effronterie, grand respect aux uns et grand embarras à tous.

Le jour de la Chandeleur venu, les maréchaux furent reçus, excepté
Harcourt, qui s'était trouvé mal, et l'abbé d'Estrées chanta la messe
comme prélat de l'ordre. Pontchartrain, fort mal avec tous les Estrées,
content d'avoir échappé au comte de Toulouse par la compassion qu'il
avait eue de sa femme, fit une niche à l'abbé d'Estrées, qu'il me conta
en s'en applaudissant fort. Quoiqu'il ne fût pas lors ni de quatre ans
depuis officier de l'ordre, il alla, comme secrétaire de la maison du
roi, lui faire remarquer que l'abbé d'Estrées, n'étant point évêque, ne
devait point s'asseoir en officiant devant lui qu'au temps où les
prêtres s'y asseyent, et n'avoir comme eux qu'un siège ployant et non
pas un fauteuil. L'avis fut goûté et toujours exécuté depuis, à la
grande amertume du pauvre abbé d'Estrées. Il fut réglé à l'occasion de
cette promotion qu'encore que les grands d'Espagne n'observent entre eux
aucun rang d'ancienneté, ils le garderaient en France, parce que les
ducs l'avaient toujours fait entre eux, et qu'étant égalés, et par
conséquent mêlés ensemble, ce mélange ne se pouvait exécuter autrement,
et cela s'est depuis toujours observé parmi eux.

Ainsi Harcourt étant malade, qui était duc plus ancien que le maréchal
de Cœuvres était grand, ce maréchal fut présenté seul par les ducs de La
Trémoille et de Chevreuse, et après avoir reçu l'ordre seul, prit sa
place après le dernier duc n'y en ayant point de moins ancien que lui
grand. Le maréchal de Villars, déclaré duc héréditaire, n'était pas
encore enregistré au parlement. Il n'avait point même de terre qui pût
être érigée\,; ce ne fut que plusieurs mois après qu'il acheta Vaux, où
M. Fouquet avait dépensé tant de millions et donné de si superbes fêtes.
Vaux relevait presque toute de Nangis, avec qui il s'accommoda, pour ne
relever que du roi, suivant le privilège d'y forcer les suzerains des
duchés, et on peut croire que Nangis qui servait dans son armée, où le
marché se conclut, et qui était un de ses plus bas courtisans, de la
complexion dont il le connaissait sur la bourse, ne lui tint pas la
bride haute\,; Villars donc jusqu'à son enregistrement n'étant considéré
que comme duc à brevet, c'est-à-dire non vérifié ou enregistré, n'eut
aucun rang dans l'ordre, jusqu'à ce qu'il le fût\,; il marcha entre les
maréchaux de Chamilly et de Châteaurenauld, comme leur ancien de
maréchal de France et tous trois ensemble furent présentés par le comte
de Solre et par le marquis d'Effiat. Après avoir reçu l'ordre, ils
prirent les dernières places après tous les chevaliers, et Villars comme
eux. MM. d'Étampes et de Puysieux présentèrent après les maréchaux de
Vauban, Rosen et Montrevel qui s'assirent après avoir reçu l'ordre après
les trois autres maréchaux, et au retour de la chapelle chez le roi,
marchèrent tous six les derniers de tous, et le maréchal de Cœuvres
précéda tous les chevaliers qui n'étaient pas ducs.

Je remarque ce détail qui depuis l'institution de l'ordre a toujours été
observé et pratiqué sans aucune difficulté de même, et il se trouvera
dans la suite que cette remarque n'est pas inutile. J'ajouterai que les
preuves de Rosen ne furent que testimoniales. Torcy, qui comme
chancelier de l'ordre rapporta les preuves, ne montra que les
attestations du commandant pour le roi de Suède en Livonie, et des
premiers seigneurs et des principaux magistrats du pays, qu'il pouvait
entrer dans tous les chapitres nobles. Torcy s'appuya de l'exemple des
maréchaux de Schomberg, père et fils, dont le dernier fut duc et pair
d'Halluyn, et du cardinal de Fürstemberg, dont les preuves pour l'ordre
du Saint-Esprit ne furent que testimoniales.

Achevons de sortir de la matière de l'ordre. Le marquis de Bedmar y
avait été nommé, ses preuves admises, et il le portait en attendant
qu'il fût reçu\,; le roi avait été extrêmement content de lui, lorsqu'il
avait été gouverneur des armes aux Pays-Bas, sous l'électeur de Bavière,
gouverneur général de ces provinces depuis l'avènement de Philippe V à
la couronne d'Espagne, et encore davantage depuis que le commandement en
chef roula sur lui par intérim, tandis que l'électeur fut en Allemagne.
Bedmar, sorti de bonne heure d'Espagne, avait toujours servi au
dehors\,; il avait de l'esprit, de la grâce, du liant, des manières
douces, affables, honnêtes. Il était ouvert et poli avec un air de
liberté et d'aisance fort rare aux Espagnols\,; de la valeur et du
maniement des troupes\,; grand courtisan, qui fit son capital de plaire
aux maréchaux de Villeroy et de Boufflers, qui le vantèrent fort au roi,
à nos officiers généraux, particuliers, et de bien traiter partout les
troupes françaises. De tout cela le roi le prit en amitié, demanda et
obtint pour lui la grandesse de première classe que sa naissance
comportait fort, le fit chevalier de l'ordre, et depuis le malheur
d'Hochstedt et le retour de l'électeur aux Pays-Bas, lui procura la
vice-royauté de Sicile, que le cardinal del Giudice n'exerçait que par
intérim. Bedmar quitta donc les Pays-Bas. Il salua le roi le 2 mars, et
en fut reçu en homme comblé de ses grâces. Le 8, il fut reçu
extraordinairement chevalier de l'ordre avec Harcourt, qui le précéda
comme plus ancien duc que Bedmar n'était grand, et ils furent présentés
ensemble par le maréchal de Villeroy et le duc de Beauvilliers. Tout se
passa comme aux fêtes de l'ordre, excepté qu'il n'y eut qu'une messe
basse\,; il n'y avait presque point d'exemple de réception hors les
fêtes de l'ordre. Il se trouva pourtant que le marquis de Béthune,
l'allant porter au roi de Pologne son beau-frère, avait été reçu ainsi,
et nous verrons dans la suite le duc d'Aumont l'être de même partant
pour son ambassade extraordinaire d'Angleterre. Reprenons maintenant le
fil ordinaire.

Il se passa une assez grande action le soir du 26 décembre devant Verue.
M. de Savoie fit passer le pont de Crescentin, par un brouillard fort
épais, à la plupart des troupes qu'il avait dans ce camp, et qui, sans
entrer dans Verue, dont on se souviendra qu'elles avaient la
communication libre, vinrent envelopper les tranchées par la droite et
par la gauche, se rejoignirent à la queue, pour couper toute retraite
pendant qu'elles attaqueraient par les deux flancs et par la queue même,
et qu'en même temps la tête le serait par une sortie de la garnison.
C'est ce qu'elles exécutèrent. Chartogne, lieutenant général, et
Imécourt, maréchal de camp de tranchée, rassemblèrent tout ce qu'ils
purent pour faire face partout et se défendre\,; le premier y fut blessé
et pris, l'autre tué. Cependant l'attaque fut si bien soutenue partout,
que M. de Vendôme, qui venait de se coucher, eut le temps de faire
prendre les armes à six brigades d'infanterie, à la tête desquelles il
rechassa les ennemis de tous les postes qu'ils avaient pris\,; ils
tinrent assez dans la batterie\,; mais à la fin ils cédèrent et furent
poursuivis jusque dans le fossé. Il y eut force monde tué et blessé de
part et d'autre, mais beaucoup plus du leur. M. de Savoie était
cependant dans une des tours du donjon, attendant un meilleur succès.
Leur surprise fut grande le lendemain, lorsque, de vingt-trois pièces de
canon qu'ils avaient enclouées, ils virent et entendirent qu'on avait
trouvé le moyen d'en désenclouer vingt et une, qui tirèrent sur eux à
l'ordinaire.

Le siège de Gibraltar se poussait comme on pouvait. Six vaisseaux
anglais s'y présentèrent le 24 décembre, escortant sept frégates
destinées à y entrer et à y porter du secours. Pointis les attaqua, prit
quatre frégates, mais il ne put empêcher les trois autres d'entrer et de
porter aux assiégés mille hommes de secours, avec les munitions et les
rafraîchissements dont elles étaient chargées. Le roi d'Espagne envoya
quatre mille hommes de renfort à ce siège.

Marlborough fut reçu en Angleterre avec des acclamations et des honneurs
extraordinaires. La chambre basse lui envoya une députation. Son orateur
le harangua\,; il le fut aussi par le chancelier, lorsqu'il alla prendre
séance pour la première fois dans la chambre haute\,; ils ne voulurent
point souffrir le maréchal de Tallard dans Londres, ni près de cette
ville où il avait été longtemps ambassadeur, et avait conservé force
connaissances. Ils l'envoyèrent fort loin de là et de la mer, à
Nottingham, avec les prisonniers les plus distingués, et répandirent les
autres ailleurs. Ils eurent tous les lieux où on les mit pour prison,
avec la liberté de se promener partout, et même à la campagne, mais sans
découcher, et doucement observés de fort près.

Le grand prieur, de son côté, attaqua, le 2 février, les postes que le
général Patay gardait entre le mont Baldo et l'Adige, avec, mille
chevaux et trois bataillons en divers endroits. Ses troupes firent une
assez molle défense et furent chassées de partout. On leur prit six
drapeaux et quatre cents prisonniers, et cette expédition leur ôta la
communication avec le Véronais, d'où ils tiraient leurs vivres. Médavy
avait, le même jour, assemblé ses troupes de l'Oglio pour inquiéter les
ennemis de ce côté-là, et les empêcher de secourir leur major général
Patay. Le comte de Linange, qui commandait l'armée depuis que le prince
Eugène n'était plus en Italie, se sentant beaucoup supérieur à Médavy,
leva tous ses quartiers pour le venir combattre, sur quoi Médavy se
retira sur l'Oglio, en un poste où il ne pouvait pas l'être, et détacha
Lautrec avec cinq cents chevaux pour observer les ennemis. Il fut coupé
par un corps plus fort que le sien, pendant que le gros marchait à lui
pour l'attaquer. Dans cette presse, il remarcha en arrière pour rompre
les troupes qui l'avaient coupé, et se percer une retraite avant que de
se trouver pris en tête et en queue. Il réussit en effet, et rejoignit
Médavy avec soixante prisonniers qu'il avait faits, mais il reçut une
grande blessure dont il mourut peu de jours après à Brescia, où on
l'avait porté.

Ce fut un extrême dommage\,; il était fort bien fait, avec infiniment
d'esprit, de grâces dans l'esprit, et du savoir, une grande application,
une grande volonté et beaucoup de talents pour la guerre\,; doux, poli
et très aimable. Le traitement plus que très dur d'Ambres, son père, lui
avait {[}fait{]} prendre le parti depuis plusieurs années de ne bouger
de sa garnison et des frontières, faute de subsistance et de pouvoir
soutenir ses humeurs. Cette vie et une santé assez délicate l'avait
rendu très particulier et très studieux, et il s'était enfin fort
accoutumé à ce genre de vie, quoique fait pour la meilleure compagnie,
qu'il aimait beaucoup et dont aussi il était fort recherché.

Maulevrier, dans le dessein où nous l'avons laissé, s'était chargé de
force lettres importantes pour la princesse des Ursins et de celles de
M\textsuperscript{me} la duchesse de Bourgogne pour la reine d'Espagne.
Au succès qu'on a vu de Tessé, fait grand le lendemain de son arrivée à
Madrid, on peut juger si lui et son gendre avaient bien travaillé à
Toulouse. M\textsuperscript{me} des Ursins regarda cette visite et les
nouvelles qu'elle en reçut comme les avant-coureurs de sa délivrance, et
Tessé et son gendre livrés à elle comme des gens qu'il fallait combler,
et qui lui seraient également utiles aux deux cours. Elle gagnait tout à
l'échange de Berwick pour Tessé. Maulevrier n'oublia rien pour se rendre
considérable. Il n'avait que trop de quoi jeter de la poudre aux yeux.
M\textsuperscript{me} des Ursins y fut prise. Elle était trop bien
informée pour ignorer les visites continuelles à Marly de
M\textsuperscript{me} de Maintenon et de M\textsuperscript{me} la
duchesse de Bourgogne à Maulevrier, sous prétexte d'aller chez sa femme,
et quantité d'autres détails. Mais quand Maulevrier lui eut raconté son
roman en beau, et que Tessé en appuyait la croyance, elle ne crut
pouvoir trop acheter un homme aussi initié dans le plus intérieur et
capable de si profondes et de si hardies intrigues\,; elle lui donna
donc sa confiance ainsi qu'à Tessé, et leur assura ainsi toute celle du
roi et de la reine d'Espagne avant que d'être arrivés auprès d'eux. De
Toulouse, elle gouvernait leur esprit et leurs affaires plus
despotiquement encore, s'il se peut, et plus sans partage que le
cardinal Mazarin, chassé du royaume, ne gouverna jamais la reine mère et
les affaires de France de chez l'électeur de Cologne, où il était
retiré.

Tessé et Maulevrier, annoncés à Madrid sur le pied de ce que je viens
d'expliquer, et chargés encore des lettres de la princesse des Ursins,
trouvèrent une ouverture entière dans le roi et la reine d'Espagne. La
première conversation fut un épanchement de cour de leur part, surtout
de celle de la reine\,; c'était par eux qu'elle fondait ses plus grandes
espérances du retour de la princesse des Ursins, sans laquelle elle ne
croyait pouvoir subsister ni vivre. Tessé, pressé d'aller sur la
frontière donner ordre à tout, et par la chose même, et par les ordres
réitérés du roi, ne put différer, dès qu'il eut conféré avec Berwick à
Madrid, et fait sa couverture. Maulevrier, allé en Espagne comme un
malade aux eaux, demeura à Madrid pour suppléer à l'absence de son
beau-père dans tout ce qui regardait l'intime confidence du palais sur
M\textsuperscript{me} des Ursins. Avec de l'esprit, là connaissance
qu'il avait de notre cour, les lumières qu'il avait tirées de la
confiance de la princesse des Ursins à Toulouse, il donna à la reine des
conseils pour des démarches dont elle éprouva l'utilité. Elle,
M\textsuperscript{me} des Ursins, M\textsuperscript{me} de Maintenon,
tout marchait en cadence.

Maulevrier sut profiter de ce que la reine n'avait personne de notre
cour à qui elle pût s'ouvrir de son désir le plus ardent, ni plus
instruit, ni de qui elle fût là-dessus plus sûre. Elle prit tant de goût
à ces entretiens secrets, qu'elle fit donner les entrées à Maulevrier
par le roi d'Espagne, qui, par chez ce prince, entrait chez la reine à
toute heure. Il avait pour cautions son beau-père, M\textsuperscript{me}
la duchesse de Bourgogne et la princesse des Ursins. Avec ces avantages,
il sut pousser les privances bien loin. En sous-ordre, la reine voulait
aussi le rappel du duc de Grammont, coupable du crime irrémissible à ses
yeux d'être contraire au retour de M\textsuperscript{me} des Ursins, et
de ne l'avoir vue que froidement dans sa route. Par là il était devenu
insupportable au roi et à la reine. Les affaires les plus pressantes
périssaient entre ses mains. Il y avait plus\,: par un conseil profond,
la reine d'Espagne avait persuadé au roi son mari de choquer en tout les
volontés du roi son grand-père, et de négliger en tout ses conseils avec
affectation. Le roi s'en plaignait avec amertume. Le but était de le
lasser par là, et de lui faire comprendre qu'il n'y avait que
M\textsuperscript{me} des Ursins, bien traitée et renvoyée
toute-puissante, qui pût remettre les choses dans le premier état, et le
faire obéir en Espagne comme dans les premiers temps.

Quand tout fut bien préparé, et que le roi {[}fut{]} adouci par le temps
de l'exil, par les grâces faites aux Estrées, par les insinuations
éloignées, parles artifices des lettres qui lui venaient de Tessé, où il
n'était pas toutefois question de la princesse\,; qu'il fut jugé qu'il
était temps d'agir plus à découvert, et que le roi {[}était{]} lassé des
dépits de la reine, de la mollesse pour elle de son petit-fils et de la
résistance qu'il trouvait à tout ce qu'il proposait de plus utile et de
plus raisonnable en Espagne, où il avait longuement éprouvé avec tant de
complaisance qu'on n'y cherchait qu'à prévenir son goût et sa volonté,
surtout à lui marquer une complaisance et une obéissance parfaite, on se
garda bien de lui laisser entrevoir qu'on songeât, ni
M\textsuperscript{me} des Ursins elle-même, à aucun retour en Espagne\,;
comme pour obtenir Toulouse au lieu de l'Italie on avait pris le même
soin de l'empêcher de s'apercevoir qu'il pût être jamais question de la
revoir à Paris et à la cour. Ce changement de l'Italie à Toulouse, que
la mollesse ou le peu de lumières des ministres souffrit dans un temps
de colère, à eux si favorable pour l'empêcher, fut le salut de toute la
grandeur de leur ennemie qui, une fois en Italie et à Rome, eût été trop
éloignée d'Espagne et de France pour machiner à temps et utilement, et,
revenue là en son premier état de consistance, y serait demeurée pour
toujours. On se garda donc bien, je le répète, de laisser entrevoir au
roi aucun désir, aucune idée de retour en Espagne.

Mais Harcourt, d'une part, qui, avec art et hardiesse, s'était toujours
conservé la liberté de parler au roi des choses d'Espagne, et
M\textsuperscript{me} de Maintenon, de l'autre, lui représentèrent peu à
peu le pouvoir sans bornes de la reine d'Espagne sur le roi son mari\,;
le dépit extrême dont elle donnait des marques jusqu'à la contradiction
la plus continuelle et la plus aigre pour tout ce qui venait du roi aux
dépens de ses propres affaires, par une humeur dont elle n'était plus
maîtresse, et qui en effet avait bien sa cause dans la dureté
qu'éprouvait une personne pour qui elle avait déployé tout ce qui était
en elle pour adoucir l'ignominie de son sort\,; qu'après tout il n'était
question, pour la contenter, que d'une complaisance entièrement
étrangère et indifférente aux affaires, qui n'y pouvait rien influer, de
permettre à M\textsuperscript{me} des Ursins de venir à la cour y dire
tout ce qu'il lui plairait pour sa justification, et devenir après tout
ce qu'il lui plairait, excepté d'y demeurer et de retourner en Espagne,
retour dont la reine même ne parlait plus et se bornait à ce que son
amie pût être entendue elle-même\,; que ce qui ne se refusait pas aux
plus coupables pouvait bien s'accorder à une personne de sari sexe et de
cette qualité\,; que, quelles que fussent les fautes qu'elle eût
commises, sa chute de si haut et si prompte, l'exil où depuis si
longtemps elle en donnait le spectacle, le contraste des récompenses si
marquées du cardinal et de l'abbé d'Estrées, étaient une pénitence qui
méritait bien qu'enfin le roi, content de lui avoir fait sentir le poids
de son indignation, et à la reine d'Espagne celui de son autorité
paternelle, voulût bien marquer à une princesse, par les mains de qui on
était réduit à passer pour toutes les affaires, et qui était outrée, une
considération qui sûrement l'adoucirait, la charmerait même, et la
ferait rentrer dans le chemin d'où le dépit l'avait égarée\,; qui, s'il
était continué, pouvait, par de mauvais conseils d'humeur et de colère,
porter les affaires en de fâcheuses extrémités qui, après les malheurs
d'Hochstedt, de Gibraltar, de la révolte de la Catalogne, demandaient
des soins et une conduite qui ne pouvaient réussir que par un grand
concert.

L'archevêque d'Aix, maître consommé en intrigues, l'homme le plus hardi,
le plus entreprenant, le plus plein d'esprit et de ressources, et qui,
depuis le temps de Madame et le retour de son exil\footnote{Les Mémoires
  de cet archevêque d'Aix (Daniel de Cosnac) ont été publiés par la
  \emph{Société de l'Histoire de France} (Paris, 1852, 2 vol.~in-8). On
  y trouve tous les détails de ces intrigues.}, s'était conservé une
sorte de liberté avec le roi qu'il connaissait parfaitement, rompit les
premières glaces, et ne parla que de l'état malheureux de
M\textsuperscript{me} des Ursins, qu'une folie sans excuse (il voulait
parler de la lettre apostillée) avait précipitée dans l'abîme de
l'humiliation. Il exagéra sa douleur d'avoir déplu et de ne pouvoir être
écoutée après n'avoir été appliquée en Espagne qu'à y faire obéir le roi
et cherché en tout à lui plaire. À mesure qu'Harcourt, d'une part, et
M\textsuperscript{me} de Maintenon, de l'autre, avec qui il agissait de
concert, et à qui dans cette crise il donna d'utiles et fermes conseils,
il retournait à la charge. Le roi, dont la vérité n'approcha jamais dans
la clôture où il s'était emprisonné lui-même, fut le seul des deux
monarchies qui ne se douta du tout point que l'arrivée de
M\textsuperscript{me} des Ursins à sa cour fût le gage assuré de son
retour en Espagne et de celui d'une puissance plus grande que jamais.
Fatigué des contradictions qu'il y éprouvait, inquiet du désordre
dangereux qui en résultait aux affaires, dans un temps où leur
changement de face demandait un parfait unisson entre les deux
couronnes, lassé des instances qui lui étaient faites et des réflexions
qui lui étaient présentées, il accorda enfin la grâce qui lui était si
pressement demandée, dont les ministres se trouvèrent fort étourdis.

Harcourt profita de ce court intervalle. Il était irréconciliable avec
Torcy et avec le duc de Beauvilliers. Chamillart n'était son homme que
parce qu'il était celui de M\textsuperscript{me} de Maintenon. Il
n'aurait pas voulu moins se mêler de ses deux départements que de celui
de Torcy\,: ce n'était donc pas là où il pouvait compter de se réunir
réellement. L'esprit, le tour, la capacité du chancelier lui plaisaient.
La malignité et l'inquisition de Pontchartrain lui pouvaient être
utiles. Leur département n'avait rien qui pût le tenter ni leur en
donner ombrage\,; ils étaient ennemis déclarés de Chamillart, et le
chancelier mal avec Beauvilliers de tout temps et même avec peu de
mesure. Tout cela plaisait fort à Harcourt et lui donna le désir de se
réunir au père et au fils, avec qui il n'avait point eu d'occasion de
prises particulières. Cela pouvait lui servir pour les choses du
conseil, et ôter au roi l'idée fâcheuse qu'il ne pouvait vivre avec pas
un de ses ministres. Je fus surpris qu'il n'accueillît avec une
attention très marquée et suivie, qu'il entamât des propos avec moi pour
voir comment j'y prendrais cette recherche\,; je me tins en garde avec
un homme ennemi de ce que j'avais de plus intime et qui ne faisait rien
qu'avec des vues. Ma politesse ne lui suffit pas. L'affaire de
M\textsuperscript{me} des Ursins s'avançait dans les ténèbres. Il était
pressé de s'unir aux Pontchartrain\,; c'était sur moi qu'il avait jeté
les yeux pour la former. Il se dégoûta et tourna court sur le premier
écuyer, déjà de ses amis et qui, n'ayant pas mes raisons, devint bientôt
son homme et fit en un instant l'union qui leur convenait à tous.

Le chancelier, mal avec Beauvilliers, brouillé ouvertement avec
Chamillart, sans liaison avec Torcy, contre lequel son fils crevait de
jalousie, totalement déchu auprès de M\textsuperscript{me} de Maintenon,
avec peu d'affaires (rares et souvent plutôt embarrassantes pour lui
qu'agréables) directement avec le roi, et ne tenant plus à lui que par
l'habitude et par l'esprit et l'agrément, il fut ravi de se lier à un
homme tel qu'était Harcourt, et tel qu'il pouvait si naturellement
devenir, qui avait avec lui des aversions et des raisons communes
d'éloignement, avec qui d'ailleurs il ne pouvait entrer en compromis ni
en soupçon pour son ministère ni pour celui de son fils, lequel, abhorré
de tout le monde et de ses confrères même, ne faisait que prendre
haleine de la peur que le comte de Toulouse lui avait faite, et était
trop heureux de se pouvoir lier avec un homme aussi considérable que
l'était Harcourt au dehors, et plus encore en dedans, dont la protection
et les conseils lui pouvaient être d'un usage si utile. Mais, en faisant
cette union, Harcourt, qui tout en douceur donnait la loi, voulut à
découvert que lime des Ursins y fût comprise, et qu'il pût lui répondre
pour toujours à l'avenir de leur amitié et de leurs services.

Ce point fut gagné avec la même facilité, et toutes les grâces du
chancelier s'y déployèrent. C'était l'ennemie de ceux qu'il haïssait, ou
avec qui il vivait sans liaisons. Ni lui ni son fils n'étaient pas à
portée qu'on leur demandât de rompre des glaces. S'engager à vouloir du
bien à une personne éloignée sans moyen de la servir, était s'engager à
peu de chose\,; et si elle venait à reprendre le dessus elle leur
devenait une protection. L'union entre eux venait donc d'être conclue,
et Harcourt, le premier écuyer et les Pontchartrain s'étaient vus,
promis et convenus de leurs faits, précisément quelques jours devant que
le roi eût lâché la grande parole sur laquelle il fut dépêché un
courrier à Toulouse portant permission de venir quand elle voudrait à
Paris et à la cour. Quelque informée qu'elle fût de tout ce qui se
brassait pour elle, la joie surpassa l'espérance. Mais le coup d'œil de
son retour à la toute-puissance en Espagne, conséquent à cette
permission, ne la dérangea pas plus qu'avait fait la chute de la foudre
sur elle à Madrid\,: toujours maîtresse d'elle-même et attentive à tirer
tout le parti qu'elle pourrait de son admission à se justifier, elle
conserva l'air d'une disgraciée qui espère, mais qui est humiliée\,;
elle avait prévenu ses intimes amis de s'en tenir exactement à ce ton\,;
elle craignit surtout de laisser rien apercevoir au roi qui le fronçât
et qui le tint en garde\,; elle prit avec une grande présence d'esprit
ses mesures en Espagne\,; elle ne se précipita point de partir et partit
néanmoins assez promptement pour ne rien laisser refroidir et marquer
son empressement à profiter de la grâce qu'elle recevait et qu'elle
avait toujours tant souhaitée.

À peine le courrier fut-il parti vers elle, que le bruit de son retour
se répandit sourdement et devint public et confirmé peu de jours après.
Le mouvement qu'il produisit à la cour fut inconcevable\,; il n'y eut
que les amis intimes de M\textsuperscript{me} des Ursins qui demeurèrent
dans un état tranquille et modéré. Chacun ouvrit les yeux et comprit que
l'arrivée d'une personne si importante n'aurait rien d'indifférent. On
se prépara à une sorte de soleil levant, qui allait changer et
renouveler bien des choses dans la nature. On ne voyait que gens, à qui
on n'avait jamais ouï proférer son nom, qui se vantaient de son amitié
et qui exigeaient des compliments sur sa prochaine arrivée. On en
trouvait d'autres, liés avec ses ennemis, qui n'avaient pas honte de se
donner pour être transportés de joie et de prodiguer les bassesses à
ceux de qui ils se flattaient qu'elles seraient offertes en encens à la
princesse des Ursins. Parmi ces derniers, les Noailles se distinguèrent.
Leur union intime avec les Estrées, et par leur gendre favori avec le
duc de Grammont, ne les arrêta point\,: ils se publièrent ravis du
retour d'une personne qu'ils avaient, disaient-ils, dans tous les temps,
aimée et honorée, et qui était de leurs amies depuis toute leur vie. Ils
le voulurent persuader à ses meilleurs amis, à M\textsuperscript{me} de
Maintenon, à elle-même.

Elle arriva enfin à Paris le dimanche 4 janvier. Le duc d'Albe, qui
avait cru bien faire en s'attachant fortement aux Estrées, espéra laver
cette tache en lui prodiguant tous les honneurs qu'il put. Il alla en
cortège fort loin hors Paris, à sa rencontre avec la duchesse d'Albe, et
la mena coucher chez lui, où il lui donna une fête. Plusieurs personnes
de distinction allèrent plus ou moins loin à sa rencontre les Noailles
n'y manquèrent pas et les plus loin de tous. M\textsuperscript{me} des
Ursins eut lieu d'être surprise d'une entrée si triomphante\,: il lui
fallut capituler pour sortir de chez le duc d'Albe. Il lui importait de
se mettre en lieu de liberté. De préférence à la duchesse de Châtillon,
sa propre nièce, elle alla loger chez la comtesse d'Egmont qui ne
l'était qu'à la mode de Bretagne, mais nièce de l'archevêque d'Aix,
qu'elle avait eue autrefois longtemps chez elle avec la duchesse de
Châtillon, et qu'elle y avait mariées l'une et l'autre. Cette préférence
était bien due à la considération de l'archevêque d'Aix, qui, dans les
temps les plus orageux, n'avait trouvé rien de difficile pour son
service jusqu'à cet agréable moment. Le roi était à Marly, et nous
étions, M\textsuperscript{me} de Saint-Simon et moi, de ce voyage,
comme, depuis que Chamillart m'avait raccommodé, cela nous arrivait
souvent. Pendant le reste de ce Marly, ce fut un concours prodigieux
chez M\textsuperscript{me} des Ursins, qui, sous prétexte d'avoir besoin
de repos, ferma sa porte au commun, et ne sortit point de chez elle. M.
le Prince y courut des premiers, et, à son exemple, tout ce qu'il y eut
de plus grand et de moins connu d'elle. Quelque flatteur que fût ce
concours, elle n'en était pas si occupée qu'elle ne le fût beaucoup plus
de se mettre bien au fait de tout ce que les dépêches n'avaient pu
comporter, et de la carte présente. La curiosité, l'espérance, la
crainte, la mode, y attirait cette foule dont plus des trois quarts
n'entraient pas. Les ministres en furent alors effrayés. Torcy eut ordre
du roi de l'aller voir. Il en fut étourdi\,: il ne répliqua pas\,; en
homme qui vit la partie faite et le triomphe assuré, il obéit. La visite
se passa avec embarras de sa part, et une froideur haute de l'autre\,:
ce fut l'époque qui fit changer de ton à M\textsuperscript{me} des
Ursins. Jusque-là modeste, suppliante, presque honteuse, elle en vit et
en apprit tant, que, de répondante qu'elle s'était proposé d'être, elle
crut pouvoir devenir accusatrice et demander justice contre ceux qui,
abusant de la confiance du roi, lui avaient attiré un traitement si
fâcheux et si long, et mise en spectacle aux deux monarchies. Tout ce
qui lui arrivait passait de bien loin ses espérances\,; elle-même s'en
est étonnée avec moi plusieurs fois, et avec moi s'est moquée de force
gens, et souvent des plus considérables, ou qu'elle ne connaissait comme
point, ou qui lui avaient été fort contraires, et qui s'empressaient
bassement auprès d'elle.

Le roi revint à Versailles le samedi 10 janvier\,; M\textsuperscript{me}
des Ursins y arriva le même jour\,; elle logea à la ville chez d'Alègre.
J'allai aussitôt la voir, n'ayant pu quitter Marly à cause des bals de
presque tous les soirs. Ma mère l'avait fort vue à Paris, où
M\textsuperscript{me} de Saint-Simon et moi lui avions envoyé témoigner
notre joie et notre empressement de la voir. J'avais toujours conservé
du commerce avec elle, et j'en avais reçu en toute occasion des marques
d'amitié. Sandricourt, qui était de ma maison, et qui servait en
Espagne, duquel j'aurai un mot à dire en son temps, en avait reçu à ma
prière toutes sortes de distinctions, et elle l'avait fort recommandé
aux principaux chefs espagnols. Je fus très bien reçu. Cependant je
m'étais promis quelque chose de plus ouvert. J'y fus peu. Harcourt, qui
habilement ne l'avait pas encore vue, y arriva et me fit retirer
discrètement\,; elle m'arrêta pour me charger de quelques bagatelles
avec un air de liberté, et tout de suite reprenant toute son ouverture,
elle me dit qu'elle se promettait bien de me revoir bientôt et de causer
avec moi plus à son aise\,: j'en vis Harcourt surpris. Sortant de la
maison, j'y vis entrer Torcy. Il avait fait en sorte, dès Paris, par sa
mère, qu'elle irait souper chez lui. Elle était contente de l'avoir
humilié, {[}et qu'il fût{]} venu chez elle par ordre du roi. Il n'était
pas temps de faire des éclats et contre un ministre\,: elle n'avait
encore vu ni le roi ni M\textsuperscript{me} de Maintenon, et ce qui se
passerait avec eux devait être la boussole de sa conduite. Le lendemain
dimanche, huitième jour de son arrivée à Paris, elle dîna seule chez
elle, se mit en grand habit, et s'en alla chez le roi, avec lequel elle
fut dans son cabinet deux heures et demie tête à tête\,; de là chez
M\textsuperscript{me} la duchesse de Bourgogne, avec qui elle fut aussi
assez longtemps seule dans son cabinet. Le roi dit le soir, chez
M\textsuperscript{me} de Maintenon, qu'il y avait encore bien des choses
dont il n'avait point encore parlé à M\textsuperscript{me} des Ursins.
Le lendemain elle vit M\textsuperscript{me} de Maintenon en particulier
fort longtemps et fort à son aise. Le mardi elle y retourna et y fut
très longtemps en tiers entre elle et le roi\,; le mercredi,
M\textsuperscript{me} la duchesse de Bourgogne, qui avait dîné et joué
chez M\textsuperscript{me} de Mailly, y fit venir la princesse des
Ursins à la fin du jeu, passa seule avec elle dans un cabinet et y
demeurèrent très longtemps.

Un mois après arriva un colonel dans les troupes d'Espagne, Italien
appelé Pozzobuono, dépêché exprès et uniquement par le roi et la reine
pour venir apporter leurs remerciements au roi sur la princesse des
Ursins, et ordre au duc d'Albe d'aller avec tout son cortège lui faire
une visite de cérémonie, comme la première fois qu'il fut chez les
princesses du sang. De ce moment il fut déclaré qu'elle demeurerait ici
jusqu'au mois d'avril pour donner ordre à ses affaires et à sa santé.
C'était déjà un grand pas que d'être maîtresse d'annoncer ainsi son
séjour. Personne, à la vérité, ne doutait de son retour en Espagne, mais
la parole n'en était pas léchée\,; elle évitait de s'en expliquer, et on
peut juger qu'elle n'essuya pas là-dessus de questions indiscrètes. Elle
se mesura fort à voir Monseigneur, Madame, Monsieur et
M\textsuperscript{me} la duchesse d'Orléans et les princesses du sang\,;
donna plusieurs jours au flot du monde, puis se renferma sous prétexte
d'affaires, de santé, d'être sortie, et tant qu'elle put ne vit à Paris
que ses amis ou ses plus familières connaissances, et les gens que par
leur place elle ne pouvait refuser.

Tant d'audiences et si longues, suivies de tant de sérénité et de foule,
fit un grand effet dans le monde, et augmenta fort les empressements.
Deux jours après ma première visite à Versailles, je retournai chez
elle, je lui retrouvai avec moi son ancienne ouverture avec laquelle
elle me fit quelques reproches d'avoir été plus intimement de ses amis
avant ses affaires que depuis. Cela ne servit qu'à nous réchauffer dans
la conversation même, où elle s'ouvrit et me parut avoir envie de me
parler. Je ne laissai pas d'être en garde par rapport à M. de
Beauvilliers\,; je savais le raccommodement du chancelier, je ne la
craignais pas sur Chamillart, et je ne me souciais point de Torcy, avec
qui je n'étais en aucune liaison. Elle ne me fit point d'embarras, elle
savait trop la carte de la cour pour ignorer mon intimité avec M. de
Beauvilliers\,; et sa politesse, et je puis dire son amitié, car elle
m'en donna des marques dans tout son séjour, m'épargna là-dessus toute
délicatesse. Le nonce nous interrompit. Mais je la revis bientôt, et
elle me parla de mille choses et d'ici et d'Espagne avec confiance, et
de la cour, et d'autres qui la regardaient. Elle fit à
M\textsuperscript{me} de Saint-Simon toutes sortes d'amitiés et
d'avances, et on verra bientôt que cela ouvrit fort les yeux de toute la
cour sur nous. Laissons-la triompher et besogner à son aise, et
retournons en arrière, dont ce long et curieux récit nous a distrait.
Mais il ne faut pas oublier que cette réception du roi à
M\textsuperscript{me} des Ursins, au retour de laquelle Tessé s'était
tant livré, plut tellement au roi et à la reine d'Espagne, qu'ils
donnèrent à Tessé toutes sortes de pouvoirs et de distinctions
militaires, de confiance et de faveur personnelle, et à son gendre
toutes celles de leur cour.

\hypertarget{chapitre-xxi.}{%
\chapter{CHAPITRE XXI.}\label{chapitre-xxi.}}

1705

~

{\textsc{Pension du roi à M\textsuperscript{me} de Caylus, à condition
de quitter le P. de La Tour.}} {\textsc{- Caractère de ce père.}}
{\textsc{- Mort de Pavillon.}} {\textsc{- Brevets de retenue à Livry et
au comte d'Évreux.}} {\textsc{- Duc de Tresmes reçu à l'hôtel de
ville.}} {\textsc{- Mariage de Rupelmonde avec une fille d'Alègre.}}
{\textsc{- Caractère et audace de M\textsuperscript{me} de Rupelmonde\,;
extraction de son mari, etc.}} {\textsc{- Duc d'Aumont gagne contre le
duc d'Elbœuf une affaire piquante.}} {\textsc{- Petits exploits de La
Feuillade.}} {\textsc{- Mort de l'électrice de Brandebourg.}} {\textsc{-
Mort de Courtebonne.}} {\textsc{- Filles de Saint-Cyr.}} {\textsc{-
Mariage de M\textsuperscript{lle} d'Osmont avec Avrincourt.}} {\textsc{-
Mort de Tressan, évêque du Mans.}} {\textsc{- Tracasserie entre
Saint-Pierre et Nancré pour les Suisses de M. le duc d'Orléans.}}
{\textsc{- Brevet de retenue à Grignan.}} {\textsc{- Mariage du
chevalier de Grignan avec M\textsuperscript{lle} d'Oraison.}} {\textsc{-
Mariage de Montal avec la sœur de Villacerf, et d'Épinay avec une fille
d'O.}} {\textsc{- Rivas chassé\,; Mejorada en sa place.}} {\textsc{-
Ronquillo.}} {\textsc{- Dégoûts à Madrid du duc de Grammont, qui demande
son rappel et à la Toison.}} {\textsc{- Triomphe éclatant et solide de
la princesse des Ursins, assurée de retourner en Espagne.}} {\textsc{-
Amitié de la princesse des Ursins pour M\textsuperscript{me} de
Saint-Simon et pour moi, et ses bons offices.}} {\textsc{- Duc et
duchesse d'Albe à un bal à Marly\,; singularités.}} {\textsc{- Amelot
ambassadeur en Espagne\,; son caractère.}} {\textsc{- Orry retourne en
Espagne.}} {\textsc{- Bourg, son caractère, ses aventures, sa chétive
fortune.}} {\textsc{- Melford rappelé à Saint-Germain et déclaré duc.}}
{\textsc{- Middleton se fait catholique.}} {\textsc{- Mort de
M\textsuperscript{me} du Plessis-Bellière.}} {\textsc{- Mort, caractère
et fortune de Magalotti.}} {\textsc{- Albergotti et son caractère.}}
{\textsc{- Mort du duc de Choiseul, qui éteint son duché-pairie.}}
{\textsc{- Mort du président de Maisons.}} {\textsc{- Mort de
M\textsuperscript{lle} de Beaufremont.}} {\textsc{- Mort de Seissac.}}
{\textsc{- Mort et deuil du duc Maximilien de Bavière.}} {\textsc{- Mort
de Beuvron.}} {\textsc{- Mort du petit duc de Bretagne\,; son deuil.}}
{\textsc{- Longue goutte du roi\,; son coucher retranché au public pour
toujours.}} {\textsc{- Mort de Rubantel.}} {\textsc{- Mort de
Breteuil\,; Armenonville conseiller d'État.}} {\textsc{- Mort du fils
unique d'Alègre.}} {\textsc{- Angervilliers intendant du Dauphiné et des
armées.}} {\textsc{- Bouchu\,; son caractère\,; singularité de ses
dernières années.}}

~

Quelque occupée que pût être M\textsuperscript{me} de Maintenon du
retour et de la réception de M\textsuperscript{me} des Ursins, rien ne
la put distraire de la maladie antijanséniste. M\textsuperscript{me} de
Caylus avait mis son exil à profit. Elle était retournée à Dieu de bonne
foi\,; elle s'était mise entre les mains du P. de La Tour, qui fut
ensuite, s'il ne l'était déjà, général des pères de l'Oratoire. Ce P. de
La Tour était un grand homme, bien fait, d'un visage agréable, mais
imposant, fort connu par son esprit liant mais ferme, adroit mais fort,
par ses sermons, par ses directions. Il passait, ainsi que la plupart de
ceux de sa congrégation, pour être janséniste, c'est-à-dire réguliers,
exacts, étroits dans leur conduite, studieux, pénitents, haïs de
Saint-Sulpice et des jésuites, et par conséquent nullement liés avec
eux\,; enviés des uns dans leur ignorance, et des, autres par la
jalousie du peu de collèges et de séminaires qu'ils gouvernaient, et du
grand nombre d'amis et illustres qui les leur préféraient. Depuis que le
P. de La Tour conduisait M\textsuperscript{me} de Caylus, la prière
continuelle et les bonnes œuvres partagèrent tout son temps, et ne lui
en laissèrent plus pour aucune société\,; le jeûne était son exercice
ordinaire, et depuis l'office du jeudi saint jusqu'à la fin de celui du
samedi, elle ne sortait point de Saint-Sulpice\,; avec cela toujours
gaie, mais mesurée et ne voyant plus que des personnes tout à fait dans
la piété, et même assez rarement. Dieu répandait tant de grâces sur
elle, que cette femme si mondaine, si faite aussi pour les plaisirs et
pour faire la joie du monde, ne regretta jamais dans ce long espace que
de ne l'avoir pas quitté plus tôt, et ne s'ennuya jamais un moment dans
une vie si dure, si unie, qui n'était qu'un enchaînement sans intervalle
de prières et de pénitences. Un si heureux état fut troublé par
l'ignorance et la folie du zèle de sa tante, pour se taire sur plus
haut\,; elle lui manda que le roi ni elle ne se pouvaient accommoder
plus longtemps de sa direction du P. de La Tour\,; que c'était un
janséniste qui la perdait\,; qu'il y avait dans Paris d'autres personnes
doctes et pieuses dont les sentiments n'étaient point suspects\,; qu'on
lui laissait le choix de tous ceux-là\,; que c'était pour son bien et
pour son salut que cette complaisance était exigée d'elle\,; que c'était
une obéissance qu'elle ne pouvait refuser au roi\,; qu'elle était pauvre
depuis la mort de son mari\,; enfin que, si elle se conformait de bonne
grâce à cette volonté, sa pension de six mille livres serait augmentée
jusqu'à dix.

M\textsuperscript{me} de Caylus eut grand'peine à se résoudre\,; la
crainte d'être tourmentée prit sur elle plus que les promesses\,; elle
quitta le P. de La Tour, prit un confesseur au gré de la cour, et
bientôt ne fut plus la même\,; la prière l'ennuya, les bonnes œuvres la
lassèrent, la solitude lui devint insupportable\,; comme elle avait
conservé les mêmes agréments dans l'esprit, elle trouva aisément des
sociétés plus amusantes, parmi lesquelles elle redevint bientôt tout ce
qu'elle avait été. Elle renoua avec le duc de Villeroy pour lequel elle
avait été chassée de la cour. On verra bientôt que cet inconvénient ne
parut rien aux yeux du roi et de M\textsuperscript{me} de Maintenon, en
comparaison de celui de se sanctifier sous la conduite d'un janséniste.
Le P. de La Tour, qui excellait par un esprit de sagesse, de conduite et
de gouvernement, était guetté avec une application à laquelle rien
n'échappait, sans qu'il fît jamais un faux pas. Le roi qui, poussé par
les jésuites et Saint-Sulpice, lui cherchait noise de tout son cœur,
s'est plusieurs fois écrié avec dépit, mais avec admiration, sur la
sagesse de cet homme, avouant que depuis fort longtemps qu'il l'épiait,
il n'avait jamais pu le trouver en faute. Sa conversation était gaie,
souvent salée, amusante, mais sans sortir du caractère qu'il portait.
C'était un homme imposant et dans la plus grande considération\,; avec
tout cela ses lumières le trompèrent à la fin, et on le verra dans la
suite tomber dans un terrible panneau, où son autorité, croyant éviter
un grand mal, entraîna le cardinal de Noailles et le chancelier
d'Aguesseau, et eut de funestes suites. Le P. de La Tour était
gentilhomme de bon lieu, d'auprès d'Eu, et avait été page de
Mademoiselle.

Pavillon, neveu du célèbre évêque de Pamiers, si connu dans les affaires
du jansénisme et de la régale, mourut vieux à Paris, où il était de
l'Académie des sciences et des inscriptions, assez pauvre et point
marié. C'était un homme infirme, de beaucoup d'esprit et fort agréable,
qui avait toujours chez lui une compagnie choisie, mais excellente, où
allaient même des gens considérables, un fort honnête homme, et qui fut
fort regretté.

Livry eut en ce même temps quatre cent mille livres de brevet de retenue
sur sa charge, et le comte d'Évreux bientôt après une augmentation de
cent mille livres du sien, qui était déjà de trois cent cinquante mille
livres.

Le duc de Tresmes fut reçu en grande pompe à l'hôtel de ville, comme
gouverneur de Paris\,; il y fut harangué par le prévôt des marchands,
qui le traita toujours de \emph{Monseigneur}. M. de Montbazon et les
gouverneurs de Paris qui l'avaient précédé, avaient eu ce traitement,
qui s'était perdu ensuite. Le duc de Créqui le fit rétablir, et les ducs
de Gesvres et de Tresmes en profitèrent. La ville lui donna le même jour
un grand festin, où il mena quantité de gens de la cour et de Paris, qui
furent placés, à la droite d'une table longue, dans trente fauteuils\,;
vis-à-vis, sur trente chaises à dos, furent les échevins, les
conseillers de ville et les conviés du prévôt des marchands, qui était
seul avec le duc de Tresmes\,; et à sa gauche, au haut bout de la table,
dans deux fauteuils, le prévôt des marchands et tous les officiers de la
ville en habit de cérémonie. On parla fort de la magnificence du repas,
qui fut en poisson, parce que c'était un samedi 24 janvier. Le duc de
Tresmes jeta de l'argent au peuple en entrant et en sortant de l'hôtel
de ville.

M\textsuperscript{me} d'Alègre maria en ce même mois sa fille à
Rupelmonde, Flamand et colonel dans les troupes d'Espagne, pendant que
son mari était employé sur la frontière\,; elle s'en défit à bon marché,
et le duc d'Albe en fit la noce. Elle donna son gendre pour un grand
seigneur, et fort riche, à qui elle fit arborer un manteau ducal. Sa
fille, rousse comme une vache, avec de l'esprit et de l'intrigue, mais
avec une effronterie sans pareille, se fourra à la cour, où avec les
sobriquets de \emph{la blonde} et de \emph{vaque-à-tout}, parce qu'elle
était de toutes foires et marchés, elle s'initia dans beaucoup de
choses, fort peu contrainte par la vertu et jouant le plus gros jeu du
monde. Ancrée suffisamment, à ce qu'il lui sembla, non contente de son
manteau ducal postiche, elle hasarda la housse sur sa chaise à porteurs.
Le manteau, quoique nouvellement, c'est-à-dire depuis vingt ou
vingt-cinq ans, se souffrait à plusieurs gens, qui n'en tiraient aucun
avantage, mais pour la housse, personne n'avait encore jamais osé en
prendre sans droit. Celle-ci fit grand bruit, mais ne dura que
vingt-quatre heures. Le roi la lui fit quitter avec une réprimande très
forte.

Le roi, lassé des lettres de M\textsuperscript{me} d'Alègre, qui tantôt
pour Marly, tantôt pour une place de dame du palais, exaltait sans cesse
les grandeurs de son gendre, chargea Torcy de savoir par preuves qui
était ce M. de Rupelmonde. Les informations lui arrivèrent prouvées en
bonne forme, qui démontrèrent que le père de ce gendre de
M\textsuperscript{me} d'Alègre, après avoir travaillé de sa main aux
forges de la véritable dame de Rupelmonde, en était devenu facteur, puis
maître, s'y était enrichi, en avait ruiné les possesseurs, et était
devenu seigneur de leurs biens et de leurs terres en leur place. Torcy
me l'a conté longtemps depuis en propres termes. Mais l'avis était venu
trop tard, et avait trouvé M\textsuperscript{me} de Rupelmonde admise à
tout ce que le sont les femmes de qualité. Le roi ne voulut pas faire un
éclat.

Jamais je ne vis homme si triste que ce Rupelmonde ni qui ressemblât
plus à un garçon apothicaire. Je me souviens qu'un soir que nous étions
à Marly, et qu'au sortir du cabinet du roi M\textsuperscript{me} la
duchesse de Bourgogne s'était remise au lansquenet, où était
M\textsuperscript{me} de Rupelmonde qui y coupait, un suisse du salon
entra quelques pas et cria fort haut\,: «\,Madame Ripilmande, allez
coucher\,; votre mari est au lit qui envoie vous demander.\,» L'éclat de
rire fut universel. Le mari, en effet, avait envoyé chercher sa femme,
et le valet, comme un sot, avait dit au suisse la commission, au lieu de
demander à parler à M\textsuperscript{me} de Rupelmonde, et la faire
appeler à la porte du salon. Elle ne voulait point quitter le jeu,
moitié honteuse, moitié effrontée\,; mais M\textsuperscript{me} la
duchesse de Bourgogne la fit sortir. Le mari fut tué bientôt après. Le
deuil fini, la Rupelmonde intrigua plus que jamais, et à force d'audace
et d'insolence, de commodités et d'amourettes, parvint longtemps depuis
à être dame du palais de la reine à son mariage, et par une longue et
publique habitude avec le comte depuis duc de Grammont, à faire le
mariage de son fils unique avec sa fille rousse et cruellement laide,
sans un sou de dot.

Les ducs d'Elbœuf, père et fils, gouverneurs de Picardie, avaient une
dispute avec le maréchal et les ducs d'Aumont, gouverneurs de Boulogne
et de Boulonnais, qui était devenue fort aigre, et qui avait été plus
d'une fois sur le point de leur faire mettre l'épée à la main l'un
contre l'autre. M. d'Elbœuf disait que Boulogne et le Boulonnais étaient
du gouvernement de Picardie, et le prouvait, parce qu'il était en usage
de présenter au roi les clefs de Boulogne quand il y était venu, et d'y
donner l'ordre, M. d'Aumont présent\,; mais il prétendait de là mettre
son attache aux provisions de gouverneur de Boulogne et du Boulonnais,
et c'est ce que MM. d'Aumont lui contestaient. Le roi enfin jugea cette
affaire en ces temps-ci, et M. d'Aumont la gagna de toutes les voix du
conseil de dépêches.

La Feuillade, arrivé au commencement de janvier, présenté par
Chamillart, et reçu en conquérant, ne dédaigna pas de danser à Marly
avec nous. Il avait laissé sa petite armée en Savoie, dans les vallées
voisines, et au blocus de Montméliant. Le voyage fut court et
brillant\,; un mois après il travailla avec le roi et Chamillart chez
lime de Maintenon, comme les généraux d'armée, prit congé et s'en
retourna. Il ne tarda pas à marcher à Nice et à Villefranche, et détacha
Gévaudan pour s'emparer de Pignerol tout ouvert. Le marquis de Roye,
lieutenant général des galères, les mena devant Villefranche avec des
vaisseaux chargés de munitions\,; elle fut bientôt prise l'épée à la
main. Il fut de là à Nice, où il ouvrit la tranchée le 17 mars, et
cependant le château de Villefranche se rendit aux\,: troupes qu'il y
avait laissées. Nice se rendit le 17 avril, et la garnison se retira au
château, qu'on ne songea pas à attaquer, entre lequel et la ville on fit
une trêve indéfinie, à laquelle M. de Savoie consentit.

L'électrice de Brandebourg mourut au commencement de février. Elle était
sœur du duc d'Hanovre, fait neuvième électeur, et qui depuis a succédé à
la reine Anne à la couronne d'Angleterre. Cette princesse mérite d'être
remarquée pour n'avoir jamais approuvé que l'électeur son mari prît le
titre de roi de Prusse. On n'en prit point le deuil, parce qu'il n'y
avait point de parenté avec le roi.

Villars, après avoir travaillé avec le roi, prit congé de lui les
premiers jours de février. Il revint un mois après\,; il avait été faire
un tour sur la Moselle\,; quinze jours après il s'en alla à Metz en
attendant qu'il pût assembler son armée.

Marsin arriva d'Alsace, et Arco de Flandre, pour y retourner bientôt.

Courtebonne, lieutenant général, mourut. Il était excellent officier et
gouverneur d'Hesdin, frère de la femme de Breteuil, conseiller d'État,
mère de Breteuil que nous verrons deux fois secrétaire d'État de la
guerre. Le roi se servit de ce gouvernement pour faire plaisir à
M\textsuperscript{me} de Maintenon. Elle trayait d'ordinaire une
demoiselle ou deux de Saint-Cyr des plus prêtes à en sortir, pour se les
attacher, écrire ses lettres et la suivre partout. Le roi, qui les
voyait là sans cesse, prenait souvent de la bonté pour elles et les
mariait. M\textsuperscript{lle} d'Osmont se trouva dans ce cas-là, avec
plus d'esprit et d'agrément que la plupart des autres. On lui trouva un
parti, d'Avrincourt, qui avait quelque peu servi de colonel de dragons
en Italie. Il avait du bien en Artois\,; Hesdin lui convenait, il en
donna vingt-cinq mille écus aux enfants de Courtebonne, et on lui donna
cent mille livres sur l'hôtel de ville. Ce fut un homme d'esprit et
adroit qui, au lieu de se laisser étranger et sa femme, sut plaire et en
tirer les meilleurs partis, moyennant quoi il s'enrichit extrêmement, et
trouva moyen, même longtemps depuis la mort du roi, d'avoir un régiment
royal de cavalerie, et son gouvernement pour son fils.
M\textsuperscript{me} la duchesse de Bourgogne s'amusa fort de cette
noce, et donna la chemise pour se divertir et faire sa cour à
M\textsuperscript{me} de Maintenon.

Il mourut en même temps un autre homme qui avait fait bien des manèges
en sa vie, qui avait succédé à l'archevêque d'Aix dans la charge de
premier aumônier de Monsieur c'était Tressan, qui ne put aller plus loin
que l'évêché du Mans, et qui enfin, de guerre lasse, s'y confina et
vendit sa charge à l'abbé de Grancey.

Cela me fait souvenir d'une tracasserie qui arriva lors entre M. et
M\textsuperscript{me} la duchesse d'Orléans. Saint-Pierre, qui avait
beaucoup d'esprit et de l'intrigue, et qui, très bon marin, avait été
cassé pour n'avoir pas voulu prendre du petit Renault les leçons
publiques de marine que le roi avait ordonnées, avait amené sa femme de
Brest, plus intrigante encore que lui et fort vive. Elle avait été jolie
quoique jeune encore, et avait été fort sur le trottoir à Brest, d'où
elle était. Je ne sais qui la produisit à M\textsuperscript{me} la
duchesse d'Orléans. Elle devint sa favorite, s'établit partout à sa
suite, quoique sans emploi chez elle, et vécut comme à Brest. Elle avait
de l'esprit, de la gaieté, de la douceur. Elle plut et s'insinua fort
avec le monde sous la protection de la princesse.

Saint-Pierre était un homme froid, se piquant de lecture, de philosophie
et de sagesse. À la dévotion près, et dans le bas étage, c'était un
ménage tout comme celui de M. et de M\textsuperscript{me} d'O, de chez
qui aussi ils ne bougeaient. M. le duc d'Orléans n'en faisait pas grand
cas, et ne trouvait ni l'importance du mari à son gré, ni le fringant et
le petit état de la femme propre à figurer favorite de
M\textsuperscript{me} la duchesse d'Orléans. Ils voulaient une place à
se fourrer, à quelque prix que ce fût, qui leur donnât quelque
consistance. Liscoët mourut qui avait les Suisses de M. le duc
d'Orléans, et la place est lucrative. Saint-Pierre et sa femme se mirent
après. M\textsuperscript{me} la duchesse d'Orléans prétendit que M. le
duc d'Orléans la lui avait promise. Nancré, qui était Dreux comme le
gendre de Chamillart, était un garçon de beaucoup d'esprit, d'agrément
et fort orné\,; il avait quitté le service, lassé d'être
lieutenant-colonel, où il avait percé par ancienneté. Son père était
mort lieutenant général et gouverneur de\ldots., qui en secondes noces
avait épousé une fille de La Bazinière, sœur de la mère du président de
Mesmes, mort premier président, et intimement avec lui et avec son
beau-fils. Celui-ci s'était trouvé dans des parties de M. le duc
d'Orléans à Paris. Il était appuyé auprès de lui de l'abbé Dubois et de
Canillac, qui lui firent donner la charge. Voilà la Saint-Pierre aux
grands pleurs, son mari aux grands airs de dédain, et à dire que c'était
l'affaire de M\textsuperscript{me} la duchesse d'Orléans, qui s'en
brouilla avec M. le duc d'Orléans. Jamais elle ne l'a pardonné à
Nancré\,; jamais, ce qui est bouffon à dire, Saint-Pierre ne l'a
pardonné à M. le duc d'Orléans, quoiqu'il ait eu mieux dans la suite, et
à peine en aucun temps a-t-il pris la peine de mettre le pied chez lui.
Ce détail de Palais-Royal semble maintenant fort fade et fort peu ici en
sa place. Les suites feront voir qu'il ne devait pas être omis. Le rare
est que Saint-Pierre arracha, sans se donner la peine de s'en remuer,
quatre mille livres d'augmentation de pension d'une de six mille livres
que M\textsuperscript{me} la duchesse d'Orléans lui avait déjà obtenue,
et que M. le duc d'Orléans n'en fut pas mieux dans ses bonnes grâces.

À propos de grâces pécuniaires, Grignan, fort endetté à commander en
Provence, obtint deux cent mille livres de brevet de retenue sur sa
lieutenance générale de cette province. Lui et sa femme, se voyant sans
garçons, tourmentèrent tant le chevalier de Grignan, qu'ils lui firent
épouser M\textsuperscript{lle} d'Oraison. C'était un homme fort sage, de
beaucoup d'amis, très considéré, avec beaucoup d'esprit et du savoir.
Une goutte presque sans relâche lui fit quitter le service où il s'était
distingué, et la cour où il aurait figuré même sans place. Il était
menin de Monseigneur, des premiers qui furent faits. Il était retiré
depuis longtemps en Provence, d'où il ne sortit plus. Ce mariage fut
fort inutile, il n'en vint aucun enfant. Mais ils n'avaient pas à
craindre l'extinction de leur maison tant il subsistait encore de
branches de Castellane.

En même temps, le petit-fils de Montal, mort chevalier de l'ordre, et
qui aurait mieux été maréchal de France, épousa une sœur de Villacerf,
premier maître d'hôtel de M\textsuperscript{me} la duchesse de
Bourgogne, et M. d'O maria sa fille aînée à M. d'Épinay assez pauvre.

M\textsuperscript{me} des Ursins, triomphante à Paris fort au-dessus de
ses espérances, faisait en même temps bien des choses en Espagne. Rivas,
autrefois Ubilla, secrétaire des dépêches universelles, célèbre pour
avoir dressé le testament de Charles II, fut chassé\,; il ne s'en releva
jamais, et Mejorada fut mis en sa place. Le père de ce dernier l'avait
eue avant Rivas. Il consentit à détacher pour Ronquillo le département
de la guerre, que celui-ci refusa\,: ce dernier était corrégidor de
Madrid, avec grande réputation. Il voulait une plus haute fortune, et il
parvint en effet quelque temps après à être gouverneur du conseil de
Castille. D'un autre côté, le duc de Grammont était accablé de dégoûts.
Poussé à bout sur toutes les affaires, qui ne réussissaient que
lorsqu'il ne s'en mêlait pas, il demanda une audience à la reine,
quoique le roi fût à Madrid, dans l'espérance de réussir par elle. Il
l'obtint, lui exposa diverses choses importantes et pressées, par
rapport au siège de Gibraltar. La reine l'écouta paisiblement, puis,
avec un sourire amer, lui demanda s'il convenait à une femme de se mêler
d'affaires, et lui tourna le dos. M\textsuperscript{me} des Ursins qui,
à cause de M\textsuperscript{me} de Maintenon, ménageait les Noailles,
ne voulait pas elle-même demander son rappel. Mais, outre qu'elle ne lui
pardonnait point les choses passées, il lui était important d'avoir un
ambassadeur dont elle pût disposer. Il fallait réduire celui qui l'était
à demander son rappel lui-même, et c'est à la fin ce qui arriva. Les
Noailles, qui faisaient tout, comme on a vu, pour son fils, leur gendre,
ne se soudoient point de lui\,; mais, par honneur pour eux-mêmes, ils
désiraient au moins qu'il fût honnêtement congédié. C'est ce que la
maréchale de Noailles négocia avec la princesse des Ursins, qui lui fit
valoir la Toison qu'elle demandait comme le comble de la considération
du roi et de la reine pour eux, et tout l'effort de son amitié et de son
crédit. Elle en fit sa cour à M\textsuperscript{me} de Maintenon, pour
lui témoigner combien tout ce qui approchait de son alliance l'emportait
sur les raisons les plus personnelles, et lui en faire valoir le
sacrifice particulier que la reine d'Espagne lui faisait de tout son
mécontentement. Cette grâce fut donc assurée, mais seulement conférée
peu avant le départ du duc de Grammont.

On retourna à Marly, où il y eut force bals. On peut croire que
M\textsuperscript{me} des Ursins fut de ce voyage. Son logement fut à la
Perspective\,; rien de pareil à l'air de triomphe qu'elle y prit, à
l'attention continuelle en tout qu'eut le roi à lui faire les honneurs,
comme à un diminutif de reine étrangère à sa première arrivée, et à la
majestueuse façon aussi dont tout était reçu avec une proportion de
grâce et de respectueuse politesse dès lors fort effacée, et qui faisait
souvenir les vieux courtisans de la cour de la reine mère. Jamais elle
ne paraissait que le roi ne se montrât tout occupé d'elle, de
l'entretenir, de lui faire remarquer les choses, de rechercher son goût
et son approbation, avec un air de galanterie, même de flatterie, qui ne
faiblit point. Les fréquents particuliers qu'elle avait avec lui chez
M\textsuperscript{me} de Maintenon, et qui duraient des heures et
quelquefois le double, ceux qu'elle avait les matins fort souvent avec
M\textsuperscript{me} de Maintenon seule, la rendirent la divinité de la
cour. Les princesses l'environnaient dès qu'elle se montrait quelque
part, et l'allaient voir dans sa chambre. Rien de plus surprenant que
l'empressement servile qu'avait auprès d'elle tout ce qu'il y avait de
plus grand, de plus en place, de plus en faveur. Jusqu'à ses regards
étaient comptés\,; et ses paroles, adressées aux dames les plus
considérables, leur imprimaient un air de ravissement.

J'allais presque tous les matins chez elle\,: elle se levait toujours de
très bonne heure, et s'habillait et se coiffait tout de suite, en sorte
que sa toilette ne se voyait jamais. Je prévenais l'heure des visites
importantes, et nous causions avec la même liberté qu'autrefois. Je sus
par elle beaucoup de détails d'affaires, et la façon de penser du roi,
de M\textsuperscript{me} de Maintenon surtout, sur beaucoup de gens.
Nous riions souvent ensemble de la bassesse qu'elle éprouvait des
personnes les plus considérées, et du mépris qu'elles s'en attiraient
sans qu'elle le leur témoignât, et de la fausseté d'autres fort
considérables qui, après lui avoir fait, et nouvellement à son arrivée,
du pis qu'elles avaient pu, lui prodiguaient les protestations, et
tâchaient à lui vanter leur attachement dans tous les temps, et à faire
valoir leurs services. J'étais flatté de cette confiance de la
dictatrice de la cour. On y fit une attention qui m'attira une
considération subite, outre que force gens des plus distingués me
trouvaient les matins seul avec elle, et que les messages qui lui
pleuvaient rapportaient qu'ils m'y avaient trouvé, et très ordinairement
qu'ils n'avaient pu parler à elle. Elle m'appelait souvent dans le
salon, ou d'autres fois j'allais lui dire un mot à l'oreille, avec un
air d'aisance et de liberté fort envié et fort peu imité. Elle ne
trouvait jamais M\textsuperscript{me} de Saint-Simon sans aller à elle,
la louer, la mettre dans la conversation de ce qui était autour d'elle,
souvent de la mener devant une glace, et de raccommoder sa coiffure ou
quelque chose de son habit, comme en particulier elle aurait pu faire à
sa fille\,; assez souvent elle la tirait de la compagnie, et causait bas
à part longtemps avec elle, toujours quelques mots bas de l'une à
l'autre, et d'autres haut, mais qui ne se comprenaient pas. On se
demandait avec surprise, et beaucoup avec envie, d'où venait une si
grande amitié, dont personne ne s'était douté\,; et ce qui achevait de
tourmenter la plupart, c'est que M\textsuperscript{me} des Ursins,
sortant de la chambre de M\textsuperscript{me} de Maintenon, d'avec le
roi et elle, ne manquait guère d'aller à M\textsuperscript{me} de
Saint-Simon, si elle la trouvait dans le premier cabinet où elle avait
la liberté d'entrer avec quelques autres dames privilégiées, et la mener
en un coin et de lui parler bas. D'autres fois la trouvant dans le
salon, sortant de ces particuliers, elle en usait de même. Cela faisait
ouvrir les yeux à tout le monde, et lui attirait force civilités.

Ce qu'il y eut de plus solide fut tout le bien qu'elle dit d'elle au roi
et à M\textsuperscript{me} de Maintenon, à plusieurs reprises\,; et nous
avons su, par des voies sûres et tout à fait éloignées de
M\textsuperscript{me} des Ursins, qu'il n'y avait sortes de bons offices
qu'elle ne lui eût rendus, sans jamais les lui avoir demandés, et
souvent, et avec art et dessein, et qu'elle avait dit au roi et à
M\textsuperscript{me} de Maintenon plus d'une fois qu'ils n'avaient
aucune femme à la cour, et de tout âge, si propre, ni si faite exprès en
vertu, en conduite, en sagesse, pour être dame du palais, et dès lors
même, quoique si jeune, dame d'honneur de M\textsuperscript{me} la
duchesse de Bourgogne, si la place venait à vaquer, ni qui s'en
acquittât avec plus de sens, de dignité, ni plus à leur gré et à celui
de tout le monde. Elle en parla de même à M\textsuperscript{me} la
duchesse de Bourgogne plusieurs fois, et ne lui déplut pas, parce que
dès lors aussi cette princesse avait jeté ses vues sur elle, si la
duchesse du Lude, qui la survécut, venait à manquer. Je suis persuadé
que, outre la bonne opinion qu'avec toute la cour le roi et
M\textsuperscript{me} de Maintenon en avaient déjà, ces témoignages de
M\textsuperscript{me} des Ursins, dans la confiance qu'ils avaient prise
en elle, leur firent l'impression dont toujours depuis les effets se
sont fait sentir, et à la fin, comme on le verra en son temps, beaucoup
plus que nous n'aurions voulu. M\textsuperscript{me} des Ursins ne
m'oublia pas non plus\,; mais une femme était plus susceptible de son
témoignage, et faisait aussi plus d'impression. Cette façon d'être avec
nous et pour nous ne se ralentit point jusqu'à son départ pour
l'Espagne.

Entre plusieurs bals où M\textsuperscript{me} des Ursins fut toujours
traitée avec les mêmes distinctions, je veux dire un mot de celui où
M\textsuperscript{me} des Ursins obtint avec quelque peine que le duc et
la duchesse d'Albe fussent conviés. Je dis avec peine, parce qu'aucun
ambassadeur, ni étranger, n'avait jamais été admis à Marly, excepté
Vernon une fois, lors du mariage de M\textsuperscript{me} la duchesse de
Bourgogne, pour faire cette distinction à M. de Savoie dont il était
envoyé, et dans les suites les ambassadeurs d'Espagne.

La séance du bal dans le salon était un carré long fort vaste. Au haut
bout, c'est-à-dire du côté du salon qui séparait l'appartement du roi de
celui de M\textsuperscript{me} de Maintenon, était le fauteuil du roi,
ou les fauteuils quand le roi et la reine d'Angleterre y étaient,
laquelle était entre les deux rois. Les fils de France et M. le duc
d'Orléans étaient les seuls hommes dans ce rang, que les princesses du
sang fermaient. Vis-à-vis étaient assis les danseurs et avec eux M. le
comte de Toulouse, et dans les commencements que j'y ai dansé, M. le Duc
qui dansait encore\,; des deux côtés les dames qui dansaient, les
titrées les premières des deux côtés sans aucun mélange entre elles
d'aucune autre, non plus qu'à table avec le roi, ou avec Monseigneur, ou
chez M\textsuperscript{me} la duchesse de Bourgogne\,; derrière le roi
le service, M. le Prince quelquefois, et ce qu'il y avait de plus
distingué, et derrière encore\,; derrière les danseuses les dames qui ne
dansaient point, et derrière elles les hommes de la cour spectateurs, et
quelques autres derrière les danseurs\,; M. le Duc ne dansant plus, et
M. le prince de Conti toujours derrière les dames spectatrices. En
masque ou non c'était de même, excepté que, à visage couvert, les fils
de France se mêlaient au bas bout parmi les danseurs. Le roi
d'Angleterre et la princesse sa sœur ouvraient toujours le bal, et tant
qu'il dansait, le roi se tenait debout. Après deux ou trois fois de ce
cérémonial, le roi demeurait assis à la prière de la reine d'Angleterre.

Le duc et la duchesse d'Albe arrivèrent sur les quatre heures et
descendirent chez la princesse des Ursins, qui avait eu permission de
les mener chez M\textsuperscript{me} de Maintenon avant que le roi y
entrât\,: ce fut une grande faveur de M\textsuperscript{me} des Ursins.
M\textsuperscript{me} de Maintenon ne voyait jamais aucun étranger ni
aucun ambassadeur, et le duc et la duchesse d'Albe n'avaient pas encore
vu son visage. On fit pour eux une chose sans conséquence. Le roi fit
mettre la duchesse d'Albe au premier rang du fond, à côté et au-dessous
de M\textsuperscript{me} la princesse de Conti, pour qu'elle vît mieux
le bal, et M\textsuperscript{me} des Ursins à côté et au-dessous d'elle.
À souper on fit mettre la duchesse d'Albe auprès de
M\textsuperscript{me} la Duchesse à la table du roi, et
M\textsuperscript{me} des Ursins auprès d'elle. Le maréchal de Boufflers
fut chargé du duc d'Albe au bal, et de prier des courtisans distingués à
une table particulière qu'il tint pour le duc d'Albe, servie par les
officiers du roi. Il y en eut une autre pareille pour le duc de Perth et
pour les Anglais. Après souper, M\textsuperscript{me} la duchesse de
Bourgogne fit jouer la duchesse d'Albe au lansquenet avec elle. Le roi,
à son coucher, donna le bougeoir au duc d'Albe\footnote{Il a déjà été
  question plus haut de ce cérémonial. Le roi seul, d'après l'\emph{État
  de la France}, avait un bougeoir à deux bobèches et par conséquent à
  deux bougies. L'aumônier de jour tenait le bougeoir pendant que le roi
  faisait ses prières. Le premier valet de chambre prenait ensuite le
  bougeoir des mains de l'aumônier. Quand le roi était arrivé au
  fauteuil où il se déshabillait, il désignait une personne de
  l'assemblée pour tenir le bougeoir. C'était ordinairement un prince ou
  seigneur étranger. Le roi déshabillé, le premier valet de chambre
  reprenait le bougeoir, et les huissiers de la chambre criaient tout
  haut\,: \emph{Allons, messieurs, passe}. Alors toute la cour se
  retirait, à l'exception de ceux qui avaient droit d'assister au petit
  coucher du roi.}, et lui fit son compliment sur la peine de s'en
retourner coucher à Paris. Il parla fort à lui et à
M\textsuperscript{me} d'Albe.

Aux autres bals, M\textsuperscript{me} des Ursins se mettait auprès du
grand chambellan, et avec sa lorgnette regardait un chacun. À tout
moment le roi se tournait pour lui parler, et M\textsuperscript{me} de
Maintenon, qui à cause d'elle venait quelquefois avant le souper un
quart d'heure ou une demi-heure à ces bals, déplaçait le grand
chambellan qui se mettait derrière elle. Ainsi, elle était joignante
M\textsuperscript{me} des Ursins, et tout près du roi de l'autre côté en
arrière, et la conversation entre eux trois était continuelle\,;
M\textsuperscript{me} la duchesse de Bourgogne s'y mêlait beaucoup, et
Monseigneur quelquefois. Cette princesse aussi n'était occupée que de
M\textsuperscript{me} des Ursins, et on voyait qu'elle cherchait à lui
plaire. Ce qui parut extrêmement singulier, ce fut de voir celle-ci
paraître dans le salon avec un petit épagneul sous le bras, comme si
elle eût été chez elle. On ne revenait point d'étonnement d'une
familiarité que lime la duchesse de Bourgogne n'eût osé hasarder, encore
moins à ces bals de voir le roi caresser le petit chien, et à plusieurs
reprises. Enfin, on n'a jamais vu prendre un si grand vol.~On ne s'y
accoutumait pas, et à qui l'a vu, et connu le roi et sa cour, on en est
surpris encore quand on y pense après tant d'années. Il n'était plus
douteux alors qu'elle ne retournât en Espagne\footnote{Le retour de la
  princesse des Ursins en Espagne était résolu dès le 13 janvier 1705.
  Voy., notes à la fin du volume.}. Ses particuliers si fréquents avec
le roi et M\textsuperscript{me} de Maintenon roulaient sur les affaires
de ce pays-là.

Le duc de Grammont demandait son retour, la reine d'Espagne le pressait
avec ardeur. Le roi et M\textsuperscript{me} de Maintenon,
intérieurement blessés contre lui, et peu contents de sa gestion en ce
pays-là, ne s'y opposaient pas\,; mais il fallait choisir un
ambassadeur. Amelot fut choisi. C'était un homme d'honneur, de grand
sens, de grand travail et d'esprit. Il était doux, poli, liant, assez
ferme, de plus un homme fort sage et modeste. Il avait été ambassadeur
en Portugal, à Venise, en Suisse, et avait eu d'autres commissions au
dehors. Partout il avait réussi, s'était fait aimer, et avait acquis une
grande réputation. Il était de robe, conseiller d'État, par conséquent
point susceptible de Toison ni de grandesse. M\textsuperscript{me} des
Ursins ne crut pas pouvoir trouver mieux pour avoir sous elle un
ambassadeur sans famille et sans protection ici autre que son mérite,
qui, sous le nom de son caractère, l'aidât mieux dans toutes les
affaires, et qui, en effet, ne fût sous elle qu'un secrétaire renforcé,
qui, témoin ici de sa gloire, lui fût souple, et à l'abri du nom duquel
elle agirait avec toute autorité en Espagne et toute confiance de ce
pays-ci. Il était bien avec le roi et avec M\textsuperscript{me} de
Maintenon, à portée de recevoir d'elle des ordres et des impressions
particulières qui le retiendraient du côté des ministres. Elle s'arrêta
donc à lui, et le fit choisir, avec ordre très exprès de n'agir que de
concert avec elle, et, pour trancher le mot, sous elle. La déclaration
suivit de près la résolution prise. Amelot eut plusieurs entretiens
longs et près à près avec M\textsuperscript{me} des Ursins\,; il reçut
immédiatement du roi des ordres particuliers, plus encore de
M\textsuperscript{me} de Maintenon. Dès que la nouvelle en fut arrivée
en Espagne, le duc de Grammont fut traité avec plus de ménagement, et
fut fait chevalier de la Toison, suivant l'engagement que
M\textsuperscript{me} des Ursins en avait bien voulu prendre.

Elle obtint une autre chose bien plus difficile, parce que le roi
s'était peu à peu laissé aller à la résolution de ne lui rien refuser.
Ce fut le retour d'Orry en Espagne, sous prétexte de la grande
connaissance qu'il avait des finances de ce pays-là, et des lumières
qu'Amelot ne pouvait tirer de personne plus sûrement, ni avec plus
d'étendue et de détail que de lui sur ces matières. On se persuada que,
sous les yeux d'Amelot, il ne pourrait plus retomber dans les
manquements qui, avec ses mensonges, avaient fait son crime. Il fut donc
effacé. Amelot partit sur la fin d'avril, et Orry incontinent après,
c'est-à-dire un mois après la déclaration de son ambassade.
M\textsuperscript{me} des Ursins obtint encore d'emmener en Espagne le
chevalier Bourg, avec caractère public d'envoyé du roi d'Angleterre, et
six mille livres d'appointements payés par le roi. C'était un
gentilhomme irlandais, catholique, qui, faute de pain, s'était intrigué
à Rome et fourré chez le cardinal de Bouillon, qui alors était ami
intime de M\textsuperscript{me} des Ursins.

Bourg était homme de beaucoup d'esprit, entièrement tourné à l'intrigue,
homme d'honneur pourtant, et malade de politique et de raisonnement. Le
cardinal de Bouillon, qui l'avait trouvé propre à beaucoup de choses
secrètes, l'y avait fort employé. Il avait fait sa cour à
M\textsuperscript{me} des Ursins, qui l'avait goûté. Il y eut je ne sais
quelle petite obscure négociation sur le cérémonial entre les cardinaux
et les petits princes d'Italie. Le cardinal de Bouillon fit envoyer
Bourg vers eux avec une lettre de créance du sacré collège. Il s'élevait
aisément et avait besoin d'être contenu. Il réussit, fut connu et
caressé de plusieurs cardinaux. L'état de domestique du cardinal de
Bouillon commença à lui peser, il s'en retira avec ses bonnes grâces et
une pension. Fatigué dans les suites de ne trouver point d'emploi à
Rome, il revint en France, s'y maria à une fille de Varenne, que nous
avons vu ôter du commandement de Metz, et bientôt après s'en alla vivre
à Montpellier. Voyant le règne de M\textsuperscript{me} des Ursins en
Espagne, il alla l'y trouver et en fut très bien reçu. Elle s'en servit
en beaucoup de choses, et lui donna un accès fort libre auprès du roi et
de la reine d'Espagne. Il eut lieu de nager là en grande eau. Il aimait
les affaires et l'intrigue. Il l'entendait bien, et, avec l'esprit
diffus et quelquefois confus, il était fort instruit des intérêts des
princes, et passait sa vie en projets. Avec tout cela et ses besoins,
rien ne l'empêchait de dire la vérité à bout portant aux têtes
principales, à Orry, à M\textsuperscript{me} des Ursins, à la reine
d'Espagne et dans les suites au roi et à l'autre reine sa femme, à
Albéroni, aux ministres les plus autorisés, qui tous l'admirent dans
leur familiarité, s'en servirent au dedans, le consultèrent et
l'estimèrent, mais le craignirent assez pour ne lui jamais donner
d'emploi, ni de subsistance que fort courte. Je l'ai fort vu en Espagne
et m'en suis bien trouvé. Bourg avait eu un fils, qui mourut, et une
fille fort jolie. Il la voulut faire venir avec sa mère le trouver en
Espagne\,; elles s'embarquèrent en Languedoc et furent prises par un
corsaire. La mère se noya, la fille fut menée à Maroc, où elle montra
beaucoup d'esprit et de vertu\,; elle y fut bien traitée, mais gardée
longtemps, puis à grand'peine renvoyée en France. Bourg, quelque temps
après mon retour d'Espagne, lassé d'y espérer en vain, revint trouver sa
fille qui était à Paris dans un couvent. Il y trouva encore moins son
compte qu'en Espagne, où au moins il voyait familièrement les ministres.
Il me dit son ennui, et qu'il s'en allait à Rome avec sa fille retrouver
son amie M\textsuperscript{me} des Ursins, et son roi naturel. Il y fut
bien reçu de l'un et de l'autre, et sa fille entra fille d'honneur chez
la reine d'Angleterre\,; mais le pauvre Bourg ne trouva pas plus de
jointure à Rome qu'en France et en Espagne. Ainsi cet homme propre à
beaucoup de choses, et qui avait été de part à quantité d'importantes,
trouva toujours les portes fermées partout à la moindre fortune.

Parlant d'Anglais catholiques, le feu roi Jacques crut en mourant devoir
faire acte de miséricorde ou de justice, je ne sais trop lequel. Le
comte de Melford, frère du duc de Perth, avait été son ministre. Il
l'avait exilé à Orléans. Middleton était entré en sa place, dont
personne n'avait d'opinion. Il était protestant, plein d'esprit et de
ruse, avec force commerces en Angleterre pour le service de son maître,
disait-il\,; mais on prétendait que c'était pour le sien, et qu'il
touchait tous ses revenus. Sa femme, qui avait pour le moins autant
d'esprit que lui, et beaucoup de manège, était catholique et gouvernante
de la princesse d'Angleterre. Elle le soutint fort, par la reine avec
qui elle était fort bien. Melford était revenu à Paris. Ce ne fut qu'en
ce temps-ci qu'il fut rappelé à Saint-Germain et déclaré duc. Le feu roi
d'Angleterre l'avait ordonné ainsi en mourant. Le duc de Perth, son
frère, avait été gouverneur du roi. Middleton craignit à ce retour que
Melford ne reprît son ancienne place qu'il occupait en son absence\,; il
tourna court. Il fut trouver la reine, lui dit que la sainte vie, et
surtout la sainte mort du feu roi son mari, et l'exhortation qu'il avait
faite en mourant à ses domestiques protestants, l'avait converti. Il se
fit catholique, et reverdit en crédit et en confiance à Saint-Germain.
Melford ne fut de rien, mais lui et sa femme eurent en France le rang et
les honneurs de duc et de duchesse comme tous ceux qui l'avaient été
faits à Saint-Germain, ou qui y étaient arrivés tels.

Plusieurs personnes marquées ou connues moururent en ce même temps comme
à la fois\,:

M\textsuperscript{me} du Plessis-Bellière, la meilleure et la plus
fidèle amie de M. Fouquet, qui souffrit la prison pour lui et beaucoup
de traitements fâcheux, à l'épreuve desquels son esprit et sa fidélité
furent toujours. Elle conserva sa tête, sa santé, de la réputation, des
amis jusqu'à la dernière vieillesse, et mourut à Paris chez la maréchale
de Créqui sa fille, avec laquelle elle demeurait à Paris.

Magalotti, un de ces braves que le cardinal Mazarin avait attirés auprès
de lui, quoique fort jeune, par le privilège de la nation. Il avait vu
le roi jeune chez le cardinal, et conservé liberté avec lui. Le roi
avait pour lui de la bonté et de la distinction, qui pourtant ne le put
soustraire à la haine de M. de Louvois, acquise par son intimité avec M.
de Luxembourg. C'était un homme délicieux et magnifique, aimé et
considéré, et qui avait été toute sa vie dans les meilleures compagnies
des armées où il avait servi. Il était lieutenant général, gouverneur de
Valenciennes, et avait le régiment Royal-Italien qui vaut beaucoup\,;
dans sa vieillesse le plus beau visage du monde, et le plus vermeil,
avec des yeux italiens et vifs, et les plus beaux cheveux blancs du
monde, et portait toujours le jupon à l'italienne. Louvois, qui l'ôta du
service, l'empêcha aussi d'être chevalier de l'ordre, quoique bon
gentilhomme florentin. C'était d'ailleurs un très bon homme, avec bien
de l'esprit, de l'entendement et de l'agrément.

Albergotti, son neveu, eut le Royal-Italien. Il avait plus d'esprit que
son oncle, de grands talents pour la guerre et beaucoup de valeur, plus
d'ambition encore, et tous moyens lui étaient bons. C'était un homme
très dangereux, très intimement mauvais, et foncièrement malhonnête
homme, avec un froid dédaigneux, et des journées sans dire une parole.
Son oncle l'avait initié dans la confiance de M. de Luxembourg, et par
là dans la compagnie choisie de l'armée, qui lui fraya celle de la cour.
Il était intimement aussi avec M. le prince de Conti par la même raison,
et fort bien avec M. le Duc. Il fut accusé, et sa conduite le vérifia,
d'avoir passé d'un camp à l'autre, c'est-à-dire d'avoir toujours tenu à
un filet à M. de Vendôme, lors et depuis sa rupture avec M. de
Luxembourg, M. le prince de Conti et leurs amis, et après la mort de M.
de Luxembourg, de s'être jeté de ce côté-là sans mesure. M. de
Luxembourg fils, M. le prince de Conti et leurs amis s'en plaignaient
fort en particulier, en public ils gardèrent des dehors. Albergotti
devint un favori de M. de Vendôme, qui lui valut la protection de M. du
Maine, laquelle l'approcha de M\textsuperscript{me} de Maintenon. Je me
suis étendu sur ce maître Italien\,; on verra dans la suite qu'il était
bon de le connaître.

J'ai assez parlé en plusieurs occasions du duc de Choiseul pour n'avoir
rien à ajouter, sinon que, par sa mort, il ne vaqua qu'un collier de
l'ordre, et que ce duché-pairie fut éteint.

On a suffisamment vu, à propos du procès de préséance avec M. de
Luxembourg, quel était le président de Maisons, pour n'avoir rien à en
dire de plus, sinon qu'il mourut fort vieux en ce temps-ci, démis de sa
charge en faveur de son fils, duquel il sera fort mention dans la suite.

M\textsuperscript{lle} de Beaufremont suivit de près M. de Duras, à
propos duquel je l'ai fait connaître.

Seissac, dont j'ai suffisamment parlé aussi, finit son indigne vie, et
laissa une belle, jeune et riche veuve fort consolée, qui perdit bientôt
après le fils unique qu'elle en avait eu et hérita de tous ses biens. En
lui s'éteignit l'illustre maison de Clermont-Lodève. Comme il avait la
fantaisie de ne porter jamais aucun deuil, personne aussi ne le prit de
lui, non pas même le duc de Chevreuse, son beau-frère.

Le roi le porta quelques jours du duc Maximilien, oncle paternel de
l'électeur de Bavière, uniquement pour gratifier ce prince. Ce duc
Maximilien avait épousé une sœur de M. de Bouillon, dont il n'eut point
d'enfants, et avec qui il vivait depuis longtemps à la campagne, en
Bavière, dans une grande piété et dans une grande retraite.

M. de Beuvron, chevalier de l'ordre et lieutenant général de Normandie,
y mourut à plus de quatre-vingts ans, chez lui, à la Meilleraye, avec la
consolation d'avoir vu son fils Harcourt arrivé à la plus haute et à la
plus complète fortune, et son autre fils Sézanne en chemin d'en faire
une, et déjà chevalier de la Toison d'or. On a vu comment elle était due
aux agréments de la jeunesse du père. C'était un très honnête homme, et
très bon homme, considéré et encore plus aimé.

Enfin on perdit Mgr le duc de Bretagne d'une manière très prompte. Mgr
le duc de Bourgogne et M\textsuperscript{me} la duchesse de Bourgogne
surtout, en furent extrêmement affligés. Le roi marqua beaucoup de
religion et de résignation. Aussitôt après, c'est-à-dire le 24 avril, le
roi s'en alla à Marly, où il mena qui il lui plut, sans que personne eût
demandé. Nous en fûmes, M\textsuperscript{me} de Saint-Simon et moi. La
goutte qui y prit au roi, et qui fut extrêmement longue, y fit demeurer
plus de six semaines, et c'est depuis cette goutte qu'on ne vit plus le
roi à son coucher, qui devint pour toujours un temps de cour réservé aux
entrées. Il n'y eut point de cérémonies, sinon que le corps du petit
prince fut porté dans un carrosse du roi non drapé, environné de gardes
et de pages avec des flambeaux. Dans ce même carrosse étaient le
cardinal de Coislin à la première place, parce qu'il portait le cœur sur
un carreau sur ses genoux, M. le Duc, comme prince du sang, à côté de
lui, M. de Tresmes, comme duc, et non comme premier gentilhomme de la
chambre, au devant avec M\textsuperscript{me} de Ventadour comme
gouvernante\,; une sous-gouvernante et un aumônier du roi étaient aux
portières. Le roi, Monseigneur, ni M. et M\textsuperscript{me} la
duchesse de Bourgogne, n'en prirent point le deuil. M. le duc de Berry
et toute la cour le porta comme d'un frère. De Saint-Denis, ils
rapportèrent le cœur au Val-de-Grâce. Paris et le public fut fort touché
de cette perte.

Rubantel, vieux, retiré, disgracié, comme je l'ai rapporté en son temps,
mourut aussi à Paris quelques jours après.

Breteuil, conseiller d'État, qui avait été intendant des finances, et
dont le fils est aujourd'hui secrétaire d'État de la guerre pour la
seconde fois, ne tarda pas à les suivre\,; sa place de conseiller d'État
fut donnée à Armenonville, déjà directeur des finances. Je le remarque,
parce que nous le verrons aller bien plus haut. En même temps aussi,
d'Alègre perdit son fils unique.

Bouchu, conseiller d'État et intendant de Dauphiné, perdu de goutte et
toujours homme de plaisir, voulut quitter cette place\,; je le remarque
parce qu'elle fut donnée à Angervilliers, quoique fort jeune, et
seulement encore intendant d'Alençon. Nous le verrons secrétaire d'État
de la guerre, et aurons occasion d'en parler plus d'une fois.

Puisque j'ai parlé de Bouchu, il faut que j'achève l'étrange singularité
qu'il donna en spectacle, autant qu'un homme de son état en peut donner.
C'était un homme qui avait eu une figure fort aimable, et dont l'esprit,
qui l'était encore plus, le demeura toujours. Il en avait beaucoup, et
facile au travail, et fertile en expédients. Il avait été intendant de
l'armée de Dauphiné, de Savoie et d'Italie, toute l'autre guerre et
celle-ci. Il s'y était cruellement enrichi, et il avait été reconnu trop
tard, non du public, mais du ministère\,; homme d'ailleurs fort galant
et de très bonne compagnie. Lui et sa femme qui était Rouillé, sœur de
la dernière duchesse de Richelieu, et de la femme de Bullion, se
passaient très bien l'un de l'autre. Elle était toujours demeurée à
Paris, où il était peu touché de la venir rejoindre, et peu flatté
d'aller à des bureaux et au conseil, après avoir passé tant d'années
dans un emploi plus brillant et plus amusant. Néanmoins il n'avait pu
résister à la nécessité d'un retour honnête, et il avait mieux aimé
demander que de se laisser rappeler. Il partit pour ce retour le plus
tard qu'il lui fut possible, et s'achemina aux plus petites journées
qu'il put. Passant à Paray\footnote{Parai ou Paray-le-Monial, que les
  anciens éditeurs ont changé en Pavé, est situé dans le département de
  Saône-et-Loire. Il y avait autrefois un prieuré de bénédictins
  dépendant de Cluni.}, terre des abbés de Cluni, assez près de cette
abbaye, il y séjourna. Pour abréger, il y demeura deux mois dans
l'hôtellerie. Je ne sais quel démon l'y fixa, mais il y acheta une
place, et, sans sortir du lieu, il s'y bâtit une maison, s'y accommoda
un jardin, s'y établit et n'en sortit jamais depuis, en sorte qu'il y
passa plusieurs années, et y mourut sans qu'il y eût été possible à ses
amis ni à sa famille de l'en tirer. Il n'y avait, ni dans le voisinage,
aucun autre bien que cette maison, qu'il s'y était bâtie\,; il n'y
connaissait personne, ni là autour auparavant. Il y vécut avec des gens
du lieu et du pays, et leur faisait très bonne chère, comme un simple
bourgeois de Paray.

\hypertarget{note-i.-note-rectificative-remise-uxe0-m.-le-duc-de-saint-simon-par-m.-de-chantuxe9rac-pour-uxe9tablir-quuranie-de-la-cropte-beauvais-uxe9tait-fille-luxe9gitime-de-la-cropte-beauvais-et-de-charlotte-martel.}{%
\chapter{NOTE I. NOTE RECTIFICATIVE REMISE À M. LE DUC DE SAINT-SIMON
PAR M. DE CHANTÉRAC POUR ÉTABLIR QU'URANIE DE LA CROPTE-BEAUVAIS ÉTAIT
FILLE LÉGITIME DE LA CROPTE-BEAUVAIS ET DE CHARLOTTE
MARTEL.}\label{note-i.-note-rectificative-remise-uxe0-m.-le-duc-de-saint-simon-par-m.-de-chantuxe9rac-pour-uxe9tablir-quuranie-de-la-cropte-beauvais-uxe9tait-fille-luxe9gitime-de-la-cropte-beauvais-et-de-charlotte-martel.}}

Le récit du duc de Saint-Simon repose tout entier sur une erreur
principale. Par contrat du 23 décembre 1653, passé à Marennes, devant
Baige, notaire héréditaire de Saintonge, dont la grosse, signée
\emph{Baige}, est conservée dans les archives de la Cropte-Chantérac, M.
de La Cropte-Beauvais avait épousé Charlotte Martel, fille de Gédéon
Martel, comte de Marennes, et d'Élisabeth de La Mothe-Foucqué (voir dans
le P. Anselme, t. VIII, p.~209, la généalogie de Martel). Uranie de La
Cropte de Beauvais, née de cette union légitime, n'était donc point
bâtarde.

Ce n'était pas non plus «\,en mauvaise compagnie\,» que le comte de
Soissons l'avait connue, mais au Palais-Royal, chez Madame, dont elle
était demoiselle d'honneur (\emph{État de la France}, mdclxxviii, t.
Ier, p.~484, et \emph{Lettres de M\textsuperscript{me} de Sévigné}).
Louis XIV était devenu très amoureux d'elle, «\,mais sa vertu
inébranlable\,» lui avait résisté, et il s'était alors tourné vers sa
compagne, M\textsuperscript{lle} de Fontange (\emph{Mémoires de la
duchesse d'Orléans, princesse palatine}).

Dans un autre endroit de ses Mémoires, M. de Saint-Simon parle encore
d'une manière inexacte de la situation de la comtesse de Soissons après
la mort de son mari, quand il dit qu'elle vécut pauvre, malheureuse,
errante, etc. La comtesse de Soissons, outre son héritage paternel et
les avantages considérables de son contrat de mariage, possédait, du
chef de sa mère, les terres des comté de Marennes et baronnie de
Tonnay-Boutonne (Lettre de la comtesse de Soissons au comte de
Chantérac, son cousin, publiée dans le \emph{Bulletin de la Société de
l'Histoire de France} de janvier 1856). Elle avait de plus, de Madame,
une pension de douze mille livres, et n'avait par conséquent pas besoin
de recevoir, «\,de fois à autre, quelque gratification de M. le duc
d'Orléans.\,» Enfin ce ne fut pas le fils d'Uranie de La Cropte, mais
bien son petit-fils, qui mourut au moment où il allait épouser
l'héritière de Massa-Carrara, de la maison de Cibo. Son fils Thomas
Emmanuel Amédée de Savoie, comte de Soissons, chevalier de la Toison
d'or, etc., avait épousé Thérèse Anne Félicité, fille du prince de
Lichtenstein.

\hypertarget{note-ii.-lettre-du-maruxe9chal-de-villars-au-roi.}{%
\chapter{NOTE II. LETTRE DU MARÉCHAL DE VILLARS AU
ROI.}\label{note-ii.-lettre-du-maruxe9chal-de-villars-au-roi.}}

La lettre de Villars, que Saint-Simon avait placée parmi les Pièces de
ses Mémoires, se trouve dans les archives du Dépôt de la guerre,
vol.~1582, lettre 103. Elle a été publiée dans le tome II, pages 409 et
suivantes des \emph{Mémoires militaires relatifs à la succession
d'Espagne}, qui font partie de la collection des \emph{Documents inédits
relatifs à l'histoire de France}. En voici le texte\,:

«\,Du camp de Friedlingen, 15 octobre 1702.

«\,J'avais l'honneur de rendre compte à Votre Majesté, par une assez
longue dépêche du 14, de tout ce qui regardait la prise de Neubourg, qui
a coûté M. de La Petitière, capitaine des grenadiers de Crussol. C'est à
la valeur de cet officier et à celle du sieur Jorreau,
lieutenant-colonel de Béarn, qu'est dû l'heureux succès de cette
entreprise. M. le marquis de Biron s'y est conduit à son ordinaire. J'y
avais envoyé M. le comte du Bourg pour donner tous les ordres
nécessaires, ce qui lui a causé le malheur de ne pouvoir se trouver à la
bataille, dont M. de Choiseul aura l'honneur de donner la première
nouvelle à Votre Majesté.

«\,Je fus informé que l'armée de l'empereur, commandée par M. le prince
de Bade, se mettait en marche le 14, et quittait ses retranchements. Dès
le 13, l'infanterie de Votre Majesté et la brigade de Vivans avaient
passé le Rhin. Le prince de Neubourg nous faisant voir un mouvement fort
vif dans le camp des ennemis, l'on crut qu'il était bon de se mettre en
mesure, ou d'empêcher leur armée de troubler notre établissement dans
notre nouveau poste, ou de l'attaquer, si l'on en détachait quelque
corps d'infanterie pour aller vers Neubourg.

«\,Sa Majesté comprendra que son armée, ayant été placée au delà du Rhin
dès le 13, par les raisons que j'ai eu l'honneur de lui exposer, fut
promptement en bataille dans (devant\,?) les retranchements des ennemis.
Dans la matinée du 14, MMMM. \#8203;. Desbordes et de Chamarande
s'étaient mis à la tête de l'infanterie, laquelle marcha très
diligemment pour gagner la crête d'une montagne assez élevée.

«\,La cavalerie des Impériaux, plus forte de deux mille chevaux que la
nôtre, était en bataille dans la plaine\,; et celle de Votre Majesté fut
placée, la gauche au fort de Friedlingen, malgré un assez gros feu de
l'artillerie de ce fort, et la droite appuyée à cette montagne que
l'infanterie avait occupée.

«\,On aperçut en ce moment que l'infanterie des ennemis faisait tous ses
efforts pour gagner la crête de la hauteur, avec cette circonstance
qu'elle y montait en bataille, et que celle de Votre Majesté traversait
des vignes et des hauteurs escarpées qui ralentissaient sa marche.

«\,Je dois faire observer à Votre Majesté que l'on avait envoyé à
Neubourg deux mille hommes de son infanterie, parmi lesquels étaient
plusieurs compagnies de grenadiers, et les deux régiments de dragons de
la reine et de Gévaudan. Cependant MM. Desbordes et de Chamarande, qui
pressaient les mouvements de l'infanterie, le premier peut-être avec
trop d'ardeur, marchaient aux ennemis avec les brigades de Champagne, de
Bourbonnais, de Poitou et de la reine. Ils les trouvèrent postés dans un
bois assez épais. Les ennemis avaient leur canon, et, malgré une très
vive résistance, ils furent renversés et leur canon fut pris. Pendant ce
temps-là, M. de Magnac, qui était dans la plaine à la tête de la
cavalerie, vit celle des ennemis s'ébranler pour venir à la charge\,;
celle de Votre Majesté était dans tout l'ordre convenable. On avait, dès
le matin, recommandé aux cavaliers de ne point se servir d'armes à feu,
et de ne mettre l'épée à la main qu'à cent pas des ennemis\,; et, à la
vérité, ils n'ont pas tiré un seul coup.

«\,Les Impériaux ont fait les trois quarts du chemin\,; M. de Magnac,
suivi de M. de Saint-Maurice, qui commandait la seconde ligne, et s'est
conduit en bon et ancien officier, s'est ébranlé de deux cents pas. La
charge n'a été que trop rude par la perte de très braves officiers, dont
j'aurai l'honneur d'envoyer une liste à Votre Majesté par le premier
ordinaire.

«\,La cavalerie impériale a été entièrement renversée, sans que les
escadrons de celle de Votre Majesté se soient démentis\,; ils ont mené
les ennemis jusqu'à un défilé qui les a fait perdre de vue, sans qu'ils
se soient écartés pour le pillage ni pour faire des prisonniers.

«\,Les nouveaux régiments n'ont pas cédé aux anciens\,; et pour nommer
ceux qui se sont distingués, il n'y a qu'à voir l'ordre de bataille\,:
M. de Vivans, commandant de la cavalerie\,; M. Dauriac\,; M. de
Massenbach, colonel réformé, commandant par son ancienneté la brigade de
Condé, a fait des merveilles\,; M. le marquis du Bourg, colonel du
régiment royal\,; M. le prince de Tarente, capitaine dans ce même
régiment\,; M. de Saint-Pouange\,; Fourquevaux, qui a sept étendards des
ennemis dans son nouveau régiment\,; M. de Conflans, brigadier. En un
mot, je puis dire à Votre Majesté qu'elle peut compter que cette
cavalerie s'est surpassée, et elle peut juger de la perte des Impériaux
par la prise de trente étendards et de trois paires de timbales. Nous
voyous, par des ordres de bataille pris aux ennemis, qu'ils avaient
cinquante-six escadrons. Votre Majesté en avait trente-quatre, les six
de la reine et de Gévaudan ayant été détachés la veille pour marcher
vers Neubourg.

«\,Notre infanterie avait défait et renversé, par trois charges
différentes, celle des ennemis, et pris leur canon\,; mais sa trop
grande ardeur, jointe à la mort de M. Desbordes, lieutenant général, et
à celle de M. de Chavannes, brigadier, la porta à sortir dans la plaine,
après avoir chassé les ennemis du bois, et à perdre ainsi son avantage.
M. de Chamarande, qui dans tout le cours de cette action s'est
parfaitement distingué, MM. de Schelleberg et du Tot, ne purent empêcher
qu'elle ne revînt. Cependant on peut juger de l'avantage qu'elle a eu
sur les ennemis, puisqu'elle leur a pris plusieurs drapeaux sans en
avoir perdu un seul.

«\,Tous les jeunes colonels y ont montré une valeur infinie. MM. de
Seignelay, de Nangis, de Coetquen, le jeune Chamarande, le comte de
Choiseul, de Raffetot, ont toujours été dans le plus grand péril et le
plus gros feu. Les ennemis ont eu plus de trois mille hommes tués sur le
champ de bataille\,; ils n'ont pas de nos prisonniers. Nous savons que
le général Stauffemberg y a été tué. On dit aussi que le comte de
Fürstemberg-Stühlingen, les comtes de Hohenlohe, Koenigseck et deux
autres colonels sont prisonniers, avec vingt-cinq autres officiers.

«\,Le comte de Hohenlohe demande de pouvoir aller à Bâle sur parole.
Nous avons été aujourd'hui sur le champ de bataille, et les endroits où
leurs bataillons ont été défaits sont marqués par quantité d'armes
abandonnées.

«\,Cependant le temps qu'il a fallu pour remettre quelque ordre dans
notre infanterie a sauvé celle des ennemis. Le chevalier de Tressemane,
major général, y a parfaitement bien servi, aussi bien que le sieur de
Beaujeu, maréchal des logis de la cavalerie. On a poussé les ennemis une
lieue et demie au delà du champ de bataille, sur lequel l'armée de Votre
Majesté a campé. On croyait quatre petites pièces de canon égarées, mais
elles ont été retrouvées ce matin. Jusqu'à présent on n'en a que deux de
celles des ennemis\,; mais j'en ai vu sept ou huit autres derrière notre
infanterie. Il est rare et heureux, dans une affaire aussi rude et aussi
disputée, que l'armée de Votre Majesté n'ait perdu ni drapeaux, ni
étendards, ni timbales, et que l'on en ait plus de trente-quatre de ceux
des ennemis. Voilà, Sire, le compte que je dois avoir l'honneur de
rendre à Votre Majesté d'un avantage bien ordinaire à ses armes toujours
victorieuses.

«\,Nous apprenons dans le moment que le comte (le Fürstemberg est mort
de ses blessures. Ce serait une grande perte pour l'empereur et pour M.
le prince de Bade, dont il était l'homme de confiance.\,»

\hypertarget{note-iii.-retour-de-la-princesse-des-ursins-en-espagne.}{%
\chapter{NOTE III. RETOUR DE LA PRINCESSE DES URSINS EN
ESPAGNE.}\label{note-iii.-retour-de-la-princesse-des-ursins-en-espagne.}}

Les papiers du duc de Noailles, conservés en partie à la bibliothèque
impériale du Louvre, fournissent d'utiles renseignements pour contrôler
les Mémoires de Saint-Simon, principalement en ce qui concerne les
affaires d'Espagne. Voici, entre autres, deux lettres se rattachant au
retour de la princesse des Ursins, dont Saint-Simon parle. La première
est une dépêche de Louis XIV au duc de Grammont, ambassadeur en Espagne,
et la seconde une lettre du duc de Grammont au maréchal de Noailles.

\emph{Dépêche de Louis XIV au duc de Grammont\footnote{La copie de cette
  lettre se trouve à la bibl. imp. du Louvre, ms. F, 325, t. XXI, lettre
  4.}}

«\,Versailles, le 13 janvier 1705.

«\,Mon cousin, depuis que j'ai parlé à la princesse des Ursins, il m'a
paru nécessaire de la renvoyer en Espagne, et d'accorder enfin cette
grâce aux instances pressantes du roi mon petit-fils et de la reine.
J'ai jugé en même temps qu'il convenait au bien de mon service de vous
charger de donner à la reine une nouvelle qu'elle désire avec autant
d'empressement. Ainsi je fais partir le courrier qui sera chargé de
cette dépêche avant même que d'annoncer à la princesse des Ursins ce que
je veux faire pour elle. Je ne vous prescris point ce que vous avez à
dire sur ce sujet. Il vous donne assez de moyens par lui-même de faire
connaître au roi et à la reine d'Espagne la tendresse que j'ai pour eux,
et combien je désire de contribuer à leur satisfaction.

«\,Je dirai encore à la princesse des Ursins que vous m'avez toujours
écrit en sa faveur. Je suis persuadé qu'elle connaît l'importance dont
il est, pour le bien des affaires et pour elle-même, de bien vivre avec
vous, et qu'elle n'oubliera rien pour maintenir cette bonne
intelligence. Si vous en jugez autrement, je serai bien aise que vous me
mandiez, avec toute la vérité que je sais que vous ne me déguisez
jamais, ce que vous en pensez, et même si vous croyez qu'il ne vous
convienne pas de demeurer en Espagne après son retour.

«\,Cette sincérité de votre part confirmera ce que j'ai vu en toutes
occasions de votre zèle pour mon service et de votre attachement
particulier à ma personne. Vous devez croire aussi que ces sentiments me
sont toujours présents, et que je serai bien aise de vous faire
connaître en toutes occasions combien ils me sont agréables.

«\,Je renverrai incessamment le courrier par qui j'ai reçu votre lettre
du 1er de ce mois, et je vous ferai savoir par son retour mes intentions
sur ce qui regarde le siège de Gibraltar. Sur ce,\,» etc.

\emph{Lettre du duc de Grammont au maréchal de Noailles sur
M\textsuperscript{me} des Ursins\footnote{Bibl, imp. du Louvre, ms. F,
  325, t. XXI, lettre 8.}}

«\,15 janvier 1705.

«\,Vous me demandez, monsieur, de la franchise et un développement de
cœur au sujet de M\textsuperscript{me} des Ursins. Je vais vous
satisfaire\,; car je vous honore et vous aime trop pour y manquer. Je
commencerai par vous détailler quelle est ma situation à cet égard. Le
roi me mande, par sa lettre du 30 novembre dernier, qu'il a permis à
M\textsuperscript{me} des Ursins de venir à la cour, mais que son retour
ici serait très contraire à son service. M. de Maulevrier, qui vient de
quitter le maréchal de Tessé, sort de me dire qu'il est vrai que M. de
Tessé a donné des espérances à la reine du retour de
M\textsuperscript{me} des Ursins auprès d'elle\,; mais tout ce qu'il a
fait à cet égard, il l'a fait par ordre. Si j'ajoutais une foi entière à
ce qu'il m'a fait dire, la chose serait décidée\,; mais comme mon ordre
est contraire, et que vous voulez que je vous dise précisément ce que je
pense sur ce retour, je vais le faire avec toute la vérité dont je suis
capable.

«\,S'il était dans la nature de M\textsuperscript{me} des Ursins de
pouvoir revenir ici avec un esprit d'abandon et de dévouement entier aux
volontés et aux intérêts du roi, et que l'ambassadeur de Sa Majesté, je
ne dis pas moi, mais qui que ce pût être, et elle, ne fussent qu'un, et
que tous deux agissent de concert sur toutes choses, sans bricoles
quelconques, et que, par ce moyen, la reine d'Espagne ne se mêlant plus
de rien que de ce que l'on voudrait, et qu'il pût paraître par là aux
Espagnols que ce n'est plus la reine et sa faction qui gouvernent
l'Espagne, qui est la chose du monde qu'ils ont le plus en horreur, et
la plus capable de leur faire prendre un parti extrême, rien alors,
selon moi, ne peut être meilleur que de faire revenir
M\textsuperscript{me} des Ursins\,; mais comme ce que je dis là n'est
pas la chose du monde la plus certaine, et que le roi d'Espagne me l'a
dit, et qu'il craint de retomber où il s'est trouvé, le tout bien
compensé, je crois que c'est coucher gros et risquer beaucoup que de s'y
commettre, et je dois vous dire que les trois quarts de l'Espagne seront
au désespoir, que les factions renouvelleront de jambes, et que, de tous
les Espagnols, celui qui sera le plus fâché intérieurement sera le roi
d'Espagne, de se revoir tomber dans le temps passé, qui est sa bête.

«\,La reine d'Espagne le force d'écrire sur un autre ton, et il ne peut
le lui refuser, parce qu'il est doux et qu'il ne veut point de
désordre\,; mais en même temps il me charge par la voie secrète d'écrire
au roi naturellement ce qu'il pense, et il le lui confirme par la lettre
ci-jointe de sa main, que je vous envoie\footnote{Lettre du 15 janvier
  1705.}. En un mot, monsieur, le roi ne sera jamais maître de ce
pays-ci qu'en décidant sur tout par lui-même, qui est tout ce que le roi
son petit-fils désire, pour se tirer de l'esclavage où il est, d'avoir
une espèce de \emph{salve l'honor} à l'égard de la reine\,; et les
Espagnols ne demandent autre choie que d'être gouvernés par leur roi. Je
vous parlerais cent ans que je ne vous dirais pas autre chose\,; c'est
ce que vous pouvez dire au roi tête à tête, sans que cela aille au
conseil, par les raisons que je vous ai déjà dites. Je vous mande la
vérité toute nue, et comme si j'étais prêt à paraître devant mon Dieu.
C'est ensuite au roi, qui a meilleur esprit que tous tant que nous
sommes, de prendre sur cela le parti qui lui conviendra.

. . . . . . . . . . . . . . . . . . . . . . . . . . . . . . . . . . . .
. . . . . . .

«\,Il faut que le roi porte par une autorité absolue le correctif
nécessaire. Toute l'Espagne parle comme moi, et c'est à la veille de
débonder si le gouvernement despotique de la reine subsiste, et il n'est
ni petit ni grand qui n'en ait par-dessus la tête, et le roi d'Espagne
et tout ce que vous connaissez ici d'honnêtes gens ne respirent que les
ordres absolus du roi pour s'y soumettre aveuglément. Mon honneur, ma
conscience, mon zèle et nia fidélité intègre et incorruptible pour le
bien du service de mon maître, m'obligent à lui parler de la sorte\,;
quiconque sera capable de lui parler autrement le trompera avec
indignité. L'Espagne est perdue sans ressource ci le gouvernement reste
comme il est, et que le roi notre maître n'en prenne pas seul le timon.
Le cardinal Portocarrero, Mancera, Montalte, San-Estevan, Monterey,
Montellano\footnote{Voy., dans le tome III des Mémoires de Saint-Simon,
  p. 4 et suiv., le caractère des principaux membres du conseil de
  Philippe V.}, et généralement tout ce qu'il y a de meilleur et de
véritablement attaché à la monarchie, concertent tous le moyen d'en
parler au roi et de lui en parler clairement. Que le roi ne se laisse
donc pas abuser par les discours, et qu'il s'en tienne à la vérité, que
j'ai l'honneur de lui mander par vous. Le marquis de Monteléon, qui est
un homme plein d'honneur et d'esprit, part incessamment pour vous aller
confirmer de bouche ce que j'ai l'honneur de mander au roi.

«\,De l'argent, nous en allons avoir, même considérablement, et l'on
vient de faire une affaire de quatorze millions de livres, qu'on
n'imaginait pas qui s'osât jamais tenter, et que, depuis Charles-Quint,
nul homme n'avait eu la hardiesse de proposer. Nous aurons la plus belle
cavalerie qu'on puisse avoir\,; quant à l'infanterie, l'on ne perd pas
un instant à songer aux moyens de la remettre\,; il y aura des fonds
fixes et affectés pour la guerre, qui seront inaltérables\,; et si nous
pouvons reprendre Gibraltar, on sera en état de faire une campagne
heureuse. J'espère pareillement venir à bout du commerce des Indes.
Après cela, si le roi imagine que quelqu'un fasse mieux à ma place, je
m'estimerai très heureux de me retirer, et je ne lui demande pour toute
récompense que de me rapprocher de sa personne, d'avoir encore le
plaisir, avant de mourir, de lui embrasser les genoux, et de songer
ensuite à finir comme un galant homme le doit faire.

«\,Tout ce que je vous demande là, monsieur, est d'une si terrible
conséquence pour le roi d'Espagne et pour moi, que je vous supplie qu'il
n'y ait que le roi, et vous, et M\textsuperscript{me} de Maintenon qui
le sachent. J'ai raison, monsieur, de vous en parler de la sorte. Tout
ce qui regarde la reine d'Espagne lui revient dans l'instant, je n'en
puis douter\,; ainsi les précautions doivent renouveler de jambes.
Depuis le retour de M\textsuperscript{me} des Ursins, vous ne sauriez
avoir trop d'attention et trop de secret sur ce que j'ai l'honneur de
vous dire.

«\,Monteléon part qui vous mettra bien nettement au fait de toutes ces
petites bagatelles.

«\,Si le roi savait à fond la manière fidèle et pleine d'esprit dont le
P. Daubenton le sert, et de laquelle j'ai toujours été témoin oculaire,
il ne se peut que Sa Majesté ne lui en sût un gré infini\,: je dois ce
témoignage à la vérité et au zèle d'un sujet bien attaché par le cœur à
son maître.\,»

\end{document}
